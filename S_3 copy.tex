\subsection{$S_3$ bound}
\begin{proposition}[Preliminary bound on $S_3$]\label{s_3bound}
    We have \[
    S_3 \lesssim (qT)^2 E(\mathcal{S})^{1/2} |\mathcal{S}|^{1/2}.
    \]
\end{proposition}
Recall that \begin{align*}
	I_m\defeq\frac{N^3}{q^3}\sum_{\substack{(t_1,\chi_1),\\(t_2,\chi_2),\\(t_3,\chi_3)\in\mathcal{S}}} &\sum_{x\in (\mathbb{Z}/q\mathbb{Z})^3}\chi_1\bar{\chi}_3(x_1)\chi_2\bar{\chi}_1(x_2)\chi_3\bar{\chi}_2(x_3) e\left(\frac{-x\cdot m}{q}\right)\\
	\times \ &\hat{h}_{t_1-t_3}\left(\frac{Nm_1}{q}\right)\hat{h}_{t_2-t_1}\left(\frac{Nm_2}{q}\right)\hat{h}_{t_3-t_2}\left(\frac{Nm_3}{q}\right).
\end{align*}

By non-stationary phase, $I_m$ is negligible for the terms $qT/N\lesssim |m|$, so 
\begin{equation}
    S_3 = \sum_{0<|m_1|,|m_2|,|m_3|\lesssim qT/N} I_m + O(T^{-100}).
\end{equation}
Furthermore, by symmetry in $m_1$, $m_2$, $m_3$, we can consider the terms $|m_1|\leq |m_2|\leq |m_3|$ at a cost of a factor of $6$.
We first consider the triple summation in $\mathbb{Z}/q$. Define \begin{align*}
    A_{m,\chi_1,\chi_2,\chi_3}&\defeq \sum_{x\in (\mathbb{Z}/q\mathbb{Z})^3}\chi_1\bar{\chi}_3(x_1)\chi_2\bar{\chi}_1(x_2)\chi_3\bar{\chi}_2(x_3) e\left(\frac{-x\cdot m}{q}\right),
\end{align*}
which is the inner summation in $I_m$.
Notice that $\chi(0)=0$, so we can define the summation in $A$ to run over $(\mathbb{Z}/q\mathbb{Z})^\times$ to get 
\begin{align*}
    A_{m,\chi_1,\chi_2,\chi_3}&= \sum_{x\in ((\mathbb{Z}/q\mathbb{Z})^\times)^3}\chi_1\bar{\chi}_3(x_1)\chi_2\bar{\chi}_1(x_2)\chi_3\bar{\chi}_2(x_3) e\left(\frac{-x\cdot m}{q}\right)\\
\end{align*}
We now make the substitution $y_1=x_1x_3^{-1}, y_2=x_2x_3^{-1} \mod q$ for the summation over $x$.
We thus rewrite the sum over $x$ as 
\begin{align*}
    &\sum_{y_1,y_2,x_3\in (\mathbb{Z}/q\mathbb{Z})^\times}
    \chi_1(y_1y_2^{-1})\chi_2(y_2)\chi_3(y_1^{-1})e\left(\frac{-(y_1m_1+y_2m_2+m_3)x_3}{q}\right)\\
    =&
    \sum_{y_1,y_2\in (\mathbb{Z}/q\mathbb{Z})^\times}\chi_1(y_1)\bar{\chi}_1(y_2)\chi_2(y_2)\bar{\chi}_3(y_1)\sum_{x_3\in (\mathbb{Z}/q\mathbb{Z})^\times}e\left(\frac{-(y_1m_1+y_2m_2+m_3)x_3}{q}\right)\\
    =&	\sum_{y_1,y_2\in \mathbb{Z}/q\mathbb{Z}}\chi_1(y_1)\bar{\chi}_1(y_2)\chi_2(y_2)\bar{\chi}_3(y_1)\left(-1+\sum_{x_3\in (\mathbb{Z}/q\mathbb{Z})}e\left(\frac{-(y_1m_1+y_2m_2+m_3)x_3}{q}\right)\right)\\
    =&\sum_{y_1,y_2\in\mathbb{Z}/q\mathbb{Z}}\chi_1(y_1)\bar{\chi}_1(y_2)\chi_2(y_2)\bar{\chi}_3(y_1) \Big(q \mathbb{I}_{q|(y_1m_1+y_2m_2+m_3)}-1\Big)\\
    =&  - (q-1)^2 \delta_{\chi_1\chi_2}\delta_{\chi_1\chi_3}+q \sum_{\substack{y_1,y_2 \in\mathbb{Z}/q\mathbb{Z} \\ y_1m_1+y_2m_2+m_3\equiv 0 \mod q}}\chi_1(y_1)\bar{\chi}_1(y_2)\chi_2(y_2)\bar{\chi}_3(y_1).
\end{align*}
Based on the Kronecker delta products, the first term in the last expression is non-zero exactly when $\chi_1=\chi_2=\chi_3$. We turn to the second term with double summation. 
We first suppose that $(m_1,q)=1$. Then there is exactly one value of $y_1$ for each value of $y_2$ in the summation, namely \[
y_1\equiv \frac{y_2m_2+m_3}{-m_1} \mod q.
\]
Similarly, if $(m_2,q)=1$, we can rearrange the summation to get one value of $y_2$ for each value of $y_1$,\[
y_2\equiv \frac{y_1m_1+m_3}{-m_2} \mod q.
\]
Finally, if $m_1$ and $m_2$ are multiples of $q$, then the summation is non-empty only if $m_3$ is also a multiple of $q$. In this case, the summation is over all terms $y_1,y_2$ modulo $q$ and evaluates to \[
q (q-1)^2\delta_{\chi_1\chi_2}\delta_{\chi_1\chi_3}
\] 

With this in mind, we break $I_m=I_{m,1}+I_{m,2}$,\begin{align*}
I_{m,1}\defeq&\frac{N^3}{q^3}\sum_{\substack{(t_1,\chi_1),\\(t_2,\chi_2),\\(t_3,\chi_3)\in\mathcal{S}\\
\chi_1=\chi_2=\chi_3}}(q-1)^2 \hat{h}_{t_1-t_3}\left(\frac{Nm_1}{q}\right)\hat{h}_{t_2-t_1}\left(\frac{Nm_2}{q}\right)\hat{h}_{t_3-t_2}\left(\frac{Nm_3}{q}\right),\\
I_{m,2}\defeq&\frac{N^3}{q^2}\sum_{\substack{(t_1,\chi_1),\\(t_2,\chi_2),\\(t_3,\chi_3)\in\mathcal{S}}} \sum_{\substack{y_1,y_2 \in\mathbb{Z}/q\mathbb{Z} \\ y_1m_1+y_2m_2+m_3\equiv 0 \mod q}}\chi_1(y_1)\bar{\chi}_1(y_2)\chi_2(y_2)\bar{\chi}_3(y_1)\\
&\times \hat{h}_{t_1-t_3}\left(\frac{Nm_1}{q}\right)\hat{h}_{t_2-t_1}\left(\frac{Nm_2}{q}\right)\hat{h}_{t_3-t_2}\left(\frac{Nm_3}{q}\right).
\end{align*}

\subsubsection{Sum in $I_{m,1}$}
\begin{proposition} [Sum in $I_{m,1}$ bound] \label{im1bound}
    We have \[
    \sum_{m_1,m_2,m_3\neq 0} I_{m,1}\lesssim (qT)^2E(\mathcal{S})|\mathcal{S}|^{1/2}.
    \]
\end{proposition}

For a Dirichlet character $\chi$, we define \[
R_\chi (v) \defeq\sum_{(t_1,\chi_1)\in \mathcal{S}, \chi_1=\chi} v^{it_1}.
\]
We then consider the expression \[
	\hat{h}_{t_1-t_3}\left(\frac{Nm_1}{q}\right)\hat{h}_{t_2-t_1}\left(\frac{Nm_2}{q}\right)\hat{h}_{t_3-t_2}\left(\frac{Nm_3}{q}\right).
\]
Expanding the Fourier transform as an integral, this expression equals \begin{align*}
	 &\int_{\reals^3}\mathbf{\tilde{\omega}}(\mathbf{u})u_1^{i(t_1-t_3)}u_2^{i(t_2-t_1)}u_3^{i(t_3-t_2)}e\left(\frac{-N\mathbf{m}\cdot \mathbf{u}}{q}\right)d\mathbf{u}\\
	 =&\int_{\reals^3}\mathbf{\tilde{\omega}}(\mathbf{u})\left(\frac{u_1}{u_2}\right)^{it_1}\left(\frac{u_2}{u_3}\right)^{it_2}\left(\frac{u_3}{u_1}\right)^{it_3}
	 e\left(\frac{-N\mathbf{m}\cdot \mathbf{u}}{q}\right)d\mathbf{u}
	 \end{align*}
	 where $\tilde{\omega}(u_1,u_2,u_3)\defeq \omega(u_1)^2\omega(u_2)^2\omega(u_3)^2$ is compactly supported. The observation is that the choice of $u_1/u_2$ and $u_2/u_3$ fixes $u_3/u_1$, so this triple integral can be rewritten in two variables. We change variables $v_1=u_1/u_3,v_2=u_2/u_3$ for the integral, which is well defined on the support of $\tilde{\omega}$. This gives us a Jacobian of $u_3^2$ and equals
	 \begin{equation}\label{im1integral}
        \begin{split}
	&\int_{\reals^3}\tilde{\omega}(v_1u_3,v_2u_3,u_3) {\left(\frac{v_1}{v_2}\right)}^{it_1} {\left(v_2\right)}^{it_2}{\left(\frac{1}{v_1}\right)}^{it_3} u_3^2 \ e\left(\frac{-N(v_1m_1+v_2m_2+m_3)u_3}{q}\right)\ dv_1\ dv_2\ du_3\\
	=&\int_{\reals^2}\int_\reals u_3^2 \ \tilde{\omega}(v_1u_3,v_2u_3,u_3) e\left(\frac{-N(v_1m_1+v_2m_2+m_3)u_3}{q}\right)  du_3 \ {\left(\frac{v_1}{v_2}\right)}^{it_1} {\left(v_2\right)}^{it_2}{\left(\frac{1}{v_1}\right)}^{it_3}  \ dv_1\ dv_2.\\
     \end{split}
\end{equation}
Therefore, \begin{equation*}
    I_{m,1} = \frac{N^3(q-1)^2}{q^3}\int_{\reals^2} \tilde{I}_{u_3}(v_1,v_2) \sum_{\chi} R_{\chi}{\left(\frac{v_1}{v_2}\right)}R_{\chi} {\left(v_2\right)}R_{\chi}{\left(\frac{1}{v_1}\right)} \ dv_1\ dv_2,
\end{equation*}
    where\[
     \tilde{I}_{u_3}(v_1,v_2)\defeq \tilde{I}_{u_3}(v_1,v_2,m)\defeq \int_\reals u_3^2 \ \tilde{\omega}(v_1u_3,v_2u_3,u_3) e\left(\frac{-N(v_1m_1+v_2m_2+m_3)u_3}{q}\right)  du_3\]
and $\sum_\chi$ runs over all Dirichlet characters mod $q$.
The innermost integral $\tilde{I}_{u_3}(v_1,v_2)$ has cancellation property. By the principle of non-stationary phase through repeated integration by parts, this integral is $O_{\epsilon, A}(T^{-A})$ for any $|v_1m_1+v_2m_2+m_3|>qT^\epsilon/N$. Therefore, we can truncate the domain of the integrals in $v_1$ and $v_2$ to  $|v_1m_1+v_2m_2+m_3|\lesssim q/N$ with negligible error. On this domain, the innermost integral in $u_3$ is $O(1)$ by the trivial bound. Moreover, by the compact support of $\tilde{\omega}$ on $[1,2]\times [1,2]\times [1,2]$  the integrand of innermost integral is non-zero only if \[
v_1u_3,v_2u_3,u_3\sim N.
\]
Importantly, this requires $1/2 \leq v_1,v_2 \leq 2$, so we can further restrict the outermost integrals to this region. 
We therefore have: 
\begin{proposition}
    \[
    |I_{m,1}|\ll \frac{N^3}{q}\int\displaylimits_{\substack{
        |v_1m_1+v_2m_2+m_3|\lesssim \frac{q}{N}\\
        \frac{1}{2}\leq v_1,v_2\leq 2
}} \sum_{\chi} R_{\chi}{\left(\frac{v_1}{v_2}\right)}R_{\chi} {\left(v_2\right)}R_{\chi}{\left(\frac{1}{v_1}\right)} \ dv_1\ dv_2+O_{\epsilon}(T^{-100}).
    \]
    Moreover, if $|m_1|\leq|m_2|\leq |m_3|$, $|I_m|=O(T^{-100})$ unless $|m_2|\asymp|m_3|$.
\end{proposition}
\begin{proof}
    The first statement is a result of our work so far. The second part of the proposition follows from the integral bounds $|v_1m_1+v_2m_2+m_3|\lesssim q/N$
    and $v_1,v_2\asymp 1$. When $|m_1|\leq |m_2|$, we have $|v_1m_1+v_2m_2|=O(|m_2|)$. These force $|m_2| \asymp|m_3|$, or else the integral will be negligible as the domain of the integration will be outside the range $|v_1m_1+v_2m_2+m_3|\lesssim q/N$.
\end{proof}
    \iffalse
    \begin{proposition}
        \label{doubleintegrals3}
        \[
        |I_m|\ll \phi(q) \frac{N^3}{q^3}  
        \sum_{y_1,y_2\in (\mathbb{Z}/q\mathbb{Z})^\times}\ \int\displaylimits_{\substack{
            |v_1m_1+v_2m_2+m_3|\lesssim \frac{q}{N}\\
            \frac{1}{2}\leq v_1,v_2\leq 2
        }} \left| R\left(\frac{v_2}{v_1},y_2,y_1\right)
        R(v_2,y_2)R\left(v_1,y_1\right)\right| dv_1 \ dv_2  + O(T^{-100}).\\
        \]
        Moreover, if $|m_1|\leq|m_2|\leq |m_3|$, $|I_m|=O(T^{-100})$ unless $|m_2|\asymp|m_3|$.
    \end{proposition}
    \begin{proof}
        
    Recall
    \begin{align*}
        I_m=\frac{N^3}{q^3}\sum_{\substack{(t_1,\chi_1),\\(t_2,\chi_2),\\(t_3,\chi_3)\in\mathcal{S}}} &\sum_{x\in (\mathbb{Z}/q\mathbb{Z})^3}\chi_1\bar{\chi}_3(x_1)\chi_2\bar{\chi}_1(x_2)\chi_3\bar{\chi}_2(x_3) e\left(\frac{-x\cdot m}{q}\right)\\
        \times \ &\hat{h}_{t_1-t_3}\left(\frac{Nm_1}{q}\right)\hat{h}_{t_2-t_1}\left(\frac{Nm_2}{q}\right)\hat{h}_{t_3-t_2}\left(\frac{Nm_3}{q}\right).
    \end{align*}
    Expanding the integrals, 
    \begin{align*}
        I_m=\frac{N^3}{q^3}\sum_{\substack{(t_1,\chi_1),\\(t_2,\chi_2),\\(t_3,\chi_3)\in\mathcal{S}}} &\sum_{x\in (\mathbb{Z}/q\mathbb{Z})^3}\chi_1\bar{\chi}_3(x_1)\chi_2\bar{\chi}_1(x_2)\chi_3\bar{\chi}_2(x_3) e\left(\frac{-x\cdot m}{q}\right)\\
        \times \ &
        \int_{\reals^3}\mathbf{\tilde{\omega}}(\mathbf{u})u_1^{i(t_1-t_3)}u_2^{i(t_2-t_1)}u_3^{i(t_3-t_2)}e\left(\frac{-N\mathbf{m}\cdot \mathbf{u}}{q}\right)d\mathbf{u},
    \end{align*}
    where $\tilde{\omega}(\mathbf{u})=\omega(u_1)^2\omega(u_2)^2\omega(u_3)^2$ is compactly supported.
    We now make the substitution $y_1=x_1x_3^{-1}, y_2=x_2x_3^-1 \mod q$ for the summation over $x$, and $v_1=u_1/u_3,v_2=u_2/u_3$ for the integral on the support of $\tilde{\omega}$.
    We thus rewrite the sum over $x$ as 
    \begin{align*}
        &\sum_{y_1,y_2,x_3\in (\mathbb{Z}/q\mathbb{Z})^\times}
        \chi_1(y_1y_2^{-1})\chi_2(y_2)\chi_3(y_1^{-1})e\left(\frac{-(y_1m_1+y_2m_2+m_3)x_3}{q}\right)\\
        =&
        \sum_{y_1,y_2\in (\mathbb{Z}/q\mathbb{Z})^\times}\chi_1(y_1)\bar{\chi}_1(y_2)\chi_2(y_)\bar{\chi}_3(y_1)\sum_{x_3\in (\mathbb{Z}/q\mathbb{Z})^\times}e\left(\frac{-(y_1m_1+y_2m_2+m_3)x_3}{q}\right),
    \end{align*}
    where we can use the trivial bound $\phi(q)$ for the innermost sum.
    We also rewrite triple integral as 
    \begin{align*}
        &\int_{\reals^3}\tilde{\omega}(v_1u_3,v_2u_3,u_3) {\left(\frac{v_1}{v_2}\right)}^{it_1} {\left(v_2\right)}^{it_2}{\left(\frac{1}{v_1}\right)}^{it_3} u_3^2 \ e\left(\frac{-N(v_1m_1+v_2m_2+m_3)u_3}{q}\right)\ dv_1\ dv_2\ du_3\\
        =&\int_{\reals^2}\int_\reals u_3^2 \ \tilde{\omega}(v_1u_3,v_2u_3,u_3) e\left(\frac{-N(v_1m_1+v_2m_2+m_3)u_3}{q}\right)  du_3 \ {\left(\frac{v_1}{v_2}\right)}^{it_1} {\left(v_2\right)}^{it_2}{\left(\frac{1}{v_1}\right)}^{it_3}  \ dv_1\ dv_2.\\
    \end{align*}
    The integrand of the innermost integral is non-zero only if \[
        v_1u_3,v_2u_3,u_3\sim N.
    \]
    Importantly, this requires $1/2 \leq v_1,v_2 \leq 2$, so we can truncate the outermost integrals to these regions. Moreover, by repeated integration by parts, this integral is $O_{\epsilon, A}(T^{-A})$ for any $|v_1m_1+v_2m_2+m_3|>qT^\epsilon/N$.
    So 
    \begin{align*}
        |I_m|\ll &\phi(q) \frac{N^3}{q^3}  
        \sum_{y_1,y_2\in (\mathbb{Z}/q\mathbb{Z})^\times}\left|\ \int\displaylimits_{\substack{
            |v_1m_1+v_2m_2+m_3|\lesssim \frac{q}{N}\\
            \frac{1}{2}\leq v_1,v_2\leq 2
        }} R\left(\frac{v_1}{v_2},y_1,y_2\right)
        R(v_2,y_2)R\left(\frac{1}{v_1},1,y_1\right) dv_1 \ dv_2 \right|\\ &+ O(T^{-100}).\\
    \end{align*}
    Since $|R(v_1^{-1},1,y_1)|=|R(v_1,y_1)|$, $| R\left(\frac{v_1}{v_2},y_1,y_2\right)| = R\left(\frac{v_2}{v_1},y_2,y_1\right)|$, we have the first part of the proposition.
    The second part of the proposition follows from the integral bounds $|v_1m_1+v_2m_2+m_3|\lesssim q/N$
    and $v_1,v_2\asymp 1$. These force $|m_2| \asymp|m_3|$, or else the integral will be zero.

    \end{proof}
\fi
\iffalse
We define \begin{align*}
	A_{m,\chi_1,\chi_2,\chi_3}&\defeq \sum_{x\in (\mathbb{Z}/q\mathbb{Z})^3}\chi_1\bar{\chi}_3(x_1)\chi_2\bar{\chi}_1(x_2)\chi_3\bar{\chi}_2(x_3) e\left(\frac{-x\cdot m}{q}\right)\\
	R(v,n_1,n_2)&\defeq \sum_{(t,\chi)\in \mathcal{S}} 
	\chi({n_1})\bar{\chi}(n_2)v^{it},\\
	R(v,n)&\defeq R(v,n,1).
\end{align*}
\fi
When $|m_2|\asymp|m_3|$, the domain of integration can be written as\begin{align*}
    |v_1m_1+v_2m_2+m_3|\lesssim \frac{q}{N} \implies \left|v_2 - \frac{v_1m_1+m_3}{-m_2}\right|\lesssim \frac{q}{|m_2|N} \asymp \frac{q}{|m_3|N}.
\end{align*}
Thus, if we fix $v_1$, the integration in $v_2$ is in a ${q}/{(|m_3|N)}$-small neighborhood of $\frac{v_1m_1+m_3}{-m_2}$. In principle, we can estimate the value of the integral by evalulating at 
$v_2 = {v_1m_1+m_3}/{-m_2}$ and multiplying it by $q/(|m_3|N)$. This is made precise by splitting the sum across $m_3$ into dyadic interals $M$, and smoothing over ranges of ${q}/{(MN)}$.

Let $\tilde{\phi}$ be a smooth bump function such that equals $\tilde{\phi}=1$ on $|x|\lesssim 1$ and is supported in $|x|\lesssim 1$, with a larger constant, so that $\|\tilde{\phi}^{(j)}\lesssim_j 1\|$ for all $j$.

We define \[
\tilde{R}_{\chi,M}(v) \defeq \left( \int \frac{NM}{q}\tilde{\phi}\left(\frac{NM}{q}(v-v')\right)|{R}_{\chi,M}(v')|^2 dv'\right)^{1/2}.
\]
\begin{proposition} \label{dyadics_31}
    Let \[
    \tilde{I}_{m,1}\defeq 
    %\frac{N^3}{q}
    \int\displaylimits_{v_1\asymp 1} \sum_{\chi} \left|R_\chi\left(v_1\right) \tilde{R}_{\chi,M}\left(\frac{m_1v_1+m_3}{-m_2v_1}\right)
\tilde{R}_{\chi,M}\left(\frac{m_1v_1+m_3}{-m_2}\right)\right|dv_1.
    \]
    Then there is a choice of $0<M_1\leq M \lesssim qT/N$ such that \[
        \sum_{0<|m_1|,|m_2|,|m_3|\lesssim \frac{qT}{N}} I_{m,1}\lesssim \frac{N^2}{M}\sum_{|m_1|\sim M_1,|m_2|,|m_3|\sim M}\tilde{I}_{m,1}+O_\epsilon(T^{-100}).
    \]
\end{proposition}

\begin{proof}
    At the cost of a factor of $6$, we consider the terms $|m_1|\leq |m_2|\asymp |m_3|$, and $|m_2|\leq|m_3|$. Expanding the sum over $m_1,m_2,m_3$ dyadically, we get for some $M_1\leq M \lesssim qT/N $ \begin{align*}
        \sum_{0<|m_1|,|m_2|,|m_3|\lesssim \frac{qT}{N}} I_{m,1}\lesssim \sum_{\substack{|m_1|\sim M_1,\\|m_2|,|m_3|\sim M}}|I_{m,1}| + O_\epsilon(T^{-100})
    \end{align*}
    For these values of $M$ and $M_1$, we now consider as $|R_{\chi}(v)|=|R_\chi(v^{-1})|$,
    \begin{align*} 
        &\int\displaylimits_{\substack{
        |v_1m_1+v_2m_2+m_3|\lesssim \frac{q}{N}\\
        \frac{1}{2}\leq v_1,v_2\leq 2
            }}\left|R_{\chi} {\left(\frac{v_1}{v_2}\right)}R_{\chi} {\left(v_2\right)}R_{\chi}{\left(\frac{1}{v_1}\right)}\right| \ dv_1\ dv_2\\
        =&\int\displaylimits_{\substack{
            |v_1m_1+v_2m_2+m_3|\lesssim \frac{q}{N}\\
            \frac{1}{2}\leq v_1,v_2\leq 2
    }}\left|R_{\chi} {\left(\frac{v_2}{v_1}\right)}R_{\chi} {\left(v_2\right)}R_{\chi}{\left({v_1}\right)}\right| \ dv_1\ dv_2\\
        \ll& 
        \int_{v_1\asymp 1}\left|R_{\chi}{\left({v_1}\right)}\right|
        \int\displaylimits_{\left|v_2 - \frac{v_1m_1+m_3}{-m_2}\right|\lesssim \frac{q}{|m_2|N}} \left| R_{\chi}\left(\frac{v_2}{v_1}\right)
        R_{\chi}(v_2)\right|  dv_2 \ dv_1\\
        \ll& 
        \int_{v_1\asymp 1}\left|R_{\chi}{\left({v_1}\right)}\right|
        \int\displaylimits_{\left|v_2 - \frac{v_1m_1+m_3}{-m_2}\right|\lesssim \frac{q}{MN}} \left| R_{\chi}\left(\frac{v_2}{v_1}\right)
        R_{\chi}(v_2)\right|  dv_2 \ dv_1\\
    \end{align*}
    The inner integral, by Cauchy-Schwarz,
    is \begin{align*}
        \leq & \Bigg(\int\displaylimits_{\left|v_2 - \frac{v_1m_1+m_3}{-m_2}\right|\lesssim \frac{q}{MN}} \left| R_{\chi}\left(\frac{v_2}{v_1}\right)\right|^2 \ dv_2 \Bigg)^{1/2}
        \Bigg(
        \int\displaylimits_{\left|v_2 - \frac{v_1m_1+m_3}{-m_2}\right|\lesssim \frac{q}{MN}} \left|
        R_{\chi}(v_2)\right|^2  dv_2\Bigg)^{1/2}\\
        \ll& \frac{q}{MN} \tilde{R}_{\chi,M}\left(\frac{v_1m_1+m_3}{-m_2v_1},y_2,y_1\right)  \tilde{R}_{\chi,M}\left(\frac{v_1m_1+m_3}{-m_2}\right)
    \end{align*}
    where in the last step, we used $v_1\asymp 1$ for the change of variables $v'=v_2/v_1$.
    Thus,
    \[
    |I_{m,1}|\lesssim \frac{N^2}{M}\tilde{I}_{m,1}.
    \]
    The proposition follows from this claim.
    \end{proof}
    
    \begin{lemma} \label{secondmoment1}
        Let $\mathcal{S}=\{(t_j,\chi_j)\}$, and the $t$'s are contained in an interval of length $T$, and are $T^\epsilon$-separated for the same character. Then \[
            \sum_{\chi} \int_{v\asymp 1} 
            \left|R_{\chi}\left(v\right)\right|^2dv \ll |\mathcal{S}|+ O_{\epsilon}(T^{-100}).
        \]
    \end{lemma}
    \begin{proof}
        
        Let $\psi$ be a bump function supported on $v\asymp 1$ and equals $1$ on the domain of integration in the lemma. Then
       \begin{align*}
        &\sum_{\chi} \int_{v\asymp 1} 
            \left|R_{\chi}\left(v\right)\right|^2dv\\
            =&\int_{v\asymp 1} \psi(v)\sum_{\substack{(t_1,\chi_1),(t_2,\chi_2)\in \mathcal{S}\\ \chi_1=\chi_2}}v^{i(t_1-t_2)}dv \\
            \leq&\int 
        \psi(v)\sum_{\substack{(t_1,\chi_1),(t_2,\chi_2)\in \mathcal{S}\\ \chi_1=\chi_2}}v^{i(t_1-t_2)}dv
        \end{align*}
        In the sum, the terms $t_1=t_2$ contribute $O(|\mathcal{S}|)$. If $t_1\neq t_2$, then $|t_1-t_2|\geq T^\epsilon$. The integral in this case is $O_\epsilon(T^{-100})$ and is negligible.
    \end{proof}
    \begin{lemma}\label{fourthmoment1}
        Let $E(\mathcal{S})=\#\{(t_1,\chi_1),(t_2,\chi_2),(t_3,\chi_3),(t_4,\chi_4)\in \mathcal{S}  :  |t_1+t_2-t_3-t_4|\lesssim 1, \chi_1\chi_2=\chi_3\chi_4\}$. Then \[
            \sum_{\chi} \int_{v\asymp 1} 
            \left|R_{\chi}\left(v\right)\right|^4dv \lesssim E(\mathcal{S}) + O_{\epsilon}(T^{-100}).
        \]
    \end{lemma}
    \begin{proof}
        We have
        \begin{align*}
            \sum_{\chi} \int_{v\asymp 1} 
            \left|R\left(v,y\right)\right|^4dv = & 
            \sum_{\substack{(t_1,\chi_1),(t_2,\chi_2),\\ (t_3,\chi_3),(t_4,\chi_4)\in \mathcal{S}\\ \chi_1=\chi_2=\chi_3=\chi_4}} \int_{v\asymp 1} v^{i(t_1+t_2-t_3-t_4)} dv.
        \end{align*}
        Similar to the previous proof, we can construct a bump function for the integral in the same way, and restrict the summation to the terms $|t_1+t_2-t_3-t_4|\leq T^\epsilon$ with an error of $O_\epsilon(T^{-100})$. The remaining terms in the summation contribute $O(E(\mathcal{S}))$, as $\chi_1=\chi_2=\chi_3=\chi_4$ implies $\chi_1\chi_2=\chi_3\chi_4$.
    \end{proof}
    \begin{lemma}\label{fourthmoment_smooth1}
        Let $E(\mathcal{S})=\#\{(t_1,\chi_1),(t_2,\chi_2),(t_3,\chi_3),(t_4,\chi_4)\in \mathcal{S}  :  |t_1+t_2-t_3-t_4|\lesssim 1, \chi_1\chi_2=\chi_3\chi_4\}$. Then \[
            \sum_{\chi} \int_{v\asymp 1} 
            \left|\tilde{R}_{M,\chi}\left(v\right)\right|^4dv \lesssim E(\mathcal{S}) + O_{\epsilon}(T^{-100}).
        \]
    \end{lemma}
    \begin{proof}
        We apply Cauchy-Schwarz to \begin{align*}
            \int_{v\asymp 1} 
            \left|\tilde{R}_{M,\chi}(v)\right|^4dv  \lesssim& \int_{v\asymp 1} 
            \left(\int_{|u-v|\lesssim q/NM}
            \frac{NM}{q}|R_{\chi}(u)|^2 du\right)^2
            dv \\
            \lesssim& \frac{NM}{q} \int_{v\asymp 1} 
           \int_{|u-v|\lesssim q/NM}
            |R(u)|^4 du \ 
            dv\\
            \lesssim&  
            \int_{u\asymp 1}
             |R_{\chi}(u)|^4 du.
        \end{align*}
        Lemma \ref{fourthmoment1} completes the proof.
    \end{proof}
\begin{proof}[Proof of Proposition \ref{im1bound}]\label{im1boundproof}
    We apply H\"older's inequality to $\tilde{I}_{m,1}$, so that \begin{align*}  
        &\int\displaylimits_{v_1\asymp 1} \sum_{\chi} \left|R_\chi\left(v_1\right) \tilde{R}_{\chi,M}\left(\frac{m_1v_1+m_3}{-m_2v_1}\right)
        \tilde{R}_{\chi,M}\left(\frac{m_1v_1+m_3}{-m_2}\right)\right|dv_1 \\\leq& 
        \Bigg(\int\displaylimits_{v_1\asymp 1} \sum_{\chi} |R_\chi\left(v_1\right)|^2 dv_1 \Bigg)^{1/2}
        \Bigg(\int\displaylimits_{v_1\asymp 1} \sum_{\chi} \left|\tilde{R}_{\chi,M}\left(\frac{m_1v_1+m_3}{-m_2v_1}\right)\right|^4dv_1 \Bigg)^{1/4}
        \Bigg(\int\displaylimits_{v_1\asymp 1} \sum_{\chi}\left| \tilde{R}_{\chi,M}\left(\frac{m_1v_1+m_3}{-m_2}\right)\right|^4dv_1 \Bigg)^{1/4}\\
        \lesssim & E(\mathcal{S})^{1/2}|\mathcal{S}|^{1/2},
    \end{align*}
    where in the last step, we apply change of variables $u=(m_1v_1+m_3)/(-m_2v_1)$ and $u=(m_1v_1+m_3)/(-m_2)$ respectively, with a Jacobian factor of $\asymp 1$.
    We now sum this over $M_1M^2$ terms, as in Proposition \ref{dyadics_31},
    so that \begin{align*}
        \sum_{0<|m_1|,|m_2|,|m_3|\lesssim \frac{qT}{N}} I_{m,1}\lesssim& \frac{N^2}{M}\sum_{|m_1|\sim M_1,|m_2|,|m_3|\sim M}\tilde{I}_{m,1}+O_\epsilon(T^{-100})\\
        \lesssim & N^2M_1ME(\mathcal{S})^{1/2}|\mathcal{S}|^{1/2}\\
        \lesssim & (qT)^2E(\mathcal{S})^{1/2}|\mathcal{S}|^{1/2}.
    \end{align*}
\end{proof}

\subsubsection{Sum in $I_{m,2}$}
\begin{proposition} [Sum in $I_{m,2}$ bound] \label{im2bound}
    We have \[
    \sum_{m_1,m_2,m_3\neq 0} I_{m,2}\lesssim (qT)^2E(\mathcal{S})|\mathcal{S}|^{1/2}.
    \]
\end{proposition}
Recall that
\begin{align*}I_{m,2}\defeq&\frac{N^3}{q^2}\sum_{\substack{(t_1,\chi_1),\\(t_2,\chi_2),\\(t_3,\chi_3)\in\mathcal{S}}} \sum_{\substack{y_1,y_2 \in\mathbb{Z}/q\mathbb{Z} \\ y_1m_1+y_2m_2+m_3\equiv 0 \mod q}}\chi_1(y_1)\bar{\chi}_1(y_2)\chi_2(y_2)\bar{\chi}_3(y_1)\\
    &\times \hat{h}_{t_1-t_3}\left(\frac{Nm_1}{q}\right)\hat{h}_{t_2-t_1}\left(\frac{Nm_2}{q}\right)\hat{h}_{t_3-t_2}\left(\frac{Nm_3}{q}\right).
    \end{align*}
The sum in $y_1,y_2$ behaves differently depending on $(m_1, q)$ and $(m_2, q)$.
\begin{proposition}
    We have
    \begin{align*}
        &\sum_{\substack{y_1,y_2 \in\mathbb{Z}/q\mathbb{Z} \\ y_1m_1+y_2m_2+m_3\equiv 0 \mod q}}\chi_1(y_1)\bar{\chi}_1(y_2)\chi_2(y_2)\bar{\chi}_3(y_1)\\=&
        \begin{dcases*}
            \sum_{\substack{y_1\in (\mathbb{Z}/q\mathbb{Z})^\times\\y_2=\frac{y_1m_1+m_3}{-m_2}}}\chi_1(y_1)\bar{\chi}_1(y_2)\chi_2(y_2)\bar{\chi}_3(y_1), &if $(m_1,q)=(m_2,q)=1$;\\
            \sum_{\substack{y_1\in (\mathbb{Z}/q\mathbb{Z})^\times}}\chi_1(y_1)\bar{\chi}_1(-m_3m_2^{-1})\chi_2(-m_3m_2^{-1})\bar{\chi}_3(y_1)& if $(m_1,q)=q, (m_2,q)=1$;\\
            \sum_{\substack{y_2\in (\mathbb{Z}/q\mathbb{Z})^\times}}\chi_1(-m_3m_1^{-1})\bar{\chi}_1(y_2)\chi_2(y_2)\bar{\chi}_3(-m_3m_1^{-1})& if $(m_1,q)=1,(m_2,q)=q$;\\
            (q-1)^2\delta_{\chi_1\chi_2}\delta_{\chi_1\chi_3}\mathbb{I}_{q|m_3}, & if $ (m_1,q)=(m_2,q)=q.$\\
        \end{dcases*}
    \end{align*}
\end{proposition} 
\begin{proof}
    If $(m_1,q)=(m_2,q)=1$, we have \begin{align*}
        \sum_{\substack{y_1,y_2 \in\mathbb{Z}/q\mathbb{Z} \\ y_1m_1+y_2m_2+m_3\equiv 0 \mod q}}\chi_1(y_1)\bar{\chi}_1(y_2)\chi_2(y_2)\bar{\chi}_3(y_1)=
        \sum_{\substack{y_1\in (\mathbb{Z}/q\mathbb{Z})^\times\\y_2=\frac{y_1m_1+m_3}{-m_2}}}\chi_1(y_1)\bar{\chi}_1(y_2)\chi_2(y_2)\bar{\chi}_3(y_1).
    \end{align*}
    
    If $(m_1,q)=q, \ (m_2,q)=1$, we have \begin{align*}
        \sum_{\substack{y_1,y_2 \in\mathbb{Z}/q\mathbb{Z} \\ y_1m_1+y_2m_2+m_3\equiv 0 \mod q}}\chi_1(y_1)\bar{\chi}_1(y_2)\chi_2(y_2)\bar{\chi}_3(y_1)=&
        \sum_{\substack{y_1\in (\mathbb{Z}/q\mathbb{Z})^\times\\y_2=\frac{y_1m_1+m_3}{-m_2}}}\chi_1(y_1)\bar{\chi}_1(y_2)\chi_2(y_2)\bar{\chi}_3(y_1)\\
        =&\sum_{\substack{y_1\in (\mathbb{Z}/q\mathbb{Z})^\times}}\chi_1(y_1)\bar{\chi}_1(-m_3m_2^{-1})\chi_2(-m_3m_2^{-1})\bar{\chi}_3(y_1).
    \end{align*}
    Similarly, for $(m_1,q)=1, \ (m_2,q)=q$, we have\begin{align*}
        \sum_{\substack{y_1,y_2 \in\mathbb{Z}/q\mathbb{Z} \\ y_1m_1+y_2m_2+m_3\equiv 0 \mod q}}\chi_1(y_1)\bar{\chi}_1(y_2)\chi_2(y_2)\bar{\chi}_3(y_1)=&
        \sum_{\substack{y_2\in (\mathbb{Z}/q\mathbb{Z})^\times\\y_1=\frac{y_2m_2+m_3}{-m_1}}}\chi_1(y_1)\bar{\chi}_1(y_2)\chi_2(y_2)\bar{\chi}_3(y_1)\\
        =&\sum_{\substack{y_2\in (\mathbb{Z}/q\mathbb{Z})^\times}}\chi_1(-m_3m_1^{-1})\bar{\chi}_1(y_2)\chi_2(y_2)\bar{\chi}_3(-m_3m_1^{-1}).
    \end{align*}

    If $(m_1,q)=(m_2,q)=q$, we have, by the orthogonality of characters, \begin{align*}
        \sum_{\substack{y_1,y_2 \in\mathbb{Z}/q\mathbb{Z} \\ y_1m_1+y_2m_2+m_3\equiv 0 \mod q}}\chi_1(y_1)\bar{\chi}_1(y_2)\chi_2(y_2)\bar{\chi}_3(y_1)=(q-1)^2\delta_{\chi_1\chi_2}\delta_{\chi_1\chi_3}\mathbb{I}_{q|m_3}
    \end{align*}

\end{proof}
 Moreover, We define \begin{align*}
	R(v,n_1,n_2)&\defeq \sum_{(t,\chi)\in \mathcal{S}} 
	\chi({n_1})\bar{\chi}(n_2)v^{it},\\
	R(v,n)&\defeq R(v,n,1).
\end{align*}, and repeat the argument in equation \ref{im1integral} and the previous proposition to get \iffalse \begin{align*}
    I_{m,2} \ll \begin{dcases*}
        \frac{N^3}{q^2}\sum_{\substack{y_1\in (\mathbb{Z}/q\mathbb{Z})^\times\\y_2=\frac{y_1m_1+m_3}{-m_2}}} \int\displaylimits_{\substack{
            |v_1m_1+v_2m_2+m_3|\lesssim \frac{q}{N}\\
            \frac{1}{2}\leq v_1,v_2\leq 2
        }} \left| R\left(\frac{v_2}{v_1},y_2,y_1\right)
        R(v_2,y_2)R\left(v_1,y_1\right)\right| dv_1 \ dv_2, &if $(m_1,q)=(m_2,q)=1$;\\
        \frac{N^3}{q^2}\sum_{\substack{y_1\in (\mathbb{Z}/q\mathbb{Z})^\times}}\int\displaylimits_{\substack{
            |v_1m_1+v_2m_2+m_3|\lesssim \frac{q}{N}\\
            \frac{1}{2}\leq v_1,v_2\leq 2
        }} \left| R\left(\frac{v_2}{v_1},-m_3m_2^{-1},y_1\right)
        R(v_2,-m_3m_2^{-1})R\left(v_1,y_1\right)\right| dv_1 \ dv_2
        & if $(m_1,q)=q, (m_2,q)=1$;\\
        \frac{N^3}{q^2}\sum_{\substack{y_2\in (\mathbb{Z}/q\mathbb{Z})^\times}}\int\displaylimits_{\substack{
            |v_1m_1+v_2m_2+m_3|\lesssim \frac{q}{N}\\
            \frac{1}{2}\leq v_1,v_2\leq 2
        }} \left| R\left(\frac{v_2}{v_1},y_2,-m_3m_1^{-1}\right)
        R(v_2,y_2)R\left(v_1,-m_3m_1^{-1}\right)\right| dv_1 \ dv_2& if $(m_1,q)=1,(m_2,q)=q$;\\
        N^3\mathbb{I}_{q|m_3} \sum_{\chi}\int\displaylimits_{\substack{
            |v_1m_1+v_2m_2+m_3|\lesssim \frac{q}{N}\\
            \frac{1}{2}\leq v_1,v_2\leq 2
    }} \left| R_{\chi}{\left(\frac{v_2}{v_1}\right)}R_{\chi} {\left(v_2\right)}R_{\chi}{\left(v_1\right)} \right|\ dv_1\ dv_2, & if $ (m_1,q)=(m_2,q)=q.$\\
    \end{dcases*}
\end{align*}
\fi

\begin{align*}
    I_{m,2} \ll \begin{dcases*}
        \frac{N^3}{q^2}\sum_{\substack{y_1\in (\mathbb{Z}/q\mathbb{Z})\\y_2=\frac{y_1m_1+m_3}{-m_2}}} \ \int\displaylimits_{\substack{
            |v_1m_1+v_2m_2+m_3|\lesssim \frac{q}{N}\\
            \frac{1}{2}\leq v_1,v_2\leq 2
        }} \left| R\left(\frac{v_2}{v_1},y_2,y_1\right)
        R(v_2,y_2)R\left(v_1,y_1\right)\right| dv_1 \ dv_2, &if $(m_1,q)=(m_2,q)=1$;\\
        \frac{N^3}{q^2}\sum_{\substack{y_1\in (\mathbb{Z}/q\mathbb{Z})}}\ \int\displaylimits_{\substack{
            |v_1m_1+v_2m_2+m_3|\lesssim \frac{q}{N}\\
            \frac{1}{2}\leq v_1,v_2\leq 2
        }} \left| R\left(\frac{v_2}{v_1},-\frac{m_3}{m_2},y_1\right)
        R(v_2,-\frac{m_3}{m_2})R\left(v_1,y_1\right)\right| dv_1 \ dv_2
        ,& if $(m_1,q)=q, (m_2,q)=1$;\\
        \frac{N^3}{q^2}\sum_{\substack{y_2\in (\mathbb{Z}/q\mathbb{Z})}}\ \int\displaylimits_{\substack{
            |v_1m_1+v_2m_2+m_3|\lesssim \frac{q}{N}\\
            \frac{1}{2}\leq v_1,v_2\leq 2
        }} \left| R\left(\frac{v_2}{v_1},y_2,-\frac{m_3}{m_1}\right)
        R(v_2,y_2)R\left(v_1,-\frac{m_3}{m_1}\right)\right| dv_1 \ dv_2,& if $(m_1,q)=1,(m_2,q)=q$;\\
        N^3\mathbb{I}_{q|m_3} \sum_{\chi}\ \int\displaylimits_{\substack{
            |v_1m_1+v_2m_2+m_3|\lesssim \frac{q}{N}\\
            \frac{1}{2}\leq v_1,v_2\leq 2
    }} \left| R_{\chi}{\left(\frac{v_2}{v_1}\right)}R_{\chi} {\left(v_2\right)}R_{\chi}{\left(v_1\right)} \right|\ dv_1\ dv_2, & if $ (m_1,q)=(m_2,q)=q.$\\
    \end{dcases*}
\end{align*}

Therefore, we split the summation of $I_{m,2}$ in $m$ into four pieces. $\Sigma_{11}$ sums over all $(m_1,q)=(m_2,q)=1$, $\Sigma_{1,q}$ sums over $(m_1,q)=1, (m_2,q)=q$, $\Sigma_{q,1}$ sums over $(m_1,q)=q,(m_2,q)=1$, and $\Sigma_{q,q}$ sums over $(m_1,q)=(m_2,q)=q$. Similar to how we dealt with $I_{m,1}$ terms, we can only consider $0<|m_1|,|m_2|,|m_3|\lesssim qT/N$ terms in the sum with an error of $O_\epsilon({T^{-100}})$ from applying Non-stationary phase.

Notice that the domain (and the overall structure) of the integrands resemble that of $I_{m,1}$, so we can define \[
\tilde{R}_M(v,y_1,y_2) \defeq \left( \int \frac{NM}{q}\tilde{\phi}\left(\frac{NM}{q}(v-v')\right)|R(v',y_1,y_2)|^2 dv'\right)^{1/2}.
\]
We can therefore split the summation in $0<|m_1|,|m_2|,|m_3|\lesssim qT/N$ in dyadic intervals, and for the choice of $M_1$ and $M$, replace the double integrals of $R$ in $v_1$ and $v_2$ with $\tilde{R}_M$ evaluated at $v_2=(v_1m_1+m_3/(-m_2))$. This leads to the following proposition.
\begin{proposition}\label{dyadics_32}
    Let \[
    \tilde{I}_{m,2}\defeq  \begin{dcases*}
        \frac{N^3}{q^2}\sum_{\substack{y_1\in (\mathbb{Z}/q\mathbb{Z})\\y_2=\frac{y_1m_1+m_3}{-m_2}}} \ \int\displaylimits_{v_1\asymp 1}  \left| \tilde{R}_M\left(\frac{m_1v_1+m_3}{-m_2v_1},y_2,y_1\right)
        \tilde{R}_M(\frac{m_1v_1+m_3}{-m_2},y_2)R\left(v_1,y_1\right)\right| dv_1, \\ \textrm{if } (m_1,q)=(m_2,q)=1;\\ \\
        \frac{N^3}{q^2}\sum_{\substack{y_1\in (\mathbb{Z}/q\mathbb{Z})}} \ \int\displaylimits_{v_1\asymp 1}  \left| \tilde{R}_M\left(\frac{m_1v_1+m_3}{-m_2v_1},-\frac{m_3}{m_2},y_1\right)
        \tilde{R}_M(\frac{m_1v_1+m_3}{-m_2},-\frac{m_3}{m_2})R\left(v_1,y_1\right)\right| dv_1 
        ,\\\textrm{if }(m_1,q)=q, (m_2,q)=1;\\ \\ 
        \frac{N^3}{q^2}\sum_{\substack{y_2\in (\mathbb{Z}/q\mathbb{Z})}}\ \int\displaylimits_{v_1\asymp 1}  \left| \tilde{R}_M\left(\frac{m_1v_1+m_3}{-m_2v_1},y_2,-\frac{m_3}{m_1}\right)
        \tilde{R}_M(\frac{m_1v_1+m_3}{-m_2},y_2)R\left(v_1,-\frac{m_3}{m_1}\right)\right| dv_1 ,\\\textrm{if }(m_1,q)=1,(m_2,q)=q;\\ \\ 
        N^3\mathbb{I}_{q|m_3} \sum_{\chi}\ \int\displaylimits_{v_1\asymp 1}  \left| \tilde{R}_{\chi,M}{\left(\frac{m_1v_1+m_3}{-m_2v_1}\right)}\tilde{R}_{\chi,M} {\left(\frac{m_1v_1+m_3}{-m_2}\right)}R_{\chi}{\left(v_1\right)} \right|\ dv_1,\\  \textrm{if }  (m_1,q)=(m_2,q)=q.
    \end{dcases*}
    \]
    Then there is a choice of $0<M_1\leq M \lesssim qT/N$ such that \[
        \sum_{0<|m_1|,|m_2|,|m_3|\lesssim \frac{qT}{N}} I_{m,2}\lesssim \frac{q}{NM}\sum_{\substack{|m_1|\sim M_1,\\|m_2|,|m_3|\sim M}}\tilde{I}_{m,2}+O_\epsilon(T^{-100}).
    \]
\end{proposition}
\begin{remark}
    Notice we can recover
    \[
    \tilde{I}_{m_2}= \frac{N^3}{q^2} \sum_{\substack{y_1,y_2 \in\mathbb{Z}/q\mathbb{Z} \\ y_1m_1+y_2m_2+m_3\equiv 0 \mod q}}\ \int_{v_1\asymp 1} \left| \tilde{R}_M\left(\frac{m_1v_1+m_3}{-m_2v_1},y_2,y_1\right)
    \tilde{R}_M(\frac{m_1v_1+m_3}{-m_2},y_2)R\left(v_1,y_1\right)\right| dv_1.
    \]
\end{remark}
\begin{proof}
    The proof follow the exact same structure as Proposition \ref{dyadics_31}.
\end{proof}
\iffalse
We define \[
\tilde{R}_M(v,y_1,y_2) \defeq \left( \int \frac{NM}{q}\tilde{\phi}\left(\frac{NM}{q}(v-v')\right)|R(v',y_1,y_2)|^2 dv'\right)^{1/2}.
\]
\begin{proposition} \label{dyadics_3}
    Let \[
    \tilde{I}_{m_2}\defeq \sum_{y_1,y_2\in (\mathbb{Z}/q\mathbb{Z})^\times}\ \int_{v_1\asymp 1} \left| \tilde{R}_M\left(\frac{m_1v_1+m_3}{-m_2v_1},y_2,y_1\right)
    \tilde{R}_M(\frac{m_1v_1+m_3}{-m_2},y_2)R\left(v_1,y_1\right)\right| dv_1.
    \]
    There is a choice of $0<M_1\leq M \lesssim qT/N$ such that \[
        S_3\lesssim \phi(q)\frac{N^2}{Mq^2}\sum_{|m_1|\sim M_1,|m_2|,|m_3|\sim M}\tilde{I}_m+O(T^{-100}).
    \]
\end{proposition}


\begin{proof}
By Proposition \ref{doubleintegrals3}, we consider the terms $|m_1|\leq |m_2|\leq|m_3|$ at the cost of a factor of $6$, and $|m_2|\asymp|m_3|$. Expanding the sum over $m_1,m_2,m_3$ dyadically, we get for some $M_1\leq M \lesssim qT/N $ \begin{align*}
    S_3\lesssim \sum_{|m_1|\sim M_1,|m_2|,|m_3|\sim M}|I_m| + O(T^{-100})
\end{align*}
We now consider
\begin{align*} 
   & \int\displaylimits_{\substack{
        |v_1m_1+v_2m_2+m_3|\lesssim \frac{q}{N}\\
        \frac{1}{2}\leq v_1,v_2\leq 2
    }} \left| R\left(\frac{v_2}{v_1},y_2,y_1\right)
    R(v_2,y_2)R\left(v_1,y_1\right)\right| dv_1 \ dv_2\\
     \ll& 
    \int_{v_1\asymp 1} |R\left(v_1,y_1\right)|
    \int\displaylimits_{\left|v_2 - \frac{v_1m_1+m_3}{-m_2}\right|\lesssim \frac{q}{|m_2|N}} \left| R\left(\frac{v_2}{v_1},y_2,y_1\right)
    R(v_2,y_2)\right|  dv_2 \ dv_1\\
    \ll& \int_{v_1\asymp 1} |R\left(v_1,y_1\right)|
    \int\displaylimits_{\left|v_2 - \frac{v_1m_1+m_3}{-m_2}\right|\lesssim \frac{q}{MN}} \left| R\left(\frac{v_2}{v_1},y_2,y_1\right)
    R(v_2,y_2)\right|  dv_2 \ dv_1\\
\end{align*}
when $|m_2|\asymp M$. The inner integral, by Cauchy-Schwarz,
is \begin{align*}
    \leq & \left(\int\displaylimits_{\left|v_2 - \frac{v_1m_1+m_3}{-m_2}\right|\lesssim \frac{q}{MN}} \left| R\left(\frac{v_2}{v_1},y_2,y_1\right)\right|^2 \ dv_2 \right)^{1/2}
    \left(
    \int\displaylimits_{\left|v_2 - \frac{v_1m_1+m_3}{-m_2}\right|\lesssim \frac{q}{MN}} \left|
    R(v_2,y_2)\right|^2  dv_2\right)^{1/2}\\
    \ll& \frac{q}{MN} \tilde{R}_M\left(\frac{v_1m_1+m_3}{-m_2v_1},y_2,y_1\right)  \tilde{R}_M(\frac{v_1m_1+m_3}{-m_2},y_2)
\end{align*}
where in the last step, we used $v_1\asymp 1$.
Thus, for $|m_2|\sim M$,
\[
|I_m|\lesssim \phi(q)\frac{N^2}{Mq^2}\tilde{I}_m.
\]
The proposition follows from this claim.
\end{proof}

\fi
To apply H\"older's inequality, we need to find bounds on the second and fourth moments of $R(v,y_1,y_2)$.
\begin{lemma} \label{secondmoment}
    Let $\mathcal{S}=\{(t_j,\chi_j)\}$, such that $\chi_j$ is a character mod $q$, and the $t$'s are contained in an interval of length $T$, and are $T^\epsilon$-separated for the same character. Then uniformly in $y_2\in \mathbb{Z}/q\mathbb{Z}$, \[
        \sum_{y_1\in (\mathbb{Z}/q\mathbb{Z})} \int_{v\asymp 1} 
        \left|R\left(v,y_1,y_2\right)\right|^2dv \ll_{\epsilon} \phi(q)|\mathcal{S}|,
    \]
     and \[
        \sum_{y_1\in (\mathbb{Z}/q\mathbb{Z})} \int_{v\asymp 1} 
        \left|R\left(v,y_1,y_2\right)\right|^4dv  \lesssim \phi(q)E(\mathcal{S}).
    \]
\end{lemma}
\begin{proof}
    For the second moment, we have \[
    |R(v,y_1,y_2)|^2 = \sum_{(t_1,\chi_1),(t_2,\chi_2)\in \mathcal{S}}
    \chi_1\bar{\chi}_2(y_1)\bar{\chi}_1{\chi}_2(y_2)v^{i(t_1-t_2)}.
    \]
    By the orthogonality of characters,\[
        \sum_{y_1\in (\mathbb{Z}/q\mathbb{Z})}|R(v,y_1,y_2)|^2 = \phi(q) \sum_{(t_1,\chi_1),(t_2,\chi_2)\in \mathcal{S}} \delta_{\chi_1\chi_2}\bar{\chi}_1{\chi}_2(y_2)v^{i(t_1-t_2)} = \phi(q)\sum_{(t_1,\chi_1),(t_2,\chi_2)\in \mathcal{S}} \delta_{\chi_1\chi_2}\mathbb{I}_{(y_2,q)=1} v^{i(t_1-t_2)},
    \]
    so it is enough to consider the second moment of $R(v,y,1)$.
    Let $\psi$ be a bump function supported on $v\asymp 1$ and equals $1$ on the domain of integration in the lemma.
   Then, \begin{align*}
        \sum_{y\in (\mathbb{Z}/q\mathbb{Z})^\times} \int_{v\asymp 1} 
        \left|R\left(v,y\right)\right|^2dv 
        \leq&\sum_{y\in (\mathbb{Z}/q\mathbb{Z})^\times} \int 
        \psi(v)\left|R\left(v,y\right)\right|^2dv 
        \\=&
        \phi(q)\int \psi(v)
        \sum_{\substack{(t_1,\chi_1),(t_2,\chi_2)\in \mathcal{S}\\ \chi_1=\chi_2}}v^{i(t_1-t_2)}
        dv\\
        =&
        \phi(q)\sum_{\substack{(t_1,\chi_1),(t_2,\chi_2)\in \mathcal{S}\\ \chi_1=\chi_2}}\int \psi(v)
        v^{i(t_1-t_2)}
        dv.
    \end{align*}
    In the sum, the terms $t_1=t_2$ contribute $O(|\mathcal{S}|)$. If $t_1\neq t_2$, then $|t_1-t_2|\geq T^\epsilon$. The integral in this case is $O_\epsilon(T^{-100})$ and is negligible.

    Similarly for the fourth moment, it is enough to consider $R(v,y)$. We have \[
    |R(v,y)|^4 = \sum_{\substack{(t_1,\chi_1),(t_2,\chi_2),\\ (t_3,\chi_3),(t_4,\chi_4)\in \mathcal{S}}}
    \chi_1{\chi}_2\bar{\chi_3}\bar{\chi_4}(y)v^{i(t_1+t_2-t_3-t_4)}.
    \]
    So again by the orthogonality of characters, \begin{align*}
        \sum_{y\in (\mathbb{Z}/q\mathbb{Z})^\times} \int_{v\asymp 1} 
        \left|R\left(v,y\right)\right|^4dv = & \phi(q)
        \sum_{\substack{(t_1,\chi_1),(t_2,\chi_2),\\ (t_3,\chi_3),(t_4,\chi_4)\in \mathcal{S}\\ \chi_1\chi_2=\chi_3\chi_4}} \int_{v\asymp 1} v^{i(t_1+t_2-t_3-t_4)} dv.
    \end{align*}
    Similar to the previous proof, we can introduce the bump function $\psi$ for the integral, and restrict the summation to the terms $|t_1+t_2-t_3-t_4|\leq T^\epsilon$ with an error of $O_\epsilon(T^{-100})$. The remaining terms in the summation contribute $O(E(\mathcal{S}))$.
\end{proof}
\begin{lemma}\label{fourthmoment_smooth}
    Let $E(\mathcal{S})=\#\{(t_1,\chi_1),(t_2,\chi_2),(t_3,\chi_3),(t_4,\chi_4)\in \mathcal{S}  :  |t_1+t_2-t_3-t_4|\leq 1, \chi_1\chi_2=\chi_3\chi_4\}$. Then \[
        \sum_{y_1\in (\mathbb{Z}/q\mathbb{Z})^\times} \int_{v\asymp 1} 
        \left|\tilde{R}_M\left(v,y_1,y_2\right)\right|^4dv  \lesssim \phi(q)E(\mathcal{S}).
    \]
\end{lemma}
\begin{proof}
    It is enough to show the case where $y_2=1$.
    We apply Cauchy-Schwarz to \begin{align*}
        \int_{v\asymp 1} 
        \left|\tilde{R}_M\left(v,y\right)\right|^4dv  \lesssim& \int_{v\asymp 1} 
        \left(\int_{|u-v|\lesssim q/NM}
        \frac{NM}{q}|R(u)|^2 du\right)^2
        dv \\
        \lesssim& \frac{NM}{q} \int_{v\asymp 1} 
       \int_{|u-v|\lesssim q/NM}
        |R(u)|^4 du \ 
        dv\\
        \lesssim&  
        \int_{u\asymp 1}
         |R(u)|^4 du.
    \end{align*}
    Lemma \ref{secondmoment} completes the proof.
\end{proof}
\begin{proof}[{Proof of Proposition \ref{im2bound}}]
    \textbf{Case 1:} $\Sigma_{11}$\\
    We apply H\"older's inequality on $\tilde{I}_{m,2}$. When $(m_1,q)=(m_2,q)=1$, we have \begin{align*}
        \frac{q^2}{N^3}\tilde{I}_{m,2}\leq &
       \Bigg( \sum_{\substack{y_1\in (\mathbb{Z}/q\mathbb{Z})\\y_2=\frac{y_1m_1+m_3}{-m_2}}} \ \int\displaylimits_{v_1\asymp 1}  \left| \tilde{R}_M\left(\frac{m_1v_1+m_3}{-m_2v_1},y_2,y_1\right)\right|^4\Bigg)^{1/4}
       \Bigg( \sum_{\substack{y_1\in (\mathbb{Z}/q\mathbb{Z})\\y_2=\frac{y_1m_1+m_3}{-m_2}}} \ \int\displaylimits_{v_1\asymp 1} \left|\tilde{R}_M(\frac{m_1v_1+m_3}{-m_2},y_2)\right|^4\Bigg)^{1/4}\\
        & \quad \quad    \times     \Bigg(\sum_{\substack{y_1\in (\mathbb{Z}/q\mathbb{Z})\\y_2=\frac{y_1m_1+m_3}{-m_2}}} \ \int\displaylimits_{v_1\asymp 1} \left|R\left(v_1,y_1\right)\right| ^2\Bigg)^{1/2}
    \end{align*}
    The second and third terms in the product simplify to $\lesssim (q-1)^{1/4}E(\mathcal{S})^{1/4}$ and $\ll_\epsilon (q-1)^{1/2}|\mathcal{S}|^{1/2}$ respectively by Lemmas \ref{secondmoment} and \ref{fourthmoment_smooth}. For the first term, we consider, for $y_1$ coprime to $q$, \[
    \chi(y_1)\bar{\chi}\left(\frac{y_1m_1+m_3}{-m_2}\right) = \bar{\chi}\left(\frac{y_1m_1+m_3}{-m_2y_1}\right)=\bar{\chi}\left(\frac{m_1}{-m_2}+\frac{m_3}{-y_1m_2}\right). 
    \]
    Therefore if $(q,m_3)=1$, the first term can be rewritten as \[
        \leq \Bigg( \sum_{y_1\in (\mathbb{Z}/q\mathbb{Z})} \ \int\displaylimits_{v_1\asymp 1}  \left| \tilde{R}_M\left(\frac{m_1v_1+m_3}{-m_2v_1},1,y_1\right)\right|^4\Bigg)^{1/4} \lesssim(q-1)^{1/4} E(\mathcal{S})^{1/4}.
    \]
    Therefore, the sum \[
    \sum_{\substack{|m_1|\sim M_1\\|m_2|,|m_3|\sim M\\
    m_1,m_2,m_3 \textrm{ coprime to } q}} \tilde{I}_{m,2} \lesssim \frac{N^3}{q}M_1M^2q E(\mathcal{S})^{1/2}|\mathcal{S}|^{1/2}.
    \]
    If $(q,m_3)=q$, then the first term is \[
        (q-1)^{1/4}\Bigg(\int\displaylimits_{v_1\asymp 1}  \left| \tilde{R}_M\left(\frac{m_1v_1+m_3}{-m_2v_1},\frac{m_2}{m_1}\right)\right|^4\Bigg)^{1/4}\ll (q-1)^{1/4}\Bigg(\int\displaylimits_{u\asymp 1}  \left| \tilde{R}_M\left(u,\frac{m_2}{m_1}\right)\right|^4\Bigg)^{1/4}
    \]by a change of variables, and extending the domain $u\asymp 1$ for the bound to be uniform across all $m_1\sim M_1$. Now we can apply H\"older's inequality for the sum of this in $m_1$ \[
       \sum_{\substack{|m_1|\sim M_1\\ m_1\textrm{ coprime to }q}} \Bigg(\int\displaylimits_{u\asymp 1}  \left| \tilde{R}_M\left(u,\frac{m_2}{m_1}\right)\right|^4\Bigg)^{1/4}\Bigg(\leq \sum_{\substack{|m_1|\sim M_1\\ m_1\textrm{ coprime to }q}} \int\displaylimits_{u\asymp 1}  \left| \tilde{R}_M\left(u,\frac{m_2}{m_1}\right)\right|^4\Bigg)^{1/4} M_1^{3/4} \lesssim M_1 E(\mathcal{S}).
    \]
    This gives \[
    \sum_{\substack{|m_1|\sim M_1\\|m_2|,|m_3|\sim M\\
    m_1,m_2 \textrm{ coprime to } q}} \tilde{I}_{m,2} \lesssim \frac{N^3}{q}M_1M^2 E(\mathcal{S})^{1/2}|\mathcal{S}|^{1/2}.
    \]

    \textbf{Case 2,3:} $\Sigma_{q,1}$ and $\Sigma_{1,q}$. 
    \\
    The arguments are similar. We can apply H\"older's inequality on the integral ($1=1/4+1/4+1/2$). For Case 2, we apply orthogonality cancellation in $y_1$ for the first and third term. Cancellation in the second term can be achieved by in summation of $m_3$, which is exactly the same argument for the $(q,m_1)=(q,m_2)=1, (q,m_3)=q$. \\
    For Case 3, the first two terms after applying H\"older's inequality achieves cancellation in $y_2$, and the last term achieves cancellation in $m_3$. Combined, these gives the same bounds as Case 1.

    \textbf{Case 4:} $\Sigma_{qq}$\\
    We repeat the argument for \hyperref[im1boundproof]{bounding the sum in $\tilde{I}_{m,1}$}. The sum in all $m_1,m_2,m_3$ multiples of $q$ gives
    \[  
        \Sigma_{qq}\lesssim \frac{N^3}{q^3}M_1M^2E(\mathcal{S})^{1/2}|\mathcal{S}|^{1/2}.
    \]
    Therefore, by Proposition \ref{dyadics_32}, we have \[
    \sum_{0<|m_1|,|m_2|,|m_3|\lesssim \frac{qT}{N}} I_{m,2} \lesssim \frac{q}{NM}\frac{N^3}{q}M_1M^2E(\mathcal{S})^{1/2}|\mathcal{S}|^{1/2}\lesssim (qT)^2E(\mathcal{S})^{1/2}|\mathcal{S}|^{1/2}.
    \]
\end{proof}
Proposition \ref{s_3bound} follows directly from Propositions \ref{im1bound} and \ref{im2bound}.