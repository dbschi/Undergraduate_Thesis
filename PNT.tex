\section{The Prime Number Theorem}

\begin{theorem}[Prime Number Theorem]
    Let $\Pi(N)=\sum_{p\leq N} 1$. Then \[
        \Pi(N) = (1+o(1))\frac{N}{\log N}.
    \]
\end{theorem}

In this section we will prove the Prime Number Theorem. This result is a minor goal of this paper.
The Prime Number theorem serves as a starting point for
studying primes in short intervals, and sets the stage for zero-density theorems.
\begin{definition}[Von Mangoldt Function]
    The \textbf{Von Mangoldt function} $\Lambda$ is defined as follows:
    \[
        \Lambda(n) = \begin{cases}
            \log p \textrm{, if $n = p^k$ for some $k\in\naturals$}\\
            0 \textrm{, else}
        \end{cases}
    \]
\end{definition}
The sum of the Von Mangoldt function $\sum\Lambda (n)$ is a more natural way to express
a prime counting function in the language of $\zeta$. To see why, consider the expression
\begin{align*}
    \frac{\zeta '(s)}{\zeta(s)} &= (\log\zeta(s))'\\
    &= \left[ -\sum_{p} \log\left(1-p^{-s}\right)\right]^{\prime}\\
    &= - \sum_{p} \frac{p^s \log p }{1-p^{-s}}\\
    &= -\sum_{p} \log p \sum_{k\in\naturals} p^{-ks}\\
    &= - \sum_{n\in\naturals} \frac{\Lambda(n)}{n^s}
\end{align*}
on $\Re(s)>1$ where the sum and products are absolutely convergent.
\begin{proposition}\label{mangoldtpnt}
    $\sum_{n\leq N} \Lambda(n) = (1+o(1))N$ implies the Prime Number Theorem.
\end{proposition}
\begin{proof}
    On one hand, we have \begin{align*}
        \sum_{n\leq N }\Lambda(n) &\leq \sum_{p \leq N }\Lambda(N)
        \\ &\leq \Pi(x){\log x}.
    \end{align*}
    And for $\epsilon>0$,
    \begin{align*}
        \sum_{n\leq N }\Lambda(n) &\geq \sum_{N^{1-\epsilon}\leq n\leq N }\Lambda(n)
        \\ &\geq \sum_{N^{1-\epsilon}\leq p \leq N }(1-\epsilon) \log(N)\\
        \\&= \Pi(N)\log(N) + O(N^{1-\epsilon}log N).
    \end{align*}
\end{proof}
Moreover, the sum of the Von Mangoldt function can be related to the zeros of the zeta function.
Let $\phi$ be smooth and rapidly decaying at infinity, and $\tilde{\phi}$ be its Mellin transform. Let $N\in\naturals$ and $c\geq 2$. Then \begin{equation} \label{mellinonzeta}
    \begin{split}
    \sum_{n\in\naturals} \Lambda(n) \phi\left(\frac{n}{N}\right) &=
    \sum_{n\in\naturals} \Lambda(n) \frac{1}{2\pi i}\int_{c-i\infty}^{c+i\infty}
    \tilde{\phi}(s)\left(\frac{n}{N}\right)^{-s} ds \\
    &= \frac{1}{2\pi i}\int_{c-i\infty}^{c+i\infty}
    \tilde{\phi}(s)\sum_{n\in\naturals} \Lambda(n) \left(\frac{n}{N}\right)^{-s} ds \\
    &= \frac{-1}{2\pi i}\int_{c-i\infty}^{c+i\infty}
    \tilde{\phi}(s) N^s \frac{\zeta'(s)}{\zeta(s)} ds \\
    \end{split}
\end{equation}

By the rapid decay of $\tilde\phi$, we change the line of integration from $c$ to $-\infty$, we get residue contributions from
a pole at $s=1$, $s=0$, as well as all $\rho$ such that $\zeta(\rho)=0$ on the critical strip,
and all the trivial zeros. 
Morally, we can take the indicator function $\phi=1$ on $[0,1]$. 
\begin{equation}\label{preexplicit}
    \sum_{n\leq N} \Lambda(n)=
    \frac{-1}{2\pi i}\int_{c-i\infty}^{c+i\infty}
    \frac{1}{s} N^s \frac{\zeta'(s)}{\zeta(s)} ds
\end{equation}
This gives \begin{equation}
    \begin{split}
        \sum_{n\leq N} \Lambda(n)&= N - \sum_{\rho} \frac{N^\rho}{\rho} -
        \frac{\zeta'(0)}{\zeta(0)} + \sum_{k\in\naturals}\frac{N^{-2k}}{2k} \\
        &= N - \sum_{\rho} \frac{N^\rho}{\rho} -
        \frac{\zeta'(0)}{\zeta(0)} + \frac{1}{2}\log\left(1-N^{-2}\right). 
    \end{split}
\end{equation}
This formula, due to von Mangoldt, can be derived with more care about the convergence in the sum:
The sum over zeros $\rho$ is not absolutely convergent, and is ordered in increasing $|\Im(\rho)|$.
\begin{theorem}[Riemann-von Mangoldt explicit formula]
    Let $N>1$ be not a prime power. Then\begin{equation}
    \sum_{n\leq N} \Lambda(n) = N - \lim_{T\to \infty}\sum_{|\Im{(\rho)}|\leq T} \frac{N^\rho}{\rho} -
    \frac{\zeta'(0)}{\zeta(0)} + \frac{1}{2}\log\left(1-N^{-2}\right). 
    \end{equation}
\end{theorem}
In practice, we truncate the integral in \label{preexplicit} up to height $T$ to obtained a truncated version of the explicit formula.
\begin{theorem}
    Let $N>1$. Then\begin{equation}
        \sum_{n\leq N} \Lambda(n) = N - \sum_{|\Im{(\rho)}|\leq T} \frac{N^\rho}{\rho} + O(\frac{N\log N \log T}{T} + \log T). 
        \end{equation}
\end{theorem}
The term $N$ in the explicit formula is already suggestive of the Prime Number Theorem. 
The major error term comes from $N^\rho$ in the sum, so bounding the $\Re(\rho)$ becomes the most important part in reducing the error term in the prime number theorem. 
This in turn is equivalent to bounding $\Re({\rho})$, and the best case is when all the zeros have real part $1/2$.
Assuming the Riemann Hypothesis, we consider the sum over the non trivial zeros \begin{align*}
\left|\sum_{|\Im(\rho)\leq T|}\frac{N^\rho}{\rho}\right| &\leq N^{1/2} \sum_{\rho}\left|{\frac{1}{\rho}}\right|.    
\end{align*}
We know there are $\sim \log T$ zeros of height $[T,T+1)$, thus the integral $\sum |\rho^{-1}|$ behaves as \[
\sum_{n\leq T} \frac{\log n}{n} = O(\log^2 T).
\]
Taking $N=T$ in the truncated explicit formula, we obtain \begin{equation} \label{RHPNT}
    \sum_{n\leq N} \Lambda(n) = N + O(N^{1/2}\log^2 N).
\end{equation}
Which implies the prime number theorem.
\begin{remark}
    The prime number theorem with the error term in \label{RHPNT} can be shown to be equivalent to the Riemann Hypothesis.
\end{remark}
The prime number theorem is also true without assuming the strong Riemann Hypothesis. 
To show this, it is sufficient to show that there are no zeros with real part $1$, so the terms in the sum contributes $O(N^{1-\epsilon})$
which will be dominated by $N$.
\begin{theorem}\label{nozerosatone}
    Let $t\in\reals$. Then $\zeta(1+it)\neq 0$.
\end{theorem}
\begin{proof}[Proof of the Prime Number Theorem]
    Condition on proving Theorem \ref{nozerosatone}, we use it to derive the prime number theorem.
Let $\phi=\phi_{N,T}$ be a bump function that equals $1$ on the interval $[2,N]$ and supported on $[3/2, N+N/T]$.
By construction we can also make $\phi^{(j)}(x)=O_j(1)$ and $\phi^{(j)}(x)=O_j(T/x)^j$ on the intervals $[3/2, 2]$ and $[N,N+N/T]$ respectively.
Then \begin{align*}
    \sum_{n\leq N}\Lambda(n) &\leq \sum_{n} \Lambda(n)\phi(n)\\
    &= \frac{-1}{2\pi i}\int_{c-i\infty}^{c+i\infty}\tilde{\phi}\frac{\zeta'(s)}{\zeta(s)}ds\\
    &= \tilde{\phi}(1) - \sum_{\rho} \tilde{\phi}(\rho) - \sum_{n} \tilde{\phi}(-2n)\\
\end{align*}
The first term \begin{align*}
    \tilde{\phi}(1) &= \int_{0}^{\infty} \phi(x) dx \\
    &= N+ O(N/T) 
\end{align*}
gives the term we want from the PNT. In the third term, we rewrite by Monotone Convergence \begin{align*}
    \sum_{n} \tilde{\phi}(-2n) &= \sum_{n} \int_{0}^{\infty} \phi(x) x^{-2n-1} dx \\
    &=\int_{0}^{\infty} \phi(x) \sum_{n} x^{-2n-1} dx \\
    &= \int_{0}^{\infty} \phi(x) \frac{1}{x^3-x} dx \\
    &= O(1)
\end{align*} 
Finally, to bound the second term, we define a parameter $\bar{T} = \bar{T}(T)$ and split the sum into\[
\sum_{|\Im{\rho}|\leq \bar{T}}\tilde{\phi}(\rho) + \sum_{|\Im{\rho}| > \bar{T}}\tilde{\phi}(\rho)
\]
In the first summation, we let $\epsilon = \epsilon_{\bar{T}}$ 
such that there are no zeros in the region
$\Re({s})>1-\epsilon, |\Im{(s)}|\leq \bar{T}$, then \begin{align*}
    \sum_{|\Im{\rho}|\leq \bar{T}}\tilde{\phi}(\rho) &= \sum_{|\Im{\rho}|\leq \bar{T}} \int_{0}^{\infty}\phi(x)x^{\rho-1}dx\\
    &= O_{T}(N^{1-\epsilon}).
\end{align*}
In the second summation, we apply integration by parts to show that \begin{align*}
    |\int_{0}^{\infty}\phi(x)x^{\rho-1}dx| &= |\frac{1}{\rho(\rho+1)}\int_{0}^{\infty}\phi''(x)x^{\rho+1}dx|\\
    &= O(\frac{1}{|\rho|^2} \frac{T^2}{N^2} \frac{N}{T} N^2)\\
    &=O(\frac{1}{|\rho|^2} TN)
\end{align*}
The sum $\frac{1}{|\rho|^2}$ behaves as $\sum_n \log n n^{-2}$, so we can pick $\bar{T}$ large enough depending on $T$
to make the contribution of $\sum_{\Im(\rho)>\bar{T}} |\rho|^{-2}$ to be $O(T^{-2})$.
So that 
\begin{align*}
    \sum_{n\leq N}\Lambda(n)&\leq N +O(N/T)+ O_T(N^{1-\epsilon})\\
    = N+O(N/T)
\end{align*}
for $N=N(T)$ sufficiently large.
Similarly, repeating the same argument on $\phi=\phi_{N,T}$ equals $1$ on the interval $[2,N-N/T]$ and supported on $[3/2, N]$
gives \[
    \sum_{n\leq N}\Lambda(n)\geq N+O(N/T).
\]
Sending $T\to\infty$ gives the PNT.
\end{proof}

\begin{proof}[Proof of Theorem \ref{nozerosatone}]
    Let $\sigma > 1$. We consider the expressions \begin{align*}
        \Re\left(\frac{\zeta'}{\zeta}(\sigma+it)\right) = -\sum_{n} \frac{\Lambda(n)}{n^\sigma}\cos (t\log n) 
    \end{align*}
    and \[
    2(1+\cos \theta)^2 =2+4\cos \theta + 2\cos^2 \theta = 3+4\cos\theta + \cos 2\theta.
    \]
    So that \begin{align*}
        \Re\left(3\frac{\zeta'}{\zeta}(\sigma)+ 4\frac{\zeta'}{\zeta}(\sigma+it) +\frac{\zeta'}{\zeta}(\sigma+2it)\right)
        &= -\sum_{n} \frac{\Lambda(n)}{n^\sigma}(3+4\cos (t\log n)+\cos (2t\log n))\\
        &= -\sum_{n} \frac{\Lambda(n)}{n^\sigma} 2(1+\cos t\log n)^2\\
        &\leq 0.
    \end{align*}
    Now for the sake of contradiction, we let $\zeta(1+it)=0$ be a zero of order $d$, and since we know $\zeta$ has a pole of order $1$ at $s=1$, we can let $t\neq 0$. 
    Consider the function $f(s)=\zeta(s)^3\zeta(s+it)^4\zeta(s+2it)$.
    By the computation above, $\Re(f'/f)\leq 0$ when $\Re(s)> 1$.
    But we also have that $f$, by construction, has a zero of order $k\geq 4d-3>0$ at $s=1$.
    So that $\Re(f'/f) = k/(s-1)+ \textrm{a holomorphic part}$. Now taking $s\to 1^{+}$, $\Re(f'/f)\to +\infty$, contradicting $\Re(f'/f)\leq 0$.
\end{proof}
