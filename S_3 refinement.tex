\section{Refinement of $S_3$ bound}

\begin{proposition}[Refinement of $S_3$]
    \label{refinements_3}
    We have \[
    S_3\lesssim \phi(q)^{7/2}T^2|\mathcal{S}|^{3/2}+\phi(q)^{3}\frac{NT}{q}|\mathcal{S}|^{1/2}E(\mathcal{S})^{1/2}.
     \]
\end{proposition}
The refinement of the previous bound relies on the result for summation over affine transformation by Guth and Maynard.
\begin{lemma} \label{affineGM}
    Let $M>0$. Let $f(u)\geq 0$, supported on $u\asymp 1$, and $|\hat{f}(\xi)|\lesssim_j (|\xi|/T)^j$ for all $j$. Then \[
        \sup_{0<M_1,M_2,M_3<M} \int\Bigg( \sum_{\substack{|m_1|\sim M_1\\|m_2|\sim M_2 \\ |m_3|\ll M_3}} f\left(\frac{m_1 u+m_3}{m_2}\right)\Bigg)^2 \ du \lesssim M^6 \|f\|_{L_1}^2+M^4\|f\|_{L_2}^2.
    \] 
\end{lemma}
This is Proposition 9.1 from [GM].
For the sum in $\tilde{I}_{m,1}$, we can directly apply this lemma, exchange the summation in $\sum_\chi$ and $\sum_m$ to get \begin{align*}
    &\sum_{\substack {|m_1|\sim M_1,|m_2|,|m_3|\sim M}}\tilde{I}_{m,1}\\
   &  =\sum_{\chi}\int\displaylimits_{v_1\asymp 1}  \left|R_\chi\left(v_1\right)\right|\sum_{\substack {|m_1|\sim M_1,\\|m_2|,|m_3|\sim M}}\left| \tilde{R}_{\chi,M}\left(\frac{m_1v_1+m_3}{-m_2v_1}\right)
    \tilde{R}_{\chi,M}\left(\frac{m_1v_1+m_3}{-m_2}\right)\right|dv_1\\
    &\stackon{CS}{\leq} \Big( \sum_{\chi}\int\displaylimits_{v_1\asymp 1}  \left|R_\chi\left(v_1\right)\right|^2 dv_1 \Bigg)^{1/2} \Bigg(\sum_{\chi}\int\displaylimits_{v_1\asymp 1}\Bigg(\sum_{\substack {|m_1|\sim M_1,\\|m_2|,|m_3|\sim M}}\left| \tilde{R}_{\chi,M}\left(\frac{m_1v_1+m_3}{-m_2v_1}\right)
    \tilde{R}_{\chi,M}\left(\frac{m_1v_1+m_3}{-m_2}\right)\right|\Bigg)^2dv_1\Bigg)^{1/2}\\
    &\stackon{CS}{\lesssim} |\mathcal{S}|^{1/2} \Bigg(\sum_{\chi}\int\displaylimits_{v_1\asymp 1}\Bigg(\sum_{\substack {|m_1|\sim M_1,\\|m_2|,|m_3|\sim M}}\left| \tilde{R}_{\chi,M}\left(\frac{m_1v_1+m_3}{-m_2v_1}\right)\right|^2 \sum_{\substack {|m_1|\sim M_1,\\|m_2|,|m_3|\sim M}}\left|
    \tilde{R}_{\chi,M}\left(\frac{m_1v_1+m_3}{-m_2}\right)\right|^2\Bigg)^2dv_1\Bigg)^{1/2}\\
    &\stackon{CS}{\leq}|\mathcal{S}|^{1/2} \Bigg(\sum_{\chi}
    \int\displaylimits_{v_1\asymp 1}\Bigg(\sum_{\substack {|m_1|\sim M_1,\\|m_2|,|m_3|\sim M}}\left| \tilde{R}_{\chi,M}\left(\frac{m_1v_1+m_3}{-m_2v_1}\right)\right|^2\Bigg)^2 dv_1 \Bigg)^{1/4}\Bigg(\sum_{\chi}
    \int\displaylimits_{v_1\asymp 1}\Big(
    \sum_{\substack {|m_1|\sim M_1,\\|m_2|,|m_3|\sim M}}\left|
    \tilde{R}_{\chi,M}\left(\frac{m_1v_1+m_3}{-m_2}\right)\right|^2\Bigg)^2dv_1\Bigg)^{1/4}.
\end{align*} 
We can bound these terms using Proposition \ref{affineGM}. Note that $R_{\chi,M}^2$ and $R_M^2$ have rapid Fourier decay as the convolution of a smooth bump function and $R$, therefore meets the condition in the Proposition.
For the second term, we can make the change of variables $u=1/v_1$ with a Jacobian of $\asymp 1$, so that \[
\frac{m_1v_1+m_3}{-m_2v_1}= \frac{m_3u + m_1}{-m_2}.
\]
We also have \begin{align*}
    \sum_{\chi} \|R_\chi^2\|_{L_1}^2\stackon{CS}{\leq}(\sum_{\chi} \|R_\chi^2\|_{L_1})^2 \lesssim |\mathcal{S}|^2,
\end{align*}
and\begin{align*}
    \sum_{\chi} \|R_\chi^2\|_{L_2}^2=\sum_{\chi} \|R_\chi^4\|_{L_1} \lesssim E(\mathcal{S})^2.
\end{align*}
Combined, this gives a bound of \[
    \frac{N^2}{M}\sum_{\substack {|m_1|\sim M_1,|m_2|,|m_3|\sim M}}\tilde{I}_{m,1}\lesssim \frac{N^2}{M} |\mathcal{S}|^{1/2} (M^6|\mathcal{S}|^{2}+M^4E(\mathcal{S}))^{1/2}\ll N^2 M^2|\mathcal{S}|^{3/2} + NM |\mathcal{S}|^{1/2}E(\mathcal{S})^{3/2},
\]
which is $\lesssim (qT)^{2}|\mathcal{S}|^{3/2}+(qT)|\mathcal{S}|^{1/2}E(\mathcal{S})^{3/2}.$


For $I_{m,2}$, we can exchange the summations in $y_1,y_2$ and in $m$ to get the sum \begin{align*}
    &\sum_{\substack {|m_1|\sim M_1,\\|m_2|,|m_3|\sim M}}\tilde{I}_{m,2}\\=&\frac{N^3}{q^2} \sum_{y_1,y_2 \in(\mathbb{Z}/q\mathbb{Z})^\times} \sum_{\substack {|m_1|\sim M_1,\\|m_2|,|m_3|\sim M\\ y_1m_1+y_2m_2+m_3\equiv 0\\ \mod q}}\int_{v_1\asymp 1} \left| \tilde{R}_M\left(\frac{m_1v_1+m_3}{-m_2v_1},y_2,y_1\right)
    \tilde{R}_M(\frac{m_1v_1+m_3}{-m_2},y_2)R\left(v_1,y_1\right)\right| dv_1.
\end{align*}
In principle, the condition $y_1m_1+y_2m_2+m_3\equiv 0$ fixes the residue class of $m_3$ based on $m_1$ and $m_2$. Therefore, we would expect a factor of $1/q^2$ if we add this summation condition to Proposition \ref{affineGM}. 
\begin{lemma}\label{affinetrans}
    Let $M>0$. Let $f(u)\geq 0$, supported on $u\asymp 1$, and $|\hat{f}(\xi)|\lesssim_j (|\xi|/T)^j$ for all $j$. Then \[
        J(f)\defeq \sup_{0<M_1,M_2,M_3<M} \int\Bigg( \sum_{\substack{|m_1|\sim M_1,|m_2|\sim M_2 , |m_3|\ll M_3\\ y_1m_1+y_2m_2+m_3\equiv 0 \mod q}} f\left(\frac{m_1 u+m_3}{m_2}\right)\Bigg)^2 \ du \lesssim M^6 \|f\|_{L_1}^2/q^2+M^4\|f\|_{L_2}^2/q^2.
    \] 
\end{lemma}
\textcolor{red}{Testing proof here...}

Let $\psi$ be a smooth bump on $|x|\lesssim 1$, such that $\psi(m_3/M_3)=1$ where $m_3\lesssim M_3$ in the summation conditiion. Furthermore, let \[
    \bar{y}\defeq \bar{y}(m_1,m_2,y_1,y_2)\defeq -m_1y_1-m_2y_2,
\]so that the condition $y_1m_1+y_2m_2+m_3\equiv 0 \mod q$ becomes $m_3=kq+\bar{y}$ for integer $k$.
For the choice of $M_1,M_2,M_3$ that the supremum in the lemma is achieved, we define \[
    g(u)\defeq\sum_{m_1\sim M_1,m_2\sim M_2} \sum_{k\in \mathbb{Z}} \psi\left(\frac{kq+\bar{y}}{M_3}\right)f\left(\frac{m_1 u+kq+\bar{y}}{m_2}\right),
\]so that we have \[
J(f)\leq \int |g(u)|^2 du \quad \stackeq{Plancheral} \quad  \int |\hat{g}(\xi)|^2 d\xi.
\]
We now compute, using the change of variables $u'=u+(kq+y)/m_1$, \begin{align*}
    \hat{g}(\xi)=&\int \sum_{m_1\sim M_1,m_2\sim M_2} \sum_{k\in \mathbb{Z}} \psi\left(\frac{kq+\bar{y}}{M_3}\right)f\left(\frac{m_1 u+kq+\bar{y}}{m_2}\right) e(-u\xi) du\\
    =& \sum_{m_1\sim M_1,m_2\sim M_2}\int \sum_{k\in \mathbb{Z}} \psi\left(\frac{kq+\bar{y}}{M_3}\right)f\left(\frac{m_1 u+kq+\bar{y}}{m_2}\right) e(-u\xi) du\\
    =& \sum_{m_1\sim M_1,m_2\sim M_2}\int \sum_{k\in \mathbb{Z}} \psi\left(\frac{kq+\bar{y}}{M_3}\right)f\left(\frac{m_1 u'}{m_2}\right) e(-u'\xi) e\left(\frac{kq+\bar{y}}{m_1}\xi\right)du'\\
    =& \sum_{m_1\sim M_1,m_2\sim M_2} \left(\int f\left(\frac{m_1 u'}{m_2}\right) e(-u'\xi) du'\right)\left(\sum_{k\in \mathbb{Z}} \psi\left(\frac{kq+\bar{y}}{M_3}\right) e\left(\frac{kq+\bar{y}}{m_1}\xi\right)\right)\\
    =& \sum_{m_1\sim M_1,m_2\sim M_2} \frac{m_2}{m_1}\hat{f}\left(\frac{m_2}{m_1}\xi\right)\left(\sum_{k\in \mathbb{Z}} \psi\left(\frac{kq+\bar{y}}{M_3}\right) e\left(\frac{kq+\bar{y}}{m_1}\xi\right)\right).\\
\end{align*}
The summation in $k$ can be rewritten through Poisson summation to get \[
    \sum_{k\in \mathbb{Z}} \psi\left(\frac{kq+\bar{y}}{M_3}\right) e\left(\frac{kq+\bar{y}}{m_1}\xi\right) = \sum_{\xi'} \frac{M_3}{q}\hat{\psi}\left(M_3\left(\frac{\xi}{m_1}-\frac{\xi'}{q}\right)\right)e\left(\frac{y\xi'}{q}\right).
\]
Therefore,
\[
\hat{g}(\xi) = \sum_{\substack{m_1\sim M_1\\m_2\sim M_2}}\sum_{\xi'}\frac{m_2}{m_1}\frac{M_3}{q}\hat{f}\left(\frac{m_2}{m_1}\xi\right)\hat\psi\left(M_3\left(\frac{\xi}{m_1}-\frac{\xi'}{q}\right)\right)e\left(\frac{-y_1m_1\xi'}{q}\right)e\left(\frac{-y_2m_2\xi'}{q}\right).
\]
By the principle of stationary phase, \[
\hat\psi\left(M_3\left(\frac{\xi}{m_1}-\frac{\xi'}{q}\right)\right)\]
is negligible except when  \[
\left|M_3\left(\frac{\xi}{m_1}-\frac{\xi'}{q}\right)\right| \lesssim 1 \implies \left|\frac{q\xi}{m_1}-\xi'\right|\lesssim \frac{q}{M_3},\]
so we can truncate the summation in $\xi'$ to be in this range.

We split \[
\int |\hat{g}(\xi)|^2 d\xi = A+B+C, 
\]
where
\begin{align*}
    A&\defeq \int_{|\xi|\leq T^\delta \frac{M_1}{M_3}}     |\hat{g}(\xi)|^2 d\xi\\
    B&\defeq \int_{T^\delta \frac{M_1}{M_3}\leq|\xi|\leq T^2}     |\hat{g}(\xi)|^2 d\xi\\
    C&\defeq \int_{T^2\leq |\xi|}     |\hat{g}(\xi)|^2 d\xi\\
\end{align*}
$C$ is negligible based on rapid decay in $\hat{f}(\xi m_2/m_1)$.