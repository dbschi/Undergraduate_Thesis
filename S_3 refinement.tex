\section{Summations over affine transformations result}
The basic bound for $S_3$ in the previous section is slightly too loose for improving on the current bound. To further improve on the bound, we need to introduce a result that exploits equidistribution over the affine transformations. Morally, the two terms in the following proposition represents the average and the variance of the distribution of $f$ respectively.

\iffalse
\begin{proposition}[Refinement of $S_3$]
    \label{refinements_3}
    We have \[
    S_3\lesssim \phi(q)^{7/2}T^2|\mathcal{S}|^{3/2}+\phi(q)^{3}\frac{NT}{q}|\mathcal{S}|^{1/2}E(\mathcal{S})^{1/2}.
     \]
\end{proposition}
The refinement of the previous bound relies on the result for summation over affine transformation by Guth and Maynard.
\begin{lemma} \label{affineGM}
    Let $M>0$. Let $f(u)\geq 0$, supported on $u\asymp 1$, and $|\hat{f}(\xi)|\lesssim_j ((1+|\xi|)/T)^{-j}$ for all $j\in \mathbb{N}$. Then \[
        \sup_{0<M_1,M_2,M_3<M} \int\Bigg( \sum_{\substack{|m_1|\sim M_1\\|m_2|\sim M_2 \\ |m_3|\ll M_3}} f\left(\frac{m_1 u+m_3}{m_2}\right)\Bigg)^2 \ du \lesssim M^6 \|f\|_{L_1}^2+M^4\|f\|_{L_2}^2.
    \] 
\end{lemma}
This is Proposition 9.1 from [GM]. The reduction in \[
\sum_{\substack{y_1,y_2 \in\mathbb{Z}/q\mathbb{Z} }}\int_{v_1\asymp 1}\Big[\sum_{\substack{|m_1|\sim M_1,|m_2|,|m_3|\sim M\\ y_1m_1+y_2m_2+m_3\equiv 0 \mod q_0}}\Big|\tilde{R}_M\left(\frac{m_1v_1+m_3}{-m_2v_1},y_2,y_1\right)\Big|^2\Big]^2dv_1\]
is somewhat lossy, so we aim to refine this bound in this section.
In principle, the condition $y_1m_1+y_2m_2+m_3\equiv 0$ fixes the residue class of $m_3$ based on $m_1$ and $m_2$. Therefore, we would expect a factor of $1/q^2$ if we add this summation condition to Proposition \ref{affineGM}. 
\fi
\begin{proposition}\label{affinetrans}
    Let $M>0$. Let $f(u,y):\reals\times \mathbb{Z}/d \to \reals$ be non-negative, supported on $u\asymp 1$ uniformly in $y$, and $|\hat{f}(\xi,x)|\lesssim_j (|\xi|/(qT)^2)^{-j}$ for all $j$. Then \begin{align*}
        J(f)\defeq &\sup_{0<M_1,M_2,M_3<M}\sum_{y \mod d} \int\Bigg( \sum_{\substack{|m_1|\sim M_1,|m_2|\sim M_2 , |m_3|\ll M_3\\ (m_1,d)=(m_2,d)=1}} f\left(\frac{m_1 u+m_3}{m_2},(m_1 y+m_3)m_2^{-1}\right)\Bigg)^2 \ du\\
        \lesssim&\frac{M^6}{d}\Big(\sum_{y \mod d } \int f(u,y) du\Big)^2+{M^4}\sum_{y\mod d} \int f(u,y)^2 du.
    \end{align*}
\end{proposition}

\begin{lemma}[Iterative bound for $J(f)$] \label{iterative}
    There exists a bump function $\phi(x)$ supported on $|x|\lesssim 1$, such that for \[
    \tilde{f}(u,y) \defeq \int T \phi(T(u'-u)) f(u',y) du'
    \]
    we have \[
    J(f)\lesssim \frac{M^6}{d}\|f\|_{L_1}^2 + M^2 \|f\|_{L_2} J(\tilde{f}).
    \]
\end{lemma}

\begin{proof}[Proof of Proposition \ref{affinetrans} from Lemma \ref{iterative}]
    The argument is exactly the same as Lemma 9.2 in \cite{GM2024}.
\end{proof}
Let $\psi$ be a smooth bump on $|x|\lesssim 1$, such that $\psi(m_3/M_3)=1$ where $m_3\lesssim M_3$ in the summation conditiion. 
For the choice of $M_1,M_2,M_3$ that the supremum in the lemma is achieved, we define \[
    g(u,y)\defeq\sum_{\substack{m_1\sim M_1,m_2\sim M_2\\ (m_1,d)=(m_2,d)=1}} \sum_{m_3\in \mathbb{Z}} \psi\left(\frac{m_3}{M_3}\right)f\left(\frac{m_1 u+m_3}{m_2} , (m_1 y+m_3)m_2^{-1}\right),
\]so that we have \[
J(f)\leq \sum_{y\mod d} \int |g(u,y)|^2 du \quad \stackeq{Plancheral} \quad  \frac{1}{d}\sum_{x\mod d} \int |\hat{g}(\xi,x)|^2 d\xi.
\]
We now compute, using the change of variables $\tilde{u}=u+m_3/m_1$ and $\tilde{y}\equiv y + m_3 m_1^{-1} (d)$, \begin{align*}
    \hat{g}(\xi,x)=&\sum_{y=0}^{d-1}\int 
    e\left(\frac{-xy}{d}\right)e(-u\xi)
    \sum_{\substack{m_1\sim M_1,m_2\sim M_2\\ (m_1,d)=(m_2,d)=1}} \sum_{m_3\in \mathbb{Z}} \psi\left(\frac{m_3}{M_3}\right)f\left(\frac{m_1 u+m_3}{m_2} , (m_1 y+m_3)m_2^{-1}\right) du\\
    =& 
    \sum_{\substack{m_1\sim M_1,m_2\sim M_2\\ (m_1,d)=(m_2,d)=1}} \sum_{m_3\in \mathbb{Z}} \sum_{\tilde{y}=0}^{d-1}\int 
    e\left(\frac{-x\tilde{y}}{d}\right) e\left(\frac{[{m_3x}{m_1}^{-1}]}{d}\right)e(-\tilde{u}\xi) e\left(\frac{m_3\xi}{m_1}\right)\psi\left(\frac{m_3}{M_3}\right)f\left(\frac{m_1 \tilde{u}}{m_2} , (m_1 \tilde{y})m_2^{-1}\right) d\tilde{u}\\
    =& \sum_{\substack{m_1\sim M_1,m_2\sim M_2\\ (m_1,d)=(m_2,d)=1}} \sum_{\tilde{y}=0}^{d-1}\int e\left(\frac{-x\tilde{y}}{d}\right)e(-\tilde{u}\xi) f\left(\frac{m_1 \tilde{u}}{m_2} , (m_1 \tilde{y})m_2^{-1}\right) d\tilde{u}
    \sum_{m_3\in \mathbb{Z}}
    \psi\left(\frac{m_3}{M_3}\right)e\left(\frac{[{m_3x}{m_1}^{-1}]}{d}\right)e\left(\frac{m_3\xi}{m_1}\right)\\
    =&\sum_{\substack{m_1\sim M_1,m_2\sim M_2\\ (m_1,d)=(m_2,d)=1}} \frac{m_2}{m_1}\hat{f}\left(\frac{m_2}{m_1}\xi, m_2 m_1^{-1}x\right)
    \sum_{m_3\in \mathbb{Z}}
    \psi\left(\frac{m_3}{M_3}\right)e\left(\frac{[{m_3x}{m_1}^{-1}]}{d}\right)e\left(\frac{m_3\xi}{m_1}\right)\\
    \stackon{Poisson}{=}&\sum_{\substack{m_1\sim M_1,m_2\sim M_2\\ (m_1,d)=(m_2,d)=1}} \frac{m_2}{m_1}\hat{f}\left(\frac{m_2}{m_1}\xi, m_2 m_1^{-1}x\right)
    \sum_{l\in \mathbb{Z}}M_3
    \hat{\psi}\left(M_3\left(l-\frac{\xi}{m_1}-\frac{[m_1^{-1}x]}{d}\right)\right)
\end{align*}
By non-stationary phase, $\hat{\psi}$ has rapid decay and thus we can truncate the summation in $l$ to \[
    M_3\left(l-\frac{\xi}{m_1}-\frac{[m_1^{-1}x]}{d}\right) \lesssim 1 \implies  m_1\left(l-\frac{[m_1^{-1}x]}{d}\right)- \xi \lesssim \frac{M_1}{M_3}
\]
with negligible error.
Thus we have \begin{align*}
    |\hat{g}(\xi,x)|\lesssim 
    \sum_{\substack{m_1\sim M_1\\(m_1,d)=1}} 
    \sum_{l : m_1\left(l-\frac{[m_1^{-1}x]}{d}\right)- \xi \lesssim \frac{M_1}{M_3} } M_3\Bigg|
    \sum_{\substack{m_2\sim M_2\\(m_2,d)=1}} \frac{m_2}{m_1}\hat{f}\left(\frac{m_2}{m_1}\xi, m_2 m_1^{-1}x\right) 
    \Bigg|+ O(T^{-100})
\end{align*}
where the choice of the representative $[m_1^{-1}x]$ is arbitrary, as it only induces a shift in the the range of $l$ by an integer.

Let $\eta>0$. We break \[
\sum_{x\mod d} \int |\hat{g}(\xi,x)|^2 d\xi
\]
into five pieces. \begin{enumerate}
    \item [Ai] $\int_{|\xi|<\frac{M_1}{M_3}T^\eta} |\hat{g}(\xi,0)|^2 d\xi$
    \item [Aii]$\sum_{x\neq 0 \mod d}\int_{|\xi|<\frac{M_1}{M_3}T^\eta} |\hat{g}(\xi,x)|^2 d\xi$
    \item [Bi] $\int_{\frac{M_1}{M_3}\leq|\xi|<(qT)^4} |\hat{g}(\xi,0)|^2 d\xi$
    \item [Bii]$\sum_{x\neq 0 \mod d}\int_{\frac{M_1}{M_3}\leq|\xi|<(qT)^4} |\hat{g}(\xi,x)|^2 d\xi$
    \item [C]  $\sum_{x \mod d}\int_{|\xi|\geq (qT)^4} |\hat{g}(\xi,x)|^2 d\xi$
\end{enumerate}
By the decay in $\hat{f}$ and $T>d^{1+\epsilon}, M^{1+\epsilon}$, C is negligible. 

\begin{lemma}[Ai bound]\label{affinemainterm}
    We have \[
    \rm{Ai} \lesssim T^{3\eta}M^6\left(\sum_y \int f(u,y) du\right)^2
    \]
\end{lemma}
\begin{proof}
    We have
    \[
        |\hat{g}(\xi,0)|\lesssim 
        \sum_{\substack{m_1\sim M_1\\(m_1,d)=1}} 
        \sum_{l : m_1l- \xi \lesssim \frac{M_1}{M_3} }M_3\Bigg|
        \sum_{\substack{m_2\sim M_2\\(m_2,d)=1}} \frac{m_2}{m_1}\hat{f}\left(\frac{m_2}{m_1}\xi, 0\right) 
        \Bigg|+ O(T^{-100}),
    \]
    so the possible non-zero contributions of $l$ satisfy $l\lesssim T^\eta/M_3$. Therefore, for $\xi\leq T^\eta M_1/M_3$\begin{align*}
    |\hat{g}(\xi,0)|^2 \lesssim& \Bigg|\Big(M_3+T^\eta\Big)\sum_{\substack{m_1\sim M_1\\(m_1,d)=1}} 
    \Big|
        \sum_{\substack{m_2\sim M_2\\(m_2,d)=1}} \frac{m_2}{m_1}\hat{f}\left(\frac{m_2}{m_1}\xi, 0\right) 
        \Big| \Bigg|^2\\
        \stackon{CS}{\ll}& M_3^2T^{2\eta} M_1\sum_{\substack{m_1\sim M_1\\(m_1,d)=1}} 
        \Big|\sum_{\substack{m_2\sim M_2\\(m_2,d)=1}} \frac{m_2}{m_1}\hat{f}\left(\frac{m_2}{m_1}\xi, 0\right) \Big|^2\\
        \lesssim & T^{2\eta}M_3^2M_2^4 |\sup_{\xi'} \hat{f}(\xi',0)|^2.
    \end{align*}
    We have $|\hat{f}(\xi',0)|\leq \sum_{y\mod d}\int |f(\xi,y)|$,
    so integrating over the range $\xi\leq M_1/M_3 T^\eta$ and taking $M_1,M_2,M_3\leq M$ gives the result.
\end{proof}


\begin{lemma}[Aii, Bi, Bii bound]\label{affineweirdterm}
    We have
    \[
    \rm{Aii}+\rm{Bi}+\rm{Bii}\lesssim dT^{2\eta}\left(M^4 \sum_{y\mod d}\int f(u,y)^2 du\right)^{1/2} (J(f))^{1/2}
    \]
\end{lemma}
\begin{proof}
    As in Ai bound, the non-zero contribution from $l$ occurs only if \[
        \Bigg|m_1\left(l-\frac{[m_1^{-1}x]}{d}\right)- \xi \lesssim \frac{M_1}{M_3} \Bigg|\implies
        \Bigg|m_1\left(dl-[m_1^{-1}]x\right) -d\xi\Bigg|\lesssim\frac{dM_1T^\eta}{M_3}
    \]
    Let $s=m_1\left(dl-[m_1^{-1}]x\right) $, so that $s$ must be an integer within $\lesssim{dM_1T^\eta}/{M_3}$ of $d\xi$. Moreover, $s\equiv -x \mod d$, so there are $\lesssim (1+M_1T^\eta/M_3)$ candidates of $s$. Since $\xi\geq T^\eta M_1/M_3$ or $x\neq 0$, we also have $s\neq 0$. Thus noticing the choice of $m_1$ and $s$ fixes the value of $l$ we have by the divisor bound \[
    \lesssim 1+\frac{M_1}{M_3}
    \]
    pairs of $(m_1,l)$ that satisfy these conditions. So in this range 
    \begin{align*}
        &|\hat{g}(\xi,x)|^2 \\
        \stackon{CS}{\ll}&T^\eta(M_1M_3+M_3^2)
        \sum_{\substack{m_1\sim M_1\\(m_1,d)=1}}  \sum_{l : m_1\left(l-\frac{[m_1^{-1}x]}{d}\right)- \xi \lesssim \frac{M_1}{M_3} } 
        \Big|
            \sum_{\substack{m_2\sim M_2\\(m_2,d)=1}} \frac{m_2}{m_1}\hat{f}\left(\frac{m_2}{m_1}\xi, m_2m_1^{-1}x\right) 
            \Big|^2,\\
    \end{align*}
    giving 
    \begin{align*}
        \rm{Aii}+{Bi}+{Bii}\leq&\sum_{x \mod d}\int_{|\xi|\leq (qT)^4} (1-\mathbb{I}_{x=0}*\mathbb{I}_{\xi\leq M_1T^{\eta}/M_3})|\hat{g}(\xi,x)|^2 d\xi\\
        {\lesssim}&T^\eta(M_1M_3+M_3^2)
        \sum_{\substack{m_1\sim M_1\\(m_1,d)=1}} \sum_{x \mod d} \int_{|\xi|<(qT)^4}\sum_{l : m_1\left(l-\frac{[m_1^{-1}x]}{d}\right)- \xi \lesssim \frac{M_1}{M_3} } 
        \Big|
            \sum_{\substack{m_2\sim M_2\\(m_2,d)=1}} \frac{m_2}{m_1}\hat{f}\left(\frac{m_2}{m_1}\xi, m_2m_1^{-1}x\right) 
            \Big|^2d\xi\\
            \leq&T^\eta(M_1M_3+M_3^2)
            \sum_{\substack{m_1\sim M_1\\(m_1,d)=1}} \sum_{x \mod d} \sum_{l} 
            \int\displaylimits_{\Big|m_1\left(l-\frac{[m_1^{-1}x]}{d}\right)- \xi \Big|\lesssim \frac{M_1}{M_3}}
            \Big|
                \sum_{\substack{m_2\sim M_2\\(m_2,d)=1}} \frac{m_2}{m_1}\hat{f}\left(\frac{m_2}{m_1}\xi, m_2m_1^{-1}x\right) 
                \Big|^2 d\xi.\\
    \end{align*}
    We now remove the dependence on $m_1$. First $\tilde{x} \equiv x m_1^{-1}$, so that the summation condition in $x$ and $l$ becomes \[
        \sum_{\tilde{x} \mod d} \sum_{l : m_1\left(l-\frac{\tilde{x}}{d}\right)- \xi \lesssim \frac{M_1}{M_3} }.
    \]
    Moreover, for each $m_1$, we now make the change of variables \[
    \xi = m_1\Big(l-\frac{\tilde{x}}{d}\Big)+\tau\frac{m_1}{M_3}
    \]
    so by slightly extending the domain of integration we can find uniformly $|\tau|\lesssim 1$ with Jacobian $\asymp M_1/M_3$.
    Combined we get
    \begin{align*}
        \rm{Aii}+{Bi}+{Bii} \lesssim &
        \frac{M_1}{M_3}(M_3^2+M_3M_1)T^{\eta}
        \sum_{\substack{m_1\sim M_1\\(m_1,d)=1}}\sum_{\tilde{x} \mod d} \sum_{l  } 
        \int_{|\tau|\lesssim 1}
        \Big|
            \sum_{\substack{m_2\sim M_2\\(m_2,d)=1}} \frac{m_2}{m_1}\hat{f}\left(m_2\Big(l-\frac{\tilde{x}}{d}\Big)+\frac{m_2}{M_3}\tau, m_2\tilde{x}\right) 
            \Big|^2 d\tau\\
            \lesssim (M_1+M_3)T^{\eta} \Sigma_1,
    \end{align*}
    where\[
    \Sigma_1 \defeq \sum_{\tilde{x} \mod d} \sum_{l  } 
    \int_{|\tau|\lesssim 1}
    \Big|
        \sum_{\substack{m_2\sim M_2\\(m_2,d)=1}} {m_2}\hat{f}\left(m_2\Big(l-\frac{\tilde{x}}{d}\Big)+\frac{m_2}{M_3}\tau, m_2\tilde{x}\right) 
        \Big|^2d\tau.
    \]
    Since $\hat{f}(\xi,x)$ is negligble unless $|\xi|\lesssim T$, we can introduce a bump function $\psi_2(x)$ supported on $|x|\lesssim 1$ to get \[
        \Sigma_1 = \sum_{\tilde{x} \mod d} \sum_{l  } \psi_2\left(\frac{M_2(l-\tilde{x}/d)}{T}\right)
        \int_{|\tau|\lesssim 1}
        \Big|
            \sum_{\substack{m_2\sim M_2\\(m_2,d)=1}} {m_2}\hat{f}\left(m_2\Big(l-\frac{\tilde{x}}{d}\Big)+\frac{m_2}{M_3}\tau, m_2\tilde{x}\right) 
            \Big|^2d\tau+O(T^{-100}).
    \]
    Expanding the square, we have \begin{align*}
        \Sigma_1= & \sum_{\tilde{x} \mod d} \sum_{l  }  \psi_2\left(\frac{M_2(l-\tilde{x}/d)}{T}\right)
        \int_{|\tau|\lesssim 1}
        \int\int \sum_{\substack{m_2,m_2'\sim M_2\\(m_2,d)=(m_2',d)=1}} \sum_{y,y'\mod d}
         {m_2} m_2' f(u,y)f(u',y')  \\ & \quad \quad \times e\left(\frac{\tilde{x}(m_2'y'-m_2y)}{d}\right) 
         e\left((m_2'u'-m_2u)\Big(l-\frac{\tilde{x}}{d}\Big)\right) e\left((m_2'u'-m_2u)\Big(\frac{\tau}{M_3}\Big)\right) du du' d\tau\\
        &= \sum_{\tilde{x} \mod d} \int\int  \sum_{y,y'\mod d} \sum_{\substack{m_2,m_2'\sim M_2\\(m_2,d)=(m_2',d)=1}} {m_2} m_2' f(u,y)f(u',y') 
        e\left(\frac{\tilde{x}(m_2'y'-m_2y)}{d}\right) 
        \Sigma_{2}\Sigma_3 du du'
           .
    \end{align*}
    where \begin{align*}
        \Sigma_2\defeq &\int_{|\tau|\lesssim 1}  e\left((m_2'u'-m_2u)\Big(\frac{\tau}{M_3}\Big)\right)  d\tau\\
        \Sigma_3\defeq &\sum_{l}  \psi_2\left(\frac{M_2(l-\tilde{x}/d)}{T}\right)e\left((m_2'u'-m_2u)l\right)e\left((m_2u-m_2'u')\Big(\frac{\tilde{x}}{d}\Big)\right)
    \end{align*}
   On $\Sigma_3$, we apply Poisson summation to get \begin{align*}
        \Sigma_3 \quad \stackon{Poisson}{=}& \quad  \frac{T}{M_2}\sum_k \hat{\psi}_2 \left(\frac{k+m_2-m_2'u'}{M_2/T}\right) e\left(-\frac{k\tilde{x}}{d}\right)\\
        =& \frac{T}{M_2}\sum_{\substack{k\\
        |k+m_2u-m_2'u'|\lesssim M_2/T}} \hat{\psi}_2 \left(\frac{k+m_2u-m_2'u'}{M_2/T}\right)+O(T^{-100}),
    \end{align*}
    where we truncated in the last step by the rapid decay of $\hat\psi_2$.
    We can directly compute the remaining dependecies on $\tilde{x}$ \[
        \sum_{\tilde{x} \mod d} e\left(\frac{\tilde{x}(m_2'y'-m_2y)}{d}\right)e\left(-\frac{k\tilde{x}}{d}\right)=d \mathbb{I}_{d|(m_2'y'-m_2y-k)}.
    \]
    Finally, we trivially bound \[
        |\Sigma_2|\lesssim 1.
     \]
     This gives us\begin{align*}
        \Sigma_1\lesssim
        d \int  {M_2}T \sum_{y\mod d}\sum_{\substack{m_2,m_2'\sim M_2\\(m_2,d)=(m_2',d)=1
        }} \sum_{\substack{k}}  f(u,y)\sum_{\substack{y'\mod d\\d|(m_2'y'-m_2y-k)}} \int_{\substack{
        |k+m_2u-m_2'u'|\lesssim M_2/T}} f(u',y')  du' du.
     \end{align*}
    The condition $d|(m_2'y'-m_2y-k)$ forces $y'= (m_2y+k)m_2'^{-1}$.
    For the second condition, we notice that $u'$ is in a $\lesssim 1/T$ neighborhood of $(m_2u+k)/m_2'$, so let $\phi(x)$ be a bump function supported on $|x|\lesssim 1$ \[
    \tilde{f}(u,y) \defeq \int T \phi(T(u'-u)) f(u',y) du'
    \]
    if needed, we extend the support of $\phi$ by a constant such that 
    \begin{align*}
        \Sigma_1\lesssim
        d \int  {M_2} \sum_{y\mod d}\sum_{\substack{m_2,m_2'\sim M_2\\(m_2,d)=(m_2',d)=1
        }} \sum_{\substack{k}}  f(u,y) \tilde{f}\left(\frac{m_2u+k}{m_2'},(m_2y+k)m_2'^{-1}\right)  du.
     \end{align*}
     Thus putting this bound for $\Sigma_1$ we obtain
     \begin{align*}
        \rm{Aii}+{Bi}+{Bii}\lesssim& (M_1+M_3)T^{\eta} d \int  {M_2} \sum_{y\mod d}\sum_{\substack{m_2,m_2'\sim M_2\\(m_2,d)=(m_2',d)=1
        }} \sum_{\substack{k}}  f(u,y) \tilde{f}\left(\frac{m_2y+k}{m_2'},(m_2y+k)m_2'^{-1}\right)  du\\
        \stackon{CS}{\ll}& M^2T^{\eta}d \Bigg(\sum_{y\mod d}\int f(u,y)^2 du \Bigg) ^{1/2}
        \Bigg(\sum_{y\mod d} \int \Big(\sum_{\substack{m_2,m_2'\sim M_2\\(m_2,d)=(m_2',d)=1
        }} \sum_{\substack{k}} \tilde{f}\left(\frac{m_2u+k}{m_2'},(m_2y+k)m_2'^{-1}\right)\Big)^2 du\Bigg) ^{1/2}
    \end{align*}
    Finally, noticing that $\tilde{f}$ has support on $y\asymp 1$ we can restrict the summation in $k$ to be $|k|\ll M_2$, so that 
    \begin{align*}
        \rm{Aii}+{Bi}+{Bii}\lesssim M^2T^{\eta}d \Bigg(\sum_{y\mod d}\int f(u,y)^2 du \Bigg) ^{1/2}
        \Bigg(J(\tilde{f})\Bigg) ^{1/2}.
    \end{align*}
\end{proof}
\begin{proof}[Proof of Lemma \ref{iterative}]
    Recall that\[
J(f)\leq \frac{1}{d}\sum_{x\mod d} \int |\hat{g}(\xi,x)|^2 d\xi.
\]
By the decay in $\hat{f}$, the range $|\xi|\geq T^2$ is negligble. The remaining parts are bounded by lemmas \ref{affinemainterm} and \ref{affineweirdterm} to get \[
J(f)\lesssim \frac{1}{d}T^{3\eta}M^6\left(\sum_y \int f(u,y) du\right)^2+ M^2T^{\eta} \Bigg(\sum_{y\mod d}\int f(u,y)^2 du \Bigg) ^{1/2}
\Bigg(J(\tilde{f})\Bigg) ^{1/2}.
\]
Letting $\eta \to 0$ sufficiently slowly gives the result.
\end{proof}


\begin{corollary}\label{cor: affine}
    Let $f$ be as in Proposition \ref{affinetrans}, except $f$ is now supported on $u \asymp L$, and $\tilde{M}={ML}$. Then \begin{align*}
        J_{M}(f)\defeq &\sup_{\substack{0<M_1,M_3<\tilde{M}, \\ 0< M_2<{M}}}\sum_{y \mod d} \int\Bigg( \sum_{\substack{|m_1|\sim M_1,|m_2|\sim M_2 , |m_3|\ll M_3\\ (m_1,d)=(m_2,d)=1}} f\left(\frac{m_1 u+m_3}{m_2},(m_1 y+m_3)m_2^{-1}\right)\Bigg)^2 \ du\\
        \lesssim&\frac{M^4\tilde{M}^2}{d}\Big(\sum_{y \mod d } \int f(u,y) du\Big)^2  + {M}^2\tilde{M}^2\sum_{y\mod d} \int f(u,y)^2 du\\
        =&\frac{\tilde{M}^6}{L^4d}\Big(\sum_{y \mod d } \int f(u,y) du\Big)^2  + \frac{\tilde{M}^4}{L^2}\sum_{y\mod d} \int f(u,y)^2 du
        .
    \end{align*}
\end{corollary}
\begin{proof}
    By the same argument as above, we keep track of the $M_2$'s to get 
    \[
        J_{{M}}(f) \lesssim \frac{M^4\tilde{M}^2}{d} \Big(\sum_{y \mod d } \int f(u,y) du\Big)^2 + \Big(M^2\tilde{M}^2\sum_{y\mod d} \int f(u,y)^2 du\Big)^{1/2}(J_{M}(\tilde{f}))^{1/2},
    \] where $J(\tilde{f})$ uses the bounds $m_1,m_2\sim M_2, m_3 \ll LM_2\leq \tilde{M}$.
\end{proof}
