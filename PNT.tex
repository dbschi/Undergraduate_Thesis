\section{The Prime Number Theorem}

\begin{theorem}[Prime Number Theorem]
    Let $\Pi(x)=\sum_{p\leq n} 1$. Then \[
        \Pi(x) = (1+o(1))\frac{x}{\log x}.
    \]
\end{theorem}

In this section we will prove the Prime Number Theorem.
However, note that this result is a minor goal of this paper.
The Prime Number Theorem provides an illustration of multiple ideas
and techniques of manipulating the zeta function.
Furthermore, this theorem serves as a starting point for
studying primes in short intervals, and demonstrates why the proof of
RH (or the weaker LH) is widely sought after, thus sets the stage for zero-density theorems.
\begin{definition}[Von Mangoldt Function]
    The \textbf{Von Mangoldt function} $\Lambda$ is defined as follows:
    \[
        \Lambda(n) = \begin{cases}
            \log p \textrm{, if $n = p^k$ for some $k\in\naturals$}\\
            0 \textrm{, else}
        \end{cases}
    \]
\end{definition}
The sum of the Von Mangoldt function $\sum\Lambda (n)$ is a more natural way to express
a prime counting function in the language of $\zeta$. To see why, consider the expression
\begin{align*}
    \frac{\zeta '(s)}{\zeta(s)} &= (\log\zeta(s))'\\
    &= \left[ -\sum_{p} \log\left(1-p^{-s}\right)\right]^{\prime}\\
    &= - \sum_{p} \frac{p^s \log p }{1-p^{-s}}\\
    &= -\sum_{p} \log p \sum_{k\in\naturals} p^{-ks}\\
    &= - \sum_{n\in\naturals} \frac{\Lambda(n)}{n^s}
\end{align*}
On $\Re(s)>1$ where the sum and products are absolutely convergent. We thus motivate the following technique:
Let $\phi$ be smooth and rapidly decaying at infinity, and $\tilde{\phi}$ be its Mellin transform. Let $N\in\naturals$ and $c\geq 2$. Then \begin{equation} \label{mellinonzeta}
    \begin{split}
    \sum_{n\in\naturals} \Lambda(n) \phi\left(\frac{n}{N}\right) &=
    \sum_{n\in\naturals} \Lambda(n) \frac{1}{2\pi i}\int_{c-i\infty}^{c+i\infty}
    \tilde{\phi}(s)\left(\frac{n}{N}\right)^{-s} ds \\
    &= \frac{1}{2\pi i}\int_{c-i\infty}^{c+i\infty}
    \tilde{\phi}(s)\sum_{n\in\naturals} \Lambda(n) \left(\frac{n}{N}\right)^{-s} ds \\
    &= \frac{-1}{2\pi i}\int_{c-i\infty}^{c+i\infty}
    \tilde{\phi}(s) N^s \frac{\zeta'(s)}{\zeta(s)} ds \\
    \end{split}
\end{equation}
Now take $\phi=1$ on $[0,1]$ and $0$ outside $[0-\epsilon, 1+\epsilon]$. 
Applying Dominated Convergence theorem on equation \label{mellinonzeta} gives for $\epsilon \downarrow 0$,
\[
    \sum_{n\leq N} \Lambda(n)=
    \frac{-1}{2\pi i}\int_{c-i\infty}^{c+i\infty}
    \frac{1}{s} N^s \frac{\zeta'(s)}{\zeta(s)} ds
\]

