
\section{Energy bound}
Here we provide the generalization for the orthogonal energy bound for Guth and Maynard's result. 
\begin{proposition}\label{energybound}
    \begin{align*}
        E(\mathcal{S})
        \lesssim& |\mathcal{S}|^2N^{2-2\sigma}+ |\mathcal{S}|^3N^{1-2\sigma} + |\mathcal{S}|^{9/4}(qT)^{1/2}N^{{1-2\sigma}}.
    \end{align*}
\end{proposition}
The idea for bounding energy is similar; if $\chi_1\chi_2=\chi_3\chi_4$ and $|t_1+t_2-t_3-t_4|$ is small, we should expect $|D_N(t_1+t_2-t_3,\chi_1\chi_2\bar{\chi}_3)|\simeq |D_N(t_4,\chi_4)|>N^\sigma$.
\begin{lemma}
    \[
    D_N(t,\chi)\lesssim \int_{|u-t|\lesssim 1} |D_N(u,\chi)|du + O(T^{-100}),
    \]
    uniformly in $\chi$.
\end{lemma}
\begin{proof}
    (GM)
    \begin{align*}
        D_N(t,\chi)=\sum_n \omega\left(\frac{n}{N}\right) b_n n^{it} \psi\left(\frac{\log n}{2\pi}\right)
    \end{align*}
    For other characters, we can just redefine $b'_n=b_n\chi(n)$.
\end{proof}
\begin{lemma} We have \[
    E(\mathcal{S})\lesssim N^{-2\sigma}\sum_{n_1,n_2\sim N}\left|R\left(\frac{n_1}{n_2},n_1,n_2\right) \right|^3.
    \]
    
\end{lemma}
\begin{proof}
    We have \begin{align*}    
    E(\mathcal{S})=& \sum_{\substack{(t_1,\chi_1),(t_2,\chi_2),\\ (t_3,\chi_3),(t_4,\chi_4)\in \mathcal{S}\\|t_1+t_2-t_3-t_4|\leq 1\\ \chi_1\chi_2=\chi_3\chi_4}} 1 
    \leq N^{-2\sigma}\sum_{\substack{(t_1,\chi_1),(t_2,\chi_2),\\ (t_3,\chi_3),(t_4,\chi_4)\in \mathcal{S}\\ |t_1+t_2-t_3-t_4|\leq 1\\ \chi_1\chi_2=\chi_3\chi_4}} |D_N(t_4,\chi_4)|^2.
    \end{align*}
    Now we apply the previous lemma and Cauchy-Schwarz to get \[
        |D_N(t_4,\chi_4)|^2\lesssim \int_{|u-t_4|\lesssim 1} |D_N(u,\chi_4)|^2du \lesssim  \int_{|u-t_1-t_2+t_3|\lesssim 1} |D_N(u,\chi_1\chi_2\bar{\chi}_3)|^2du,
    \]
   
    Since $\chi_1,\chi_2,\chi_3$ fixes $\chi_4$, and the $t$'s within the same character are $T^\epsilon$ separated, there is $O(1)$ possible pairs of $(t_4,\chi_4)$ for each choice of $(t_1,\chi_1),(t_2,\chi_2),(t_3,\chi_3)$, so \begin{align*}    
        E(\mathcal{S})\lesssim& N^{-2\sigma}\sum_{\substack{(t_1,\chi_1),(t_2,\chi_2),\\ (t_3,\chi_3),(t_4,\chi_4)\in \mathcal{S}\\ |t_1+t_2-t_3-t_4|\leq 1\\ \chi_1\chi_2=\chi_3\chi_4}} \int_{|u-t_1-t_2+t_3|\lesssim 1} |D_N(u,\chi_1\chi_2\bar{\chi}_3)|^2du\\
        \lesssim&  N^{-2\sigma}\sum_{\substack{(t_1,\chi_1),(t_2,\chi_2),\\ (t_3,\chi_3)\in \mathcal{S}}} \int_{|u-t_1-t_2+t_3|\lesssim 1} |D_N(u,\chi_1\chi_2\bar{\chi}_3)|^2du\\
        =& N^{-2\sigma}\sum_{\substack{(t_1,\chi_1),(t_2,\chi_2),\\ (t_3,\chi_3)\in \mathcal{S}}} \int_{|u|\lesssim 1} |D_N(t_1+t_2-t_3 + u,\chi_1\chi_2\bar{\chi}_3)|^2du\\
        =&N^{-2\sigma}\sum_{n_1,n_2} b_{n_1}\bar{b}_{n_2} \omega\left(\frac{n_1}{N}\right)\omega\left(\frac{n_2}{N}\right)\int_{|u|\lesssim 1} \left(\frac{n_1}{n_1}\right)^{iu}R\left(\frac{n_1}{n_2},n_1,n_2\right)^2 R\left(\frac{n_2}{n_1},n_2,n_1\right)du\\
        \lesssim & N^{-2\sigma}\sum_{n_1,n_2\sim N}\left|R\left(\frac{n_1}{n_2},n_1,n_2\right)^2 R\left(\frac{n_2}{n_1},n_2,n_1\right)\right|\\
        \lesssim & N^{-2\sigma}\sum_{n_1,n_2\sim N}\left|R\left(\frac{n_1}{n_2},n_1,n_2\right) \right|^3.
        \end{align*}
\end{proof}

\begin{lemma}
    We have 
\[
        \sum_{n_1,n_2\sim N}\left|R\left(\frac{n_1}{n_2},n_1,n_2\right) \right|^2\lesssim |\mathcal{S}|N^2+ |\mathcal{S}|^2N + |\mathcal{S}|^{5/4}(qT)^{1/2}N.
    \]
\end{lemma}
\begin{proof}
    From the definition of $R$, \begin{align*}
        \sum_{n_1,n_2\sim N}\left|R\left(\frac{n_1}{n_2},n_1,n_2\right) \right|^2=& \sum_{n_1,n_2\sim N}\sum_{\substack{(t_1,\chi_1),(t_2,\chi_2)\in \mathcal{S}}}
        \chi_1(n_1)\bar{\chi}_1(n_2)\left(\frac{n_1}{n_2}\right)^{it_1}\bar{\chi}_2(n_1){\chi}_2(n_2)\left(\frac{n_1}{n_2}\right)^{-it_2}
        \\
        &=\sum_{\substack{(t_1,\chi_1),(t_2,\chi_2)\in \mathcal{S}}}\left|\sum_{n\sim N}
        \chi_1(n_1)\bar{\chi}_2(n_1)n^{i(t_1-t_2)}\right|.
    \end{align*}
    A direct application of Heath Brown's Theorem \ref{heathbrown} gives \[
        \sum_{n_1,n_2\sim N}\left|R\left(\frac{n_1}{n_2},n_1,n_2\right) \right|^2\lesssim |\mathcal{S}|N^2+ |\mathcal{S}|^2N + |\mathcal{S}|^{5/4}(qT)^{1/2}N.
    \]
\end{proof}
The trivial bound for $R\leq |\mathcal{S}|$ gives \begin{align*}
    E(\mathcal{S})\lesssim& N^{-2\sigma}\sum_{n_1,n_2\sim N}\left|R\left(\frac{n_1}{n_2},n_1,n_2\right) \right|^3\\
    \lesssim &|\mathcal{S}|N^{-2\sigma}\sum_{n_1,n_2\sim N}\left|R\left(\frac{n_1}{n_2},n_1,n_2\right) \right|^2\\
    \lesssim& |\mathcal{S}|^2N^{2-2\sigma}+ |\mathcal{S}|^3N^{1-2\sigma} + |\mathcal{S}|^{9/4}(qT)^{1/2}N^{{1-2\sigma}}.
\end{align*}
\textcolor{red}{TODO: Merge with previous part to find bound}

The arguments beyond will be adaptations from GM. 

\begin{lemma}
    We have \begin{align*}
        \sum_{n_1,n_2\sim N}\left|R\left(\frac{n_1}{n_2},n_1,n_2\right) \right|^4\lesssim ?
    \end{align*}
\end{lemma}
\begin{proof}
    Let \[
    U_B=\left\{u\in\mathbb{Z} \ : \ \#\{((t_1,\chi_1),(t_2,\chi_2))\in \mathcal{S} \ : \ \lfloor {t_1-t_2}\rfloor = u \} \sim B \right\},
    \]
    so that we split the sum in $R$ as \begin{align*}
        \left|R\left(\frac{n_1}{n_2} ,n_1,n_2\right) \right|^4 =&
        \left|
        \sum_{(t_1,\chi_1),(t_2,\chi_2)\in \mathcal{S}}
        \chi_1(n_1)\bar{\chi}_2(n_2) \left(\frac{n_1}{n_2}\right)^{i(t_1-t_2)}\right|^2 \\
        =& \left|
        \sum_{j=0}^{ \lfloor \log_2 |\mathcal{S}|\rfloor} \sum_{u\in U_{2^j}}\sum_{\substack{(t_1,\chi_1),(t_2,\chi_2)\in \mathcal{S}\\ \lfloor t_1-t_2 \rfloor = u}} 
        \chi_1(n_1)\bar{\chi}_2(n_2) \left(\frac{n_1}{n_2}\right)^{i(t_1-t_2)}\right|^2\\
        \lesssim& \sum_{j=0}^{ \lfloor \log_2 |\mathcal{S}|\rfloor} \left|
         \sum_{u\in U_{2^j}}\sum_{\substack{(t_1,\chi_1),(t_2,\chi_2)\in \mathcal{S}\\ \lfloor t_1-t_2 \rfloor = u}} 
        \chi_1(n_1)\bar{\chi}_2(n_2) \left(\frac{n_1}{n_2}\right)^{i(t_1-t_2)}\right|^2
    \end{align*}
    where we applied Cauchy-Schwarz in the last step.
    Therefore, \begin{align*}
        \sum_{n_1,n_2\sim N}\left|R\left(\frac{n_1}{n_2},n_1,n_2\right) \right|^4 \lesssim & 
        \sup_{j\leq \lfloor \log_2 |\mathcal{S}|\rfloor} \sum_{n_1,n_2\sim N} \left|
        \sum_{u\in U_{2^j}}\sum_{\substack{(t_1,\chi_1),(t_2,\chi_2)\in \mathcal{S}\\ \lfloor t_1-t_2 \rfloor = u}} 
       \chi_1(n_1)\bar{\chi}_2(n_2) \left(\frac{n_1}{n_2}\right)^{i(t_1-t_2)}\right|^2\\
       \leq& \sup_{j\leq \lfloor \log_2 |\mathcal{S}|\rfloor} \sum_{n_1,n_2\sim N} \left|
       \sum_{u\in U_{2^j}}\sum_{\substack{(t_1,\chi_1),(t_2,\chi_2)\in \mathcal{S}\\ \lfloor t_1-t_2 \rfloor = u}} 
      \chi_1(n_1)\bar{\chi}_2(n_2) \left(\frac{n_1}{n_2}\right)^{i(t_1-t_2)}\right|^2
    \end{align*}
\end{proof}