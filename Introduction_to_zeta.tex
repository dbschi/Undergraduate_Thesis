\section{Introduction to the Riemann Zeta Function}

\begin{definition}[Zeta Function]
	Let $s\in\complex$ with $\Re(s)>1$. Then \begin{equation}
	\zeta(s)=\sum_{n=1}^{\infty}\frac{1}{n^s}.
	\end{equation}
\end{definition}
The zeta function converges absolutely on $\Re(s)>1$ by comparing to the integral $\int x^{-\Re(s)} dx$.
\begin{proposition}\label{eulerproduct}
	On $\Re(s)>1$, \begin{equation}
		\zeta(s) = \prod_{p\in\mathbb{N}}\left(1-\frac{1}{p^s}\right)^{-1}.	
	\end{equation}
\end{proposition}
\begin{remark}
	This expresion also converges absolutely for $\Re(s)>1$.
\end{remark}
\begin{proof}[Sketch of proof]
	Write $s=\sigma+it$. For each $p$,\[
		\left(1-\frac{1}{p^s}\right)^{-1} = \left(\frac{1}{p^s}+\frac{1}{p^{2s}}+\frac{1}{p^{3s}}+...\right)
	\]
	converges absolutely for $\Re(s)>1$ and uniformly across all $p$. We thus take for $m>N$ \begin{align*}
		\prod_{p\leq N}\left(1-\frac{1}{p^s}\right)^{-1}  &= \prod_{p\leq N}\left(\sum_{k=1}^{m} \frac{1}{p^{ks}}+O(2^{-m\sigma})\right)\\
		&\stackeq{(*)}\sum_{n=1}^{N}\frac{1}{n^s}+O_1(\sum_{n=N+1}^{\infty} \frac{1}{n^\sigma}) + O(2^{-m\sigma})\\
		&=\zeta(s) + O_1(\sum_{n=N+1}^{\infty} \frac{1}{n^\sigma}) + O(2^{-m\sigma})
	\end{align*}
	Where we apply to Fundemental Theorem of Number Theory in (*) to show that each term $n^{-s}$ has coefficient $1$ determined by the unique prime factorization.
	As $m\to\infty$, $2^{-m\sigma} \to 0$. Then we take $N\to\infty$, the tail of the infinite sum converges to zero too.
\end{proof}

Proposition \ref{eulerproduct} shows an inherent connection of the zeta function with primes.
To further see this connection, we need to extend the zeta function.
\begin{theorem}
	$\zeta$ extends to a meromorphic function on $\complex$ with a simple pole at $s=1$. By abuse of notation, we identify the extension of the zeta function with $\zeta$ too.
	Moreover, $\xi(s)\defeq \pi^{-s/2}\Gamma(s/2)\zeta(s)$ satisfies \begin{equation}\label{symmetryeq}
		\xi(s) = \xi(1-s).
	\end{equation}
\end{theorem}
\begin{proof}
	Using \begin{align*}
		\Gamma(s)=\int_{0}^{\infty} e^{-x}x^{s-1}dx,
	\end{align*}
	We make the substitution $x=\pi n^2y$ to get \begin{align*}
		\Gamma(s)&= \int_{0}^{\infty} e^{-\pi n^2 y} (\pi n^2 y)^{s-1} \pi n^2 dy\\
		\implies \frac{\Gamma(s)}{\pi^s n^{2s}} &= \int_{0}^{\infty} e^{-\pi n^2 y} y^{s-1} dy
	\end{align*}
	So that by the Monotone Convergence Theorem,\begin{align*}
		\pi^{-s/2}\Gamma(s/2)\zeta(s) &= \sum_{n=1}^{\infty}\frac{\Gamma(s/2)}{\pi^{s/2} n^s}\\
		&= \sum_{n=1}^{\infty} \int_{0}^{\infty} e^{-\pi n^2x} x^{s/2-1} dx\\
		&=  \int_{0}^{\infty}\sum_{n=1}^{\infty}\left(e^{-\pi n^2 x}\right)  x^{s/2-1} dx.
	\end{align*}
	We now let \[
		\omega(x) = \sum_{n=1}^{\infty} e^{-\pi n^2 x}, \quad \theta(x) = \sum_{n=-\infty}^{\infty} e^{-\pi n^2 x} = 2\omega(x)+1.
	\]
	We apply Poisson Summation to \begin{align*}
		\theta(x) &= \sum_{n=-\infty}^{\infty} e^{-\pi n^2 x}\\
		&= \sum_{k=-\infty}^{\infty} \int_{-\infty}^{\infty} e^{-\pi y^2 x} e^{-2\pi i k y} dy\\
		&= \sum_{k=-\infty}^{\infty} \int_{-\infty}^{\infty} e^{-\pi y^2 x} e^{-2\pi i k y} dy\\
		&= \sum_{k=-\infty}^{\infty} \frac{1}{\sqrt{x}} \int_{-\infty}^{\infty} e^{-\pi u^2} e^{-2\pi i k u /\sqrt{x}} du\\
		&= \sum_{k=-\infty}^{\infty} \frac{1}{\sqrt{x}} e^{-\pi k^2  / x}\\
		&= \frac{1}{\sqrt{x}} \theta \left(\frac{1}{x}\right)
	\end{align*}
	using the substitution $y\sqrt{x}=u$. Replacing with $\omega$, \[
	\sqrt{x}(2\omega(x)+1) = 2\omega\left(\frac{1}{x}\right)+1
	\implies \omega\left(\frac{1}{x}\right)=\sqrt{x}\omega(x) +\frac{\sqrt{x}}{2} - \frac{1}{2}
	\]
	We thus write, using $y=1/x$, \begin{align*}
		\xi(s) &= \int_{0}^{1}\omega(x) x^{s/2-1} dx +
		\int_{1}^{\infty}\omega(x)  x^{s/2-1} dx\\
		&=\int_{1}^{\infty}\omega(1/y) y^{-s/2-1} dy +
		\int_{1}^{\infty}\omega(x)  x^{s/2-1} dx\\
		&=\int_{1}^{\infty}\left(
		\sqrt{y}\omega(y) +\frac{\sqrt{y}}{2} - \frac{1}{2}
		\right) y^{-s/2-1} dy +
		\int_{1}^{\infty}\omega(x)  x^{s/2-1} dx\\
		&= \int_{1}^{\infty}\left(
			\frac{\sqrt{y}}{2} - \frac{1}{2}
			\right) y^{-s/2-1} dy+ \int_{1}^{\infty}\omega(x)  \left(x^{s/2-1} + x^{-s/2-1/2}\right)dx\\
		&= -\frac{1}{1-s}+ \frac{1}{s}+ 
			\int_{1}^{\infty}\left(
				\frac{\sqrt{y}}{2} - \frac{1}{2}
				\right) y^{-s/2-1} dy+ \int_{1}^{\infty}\omega(x)  \left(x^{s/2-1} + x^{-s/2-1/2}\right)dx\\
		&= \frac{1}{s(1-s)}+
			\int_{1}^{\infty}\left(
				\frac{\sqrt{y}}{2} - \frac{1}{2}
				\right) y^{-s/2-1} dy+ \int_{1}^{\infty}\omega(x)  \left(x^{s/2-1} + x^{-s/2-1/2}\right)dx
	\end{align*}
	$\omega$ decays exponentially in $x$, so the integral converges and the last is well defined on $\complex$ except when $s=1$ or $s=0$.
	Finally, notice that the last expression is symmetric when $s$ is replaced with $(1-s)$, so shows equation \ref{symmetryeq}.
\end{proof}
\begin{corollary}
	$\forall n\in\naturals, \zeta(-2n)=0.$
\end{corollary}
\begin{proof}
	$\Gamma$ has a simple pole at $-n$.
\end{proof}
\begin{corollary}
	
\end{corollary}
