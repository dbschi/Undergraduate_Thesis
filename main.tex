\documentclass[openany,
ngerman,
toc=flat,
toc=chapterentrywithdots,
captions=tableabove,
listof=entryprefix,
listof=leveldown,
fontsize=12pt,
numbers=noenddot]
{book}
\usepackage{blindtext}
\usepackage{xcolor}
\usepackage{amsfonts}
\usepackage{hyperref} 
\usepackage{subcaption}
\usepackage{indentfirst}
\usepackage{graphicx}
\usepackage[normalem]{ulem}
\usepackage [english]{babel}
\usepackage [autostyle, english = american]{csquotes}
\usepackage{appendix}
\usepackage{tabularx}
\usepackage{bbm}
\usepackage{amsmath,amssymb}
\usepackage{mathtools}
\usepackage{makecell}
\graphicspath{ {./images/} }
\usepackage{enumitem}
\usepackage{soul}
\usepackage{mathtools}
\usepackage{wrapfig}
\usepackage{mathrsfs}

\usepackage{MnSymbol,wasysym}
\usepackage{centernot}
\usepackage{amsthm}
\usepackage{biblatex}
\addbibresource{citations.bib}
\allowdisplaybreaks

\MakeOuterQuote{"}

\newtheorem{theorem}{Theorem}[chapter]
\newtheorem{definition}[theorem]{Definition}
\newtheorem{conjecture}[theorem]{Conjecture}
\newtheorem{corollary}[theorem]{Corollary}
\newtheorem{lemma}[theorem]{Lemma}
\newtheorem{proposition}[theorem]{Proposition}
\newtheorem{example}[theorem]{Example}
\newenvironment{remark}[1][Remark:]{\begin{trivlist}
		\item[\hskip \labelsep {\bfseries #1}]}{\end{trivlist}}

\newcommand{\defeq}{\mathrel{\mathop:}=}
\newcommand{\stackeq}[1]{\stackrel{\mathclap{\normalfont\mbox{\normalfont\tiny {#1}}}}{=}}
\newcommand{\stackon}[2]{\stackrel{\mathclap{\normalfont\mbox{\normalfont\tiny {#1}}}}{#2}}
\newcommand\reals{\mathbb{R}}
\newcommand\field{\mathbb{F}}
\newcommand\complex{\mathbb{C}}
\newcommand\schwartz{\mathcal{S}}
\renewcommand{\mod}{ \textrm{ mod }}
\newcommand\naturals{\mathbb{N}}
\usepackage{geometry}
\geometry{
	letterpaper,
	margin=1in
}
\usepackage{blindtext}
\usepackage{tikz,tcolorbox,tikz-3dplot}
\usepackage{pgfplots}
\usepgfplotslibrary{fillbetween}
\usepackage{xcolor}
\usepackage{amsfonts}
\usepackage{hyperref} 
\usepackage{indentfirst}
\usepackage{graphicx}
\usepackage{caption}
\usepackage{subcaption}
\usepackage[normalem]{ulem}
\usepackage [english]{babel}
\usepackage [autostyle, english = american]{csquotes}
\usepackage{appendix}

\usepackage[ruled,linesnumbered]{algorithm2e}
\SetKwComment{Comment}{/* }{ */}
\DontPrintSemicolon 
\usepackage{tabularx}
\usepackage{amsmath,amssymb}
\usepackage{mathtools}
\usepackage{makecell}
\graphicspath{ {./images/} }
\usepackage{enumitem}
\usepackage{soul}
\usepackage{mathtools}
\usepackage{wrapfig}
\usepackage{mathrsfs}
\usepackage{centernot}
\usepackage{amsthm}
\usepackage{listings}
\setcounter{chapter}{-1}
\title{Large values of Dirichlet Polynomials and Zero Density Results of the Riemann Zeta function}
\author{Yung Chi Li}
%\date{January 2024}
\usepackage{titling}
\predate{}
\postdate{}
\date{}
\begin{document}

%\maketitle
\begin{titlepage}
    \begin{center}
        \vspace*{1cm}
            
        \Large
        \textbf{Large values of Dirichlet Polynomials and Zero Density Results of the Riemann Zeta Function}
            
        \vspace{0.5cm}
        
        \Large
        Yung Chi Li
            
        \vfill
            
       	Submitted in partial fulfillment of the\\ requirements of
		Honors in Mathematics\\
		
		
		Northwestern University
            
    
            
    \end{center}
\end{titlepage}
\tableofcontents
\section*{Statement Regarding Generative AI}
No part of this thesis had been aided by generative AI.

\section*{Acknowledgements}
I thank my advisor, Professor Maksym Radziwi\l\l, for his guidance on topic selection, explanation of current literature, and insights into solving hybrid Guth-Maynard. 
\chapter{Notation and Preliminaries}
\section*{Notation}
Below are the notational preferences of the author.
\begin{enumerate}
    \item $p$ always denotes a prime, and by extension $p_j,p_n$ etc.
\end{enumerate}

\section{Preliminaries}
Here we give some supplementary definitions and statements of theorems.
\begin{theorem}[M\"obius Inversion]
	The M\"obius function $\mu$ is defined for $n\in\naturals$,
	\[
		\mu(n) = \begin{cases}
			1, \textrm{ if } n=1\\
			(-1)^k, \textrm{ if }n=p_1p_2...p_k\textrm{ for distinct }p\textrm{'s}\\
			0, \textrm {otherwise}
		\end{cases}
	\]
	Suppose we have arithmetic functions $f,g$, and that\[
		f(n) = \sum_{d|n} g(d)
	\]
	Then the M\"obius Inversion formula gives 
	\[
		g(n) = \sum_{d|n} \mu(d) f\left(\frac{n}{d}\right)
	\]
\end{theorem}
\begin{example}
	On $\Re(s)>1$, let $M_N(s) = \sum_{n\leq N} \mu(n)n^{-s}$.
	Then setting $f(n)=1$ for all $n$, $g(1)=1$, $g(n)=0$ for $n\geq 2$, we multiply $M_N$ by $\zeta$ in Dirichlet series to get\[
		\zeta(s)M_N(s) = \sum_{n} \frac{a_n}{n^{-s}},
	\]
	where $a_n=g(n)$ for all $n\leq N$.
	Similarly, letting $M_N(s) = \sum_{n\leq N} \chi(n) \mu(n) n^{-s}$ for some Dirichlet character $\chi$,
	we get \[
		L(s,\chi)M_N(s) = \sum_{n} \frac{a_n\chi(n)}{n^{-s}}
	\]
	with the same $a_n$ as in the previous equation.
\end{example}

\begin{theorem}[Fourier Inversion]
	In Schwartz space, the Fourier transform of $\mathcal{F}:\schwartz(\reals^d)\to\schwartz(\reals^d)$ of $f\in\mathcal{S}(\reals^d)$ is given by
    \[
        \hat{f}(\mathbf{\xi})\defeq\mathcal{F}f(\mathbf{\xi})\defeq \int_{\reals^d} e(- \mathbf{\xi}\cdot \mathbf{x}) f(\mathbf{x}) \ d\mathbf{x}
    \]
    has inverse given by 
    \[
       f(\mathbf{x})= \mathcal{F}^{-1}\hat{f}(\mathbf{x})\defeq \int_{\reals^d} e(\mathbf{\xi}\cdot \mathbf{x}) \hat{f}(\mathbf{\xi}) \ d\mathbf{\xi}.
    \]
\end{theorem}
\begin{theorem}[Mellin Inversion]
	The Mellin transform of a function $f:(0,\infty)\to\complex$
	\[
	\tilde{f}(s) \defeq \mathcal{M}f(s) \defeq \int_{0}^{\infty}f(x) x^{s-1} \ dx
	\]
	has inverse \[
	\mathcal{M}^{-1} \tilde{f} (x) = \int_{c-i\infty}^{c+i\infty} \tilde{f}(s) x^{-s}ds
	\]
	on $a<c<b$ provided that the integral $\tilde{f}$ is absolute convergent on the strip $a<\Re(s)<b$.
\end{theorem}
\begin{theorem}[Poisson Summation]
	
\end{theorem}
	\chapter{ The Riemann Zeta Function and the Prime Number Theorem}
	%\chapter{Background and Introduction}
	\section{Introduction to the Riemann Zeta Function}
We give a quick introduction to the Zeta function in this section, including its product representation and analytic continuation.
\begin{definition}[Zeta Function]
	Let $s\in\complex$ with $\Re(s)>1$. Then \begin{equation}
	\zeta(s)=\sum_{n=1}^{\infty}\frac{1}{n^s}.
	\end{equation}
\end{definition}
The zeta function converges absolutely on $\Re(s)>1$ by comparing to the integral $\int x^{-\Re(s)} dx$.
The properties of the zeta function as they relate to the distribution primes. In particular, the Dirichlet series can be represented as a product of primes.
\begin{proposition}\label{eulerproduct}
	On $\Re(s)>1$, \begin{equation}
		\zeta(s) = \prod_{p\in\mathbb{N}}\left(1-\frac{1}{p^s}\right)^{-1}.	
	\end{equation}
\end{proposition}
\begin{remark}
	This expresion also converges absolutely for $\Re(s)>1$. Since \[
		\left(1-\frac{1}{p^s}\right)^{-1} = \frac{p^s}{p^s-1} = 1 + \frac{1}{p^s-1}
	\]
	and $\sum(p^s-1)^{-1}$ converges absolutely by comparison to the zeta function Dirichlet series.
\end{remark}
\begin{proof}[Sketch of proof]
	Write $s=\sigma+it$. For each $p$,\[
		\left(1-\frac{1}{p^s}\right)^{-1} = \left(\frac{1}{p^s}+\frac{1}{p^{2s}}+\frac{1}{p^{3s}}+...\right)
	\]
	converges absolutely for $\Re(s)>1$ and uniformly across all $p$. We thus take for $m>N$ \begin{align*}
		\prod_{p\leq N}\left(1-\frac{1}{p^s}\right)^{-1}  &= \prod_{p\leq N}\left(\sum_{k=1}^{m} \frac{1}{p^{ks}}+O(2^{-m\sigma})\right)\\
		&\stackeq{(*)}\sum_{n=1}^{N}\frac{1}{n^s}+O(\sum_{n=N+1}^{\infty} \frac{1}{n^\sigma}) + O(2^{-m\sigma})\\
		&=\zeta(s) + O(\sum_{n=N+1}^{\infty} \frac{1}{n^\sigma}) + O(2^{-m\sigma})
	\end{align*}
	Where we apply to Fundemental Theorem of Arithmetic in (*) to show that each term $n^{-s}$ has coefficient $1$ determined by the unique prime factorization.
	As $m\to\infty$, $2^{-m\sigma} \to 0$. Then we take $N\to\infty$, the tail of the infinite sum converges to zero too.
\end{proof}
\begin{proposition} \label{analyticcontinuation}
	$\zeta$ extends to a meromorphic function on $\complex$ with a simple pole at $s=1$. By abuse of notation, we identify the extension of the zeta function with $\zeta$ too.
\end{proposition}

We will prove Proposition \ref{analyticcontinuation} in two steps. First, we will extend $\zeta$ to $\sigma>0$. Then, we will describe the continuation of the zeta function to the whole plane using by its functional equation: $\zeta$ has a line of symmetry across $\Re (s)=1/2$.
\begin{proposition}\label{extension}
	Let $\xi(s)\defeq \pi^{-s/2}\Gamma(s/2)\zeta(s)$. Then \begin{equation}\label{symmetryeq}
		\xi(s) = \xi(1-s).
	\end{equation}
\end{proposition}
\begin{proof}[Extension of $\zeta$ to $\sigma>0$]
	We apply integration by parts on Dirichlet series when $\sigma>1$ \begin{align*}
		\zeta(s)&=\int_{1/2}^{\infty} x^{-s} d \lfloor x \rfloor \\
				&= s \int_{1/2}^{\infty} \lfloor x \rfloor x^{-s-1} d x\\
				&=  s \int_{1}^{\infty}  x^{-s} - \frac{\{x\}}{x^{-s-1}} d x\\
				&= \frac{s}{s-1} - s \int_{1}^{\infty} \frac{\{x\}}{x^{-s-1}} dx\\
	\end{align*}
	where in the last expression, the integral converges when $\sigma>0$, and the pole at $s=1$ arises from the first term.
\end{proof}


\begin{proof}[Proof of Proposition \ref{extension}]
	Using \begin{align*}
		\Gamma(s)=\int_{0}^{\infty} e^{-x}x^{s-1}dx,
	\end{align*}
	we make the substitution $x=\pi n^2y$ to get \begin{align*}
		\Gamma(s)&= \int_{0}^{\infty} e^{-\pi n^2 y} (\pi n^2 y)^{s-1} \pi n^2 dy\\
		\implies \frac{\Gamma(s)}{\pi^s n^{2s}} &= \int_{0}^{\infty} e^{-\pi n^2 y} y^{s-1} dy
	\end{align*}
	So that by the Monotone Convergence Theorem,\begin{align*}
		\pi^{-s/2}\Gamma(s/2)\zeta(s) &= \sum_{n=1}^{\infty}\frac{\Gamma(s/2)}{\pi^{s/2} n^s}\\
		&= \sum_{n=1}^{\infty} \int_{0}^{\infty} e^{-\pi n^2x} x^{s/2-1} dx\\
		&=  \int_{0}^{\infty}\sum_{n=1}^{\infty}\left(e^{-\pi n^2 x}\right)  x^{s/2-1} dx.
	\end{align*}
	We now let \[
		\omega(x) = \sum_{n=1}^{\infty} e^{-\pi n^2 x}, \quad \theta(x) = \sum_{n=-\infty}^{\infty} e^{-\pi n^2 x} = 2\omega(x)+1,
	\]
	and apply Poisson Summation to \begin{align*}
		\theta(x) &= \sum_{n=-\infty}^{\infty} e^{-\pi n^2 x}\\
		&= \sum_{k=-\infty}^{\infty} \int_{-\infty}^{\infty} e^{-\pi y^2 x} e^{-2\pi i k y} dy\\
		&= \sum_{k=-\infty}^{\infty} \int_{-\infty}^{\infty} e^{-\pi y^2 x} e^{-2\pi i k y} dy\\
		&= \sum_{k=-\infty}^{\infty} \frac{1}{\sqrt{x}} \int_{-\infty}^{\infty} e^{-\pi u^2} e^{-2\pi i k u /\sqrt{x}} du\\
		&= \sum_{k=-\infty}^{\infty} \frac{1}{\sqrt{x}} e^{-\pi k^2  / x}\\
		&= \frac{1}{\sqrt{x}} \theta \left(\frac{1}{x}\right)
	\end{align*}
	using the substitution $y\sqrt{x}=u$. Replacing with $\omega$, \[
	\sqrt{x}(2\omega(x)+1) = 2\omega\left(\frac{1}{x}\right)+1
	\implies \omega\left(\frac{1}{x}\right)=\sqrt{x}\omega(x) +\frac{\sqrt{x}}{2} - \frac{1}{2}.
	\]
	We thus write, using $y=1/x$, \begin{align*}
		\xi(s) &= \int_{0}^{1}\omega(x) x^{s/2-1} dx +
		\int_{1}^{\infty}\omega(x)  x^{s/2-1} dx\\
		&=\int_{1}^{\infty}\omega(1/y) y^{-s/2-1} dy +
		\int_{1}^{\infty}\omega(x)  x^{s/2-1} dx\\
		&=\int_{1}^{\infty}\left(
		\sqrt{y}\omega(y) +\frac{\sqrt{y}}{2} - \frac{1}{2}
		\right) y^{-s/2-1} dy +
		\int_{1}^{\infty}\omega(x)  x^{s/2-1} dx\\
		&= \int_{1}^{\infty}\left(
			\frac{\sqrt{y}}{2} - \frac{1}{2}
			\right) y^{-s/2-1} dy+ \int_{1}^{\infty}\omega(x)  \left(x^{s/2-1} + x^{-s/2-1/2}\right)dx\\
		&= \frac{1}{1-s} + \frac{1}{s}+  \int_{1}^{\infty}\omega(x)  \left(x^{s/2-1} + x^{-s/2-1/2}\right)dx\\
		&= \frac{1}{s(1-s)}+ \int_{1}^{\infty}\omega(x)  \left(x^{s/2-1} + x^{-s/2-1/2}\right)dx.
	\end{align*}
	$\omega$ decays exponentially in $x$, so the integral converges and the last expression is well defined on $\complex$ with simple poles at $s=1$ or $s=0$.
	Finally, notice that the last expression is symmetric when $s$ is replaced with $(1-s)$, so proves equation \ref{symmetryeq}.
\end{proof}
Finally, we extend to $\zeta(0)$ by noticing that the poles of the functional equation from $\zeta(1)$ and $\Gamma(0)$ cancel out, so the Riemann Extension Theorem can be applied.

From the functional equation, we get `trivial' zeros of the zeta function from the poles of $\Gamma$. 

\begin{corollary}
	On $\Re(s)>1$ or $\Re(s)<0$, $\zeta(s)\neq 0$, except $\forall n\in\naturals, \zeta(-2n)=0.$
\end{corollary}
\begin{proof}
	Using the product representation of $\zeta$ where it converges, none of $(1-p^{-s})^{-1} =0$, so $\zeta(s)\neq 0$ on $\Re(s)>1$.
	$\Gamma$ has no zeros and has a simple pole at $-n$ for all $n\in\naturals$, so by equation \ref{symmetryeq} we get the zeros for $\Re(s)>0$ are exactly at the negative even integers.
\end{proof}
These zeros are known as the trivial zeros of $\zeta$. The remaining zeros lie between $0\leq \Re(s) \leq 1$.
\begin{definition}[Critical Strip and Critical Line]
	We denote the region $0\leq \Re(s) \leq 1$ as the \textbf{critical strip}. We denote the line $\Re(s) = 1/2$ as the \textbf{critical line}.
\end{definition}

\begin{corollary}
	On the critical strip, if $\zeta(s)=0$, $\zeta(\overline{s})=\zeta(1-s)=\zeta(1-\overline{s})=0.$
\end{corollary}
\begin{proof}
	This follows from equation \ref{symmetryeq}, and $\zeta(\overline{s})=\overline{\zeta(s)}$ holds where the Dirichlet series converges, thus holds everywhere.
\end{proof}
The number of zeros in the critical strip can be calculated using the argument principle applied to the function $\xi$ over the box with corners $-1+iT$,$-1-iT$,$2-iT$,$2+iT$.
Applying the functional equation, we get the following result.

\begin{theorem}[Number of zeros of $\zeta$] \label{numberzero}
	The number of zeros up to height $T$\[
	\#\{\sigma+it \ | \ \zeta(\sigma+it)=0,0 \leq \sigma\leq 1, |t|\leq T \} = \frac{T}{2\pi}\log{\frac{T}{2\pi e}} + O(\log T).
	\]
\end{theorem}
Theorem \ref{numberzero} is obtained by Riemann in his famous 1859 paper on the Zeta function, where he analytically continued the Zeta function and conjectured that all the zeros lie on the critical line \cite{Riemann1859}.
\begin{conjecture}[Riemann Hypothesis] \label{RH}
	The \textbf{Riemann Hypothesis} (RH) asserts that on the critical strip, \[
	\zeta(s) = 0 \implies \Re(s) = \frac{1}{2}. 
	\]
\end{conjecture}

	\section{The Prime Number Theorem}

\begin{theorem}[Prime Number Theorem]
    Let $\Pi(N)=\sum_{p\leq N} 1$. Then \[
        \Pi(N) = (1+o(1))\frac{N}{\log N}.
    \]
\end{theorem}

In this section we will prove the Prime Number Theorem. This result is a minor goal of this paper.
The Prime Number theorem serves as a starting point for
studying primes in short intervals, and sets the stage for zero-density theorems.
\begin{definition}[Von Mangoldt Function]
    The \textbf{Von Mangoldt function} $\Lambda$ is defined as follows:
    \[
        \Lambda(n) = \begin{cases}
            \log p \textrm{, if $n = p^k$ for some $k\in\naturals$}\\
            0 \textrm{, else}
        \end{cases}
    \]
\end{definition}
The sum of the Von Mangoldt function $\sum\Lambda (n)$ is a more natural way to express
a prime counting function in the language of $\zeta$. To see why, consider the expression
\begin{align*}
    \frac{\zeta '(s)}{\zeta(s)} &= (\log\zeta(s))'\\
    &= \left[ -\sum_{p} \log\left(1-p^{-s}\right)\right]^{\prime}\\
    &= - \sum_{p} \frac{p^s \log p }{1-p^{-s}}\\
    &= -\sum_{p} \log p \sum_{k\in\naturals} p^{-ks}\\
    &= - \sum_{n\in\naturals} \frac{\Lambda(n)}{n^s}
\end{align*}
on $\Re(s)>1$ where the sum and products are absolutely convergent.
\begin{proposition}\label{mangoldtpnt}
    $\sum_{n\leq N} \Lambda(n) = (1+o(1))N$ implies the Prime Number Theorem.
\end{proposition}
\begin{proof}
    On one hand, we have \begin{align*}
        \sum_{n\leq N }\Lambda(n) &\leq \sum_{p \leq N }\Lambda(N)
        \\ &\leq \Pi(x){\log x}.
    \end{align*}
    And for $\epsilon>0$,
    \begin{align*}
        \sum_{n\leq N }\Lambda(n) &\geq \sum_{N^{1-\epsilon}\leq n\leq N }\Lambda(n)
        \\ &\geq \sum_{N^{1-\epsilon}\leq p \leq N }(1-\epsilon) \log(N)\\
        \\&= \Pi(N)\log(N) + O(N^{1-\epsilon}log N).
    \end{align*}
\end{proof}
Moreover, the sum of the Von Mangoldt function can be related to the zeros of the zeta function.
Let $\phi$ be smooth and rapidly decaying at infinity, and $\tilde{\phi}$ be its Mellin transform. Let $N\in\naturals$ and $c\geq 2$. Then \begin{equation} \label{mellinonzeta}
    \begin{split}
    \sum_{n\in\naturals} \Lambda(n) \phi\left(\frac{n}{N}\right) &=
    \sum_{n\in\naturals} \Lambda(n) \frac{1}{2\pi i}\int_{c-i\infty}^{c+i\infty}
    \tilde{\phi}(s)\left(\frac{n}{N}\right)^{-s} ds \\
    &= \frac{1}{2\pi i}\int_{c-i\infty}^{c+i\infty}
    \tilde{\phi}(s)\sum_{n\in\naturals} \Lambda(n) \left(\frac{n}{N}\right)^{-s} ds \\
    &= \frac{-1}{2\pi i}\int_{c-i\infty}^{c+i\infty}
    \tilde{\phi}(s) N^s \frac{\zeta'(s)}{\zeta(s)} ds \\
    \end{split}
\end{equation}
Morally, we can take a bump function $\phi=1$ on $[0,1]$ and supported in $[0-\epsilon, 1+\epsilon]$. 
\begin{equation}\label{preexplicit}
    \sum_{n\leq N} \Lambda(n)=
    \frac{-1}{2\pi i}\int_{c-i\infty}^{c+i\infty}
    \frac{1}{s} N^s \frac{\zeta'(s)}{\zeta(s)} ds
\end{equation}
By the rapid decay of $\tilde\phi$, we change the line of integration from $c$ to $-\infty$, we get residue contributions from
a pole at $s=1$, $s=0$, as well as all $\rho$ such that $\zeta(\rho)=0$ on the critical strip,
and all the trivial zeros. This gives \begin{equation}
    \begin{split}
        \sum_{n\leq N} \Lambda(n)&= N - \sum_{\rho} \frac{N^\rho}{\rho} -
        \frac{\zeta'(0)}{\zeta(0)} + \sum_{k\in\naturals}\frac{N^{-2k}}{2k} \\
        &= N - \sum_{\rho} \frac{N^\rho}{\rho} -
        \frac{\zeta'(0)}{\zeta(0)} + \frac{1}{2}\log\left(1-N^{-2}\right). 
    \end{split}
\end{equation}
The sum over zeros $\rho$ is not absolutely convergent, and is ordered in increasing $|\Im(\rho)|$.
This formula is in fact true.
\begin{theorem}[Riemann-von Mangoldt explicit formula]
    Let $N>1$ be not a prime power. Then\begin{equation}
    \sum_{n\leq N} \Lambda(n) = N - \lim_{T\to \infty}\sum_{|\Im{(\rho)}|\leq T} \frac{N^\rho}{\rho} -
    \frac{\zeta'(0)}{\zeta(0)} + \frac{1}{2}\log\left(1-N^{-2}\right). 
    \end{equation}
\end{theorem}
In practice, we truncate the integral in \label{preexplicit} up to height $T$, \textit{is the proof needed?} to obtained a truncated version of the explicit formula.
\begin{theorem}
    Let $N>1$. Then\begin{equation}
        \sum_{n\leq N} \Lambda(n) = N - \sum_{|\Im{(\rho)}|\leq T} \frac{N^\rho}{\rho} + O(\frac{N\log N \log T}{T} + \log T). 
        \end{equation}
\end{theorem}
The term $N$ in the explicit formula is already suggestive of the Prime Number Theorem. 
The major error term comes from $N^\rho$ in the sum, so bounding the $\Re(\rho)$ becomes the most important part in reducing the error term in the prime number theorem. 
This in turn is equivalent to bounding $\Re({\rho})$, and the best case is when all the zeros have real part $1/2$.
Assuming the Riemann Hypothesis, we consider the sum over the non trivial zeros \begin{align*}
\left|\sum_{|\Im(\rho)\leq T|}\frac{N^\rho}{\rho}\right| &\leq N^{1/2} \sum_{\rho}\left|{\frac{1}{\rho}}\right|.    
\end{align*}
We know there are $\sim \log T$ zeros of height $[T,T+1)$, thus the integral $\sum |\rho^{-1}|$ behaves as \[
\sum_{n\leq T} \frac{\log n}{n} = O(\log^2 T).
\]
Taking $N=T$ in the truncated explicit formula, we obtain \begin{equation} \label{RHPNT}
    \sum_{n\leq N} \Lambda(n) = N + O(N^{1/2}\log^2 N).
\end{equation}
Which implies the prime number theorem.
\begin{remark}
    The prime number theorem with the error term in \label{RHPNT} can be shown to be equivalent to the Riemann Hypothesis.
\end{remark}
The prime number theorem is also true without assuming the Riemann Hypothesis. 
It is sufficient to show that there are no zeros with real part $1$, so the terms in the sum contributes $O(N^{1-\epsilon})$
which will be dominated by $N$.
\begin{theorem}\label{nozerosatone}
    Let $t\in\reals$. Then $\zeta(1+it)\neq 0$.
\end{theorem}
Condition on proving Theorem \ref{nozerosatone}, we use it to derive the prime number theorem.
Let $\phi=\phi_{N,T}$ be a bump function that equals $1$ on the interval $[2,N]$ and supported on $[3/2, N+N/T]$.
By construction we can also make $\phi^{(j)}(x)=O_j(1)$ and $\phi^{(j)}(x)=O_j(T/x)^j$ on the intervals $[3/2, 2]$ and $[N,N+N/T]$ respectively.
Then \begin{align*}
    \sum_{n\leq N}\Lambda(n) &\leq \sum_{n} \Lambda(n)\phi(n)\\
    &= \frac{-1}{2\pi i}\int_{c-i\infty}^{c+i\infty}\tilde{\phi}\frac{\zeta'(s)}{\zeta(s)}ds\\
    &= \tilde{\phi}(1) - \sum_{\rho} \tilde{\phi}(\rho) - \sum_{n} \tilde{\phi}(-2n)\\
\end{align*}
The first term \begin{align*}
    \tilde{\phi}(1) &= \int_{0}^{\infty} \phi(x) dx \\
    &= N+ O(N/T) 
\end{align*}
gives the term we want from the PNT. In the third term, we rewrite by Monotone Convergence \begin{align*}
    \sum_{n} \tilde{\phi}(-2n) &= \sum_{n} \int_{0}^{\infty} \phi(x) x^{-2n-1} dx \\
    &=\int_{0}^{\infty} \phi(x) \sum_{n} x^{-2n-1} dx \\
    &= \int_{0}^{\infty} \phi(x) \frac{1}{x^3-x} dx \\
    &= O(1)
\end{align*} 
Finally, to bound the second term, we define a parameter $\bar{T} = \bar{T}(T)$ and split the sum into\[
\sum_{|\Im{\rho}|\leq \bar{T}}\tilde{\phi}(\rho) + \sum_{|\Im{\rho}| > \bar{T}}\tilde{\phi}(\rho)
\]
In the first summation, we let $\epsilon = \epsilon_{\bar{T}}$ 
such that there are no zeros in the region
$\Re({s})>1-\epsilon, |\Im{(s)}|\leq \bar{T}$, then \begin{align*}
    \sum_{|\Im{\rho}|\leq \bar{T}}\tilde{\phi}(\rho) &= \sum_{|\Im{\rho}|\leq \bar{T}} \int_{0}^{\infty}\phi(x)x^{\rho-1}dx\\
    &= O_{T}(N^{1-\epsilon}).
\end{align*}
In the second summation, we apply integration by parts to show that \begin{align*}
    |\int_{0}^{\infty}\phi(x)x^{\rho-1}dx| &= |\frac{1}{\rho(\rho+1)}\int_{0}^{\infty}\phi''(x)x^{\rho+1}dx|\\
    &= O(\frac{1}{|\rho|^2} \frac{T^2}{N^2} \frac{N}{T} N^2)\\
    &=O(\frac{1}{|\rho|^2} TN)
\end{align*}
The sum $\frac{1}{|\rho|^2}$ behaves as $\sum_n \log n n^{-2}$, so we can pick $\bar{T}$ large enough depending on $T$
to make the contribution of $\sum_{\Im(\rho)>\bar{T}} |\rho|^{-2}$ to be $O(T^{-2})$.
So that 
\begin{align*}
    \sum_{n\leq N}\Lambda(n)&\leq N +O(N/T)+ O_T(N^{1-\epsilon})\\
    = N+O(N/T)
\end{align*}
for $N=N(T)$ sufficiently large.
Similarly, repeating the same argument on $\phi=\phi_{N,T}$ equals $1$ on the interval $[2,N-N/T]$ and supported on $[3/2, N]$
gives \[
    \sum_{n\leq N}\Lambda(n)\geq N+O(N/T).
\]
Sending $T\to\infty$ gives the PNT.
\begin{proof}[Proof of Theorem \ref{nozerosatone}]
    
\end{proof}

	\chapter{Primes in Short Intervals and Zero Density Results}
	\section{Primes in Short Intervals}
We would like to answer the following question about primes in short intervals. Let $y=y(x)$. What is the smallest asymptotic behavior of $y$
such that \begin{equation}\label{shortintervalpnt}
\sum_{x\leq n \leq x+y} \Lambda(n) = (1+o(1)) y
\end{equation}
for large enough $x$? That is, what is the shortest interval such that we have the behavior of the Prime Number Theorem? If \ref{shortintervalpnt} holds for some $y$,
we say the Prime Number Theorem holds for intervals of $y$.
\begin{remark}
    This question can be rephrased into finding primes in short intervals, by including a factor of $\log x$. 
\end{remark} 
\begin{proposition}
    Assume the RH. Then the Prime Number Theorem holds in intervals of $x^{1/2+\epsilon}$.
\end{proposition}
\begin{proof}
    Assume the RH, then
    \begin{align*}
        \sum_{x\leq n \leq x+y} \Lambda(n)=
        y+O(x^{1/2}\log^2 x) = x^{1/2+\epsilon} + o(x^{1/2+\epsilon}),
    \end{align*}
    so that the sum is non-zero for large enough $x$.
\end{proof}
Recalling that the error term is related to the real part of the zeros of the Zeta function, we motivate the following definition of zero-density:
\begin{definition}
    Let $N(\sigma, T)$ denote the number of zeros of the zeta function with real part greater than $\sigma$ and imaginary part between $-T$ and $T$. That is,\[
        N(\sigma,T) = \# \{\rho = \beta + i\gamma \ | \ \beta \geq\sigma, |\gamma|\leq T\}.
    \]
\end{definition}
\begin{remark}
    The ideal scenario is that $N(\sigma,T)=0$ for all $\sigma> 1/2$. 
\end{remark}
\begin{theorem}[Chudakov] 
    There exists a constant $A$ such that $\zeta(\sigma+iT)\neq 0$ in the region \begin{equation*}
        \sigma > 1 - A\frac{\log \log T}{\log T}.
    \end{equation*}
\end{theorem}
\textit{add reference}
\begin{theorem}[Hoheisel] \label{Hoheisel}
    Let $A$ be defined as in the previous theorem.
    Suppose that $N(\sigma, T)\ll T^{a(1-\sigma)}\log^b T$ uniformly in $1/2\leq\sigma<1$ and in $T$. Then for all \[
        \theta > 1 - \frac{1}{a+b/A},
    \] the Prime Number Theorem holds in 
    intervals of $y=x^\theta$.
\end{theorem}
\begin{proof}
    First notice that $N(1/2,T)$ gets at least half of the zeros of height $T$, so $a\geq 2$.
    Let $y\ll x$. 
    We consider the expression \[
        S=S(x,y)=\frac{1}{y}\sum_{x\leq n \leq x+y} \Lambda(n).
    \]
    By the truncated version of the explicit formula in Theorem \ref{truncateexplcit}, we get
    \begin{align*}
        S &= 1 - \sum_{|\Im{(\rho)}|\leq T} \frac{(x+y)^\rho-x^\rho}{\rho y} + O(\frac{x}{yT}(\log xT) ^2) + O(\frac{\log x}{y}) . 
    \end{align*} 
    We want to show that except for the constant $1$ term, the remaining parts are $o(1)$.
    We focus on the sum over the non-trivial zeros with height less than $T$, and enumerate them $\rho_j$. For each $\rho_j=\sigma_j+it_j$, we apply the Mean Value Theorem on the function $f(x)=x^\rho_j$ to get
    \begin{align*}
        \left|\sum_{\rho_j} \frac{(x+y)^\rho-x^\rho}{\rho y}\right|&\leq \sum_{\rho_j}\left|\frac{(x+y)^{\rho_j}-x^{\rho_j}}{\rho_j y}\right|\\
        &\ll \sum_{\rho_j} x^{\sigma_j-1}\\
        &= \sum_{\rho_j} x^{\sigma_j-1} - x^{-1} + x^{-1}\\
        &= O\left(\frac{T\log T}{x}\right) + \sum_{\rho_j} x^{\sigma_j-1} - x^{-1}.
    \end{align*}
    And by replacing $x^{\sigma_j}-1$ by an integral,\begin{align*}
        \sum_{\rho_j} x^{\sigma_j-1} - x^{-1} &=\sum_{\rho_j} \int_0^{\sigma_j}  x^{u-1} \log x \ du \\ 
        &=  \int_0^{1-A\frac{\log \log T}{\log T}} \sum_{\rho_j} \mathbbm{1}_{u\leq \sigma_j}x^{u-1} \log x \ du\\
        &= \int_0^{1-A\frac{\log \log T}{\log T}} N(u,T) x^{u-1} \log x \ du
    \end{align*}
    Where in the penulitimate step we made use of Chudaokov's bound and exchanged the order of integration and summation.
    Now we can apply the hypothesis that $N(\sigma, T)\ll T^{a(1-\sigma)}\log^b T$ for $\sigma>1/2$ and trivially $N(\sigma, T)\ll T \log T \ll T^{a(1-\sigma)}\log^b T$ for $\sigma \leq 1/2$.
    This evaluates to 
    \begin{align*}
        \sum_{\rho_j} x^{\sigma_j-1} - x^{-1} & \ll \int_0^{1-A\frac{\log \log T}{\log T}} T^{a(1-u)} \ \log^b T \  x^{u-1} \log x \ du\\
        &= \log^b T \int_0^{1-A\frac{\log \log T}{\log T}} \left(\frac{T^{a}}{x}\right)^{1-u} \log x \ du\\
        &= \frac{\log x \log^{b} T}{a \log T - \log x } \left[\frac{T^a}{x}-\left(\frac{T^a}{x}\right)^{A\frac{\log \log T}{\log T}}\right]\\
    \end{align*}
    Combined with the previous bounds, we have \[
        S=1 + O\left(\frac{T\log T}{x}\right)+ O\left( \frac{\log x \log^{b} T}{a \log T - \log x } \left[\frac{T^a}{x}-\left(\frac{T^a}{x}\right)^{A\frac{\log \log T}{\log T}}\right]\right)+ O(\frac{x}{yT}(\log xT) ^2) + O(\frac{\log x}{y}) . 
    \]
    To make all terms (except for the first) to be $o(1)$, we want to set 
    $y=x^\theta$, $T=x^k$, such that $\theta,k$ satisfy \[
        k<1,\ ak<1, \ k+\theta>1, 
    \]
    so that the second, fourth and fifth terms are $o(1)$ in $x$.
    For the third term, we can simplify \begin{align*}
        \frac{\log x \log^{b} T}{a \log T - \log x } \left[\frac{T^a}{x}-\left(\frac{T^a}{x}\right)^{A\frac{\log \log T}{\log T}}\right]
        &= \frac{k^b \log^{b}x}{ak-1}  \left[x^{ak-1}-x^{(ak-1)A\frac{\log (k \log x)}{k \log x}}\right]
        \\ &\leq \frac{k^b \log^{b}x}{1-ak} x^{ak-1}  + \frac{k^b \log^{b}x}{1-ak} \exp\left((ak-1)A\frac{\log (k \log x)}{k} \right)\\
        &\leq \frac{k^b \log^{b}x}{1-ak} x^{ak-1}  + \frac{k^b \log^{b}x}{1-ak} \exp\left((ak-1)A\frac{\log (k \log x)}{k} \right)\\
        & =O(x^{ak-1})+O\left(\left(\log  x\right)^{b+\frac{(ak-1)A}{k}}\right).
    \end{align*}
    We require that the last term decays in $x$, and this happens when \[
        b+\frac{(ak-1)A}{k}< 0 \implies (aA+b)k<A \implies k < \frac{1}{a+\frac{b}{A}}
    \]
    We had $a\geq 2 >1$, so this $k$ satisfies $k<1$ and $ak<1$.
    Finally, for $k={1}/({a+bA^{-1}})- \delta/2$ we let $\theta = 1-k+\delta$ to satisfy $\theta+k>1$,
    so we can find any ${1}/({a+bA^{-1}})+\delta >\theta>1-{1}/({a+bA^{-1}})$, and \[
        \frac{1}{y}\sum_{x\leq n \leq x+y} \Lambda(n) = S = 1+o(1)
    \]for $y=x^\theta$. This completes the proof.

\end{proof}
Theorem \ref{Hoheisel} gives the classical way to relate the distribution of primes in short intervals to the density of zeros away from the real-half line. The long-standing bound for zero density is due to separate proofs of Ingham and Huxley:
\begin{theorem}[Ingham bound for zero density]
    Let $1/2\leq \sigma\leq 3/4$. We have \[
        N(\sigma,t)\lesssim T^{\frac{3(1-\sigma)}{2-\sigma}}.
        \]
\end{theorem}
\begin{theorem}[Huxley bound for zero density]
    Let $3/4\leq \sigma\leq 1$. We have \[
        N(\sigma,t)\lesssim T^{\frac{3(1-\sigma)}{3\sigma-1}}.
        \]
\end{theorem}

Combining these two bounds, we get the following zero density theorem.
\begin{theorem}[Ingham-Huxley bound for zero density]
   We have \[
    N(\sigma,t)\lesssim T^{\frac{12}{5}(1-\sigma)},
    \]
    uniformly for $1/2\leq \sigma\leq 1$
\end{theorem}
Notice that $12/5$ comes from $\sigma = 3/4$. In June 2024, Guth and Maynard published a proof that improves the Ingham-Huxley bound at $\sigma \in [7/10,8/10]$, thus improving the result of primes in short intervals (as well as many other number theoretic results). The following sections will be dedicated to Huxley's proof of zero density, as well as Guth-Maynard's ideas in the proof. Finally, adapting from Guth and Maynard, we will provide a proof of the analogous zero-density result for $L$-functions.
\begin{theorem}[Guth-Maynard bound for zero density]
    We have \[
     N(\sigma,t)\lesssim T^{\frac{30}{13}(1-\sigma)},
     \]
     uniformly for $1/2\leq \sigma\leq 1$
 \end{theorem}
 \begin{figure}[h]
    \centering
    \begin{subfigure}{0.4\textwidth}
        \includegraphics[width=\textwidth]{inghamhuxley1.png}
        \caption{The bounds for the exponent coincide at $\sigma=3/4$}
    \end{subfigure}
    \begin{subfigure}{0.4\textwidth}
        \includegraphics[width=\textwidth]{inghamhuxley2.png}
        \caption{$\sigma=3/4$ is also the bottleneck when written in Hoheisel's form.}
    \end{subfigure}

    \centering
    \begin{subfigure}{0.4\textwidth}
        \includegraphics[width=\textwidth]{gm_1.png}
        \caption{Guth-Maynard's result improves in the range at $\sigma\in[7/10,8/10].$}
    \end{subfigure}
    \begin{subfigure}{0.4\textwidth}
        \includegraphics[width=\textwidth]{gm_2.png}
        \caption{The exponent is reduced around the bottleneck region.}
    \end{subfigure}
\end{figure}
	\section{Huxley's Proof of Zero Density}
\begin{theorem}[Huxley]
    We have  \[
    N(\sigma,t)\lesssim T^{\frac{12}{5}(1-\sigma)}.
    \]
\end{theorem}

Huxley's methodology for detecting zeros as follows. Let $M_x(s) \defeq \sum_{n=1}^{x} \mu(n)n^{-s}$. Since this also converges absolutely on $\Re(s)>1$, we can write the dirichlet series of $\zeta(s) M_x (s)$ as \[
    \zeta(s) M_x (s) \defeq \sum_{n}a_n n^{-s}
\]
for some choice of $a_n=a_n(x)$. The zeros of its analytic continuation will contain the zeros of $\zeta$. This may look inefficient as we may have introduced extra zeros from $M_x$, but the tradeoff is that we can bound these $a_n$'s.
\begin{proposition}
    We have  \[\begin{cases}
        a_1 = 1, & \\
        a_n = 0, & \textrm{if $1<n\leq x$},\\
        |a_n|\leq d(n), & \textrm{if $n>x$}.
    \end{cases}\]
\end{proposition}
\begin{proof}
    For all $n\leq x$, this follows from M\"obius inversion. For $n>x$, we just apply the trivial bound $|\mu(d)|\leq 1$ on \[
    a_n  = \sum_{d|n} \mu(d) .
    \]
\end{proof}

Let $y>x$ a parameter to be choosen later, and $y\leq T^A$ for an absolute constant $A$. We apply the Mellin transform to \begin{equation*}
    \begin{split}
    \sum_{n}a_n n^{-s} e^{-n/y}=&\frac{1}{2\pi i}\sum_{n}a_n n^{-s} \int_{2-i\infty}^{2+i\infty}\Gamma(w) y^w n^{-w} dw\\
    =&\frac{1}{2\pi i}\int_{2-i\infty}^{2+i\infty}\zeta(s+w)M_x(s+w)\Gamma(w) y^w dw.
    \end{split}
\end{equation*}
If we move the line of integration to $\Re(w) = 1/2 - \Re(s)$, we get simple pole residue contributions from $\zeta$ and $\Gamma$ \begin{equation}
    \label{huxleyperron}
    \begin{split}
    e^{-1/y}+\sum_{n>x}a_n n^{-s} e^{-n/y}=\sum_{n}a_n n^{-s} e^{-n/y}=&\zeta(s)M_x(s) +M_x(1)\Gamma(1-s)y^{1-s}\\+&
    \frac{1}{2\pi}\int_{-\infty}^{\infty}\zeta(\frac{1}{2}+i\Im(s)+iw)M_x(\frac{1}{2}+i\Im(s)+iw)\\ &\cdot \Gamma\left(\frac{1}{2}-\Re(s)+it\right) y^{\frac{1}{2}-\Re(s)+it} dt.
    \end{split}
\end{equation}
We take $y$ large enough so that $e^{-1/y}$ is close to $1$. Since $M_x(s)$ is an approximation of $1/\zeta$, we should expect that the term $\zeta(s)M_x(s)$ is about $1$ most of the time and the other terms are small. However, if $s$ is a zero of $\zeta$, then $\zeta(s)M_x(s)=0$, so at least one of the following things need to happen \begin{enumerate}[label=(\roman{*})]
    \item $|\sum_{n>x}a_n n^{-s} e^{-n/y}|$ is large.
    \item The integral in $t$ is large.
    \item $|M_x(1)\Gamma(1-s)y^{1-s}|$ is large.
\end{enumerate}
We thus transform the problem of detecting zeros to counting the number of occurences of extreme values. We will later see that type (iii) zeros are negligible, so we need to bound the number of type (i) and type (ii) zeros. 

\begin{lemma}
    Let $a$ be an arithmetic function, and $D_N(s)=\sum_{n\leq N}a(n)n^{s}.$
    If $W = \{t_j\}\subseteq [0,T]$ is a one-separated set such that \[
        |D_N(it_j)| > V \ \forall j,
    \]
    then \[
        |W|\ll \frac{\log^2 T}{V^\alpha} \int_{-(\log N)^{-1}}^{(\log N)^{-1}}
        \int_{0}^{T} |D_N(x+it)|^\alpha dt \ dx
        \]
    for $\alpha > 0$.
\end{lemma}
\begin{proof}
    With a cost of $O(1)$ we can consider $W\subseteq [(\log N)^{-1}, T-(\log N)^{-1}]$.
    Since $D_N$ is analytic, $|D_N|^\alpha$ is subharmonic. Let $B(t_j)$ describe a square-box of side length $(\log N)^-1$ centered at $it_j$ in the complex plane, then \begin{align*}
        V^\alpha \sum_{j} 1 \leq \sum_{j}|D(it_j)|^\alpha \leq \log^2 N \sum_{j} \int_{B(t_j)} |D(s)|^\alpha dA \leq \log^2 T \int_{-(\log N)^{-1}}^{(\log N)^{-1}}
        \int_{0}^{T} |D_N(x+it)|^\alpha dt \ dx.
    \end{align*}
\end{proof}

\begin{corollary}
   Let $W = \{t_j\}\subseteq [0,T]$ be a one-separated set such that \[
        \left|\zeta\left(\frac{1}{2}+it_j\right)\right|>V \ \forall j,
    \]
    then \[
    |W|\ll TV^{-4} \log^{O(1)}T.
    \]
\end{corollary}
\begin{proof}
    We have \[
        \left|\zeta\left(\frac{1}{2}+it\right)\right|\ll \sum_{n\leq \sqrt{T}}n^{-1/2-it},
    \]
    so applying the previous lemma on $D(s)=\sum_{n\leq \sqrt{T}}n^{-s}$ with $\alpha=4$ gives \begin{align*}
        |W|\ll \frac{\log ^2 T}{V^4} \int_{-2(\log T)^{-1}}^{2(\log T)^{-1}}
        \int_{0}^{T} |D_N(x+it)|^4 dt \ dx
    \end{align*}
\end{proof}

\begin{lemma}[Hal\'asz Inequality]
    \label{halasz}
    Let $a$ be an arithmetic function, and $D_N(s)=\sum_{n\leq N}a(n)n^{s}$, and $G=\sum_{n\leq N} |a(n)|^2.$
    If $W = \{t_j\}\subseteq [0,T]$ is a one-separated set such that \[
        |D_N(it_j)| > V \ \forall j,
    \]
    then 
    \[
        |W|\lesssim GNV^{-2} + G^3NTV^{-6}.
        \]
\end{lemma}
\begin{proof}
    Let $-\theta_j$ be the argument of $D(it_j)$, then we have \begin{align*}
        V|W|\leq \sum_j |D(it_j)| = \sum_j e^{\theta_j}D(it_j ) = \sum_{n\leq N} \sum_j e^{i\theta_j} a(n) n^{-it_j}.
    \end{align*}
    By Cauchy-Schwarz, this summation is \[
    \leq \left(\sum_{n\leq N} |a(n)|^2 \right)^{1/2}\left(\sum_{n\leq N} \bigg|\sum_j e^{i\theta_j} n^{-it_j}\bigg|^2 \right)^{1/2}.
    \]
    The first summation in this expression is $G$, so we want to bound the latter nested summation.
    Expanding the summation gives \begin{align*}
        \sum_{n\leq N} \bigg|\sum_j e^{i\theta_j} n^{-it_j}\bigg|^2 =& 
        \sum_{n\leq N} \sum_{j_1,j_2} e^{i\theta_{j_1}-i\theta_{j_2}} n^{it_{j_1}-it_{j_2}}\\
        \leq & |W|N + \sum_{j_1,j_2} \bigg|\sum_{n\leq N}  n^{it_{j_1}-it_{j_2}}\bigg|.
    \end{align*}
    \textcolor{red}{TODO}
\end{proof}

\begin{proof}[Proof of Huxley's Zero Density Theorem]
    From equation \ref{huxleyperron}, we take $y>6$ so that $e^{-1/y}>5/6$. We also truncate the sum in $n>x$ to $x<n\leq y^2$ with an error of $1/6$ for large enough $y$. Finally, we truncate the integral in $t$ to the range $|t|\leq B\log T$ with an error of $1/6$. Thus, $s$ is a zero only if \begin{enumerate}[label=(\roman{*})]
        \item $|\sum_{x<n\leq y^2}a_n n^{-s} e^{-n/y}|\geq \frac{1}{6}$, or
        \item $\frac{1}{2\pi}|\int_{-B\log T}^{B\log T}\zeta(\frac{1}{2}+i\Im(s)+iw)M_x(\frac{1}{2}+i\Im(s)+iw) \Gamma\left(\frac{1}{2}-\Re(s)+it\right) y^{\frac{1}{2}-\Re(s)+it} dt|\geq \frac{1}{6}$, or
        \item $|M_x(1)\Gamma(1-s)y^{1-s}|\geq \frac{1}{6}$ .
    \end{enumerate}
    Of the zeros $\rho = \beta+i\gamma$ of $\zeta$ in the region, at the cost of a factor of $\log T$, we take representatives such that if $\rho_1\neq \rho_2$ then $|\rho_1-\rho_2|\geq \textcolor{red}{1}$. 
    For Class (i) zeros, we split the sum dyadically to get \begin{equation}
    \label{class1dyadic}
    \bigg| \sum_{n\sim U, n\leq y^2} a(n)  n^{-\rho}e^{n/y}\bigg| \geq O((\log T)^{-1}),
    \end{equation}
    for some $x\leq U = 2^k\leq y$. Applying Lemma \ref{halasz}, we get that the number of times that equation \ref{class1dyadic} can happen for each $U$ is \[
    \lesssim U^{2-2\sigma} + U^{4-6\sigma}T\lesssim 
    \]
    (Note that the $\log$ factors are dominated by any choice of $T^\epsilon$)
\end{proof}

	\chapter{Guth-Maynard's Proof of the Zero Density near $\sigma = 3/4$}
	%\section{Guth-Maynard's proof of Large Values of Dirichlet Polynomials}
In May 2024, Guth and Maynard published an improvement of the large values of Dirichlet polynomails estimate at $\sigma\in[7/10,8/10]$.
\begin{theorem}[Guth-Maynard Large Values Estimate]
    Let $(b_n)$ be a sequence of complex numbers such that $|b_n|\leq 1$ for all $n$, and $W=\{t_j\}_{j=1}^{|W|}$ be a $1$-separated set $\subseteq [0,T]$, such that \[
    \left|\sum_{n\sim N}b_n n^{it_j}\right|\geq V
    \]
    for each $t_j\in W$. Then \[
    |W|\lesssim N^2V^{-2}+N^{18/5}V^{-4}+TN^{12/5}V^{-4}.
    \]
\end{theorem}
Let us compare this bound to Lemma \ref{halasz}, which states \[
|W|\lesssim N^2V^{-2}+TN^4V^{-6}.
\]
In the critical case $V=N^{3/4}, N\leq T^{5/6-\epsilon}$, the original bound will give \[
|W|\lesssim N^2N^{-3/2}+TN^4N^{-9/2}\lesssim N^{1/2}+ TN^{-1/2}\lesssim TN^{-1/2},
\]
while the bound by Guth and Maynard gives \[
|W|\lesssim N^2N^{-3/2}+N^{18/5}N^{-3}+TN^{12/5}N^{-3}\lesssim N^{1/2}+TN^{-3/5}\lesssim TN^{-3/5}.
\]
This new theorem, when applied in Huxley's proof in the previous chapter, gives an improvement in the bound of zero density in the range $\sigma\in[7/10,8/10]$.
\section{Outline and Sketch of proof}
%\node[label=above/below/etc:{label}] (x) {}
The structure of the proof can be broken down as follows: We first notice that $|W|$ is bounded by the operator norm of a matrix $M$. This operator norm, using results from linear algebra, is bounded by the trace. Applying Poisson summation on the trace gives $4$ terms that are separately handled, which we will name $S_0$ to $S_3$. We will see that $S_0$ gives the `main term' that is consistent with the density hypothesis, $S_1$ is negligible, $S_2$ is bounded by a theorem by Heath-Brown, which we state below. 
\begin{theorem}[Heath-Brown]
    \label{heathbrown}
    Let $\mathcal{S}=\{(t_j,\chi_j)\}$ be one-separate, primitive characters of modulus $q$. Then 
    \[
        \sum_{\substack{(t_1,\chi_1)\\(t_2,\chi_2)}}\left|\sum_{n=1}^{N} b_n n^{-i(t_1-t_2)}\chi_1\bar{\chi}_2(n)\right|^2 \lesssim  |\mathcal{S}|N^2+ |\mathcal{S}|^2N + |\mathcal{S}|^{5/4}(qT)^{1/2}N.
    \]
\end{theorem}
The most tricky term, $S_3$ is a summation over a three-dimensional lattice. We will see that $S_3$ is bounded by what is known as the \textit{additive energy} of the set $W$, defined by\[
    E(W)\defeq \#\{t_1,t_2,t_3,t_4\in W : |t_1+t_2-t_3-t_4|\ll T^\epsilon\}.
\]
This term describes the `additive structure' of $W$.
We see that $E(W)$ is bounded below by $|W|^2$, as the condition is satisfied when $t_1=t_3$ and $t_2=t_4$. Moreover, since $W$ is $1-separated$, the choice of $t_1,t_2,t_3$ fixes $O(1)$ choices for $t_4$, so $E(W)$ is bounded above by $|W|^3$. In the extreme case that the additive structure of $W$ is high, such as when $t_j=j\alpha$ for a constant $\alpha$, the energy of the set is $O(|W|^3)$. This definition naturally arrises from taking the fourth moment of the function \[
R(v)\defeq\sum_{t\in W} v^{it}.
\] 
This gives us \[
R(v)^4=\sum_{t_1,t_2,t_3,t_4\in W}v^{i(t_1+t_2-t_3-t_4)}.
\]

The naive choice $E(W)\leq W^3$ is slightly too loose to beat the Ingham-Huxley bound. However, an orthogonal bound can be found for $E(W)$ based on Heath-Brown's theorem. Finally, the bound in $7$ and $8$ combined is enough to give an improvement in most cases, a further refinement of the $S_3$ bound was required. This relies on the averaging over the affine summations of $R$.
 \[
        \sup_{0<M_1,M_2,M_3<M} \int\Bigg( \sum_{\substack{|m_1|\sim M_1\\|m_2|\sim M_2 \\ |m_3|\ll M_3}} \left|R\left(\frac{m_1 u+m_3}{m_2}\right)\right|\Bigg)^2 \ du \lesssim M^6 \|R\|_{L_2}^4+M^4\|f\|_{L_4}^4.
\] 

We give a quick sketch of the whole proof below. In the next section, we will give a full proof of the generalized statement of the theorem that considers primitive Dirichlet characters mod $q$. The proof of Guth-Maynard can be recovered by using the special case $q=1$.

\begin{figure} [t]
    \begin{tikzpicture}
        \begin{scope}[every node/.style={circle,thick,draw}]
            \node [label={[align=right]left:{$|W|$ is bounded by the operator\\ norm of a matrix $M$}}](A) at (0,0) {1};
            \node [label={[align=right]left:{$\|M\|^2$ is bounded by \\$\textrm{Tr}((M^*M)^k)^{1/k}$}}](B) at (0,-2) {2};
            \node [label={[align=right]left:{Poisson summation on $\textrm{Tr}((M^*M)^3)$,\\split in $4$ pieces}}](C) at (0,-4) {3};
            \node [label={[align=right] left:{$S_0$ gives main \\ term $N^2V^{-2}$}}](D) at (-6,-7) {4};
            \node [label={[align=right] left:{$S_1$ is negligible}}](E) at (-2,-7) {5};
            \node [label={[align=right] left:{$S_2$ reduces to\\ Heath-Brown}}](F) at (2,-7) {6};
            \node [label={[align=right] left:{$S_3$ is bounded\\ by Energy}}](G) at (6,-7) {7};
            \node [label={[align=right]left:{Energy bounded by \\ Heath-Brown}}](H) at (6,-10) {8};
            \node [label={[align=right]left:{Refinement of \\ $S_3$ bound}}](I) at (4,-8.5) {9};
        \end{scope}
        %\draw [dotted] (a) --  (b);
        \draw [->](A) -- (B);\draw [->](B) -- (C);
        \path [->](C) edge [bend left =13](D); \draw [->](C) -- (E); \draw [->](C) -- (F); \path [->](C) edge[bend left =20] (G);
        \draw[->] (G)--(H);\draw[->,red,dotted,thick] (G)--(I);\draw[dotted, red,->,thick] (I)--(H);
        %\draw (a) -- node[midway, below left]{1}(d) -- node[midway, below right]{2}(b);
        %\draw (a)-- node[midway, above left]{2} (c) --node[midway, above right]{1}(b);
\end{tikzpicture}
\caption{Graphical representation of Guth-Maynard proof outline}
\end{figure}
\subsubsection*{0. Setup}
First, as in the theorem, we let $(b_n)$ be a sequence of complex numbers such that $|b_n|\leq 1$ for all $n$, \[
D_n(t)\defeq \sum_{n\sim N}b_n n^{it},
\] $W=\{t_j\}_{j=1}^{|W|}$ be a $T^\epsilon$-separated set $\subseteq [0,T]$, such that \[
\left|D_n(t_j)\right|\geq V
\]
for each $t_j\in W$. Notice that we now let the set be $T^\epsilon$ separated for $\epsilon>0$. This means that we will give up a factor of $T^\epsilon$ in the final bound, but this makes many computations cleaner as this $T^\epsilon$ dominates the log factors. Moreover, we can introduce a bump function $\omega$ with support in $[1,2]$ to localize the summation, and rewrite \[
    D_n(t_j)=\sum_{n}\omega\left(\frac{n}{N}\right)b_n n^{it_j}.
\]
This is added for the Poisson summation in step $3$.
\subsubsection*{1. Bounding $|W|$ with operator norm}
We view $\vec{b}=(b_n)_{n\sim N}$ as a $N$-dimensional vector, and consider the $|W|\times N$ matrix, indexed by $j$ from $1$ to $|W|$ and $n\sim N$,\[
    M_{j,n}=n^{it_j}= \omega\left(\frac{n}{N}\right)n^{i t_j}.
\]
Then we can view the $j$-th entry of the product $M\vec{b}$ as $D_n(t_j)$. In other words \[
|M\vec{b}|^2\geq V^2{|W|}.
\]
However, we can bound $|M\vec{b}|$ using the operator norm of $M$ and $|b_n|\leq 1$ to get\[
    |M\vec{b}|^2\leq \|M\|^2|\vec{b}|^2 \leq \|M\|^2 N.
\]
Combined with the previous inequality, we get \begin{equation}
    \label{basicineq}
|W|\leq \|M\|^2 NV^{-2}.
\end{equation}

\subsubsection*{2. Bounding $\|M\|$}
An immediate way to proceed is to note that $\|M\|^2$ is the largest eigenvalue of $MM^*$, which in turn is bounded by sum of eigenvalues which is the trace of $M^*M$. However, this is somewhat inefficient. Consider $N$-dimensional vector that enumerates through the eigenvalues $(\lambda_n)$ of $MM^*$, so that the trace will be the $L_1$ norm of this vector. In principle, we would like the $L_{\infty}$ norm of this vector, so we can try to take $L_k$ norms of this vector for big $k$ to get close to $L_\infty$. Using an eigenbasis for $MM^*$, we can see that the $L_k$ norm is represented by \[
\left(\sum_{n\sim N}\lambda_n ^ {k}\right)^{1/k}= \textrm{Tr}((MM^*)^k)^{1/k}.
\]
We take $k=3$, which is the highest power we can afford given the tools at our disposal. This gives

\begin{equation}
    \label{traceineq}
    |W|\leq \textrm{Tr}((MM^*)^3)^{1/3} NV^{-2}.
\end{equation}

\subsubsection*{3. Expansion of $\rm{Tr}((MM^*)^3)$}
We first compute 

\begin{align*}
    (MM^*)_{n_1,n_2} = \sum_{t\in W} \omega\left(\frac{n_1}{N}\right)\omega\left(\frac{n_2}{N}\right)
    n_1^{-it_j}n_2^{it_j}
\end{align*}
so that\begin{align*}
    \textrm{tr}((M^*M)^3)=& \sum_{t_1,t_2,t_3\in W}\sum_{n_1,n_2,n_3\sim N} 
    \omega\left(\frac{n_1}{N}\right)^2\omega\left(\frac{n_2}{N}\right)^2\omega\left(\frac{n_3}{N}\right)^2 n_1^{i(t_1-t_3)}
   n_2^{i(t_2-t_1)}
    n_3^{i(t_3-t_2)}\\=& \sum_{t_1,t_2,t_3\in W}\sum_{n_1,n_2,n_3} 
    \omega\left(\frac{n_1}{N}\right)^2\omega\left(\frac{n_2}{N}\right)^2\omega\left(\frac{n_3}{N}\right)^2 \left(\frac{n_1}{N}\right)^{i(t_1-t_3)}
    \left(\frac{n_2}{N}\right)^{i(t_2-t_1)}
    \left(\frac{n_3}{N}\right)^{i(t_3-t_2)}.
\end{align*}
Let $h_t(u)\defeq \omega(u)^2 u^{it}$, we can apply Poisson summation in the inner integral over $n_1,n_2,n_3$ to get 
\begin{equation}\label{poissongm}
    \rm{tr}((M^*M)^3)= N^3\sum_{t_1,t_2,t_3\in W}\sum_{m_1,m_2,m_3}  \hat{h}_{t_1-t_3}(Nm_1)\hat{h}_{t_2-t_1}(Nm_2)\hat{h}_{t_3-t_2}(Nm_3).
\end{equation}
What we can gain here is that $\hat{h}_t{m}$ has decay in $t$ or $m$ based on the principle of non-stationary phase. 
\begin{lemma}[Non-stationary phase]
    \label{nonstationary}
    We have for any integer $A>0$\begin{align*}
        |\hat{h}_t(\xi)|\ll_A \frac{1+|t|^A}{|\xi|^A},\\
        |\hat{h}_t(\xi)|\ll_A \frac{1+|\xi|^A}{|t|^A}. 
    \end{align*}
\end{lemma}
\begin{proof}
   We have \[
   \hat{h}_t(\xi)=\int \omega(u)^2u^{it}e^{2\pi i \xi u} du.\]
   By repeated integration by parts on $\omega(u)^2u^{it}$ and $e^{2\pi i \xi u}$, we get \[
    |\hat{h}_t(\xi)|=\left|\int (2\pi i \xi)^{-A} e^{2\pi i \xi u}  \frac{d^A}{(du)^A}\left(\omega^2(u)u^{it}\right) du\right| \ll_A \frac{1+|t|^A}{|\xi|^A}.
   \]
   A similar argument for integration by parts on $\omega(u)^2e^{2\pi i \xi u}$ and $u^{it}$ gives \[
    |\hat{h}_t(\xi)|=\left|\int \frac{1}{(it+1)(it+2)\ldots(it+A)}u^{it+A}  \frac{d^A}{(du)^A}\left(\omega^2(u)e^{2\pi i \xi u}\right) du \right|\ll_A \frac{1+|\xi|^A}{|t|^A}.
   \]
\end{proof}
This means that we can handle terms in equation \ref{poissongm} if $m_i$ is small and $t_j-t_k$ is big, or  $m_i$ is big and $t_j-t_k$ is small. With this in mind, we split the sum over $m_1,m_2,m_3$ in the equation into four parts. $S_0$, where all three $m$ terms are zero, $S_1$, where exactly one of the $m$ terms is non-zero, $S_2$, where exactly two of the $m$ terms are non-zero, and $S_3$, where all three $m$ terms are non-zero.
That is,

\[\rm{tr}((M^*M)^3)= S_0+S_1+S_2+S_3, \]
where \[
S_j = N^3\sum_{m_1,m_2,m_3, \#\{m_k=0\}=j} I_m,
\]
\[
I_m=I_{(m_1,m_2,m_3)}\defeq N^3\sum_{t_1,t_2,t_3\in W}\hat{h}_{t_1-t_3}(Nm_1)\hat{h}_{t_2-t_1}(Nm_2)\hat{h}_{t_3-t_2}(Nm_3).
\]

\subsubsection*{4. Bounding $S_0$}
$S_0$ only has one term in the sum. \[
S_0 = N^3\sum_{t_1,t_2,t_3\in W} \hat{h}_{t_1-t_3}(0)\hat{h}_{t_2-t_1}(0)\hat{h}_{t_3-t_2}(0)
\]
Now we can apply that $W$ is $T^\epsilon$ separated, so there is a trivial bound $|W|\leq T$ and $\hat{h}_{t_j-t_k}$ is negligible by the principle of non-stationary phase. So we can only consider \[
S_0 = N^3\sum_{t\in W}\hat{h}_0(0) + O(T^{-100}) = N^3|W|\|\omega\|_{L_2}^6.
\]Taking the cube root, this term gives $O(N^2V^{-2}|W|)$ in equation \ref{traceineq}. This is strikingly similar to the $N^2V^{-2}$ term that the density hypothesis conjectures. Guth and Maynard isolates this term by introducing the following lemma.
\begin{lemma}\label{gmtrace}
    Let $A$ be an $m\times n$ matrix. Then 
    \[\|A\| \leq 2\left(\rm{tr}((AA^*)^3)-\frac{\rm{tr}(AA^*)^3}{m^2}\right)^{1/6}+2\left(\frac{\rm{tr}(AA^*)}{m}\right)^{1/2}.
    \]
\end{lemma}
\begin{proof}
    This is Lemma 4.2 from Guth-Maynard. \cite{}
\end{proof}
Applying this lemma, we can compute that \[
    \rm{tr}(MM^*) =\sum_{n\sim N} \sum_{t\in W} \omega\left(\frac{n}{N}\right)^2 n^{-it} n^{it}= |W|\sum_n \omega\left(\frac{n}{N}\right)^2.
\]
Applying Poisson summation, this equals \[
    |W|\sum_m N \hat{h}_0(mN).
\]
By non-stationary phase use the rapid decay of $\hat{h}_0(\xi)$ in $\xi$ to only consider the term $m=0$ at the cost of $N^{-100}$.
Therefore, $\rm{tr}(MM^*)=|W|N\|\omega\|_{L_2}^2+O(N^{-100})$.
Lemma \ref{gmtrace} gives \[
|W|\ll NV^{-2}(N+(S_0+S_1+S_2+S_3-N^3|\omega|_{L_2}^6|W|)^{1/3}) \ll N^2V^{-2}+  NV^{-2}(S_1+S_2+S_3)^{1/3}. 
\]


\subsubsection*{5. Bounding $S_1$}
By symmetry in $m_1,m_2,m_3$, we can consider the terms where $m_3\neq 0$ at a cost of a factor of $3$.
Then \[
S_1 =3 N^3\sum_{m\neq 0} \sum_{t_1,t_2,t_3\in W}\hat{h}_{t_1-t_3}(0)\hat{h}_{t_2-t_1}(0)\hat{h}_{t_3-t_2}(mN).
\] 
This term is bounded by non-stationary phase. If $|m|>T^{1+\epsilon}/N$, then $|m|/|t_3-t_2|< T^\epsilon$, so we can truncate the sum to $|m|\leq T^{1+\epsilon}/N$ with an error of $O_\epsilon(T^{-100})$. In this range, if $t_1\neq t_3$ or $t_2\neq t_1$, then they are $T^\epsilon$ apart, then we get rapid decay in $\hat{h}_{t_1-t_3}(0)$ or $\hat{h}_{t_2-t_1}(0)$ to be $O_\epsilon{T^{-100}}$. But when $t_1=t_2=t_3$, we get decay in the last term $\hat{h}_0(mN)$. Combining all cases, this term is negligible.

\subsubsection*{6. Bounding $S_2$}

By symmetry again we can consider the terms where $m_1,m_2\neq 0,m_3=0$.Then \begin{align*}
    S_2=&3 N^3\sum_{m_1,m_2\neq 0} \sum_{t_1,t_2,t_3\in W}\hat{h}_{t_1-t_3}(m_1N)\hat{h}_{t_2-t_1}(m_2N)\hat{h}_{t_3-t_2}(0).
\end{align*}
Due to decay of the last term in $|t_3-t_2|$, we can only consider the terms $t_3=t_2$ with error $O_{\epsilon,A}(T^{-A})$. Then we can rewrite
\begin{align*}
    3 N^3\hat{h}_{0}(0)\sum_{m_1,m_2\neq 0} \sum_{t_1,t_2\in W}\hat{h}_{t_1-t_2}(m_1N)\hat{h}_{t_2-t_1}(m_2N) &= 3 N^3\hat{h}_{0}(0)\sum_{m_1,m_2\neq 0} \sum_{t_1,t_2\in W}\hat{h}_{t_1-t_2}(m_1N)\hat{h}_{t_1-t_2}(-m_2N)\\
    &= 3 N^3\hat{h}_{0}(0)\sum_{t_1,t_2\in W}\left(\sum_{m\neq 0}\hat{h}_{t_1-t_2}(mN)\right)^2.
\end{align*}
Poisson summation gives \[
N \sum_m \hat{h}_{t_1-t_2}(mN) = \sum_n h_{t_1-t_2}\Big(\frac{n}{N}\Big)=\sum_{n} \omega\left(\frac{n}{N}\right) n^{i(t_1-t_2)}.
\]
Therefore, a direct application of Heath Brown's theorem \ref{heathbrown} gives a bound for $S_2$.
    Here we have added in the terms for $m=0$, which is insignificant when $t_1$ and $t_2$ are $T^\epsilon$ separated. However, this is somewhat lossy for terms $t_1=t_2$. However, we can ignore the terms with $t_1=t_2$ at a small error due to the rapid decay in $\hat{h}_{t_1-t_2}(mN)$, apply Poisson summation, then add the terms $t_1=t_2$ again.

\subsubsection*{7. Bounding $S_3$}
$S_3$ sums over most points on the $3$-dimensional lattice. By symmetry, we can consider only the terms with $|m_1|\leq |m_2|\leq |m_3|$ with an error factor of $6$.
Recall that \[
I_m=N^3\sum_{t_1,t_2,t_3\in W}\hat{h}_{t_1-t_3}(Nm_1)\hat{h}_{t_2-t_1}(Nm_2)\hat{h}_{t_3-t_2}(Nm_3).
\]
By the the principle of non-stationary phase, we can truncate the sum across $I_m$ to $|m_1|,|m_2|,|m_3|\lesssim T/N$, at the cost of $O\epsilon(T^{-100})$, as $t_j-t_k=O(T)$.
We expand $\hat{h}$ in integral form, so that \begin{align*}
    I_{\vec{m}}&=N^3\sum_{t_1,t_2,t_3\in W}\int_{\reals^3} \omega(u_1)^2\omega(u_2)^2\omega(u_3)^2 u_1^{i(t_1-t_3)}u_2^{i(t_2-t_1)}u_3^{i(t_3-t_2)} e(-N\vec{m}\cdot \vec{u}) d\vec{u}\\ &=N^3\sum_{t_1,t_2,t_3\in W}\int_{\reals^3} \tilde{\omega}(\vec{u}) \Big(\frac{u_1}{u_2}\Big)^{it_1}\Big(\frac{u_2}{u_3}\Big)^{it_2}\Big(\frac{u_3}{u_1}\Big)^{it_3}e(-N\vec{m}\cdot \vec{u}) d\vec{u}
\end{align*}
For $\tilde{\omega}(\vec{u})=\omega(u_1)^2\omega(u_2)^2\omega(u_3)^2$. Because $\tilde{\omega}$ is supported away from $u_3=0$, we can introduce the change of variables $v_1=u_1/u_3$, $v_2=u_2/u_3$. The Jacobian is $u_3^2$, and $u_1/u_2=v_1/v_2$, so that \begin{align*}
    I_{\vec{m}}& = N^3\sum_{t_1,t_2,t_3\in W}\int_{\reals^3} u_3^2 \tilde{\omega}(v_1u_3,v_2u_3,u_3) \Big(\frac{v_1}{v_2}\Big)^{it_1} v_2^{it_2}\Big(\frac{1}{v_1}\Big)^{it_3}e(-Nu_3(m_1v_1+m_2v_2+m_3)) dv_1\ dv_2 \ du_3\\ 
    & = N^3\sum_{t_1,t_2,t_3\in W}\int_{\reals^2}\int_{\reals} u_3^2 \tilde{\omega}(v_1u_3,v_2u_3,u_3)e(-Nu_3(m_1v_1+m_2v_2+m_3))   du_3 \  \Big(\frac{v_1}{v_2}\Big)^{it_1} v_2^{it_2}\Big(\frac{1}{v_1}\Big)^{it_3}  dv_1\ dv_2 
\end{align*}
The inner integral in $u_3$ places restrictions on the domain of integration. First, the support of $\tilde{\omega}$ is $[1,2]\times [1,2]\times [1,2]$. Thus, if it is non-zero, we have $v_1u_3,v_2u_3,u_3\in [1,2]\implies v_1,v_2\in [1/2,2]$. Therefore, we can restrict the outer integral in $v_1$ and $v_2$ to this range. Next, since $v_1,v_2=O(1)$, the chain rule gives\[
\left(\frac{\partial}{\partial u_3}\right)^j{\omega}(v_1u_3,v_2u_3,u_3)\ll_j 1.\]
Therefore, we can apply the repeated integration by parts to get rapid decay of the integral in $|N(m_1v_1+m_2v_2+m_3)|$. In particular, we can truncate the integral to the range $|N(m_1v_1+m_2v_2+m_3)|\ll T^\epsilon$ at an error of $O_{\epsilon}(T^{-100})$, and use \[
    \int_{\reals} u_3^2 \tilde{\omega}(v_1u_3,v_2u_3,u_3)e(-Nu_3(m_1v_1+m_2v_2+m_3))   du_3=O(1)
\]
in this range by the compact support of $\tilde{\omega}$. This gives us \begin{equation}
    |I_{\vec{m}}|\leq \Bigg|N^3 \sum_{t_1,t_2,t_3\in W}\int\displaylimits_{\substack{
        |v_1m_1+v_2m_2+m_3|\lesssim \frac{1}{N}\\
        \frac{1}{2}\leq v_1,v_2\leq 2
    }} \Big(\frac{v_1}{v_2}\Big)^{it_1} v_2^{it_2}\Big(\frac{1}{v_1}\Big)^{it_3}  dv_1\ dv_2 \Bigg|+O_{\epsilon}(T^{-100}).
\end{equation}
Recall that in the outline we defined\[
R(v)\defeq\sum_{t\in W} v^{it}.
\]
Exchanging the summation and integral, we get the term \begin{align*}
    &\Bigg| N^3 \int\displaylimits_{\substack{
        |v_1m_1+v_2m_2+m_3|\lesssim \frac{1}{N}\\
        \frac{1}{2}\leq v_1,v_2\leq 2
    }} R\Big(\frac{v_1}{v_2}\Big) R({v_2}\Big) R\Big(\frac{1}{v_1}\Big) dv_1\ dv_2 \Bigg|\\
    \leq & N^3 \int\displaylimits_{\substack{
        |v_1m_1+v_2m_2+m_3|\lesssim \frac{1}{N}\\
        \frac{1}{2}\leq v_1,v_2\leq 2
    }} \Bigg|R\Big(\frac{v_1}{v_2}\Big) R({v_2}\Big) R\Big(\frac{1}{v_1}\Big)\Bigg| dv_1\ dv_2 \\
    =  & N^3 \int\displaylimits_{\substack{
        |v_1m_1+v_2m_2+m_3|\lesssim \frac{1}{N}\\
        \frac{1}{2}\leq v_1,v_2\leq 2
    }} \Bigg|R\Big(\frac{v_2}{v_1}\Big) R({v_2}\Big) R({v_1})\Bigg| dv_1\ dv_2,
\end{align*}
where the last step, we used \[
|R(v^{-1})| = \Big|\sum_{t\in W} v^{-it}\Big| = \Big|\sum_{t\in W} v^{it}\Big| = |R(v)|.
\]
Now we fix $v_1$, and consider the integral in $v_2$ in the range \[
    |v_1m_1+v_2m_2+m_3|\lesssim \frac{1}{N} \implies \Big|v_2 -\frac{v_1m_1+m_3}{-m_2}\Big|\lesssim \frac{1}{|m_2| N}.
\]
If we enforce the conditions $|m_1|\leq |m_2|\leq|m_3|$ and $v_2\asymp 1$, we see that the domain of integration is empty unless $|m_2|\asymp |m_3|$. Thus, we can break the sum across \[\sum_{|m_1|,|m_2|,|m_3|\lesssim T/N}\] to be \[\log T^{1+\epsilon}/N \ \rm{sup}_{U=2^j, V\leq U} \sum_{\substack{|m_1|\sim V\\  |m_2|, |m_3|\sim U}}.\]
Moreover, we integrate in $v_2$ over a very small neighborhood of width $\lesssim 1/(|m_2|N)$ around $v_2=({v_1m_1+m_3})/({-m_2})$. In principle, we can estimate this integral by taking the value of $R$ at this point to get \[
    \approx  \frac{1}{|m_2|N} \int\displaylimits_{\substack{
        \frac{1}{2}\leq v_1\leq 2
    }} \Bigg|R\Big(\frac{v_1m_1+m_3}{-v_1m_2}\Big) R(\frac{v_1m_1+m_3}{-m_2}\Big) R({v_1})\Bigg| dv_1.
\]
This is made precise by apply to a $(1/N|m_2|)$-smoothened version of $R$, stated in Proposition \ref{dyadics_3}. Finally, H\"older's inequality gives a bound of \begin{align*}
    \Big\|R\Big(\frac{vm_1+m_3}{-vm_2}\Big) \Big\|_{\substack{L_4\\ v\asymp 1}}\Big\|R\Big(\frac{vm_1+m_3}{-m_2}\Big) \Big\|_{\substack{L_4\\ v_1\asymp 1}}\Big\|R(v) \Big\|_{\substack{L_2\\ v_1\asymp 1}}
\end{align*}

The second moment of $R$ is bounded by the size of $W$. Indeed, we have \begin{align*}
    \int_{v\asymp 1} |R(v)|^2 dv =  \sum_{t_1,t_2\in W} \int_{v\asymp 1} v^{i(t_1-t_2)} dv.  
\end{align*}
If $t_1-t_2\neq 0$, then $|t_1-t_2|>T^\epsilon$, so the oscillatory integral will be negligible by the fast decay. There are $|W|$ terms satisfying $t_1=t_2$, and each contributes $O(1)$ to the sum. Similarly, the fourth moment of $R$ is bounded by the energy, recalling its definition \[
    E(W)\defeq \#\{t_1,t_2,t_3,t_4\in W : |t_1+t_2-t_3-t_4|\ll T^\epsilon\}.
\]
We get
\begin{align*}
    \int_{v\asymp 1} |R(v)|^4 dv =   \sum_{t_1,t_2,t_3,t_4\in W} \int_{v\asymp 1}  v^{i(t_1+t_2-t_3-t_4)} dv, 
\end{align*}
and the terms in the summation are negligble unless $|t_1+t_2-t_3-t_4|\ll T^\epsilon$. Assembling everything together gives an initial bound of \[
S_3\lesssim N^3\sum_{|m_1|\sim V ,|m_2|\sim V, |m_3|\sim V}\frac{1}{|m_2|N}E(W)^{1/2}|W|^{1/2}\lesssim T^2E(W)^{1/2}|W|^{1/2},
\]
recalling that $U,V\lesssim T/N$.


\subsubsection*{9. Refinement of $S_3$ bound}

Recall that in the previous section, we summed across 
\[
    \sum_{\substack{|m_1|\sim V\\ |m_2|, |m_3|\sim U}} \frac{N}{|m_2|} \int\displaylimits_{\substack{
        \frac{1}{2}\leq v_1\leq 2
    }} \Bigg|R\Big(\frac{v_1m_1+m_3}{-v_1m_2}\Big) R\Big(\frac{v_1m_1+m_3}{-m_2}\Big) R({v_1})\Bigg| dv_1
\]
By repeatedly applying Cauchy-Schwartz, we can move the summation into the integral and obtain terms that resembles the form \[
   \int_{v_1\asymp 1} \Bigg(\sum_{\substack{|m_1|\sim V\\ |m_2|, |m_3|\sim U}}  \Big| R\Big(\frac{v_1m_1+m_3}{-m_2}\Big)\Big|^2\Bigg)^2 dv_1.
\]
The term $(v_1m_1+m_3)/(-m_2)$ describes an affine transformation in $v_1$. Intuitively, thus averaging over all the affine transformations is can be expressed as the expected value of a random variable squared, where the random variable \[
X\defeq X(v_1)\defeq \sum_{\substack{|m_1|\sim V\\ |m_2|, |m_3|\sim U}}  \Big| R\Big(\frac{v_1m_1+m_3}{-m_2}\Big)\Big|^2,
\] and the measure is uniform across $v_1\asymp 1$. This can be split up to get \[
E[X^2] = E[X]^2 + \rm{Var}(X).
\] Indeed, we have Lemma \ref{affinetransgm} that gives us this estimate. 
By a change of variables, the expected value of $X$ is represented by $M^3 \int_{u\asymp 1} R(u)^2 du$.
\begin{lemma}\label{affinetransgm}
    Let $M>0$. Let $f(u)\geq 0$, supported on $u\asymp 1$, and $|\hat{f}(\xi)|\lesssim_j (|\xi|/T)^j$ for all $j$. Then \begin{align*}
        J(f)\defeq &\sup_{0<M_1,M_2,M_3<M} \int\Bigg( \sum_{\substack{|m_1|\sim M_1,|m_2|\sim M_2 , |m_3|\ll M_3}} f\left(\frac{m_1 u+m_3}{m_2}\right)\Bigg)^2 \ du\\
        \lesssim&M^6\Big( \int f(u) du\Big)^2+M^4 \int f(u)^2 du.
    \end{align*}
\end{lemma}
This is Proposition 9.1 in \cite{GM2024}.
This improves the bound on $S_3$ to\[
S_3\lesssim T^2|W|^{3/2}+TN|W|^{1/2}E(W)^{1/2}.
\]
This gives an improvement from the previous bound for the case with high energy and $N<T^{1-\delta}$.
\begin{remark}
    The proof to Lemma \ref{affinetransgm} does not use probability. Instead, it applies Plancherel's theorem on $\sum_{m_1,m_2,m_3} f$ and bounds in the Fourier domain.
    At low frequencies, this can be bounded by the $L_1$ norm of $f$. At high frequencies, the Fourier transform of $f$ decays rapidly. Finally, at mid frequencies, the integral relates to \[
     (M^4\int f^2 du)^{1/2}J(f)^{1/2},
    \]
    by Cauchy Schwarz and a change of variables to decouple $m_1$ and $m_2$, providing the required bound by backward induction.
\end{remark}

\subsubsection*{8. Bound on $E(W)$}
Finally, we give an orthogonal bound on the energy of the set $W$. The idea is that if $|t_1+t_2-t_3-t_4|$ is small, then we can approximate \[
D_N(t_4) \approx D_N(t_1+t_2-t_3).
\]
This is made precise by applying a smoothing of $D_N$ over a width of $\lesssim 1$.
Therefore, since the choice of $t_4$ is fixed by the choice of $t_1,t_2,t_3$, we have\[
E(W)V^2\leq \sum_{|t_1-t_2-t_3-t_4|\lesssim 1} |D_N(t_4)|^2 \approx \sum_{|t_1-t_2-t_3-t_4|\lesssim 1} |D_N(t_1+t_2-t_3)|^2 \leq \sum_{t_1,t_2,t_3\in W} |D_N(t_1+t_2-t_3)|^2.
\]
Now we can expand $|D_N(t_1+t_2-t_3)|^2$ to get \begin{align*}
    \sum_{t_1,t_2,t_3\in W} |D_N(t_1+t_2-t_3)|^2=& \sum_{t_1,t_2,t_3\in W}\sum_{n_1,n_2\sim N} b_{n_1} \bar{b}_{n_2} \Big(\frac{n_1}{n_2}\Big)^{i(t_1+t_2-t_3)}\\ 
    =&\sum_{n_1,n_2\sim N} b_{n_1} \bar{b}_{n_2}\sum_{t_1,t_2,t_3\in W}\Big(\frac{n_1}{n_2}\Big)^{i(t_1+t_2-t_3)}\\ 
    \leq& \sum_{n_1,n_2\sim N} \Big|R\Big(\frac{n_1}{n_2}\Big)R\Big(\frac{n_1}{n_2}\Big)R\Big(\frac{n_2}{n_1}\Big)\Big|\\ 
    \leq& \sum_{n_1,n_2\sim N} \Big|R\Big(\frac{n_1}{n_2}\Big)\Big|^3.
\end{align*}
Now we can apply the trivial bound $|R|\leq |W|$ to get that \[
    E(W)\lesssim V^{-2}|W|\sum_{n_1,n_2\sim N} \Big|R\Big(\frac{n_1}{n_2}\Big)\Big|^2.
\]
This is in turn bounded by Heath-Brown's result as \begin{align*}
    \sum_{n_1,n_2\sim N} \Big|R\Big(\frac{n_1}{n_2}\Big)\Big|^2 &=
    \sum_{n_1,n_2\sim N}\sum_{t_1,t_2\in W} \Big(\frac{n_1}{n_2}\Big)^{i(t_1-t_2)}\\ 
    &= \sum_{t_1,t_2\in W} \Big|\sum_{n\sim N} n^{i(t_1-t_2)} \Big|^2\\
    &\lesssim |W|N^2+|W|^2N+|W|^{5/4}T^{1/2}N.
\end{align*}
This is enough to give an improvement on Ingham-Huxley's result, but can be further improved using Cauchy Schwartz on the third moment \begin{align*}
    \sum_{n_1,n_2\sim N} \Big|R\Big(\frac{n_1}{n_2}\Big)\Big|^3 \leq 
    \Bigg(\sum_{n_1,n_2\sim N} \Big|R\Big(\frac{n_1}{n_2}\Big)\Big|^2 \Bigg)^{1/2} \Bigg(\sum_{n_1,n_2\sim N} \Big|R\Big(\frac{n_1}{n_2}\Big)\Big|^4\Bigg)^{1/2}.
\end{align*}
The fourth moment can be reduced back to the second moment by taking representative classes of $\lfloor t_1-t_2 \rfloor$, thus can also be bounded using Heath-Brown's result. 
	\section{Towards a Hybrid Zero Density Result}
\iffalse

\textit{Here I will document some progress with working with
the hybrid version of Guth-Maynard. This section will be removed when I send the draft to Prof. Wunsch.}
\fi
We would like to generalize Guth and Maynard's result to $L$-functions. Specifically, let $\chi$ be a \hyperref[dcharacter]{Dirichlet character}. we are interested in the the zeros defined by \[
L(s,\chi)\defeq\sum_{n} \frac{\chi(n)}{n^{-s}}
\]
on $\Re(s)>1$ and its analytic continuation on the whole complex plane. The structure of the arguments for analytic continuation, its line of symmetry along $\Re(s) = 1/2$, and the locations of trivial zeros are very similar to that of the zeta function. This motivates the Generalized Riemann Hypothesis. \begin{conjecture} [Generalized Riemann Hypothesis]
    The \text{Generalized Riemann Hypothesis} asserts that on the critical strip, \[
    L(s,\chi)=0 \implies \Re(s)=\frac{1}{2},
    \]
    for any Dirichlet character.
\end{conjecture}
This leads to even stronger for primes in short intervals. Namely, fix an integer $q$, we have that \[
\sum_{\substack{n\leq N\\ n\equiv a\mod q}}\Lambda(n) = \begin{cases}
    \frac{1}{\phi(q)} N + O(x^{2+o(1)}), & \textrm{if $\gcd(a,q)=1$}\\
    o(n),& \textrm{otherwise.}
\end{cases}
\]
This means that not only that the Prime Number Theorem holds in intervals of $x^{2+\epsilon}$, the distribution of primes in each of the residual classes (coprime to $q$) are uniform at this scale too. Noticing that we can modify Huxley's proof with \[
M_{x,\chi} = \sum_{n\leq x} \chi(n) \mu(n) n^{-s},
\]
we have \[
L(s,\chi)M_{x,\chi} = \sum_{n} a_n \chi(n) \mu(n) n^{-s}.
\]

Thus, we can reproduce a similar proof on the zero density of $L$-functions. 
\begin{definition}[Zero Density for $L$-functions]
    Let $N(\sigma,\chi,T)$ denote the number of zeros of the $L$-function $L(-,\chi)$ with real part greater than $\sigma$ and imaginary part between $-T$ and $-T$. That is \[
    N(\sigma,\chi,T) \defeq \#\{\rho=\beta+i\gamma \ | \ \beta\geq \sigma, |\gamma|\leq T\}.
    \]
    For backwards compatibility with our previous definition, we take $N(\sigma,T)\defeq N(\sigma, 1,T)$.
\end{definition}
The hybrid analogs of the zero density bounds of Ingham and Huxley are known.
\begin{theorem}[Hybrid Ingham bound for zero density]
    Let $1/2\leq \sigma\leq 3/4$. We have \[
        \sum_{\chi^* }N(\sigma,\chi^*,t)\lesssim (qT)^{\frac{3(1-\sigma)}{2-\sigma}},
        \]
        where $\sum_{\chi^*}$ sums over all the primitive characters $\chi^*$ of modulus $q$.
\end{theorem}
\begin{theorem}[Hybrid Huxley bound for zero density]
    Let $3/4\leq \sigma\leq 1$. We have \[
        \sum_{\chi^* }N(\sigma,\chi^*,t)\lesssim (qT)^{\frac{3(1-\sigma)}{3\sigma-1}},
        \]        
        where $\sum_{\chi^*}$ sums over all the primitive characters $\chi^*$ of modulus $q$.
\end{theorem}

The method for detecting zeros is very similar to Huxley's proof above with the slight change in definition of $M_{x,\chi}$. This argument then reduces to bounding the number of times large values of Dirichlet polynomials can occur. Therefore we want a result in the form:
\begin{quotation}  
Let $\mathcal{S}=\{(t_j,\chi_j)\}$ be a set such that each $\chi_j$ is a primitive Dirichlet character of modulus $q$, and $|t_j-t_k|\geq 1$ if $j\neq k$ and $\chi_j=\chi_k$. (That is, the $t$'s are $1$-separated if the characters are the same.)Let $|b_n|\leq 1$ be a sequence of numbers indexed in $n$, and suppose also that \[
\Bigg|\sum_{n\sim N} b_n\chi_j(n)n^{it_j} \Bigg|>V.
\]
We want to find a bound on $|\mathcal{S}|$.
\end{quotation}


Our result is as follows: 
\begin{theorem}
    To be determined \textrm{\smiley}.
\end{theorem}

The idea of the proof is very similar to Guth and Maynard's proof. We can define a $|\mathcal{S}\times N|$ matrix $M$ with entries
\[
    M_{t_j,\chi_j,n} = \chi_j(n)n^{it_j}
\]
for $(t_j,\chi_j)\in\mathcal{S}$ and $n\sim N$, and bound its operator norm in the exact same way: taking it to the $M^*M$ to the third power and calculating its trace. The Dirichlet characters are not nice to handle when extended to a function $\reals \to \complex$, so we break the sum into $q$ summations across each residue class $\mod q$, and apply Poisson summation on each piece of the sum. This gives us \begin{align*}
    \textrm{tr}((M^*M)^3)=\sum_{\substack{(t_1,\chi_1),\\(t_2,\chi_2),\\(t_3,\chi_3)\in\mathcal{S}}}
    \frac{N^3}{q^3}\sum_{m\in\mathbb{Z}^3}&\sum_{x\in (\mathbb{Z}/q\mathbb{Z})^3}\chi_1\bar{\chi}_3(x_1)\chi_2\bar{\chi}_1(x_2)\chi_3\bar{\chi}_2(x_3) e\left(\frac{-x\cdot m}{q}\right)\\
    \times \ &\hat{h}_{t_1-t_3}\left(\frac{Nm_1}{q}\right)\hat{h}_{t_2-t_1}\left(\frac{Nm_2}{q}\right)\hat{h}_{t_3-t_2}\left(\frac{Nm_3}{q}\right),
\end{align*}
where $h_t(x) \defeq \omega(u)^2 x^{it}$ has the same definition, thus its fourier transform has the same properties in decay. Similarly, we find that when we break the sum in $\mathbb{Z}^3$ into $S_0$ to $S_3$ in the same way, $S_0$ gives the main term, $S_1$ is negligible, and $S_2$ can be bounded by Heath-Brown's theorem. For the $S_3$ bound, we redefine \[
R(v,n_1,n_2)\defeq \sum_{(t,\chi)\in\mathcal{S}}\chi(n_1)\bar{\chi}(n_2) v^{it}.
\]
This additional structure in $\chi$ may look complicated when taking the $L_2$ or $L_4$ norm of $R$ in $v\asymp 1$, but this cancels out when taking \[
\sum_{n_1\in\mathbb{Z}/q\mathbb{Z}} R(v,n_1,n_2)^2 = \sum_{n_1\in\mathbb{Z}/q\mathbb{Z}}\sum_{\substack{(t_1,\chi_1),(t_2,\chi_2)\in\mathcal{S}}}\chi_1(n_1)\bar\chi_2(n_1)\bar\chi_1(n_2)\chi_2(n_2) v^{i(t_1-t_2)}.
\]
By the orthogonality of Dirichlet characters, the sum in $n_1$ vanishes except when $\chi_1=\chi_2$, in which case (assuming $\gcd(n_2,q)=1$) we can reduce \[
    \sum_{n_1\in\mathbb{Z}/q\mathbb{Z}} R(v,n_1,n_2)^2 =\phi(q) \sum_{\substack{(t_1,\chi_1),(t_2,\chi_2)\in\mathcal{S}\\\chi_1=\chi_2}}v^{i(t_1-t_2)}.
\]
When $\chi_1=\chi_2$, we have $t_1=t_2$ or they are $T^\epsilon$ separated, giving us decay in the calculation of the second moment of $R$ to be $\phi(q)|\mathcal{S}|$.
Similarly, we see the fourth power of $R$ cancels in the sum \[
    \sum_{n_1\in\mathbb{Z}/q\mathbb{Z}} R(v,n_1,n_2)^4 = \sum_{n_1\in\mathbb{Z}/q\mathbb{Z}}\sum_{\substack{(t_1,\chi_1),(t_2,\chi_2),\\(t_3,\chi_3),(t_4,\chi_4)\\\in\mathcal{S}}}\chi_1\chi_2\bar\chi_3\bar\chi_4(n_1)\bar\chi_1\bar\chi_2\chi_3\chi_4(n_2) v^{i(t_1+t_2-t_3-t_4)}.
\]
Only terms where $\chi_1\chi_2=\chi_3\chi_4$ can have non-zero contribution, giving us a natural definition for the energy of the set to be \[
E(\mathcal{S})\defeq \#\{(t_1,\chi_1),(t_2,\chi_2),(t_3,\chi_3),(t_4,\chi_4)\in \mathcal{S} \ | \ \chi_1\chi_2=\chi_3\chi_4, |t_1+t_2-t_3-t_4|\lesssim 1\}.
\]
This in turn can be bounded by a third moment of $R$: \[
 E(\mathcal{S})\lesssim N^{-2\sigma}\sum_{n_1,n_2\sim N}\left|R\left(\frac{n_1}{n_2},n_1,n_2\right) \right|^3.\]
 This expression is then bounded by Heath-Brown's theorem.

\subsection{Setup and Reduction of Theorem}
\begin{lemma}\label{reduction}
    Let $\mathcal{S}=\{(t_j,\chi_j)\}$ be a set such that each $\chi_j$ is a primitive Dirichlet character of modulus $q$, and $|t_j-t_k|\geq T^\epsilon$ if $j\neq k$ and $\chi_j=\chi_k$. Let $|b_n|\leq 1$ be a sequence of numbers indexed in $n$, $\omega$ be a smooth bump function that equals $1$ on $[6/5,9/5]$ and has support in $[1,2]$ (thus $\omega^{(A)}\ll_A 1$ for all $A$). Let $V=N^\sigma$, where $\sigma \in [placeholder]$, and $N=qT^\alpha$, where $\alpha \in [placeholder]$ Suppose also that \[
    |D_N(t_j,\chi_j)|\defeq\Bigg|\sum_{n\sim N} \omega\Bigg(\frac{n}{N}\Bigg)b_n\chi_j(n)n^{it_j} \Bigg|>V.
\]
for all $(t_j,\chi_j)\in\mathcal{S}$.
Then \[
|\mathcal{S}|\lesssim \smiley.
\]
\end{lemma}
\begin{proof}[Proof of main theorem from Lemma \ref{reduction}]
    We split \[
        \sum_{n\sim N} b_n\chi_j(n)n^{it_j}=\sum_{N\leq n< 6N/5} b_n\chi_j(n)n^{it_j} +\sum_{6N/5 \leq n < 9N/5} b_n\chi_j(n)n^{it_j} +\sum_{9N/5 \leq n < 2N} b_n\chi_j(n)n^{it_j}.
    \]
    So that if \[
        \Big|\sum{n\sim N} b_n\chi_j(n)n^{it_j}\Big|>V,
    \]
    then at least one of the three summations in the right has magnitude at least $V/3$.
    But then each of these sums can be written as dirichlet sums in the form of the lemma: \begin{align*}
        \sum_{N\leq n< 6N/5} b_n\chi_j(n)n^{it_j} &=  \sum_{N\leq n< 6N/5} \omega\Bigg(\frac{n}{11N/15}\Bigg) b_n\chi_j(n)n^{it_j}, \\
        \sum_{6N/5 \leq n < 9N/5} b_n\chi_j(n)n^{it_j} &=  \sum_{6N/5 \leq n < 9N/5} \omega\Bigg(\frac{n}{N}\Bigg) b_n\chi_j(n)n^{it_j}, \\
        \sum_{9N/5 \leq n < 2N}  b_n\chi_j(n)n^{it_j} &=  \sum_{9N/5 \leq n < 2N}  \omega\Bigg(\frac{n}{19N/15}\Bigg) b_n\chi_j(n)n^{it_j}. \\
    \end{align*}
    Applying the lemma on each of these three pieces and summing the three bounds gives the same bound for $\sum_{n\sim N} b_n\chi_j(n)n^{it_j}$ up to a constant factor.
\end{proof}
\subsection{The matrix $M$ and its trace expansion}
We define $M$ a $|\mathcal{S}\times N|$ matrix with entries
\[
    M_{(t_j,\chi_j),n} = \omega\Big(\frac{n}{N}\Big)\chi_j(n)n^{it_j}
\]
for $(t_j,\chi_j)\in\mathcal{S}$ and $n\sim N$.
Thus by the same reasoning that $(M\vec{b})_j=D_N(t_j,\chi_j)$,
we want to bound the size of $\mathcal{S}$ by trace of the matrix \[
\textrm{tr}((M^*M)^3).
\].

\begin{proposition} \label{setup_poisson}
    Let  \begin{align*}
        I_m\defeq\frac{N^3}{q^3}\sum_{\substack{(t_1,\chi_1),\\(t_2,\chi_2),\\(t_3,\chi_3)\in\mathcal{S}}} &\sum_{x\in (\mathbb{Z}/q\mathbb{Z})^3}\chi_1\bar{\chi}_3(x_1)\chi_2\bar{\chi}_1(x_2)\chi_3\bar{\chi}_2(x_3) e\left(\frac{-x\cdot m}{q}\right)\\
        \times \ &\hat{h}_{t_1-t_3}\left(\frac{Nm_1}{q}\right)\hat{h}_{t_2-t_1}\left(\frac{Nm_2}{q}\right)\hat{h}_{t_3-t_2}\left(\frac{Nm_3}{q}\right).
    \end{align*}
    We have \[
    |\mathcal{S}|\lesssim \frac{\phi}{q}N^2V^{-2}+ NV^{-2}\sum_{m\in\mathbb{Z}^3 - \{0\}} I_m
    \]
\end{proposition}
\begin{lemma}[Explicit Expression of Traces]
    Let \[
    P(n,(t_1,\chi_1),(t_2,\chi_2))\defeq \omega\left(\frac{n}{N}\right)^2 \left(\frac{n}{N}\right)^{i(t_2-t_1)}\chi_2\bar{\chi}_1(n).
    \]
    We have 
    \begin{align*}
        \rm{tr}(M^*M) = & \ |\mathcal{S}|\sum_{n\sim N, \gcd (n,q)=1} \omega\left(\frac{n}{N}\right)^2, \\
        \rm{tr}((M^*M)^3)=&\sum_{\substack{(t_1,\chi_1),(t_2,\chi_2),\\(t_3,\chi_3)\in\mathcal{S}}}\sum_{n_1,n_2,n_3\sim N} 
        P(n_1,(t_1,\chi_1),(t_3,\chi_3))P(n_2,(t_2,\chi_2),(t_1,\chi_1))P(n_3,(t_3,\chi_3),(t_2,\chi_2)).
    \end{align*}
\end{lemma}
\begin{proof}
    We compute the traces of $MM^*$ and $(MM^*)^3$ instead, as this makes the calculations cleaner.
    We see that \begin{align*}
        (MM^*)_{(t_j,\chi_j),(t_k,\chi_k)} = \sum_{n\sim N} \omega\left(\frac{n}{N}\right)^2 n^{i(t_k-t_j)}\bar{\chi}_j\chi_k(n) = \sum_{n\sim N}P(n,(t_k,\chi_k),(t_j,\chi_j))N^{i(t_k-t_j)},
    \end{align*}
    so that \begin{align*}
        \textrm{tr}(MM^*) =& \sum_{(t_j,\chi_j)} \sum_{n\sim N} P(n,(t_j,\chi_j),(t_j,\chi_j)) \\
        =&|\mathcal{S}|\sum_{n\sim N} |\chi_j(n)|\omega\left(\frac{n}{N}\right)^2.
    \end{align*}
    For the third power, we have 
    \begin{align*}
        (MM^*)^3_{(t_j,\chi_j),(t_k,\chi_k)} = \sum_{(t_1,\chi_1),(t_2,\chi_2)\in \mathcal{S}}   (MM^*)_{(t_j,\chi_j),(t_1,\chi_1)} (MM^*)_{(t_1,\chi_1),(t_2,\chi_2)} (MM^*)_{(t_2,\chi_2),(t_k,\chi_k)},
    \end{align*}
    so that the trace is given by \begin{align*}
        &\sum_{\substack{(t_1,\chi_1),(t_2,\chi_2),\\(t_3,\chi_3)\in\mathcal{S}}} (MM^*)_{(t_3,\chi_3),(t_1,\chi_1)} (MM^*)_{(t_1,\chi_1),(t_2,\chi_2)} (MM^*)_{(t_2,\chi_2),(t_3,\chi_3)}\\
        =&\sum_{\substack{(t_1,\chi_1),(t_2,\chi_2),\\(t_3,\chi_3)\in\mathcal{S}}} \sum_{n_1,n_2,n_3\sim N} 
        P(n_1,(t_1,\chi_1),(t_3,\chi_3))P(n_2,(t_2,\chi_2),(t_1,\chi_1))P(n_3,(t_3,\chi_3),(t_2,\chi_2)).
    \end{align*}
\end{proof}
We would like to apply Poisson summation directly to $P$ in $n_1$, $n_2$, $n_3$, respectively. However, we do not have a good way to express the Dirichlet character $\chi_j\bar\chi_k$ as a $C^\infty$ function. To get around this, we just split the sum in $n$ into $q$ pieces. 
\begin{lemma}[Poisson Summation with Periodic Arithmetic Function]
    Let $f:\reals \to \complex$ be Schwartz, and $g:\naturals\to\complex$ be an arithmetic with period $q$. Then \[
    \sum_n g(n)f(n) = \frac{1}{q}\sum_{\xi} \sum_{x \mod q} g(x) e\bigg(\frac{-x\xi}{q}\bigg)\hat{f}\bigg(\frac{\xi}{q}\bigg). 
    \]
\end{lemma}
\begin{proof}
    By absolute convergence, we have \begin{align*}
        \sum_n g(n)f(n) = \sum_n \sum_{x\mod q} g(qn+x)f(qn+x)=  \sum_{x\mod q} g(x)\sum_n f(qn+x).
    \end{align*}
    Applying Poisson summation to the last expression in $n$ gives \[
        \frac{1}{q} \sum_{x\mod q} g(x) \sum_{\xi} e\bigg(\frac{-x\xi}{q}\bigg)\hat{f}\bigg(\frac{\xi}{q}\bigg).
    \]
\end{proof}
\begin{corollary}[Poisson summation of Trace]
    Let $h_t(u)\defeq\omega(u)^2u^{it}$,
    Then \begin{align*}
        \rm{tr}(M^*M) &= |\mathcal{S}|\frac{N}{q}\sum_{m}\sum_{x\in \mathbb{Z}/q\mathbb{Z}}e\left(\frac{-xm}{q}\right)\hat{h}_0\left(\frac{Nm}{q}\right)\\
        \rm{tr}((M^*M)^3)&=\sum_{\substack{(t_1,\chi_1),\\(t_2,\chi_2),\\(t_3,\chi_3)\in\mathcal{S}}}
        \frac{N^3}{q^3}\sum_{m\in\mathbb{Z}^3}\sum_{x\in (\mathbb{Z}/q\mathbb{Z})^3}\chi_1\bar{\chi}_3(x_1)\chi_2\bar{\chi}_1(x_2)\chi_3\bar{\chi}_2(x_3) e\left(\frac{-x\cdot m}{q}\right)\\
        & \quad \quad \quad \times \ \hat{h}_{t_1-t_3}\left(\frac{Nm_1}{q}\right)\hat{h}_{t_2-t_1}\left(\frac{Nm_2}{q}\right)\hat{h}_{t_3-t_2}\left(\frac{Nm_3}{q}\right).
    \end{align*}
\end{corollary}
In the trace of $(M^*M)^3$, we split the sum in $m\defeq (m_1,m_2,m_3)$ according to the number of non-zero entries into four parts (same as Guth Maynard Sketch above) $S_0+ S_1+S_2+S_3$,
where $S_j$ runs over the values of $m$ with exactly $j$ non-zero entries.

\begin{lemma}[Isolation of main term]
    We have \begin{align*}
        \rm{tr}(M^*M) &= \frac{|\mathcal{S}|N\phi(q)}{q}\|\omega\|_{L_2}^2 + O(T^{-100}),\\
        S_0 &= \frac{|\mathcal{S}|N^3\phi(q)^3}{q^3}\|\omega\|_{L_2}^6+O(T^{-100}).
    \end{align*}
\end{lemma}
\begin{proof}
    By \hyperref[nonstationary]{Non-Stationary Phase}, we have rapid decay of $h_0(\xi)$, so we can only consider the terms with $m=0$ in the sum for $\rm{tr}(M^*M)$ at a cost of $O(T^{-100})$.
    This $m=0$ term evaluates to \[
        \frac{|\mathcal{S}|N}{q}\sum_{x\in \mathbb{Z}/q\mathbb{Z}}e\left(0\right)\hat{h}_0\left(0\right)=\frac{|\mathcal{S}|N\phi(q)}{q}\|\omega\|_{L_2}^2.
    \]
    For $S_0$, we have\[
    S_0 = \sum_{\substack{(t_1,\chi_1),\\(t_2,\chi_2),\\(t_3,\chi_3)\in\mathcal{S}}}
    \frac{N^3}{q^3}\sum_{x\in (\mathbb{Z}/q\mathbb{Z})^3}\chi_1\bar{\chi}_3(x_1)\chi_2\bar{\chi}_1(x_2)\chi_3\bar{\chi}_2(x_3)  \hat{h}_{t_1-t_3}\left(0\right)\hat{h}_{t_2-t_1}\left(0\right)\hat{h}_{t_3-t_2}\left(0\right).
    \]
    By the orthogonality of characters (as representations of $(\mathbb{Z}/q\mathbb{Z})^{\times}$),
    only the terms with $\chi_1=\chi_2=\chi_3$ can have non-zero contribution to the sum. But when $\chi_1=\chi_2=\chi_3$, $|t_i-t_j|=0$ or $|t_i-t_j|\geq T^{\epsilon}$. In the second case, we can apply \hyperref[nonstationary]{Non-Stationary Phase} to get rapid decay in $\hat{h}_{t_i-t_j}({0})$. Therefore \[
        S_0 = \sum_{(t,\chi)\in\mathcal{S}}
        \frac{N^3}{q^3}\sum_{x\in (\mathbb{Z}/q\mathbb{Z})^3} \hat{h}_{0}\left(0\right)^3+O(T^{-100})=\frac{|\mathcal{S}|N^3\phi(q)^3}{q^3}\|\omega\|_{L_2}^6+O(T^{-100}).
    \]
\end{proof}

\begin{proof}[Proof of Proposition \ref{setup_poisson}]
    We note that the inequality in equation \ref{basicineq} applies in this case, and apply Lemma \ref{gmtrace} and substitute the values we have obtained.
\end{proof}
	\section{$S_1$ bound}
\begin{proposition}\label{s1bound}
    $S_1=O\epsilon(T^{-10})$.
\end{proposition}
\begin{proof}
    Recall that \[
    S_1=\sum_{\substack{m_1,m_2,m_3\in \mathbb{Z},\\
    \#\{m_j \ | \ m_j\neq 0\}=1}} I_m.\]
By symmetry, we sum $I_m$ across all $m=(0,0,m_3\neq 0)$ at a cost of a factor of $3$.
In this case, \begin{align*}
    I_{(0,0,m_3)}=\frac{N^3}{q^3}\sum_{\substack{(t_1,\chi_1),\\(t_2,\chi_2),\\(t_3,\chi_3)\in\mathcal{S}}} &\sum_{x\in (\mathbb{Z}/q\mathbb{Z})^3}\chi_1\bar{\chi}_3(x_1)\chi_2\bar{\chi}_1(x_2)\chi_3\bar{\chi}_2(x_3) e\left(\frac{-x_3 m_3}{q}\right)\\
    \times \ &\hat{h}_{t_1-t_3}\left(0\right)\hat{h}_{t_2-t_1}\left(0\right)\hat{h}_{t_3-t_2}\left(\frac{Nm_3}{q}\right)
\end{align*}
By the orthogonality of characters, this expression vanishes when $\chi_1\neq \chi_2$ (consider summation in $x_2$) or $\chi_2\neq \chi_3$ (summation in $x_1$). Therefore, we can further restrict our summation in $\mathcal{S}$ to be over terms with $\chi_1=\chi_2=\chi_3$ to get \begin{align*}
    I_m&=\frac{N^3}{q^3}\sum_{\substack{(t_1,\chi_1),\\(t_2,\chi_2=\chi_1),\\(t_3,\chi_3=\chi_1)\in\mathcal{S}}} \ \phi(q)^2 \sum_{x_3\in (\mathbb{Z}/q\mathbb{Z})^{\times}}e\left(\frac{-x_3 m_3}{q}\right)\\
   & \quad  \times  \quad {h}_{t_1-t_3}\left(0\right)\hat{h}_{t_2-t_1}\left(0\right)\hat{h}_{t_3-t_2}\left(\frac{Nm_3}{q}\right)\\
   &=O\left(\frac{N^3\phi(q)^3}{q^3}\right) \sum_{\substack{(t_1,\chi_1),\\(t_2,\chi_2=\chi_1),\\(t_3,\chi_3=\chi_1)\in\mathcal{S}}} \hat{h}_{t_1-t_3}\left(0\right)\hat{h}_{t_2-t_1}\left(0\right)\hat{h}_{t_3-t_2}\left(\frac{Nm_3}{q}\right)\\
\end{align*}
So we trivially bound $S_1$ by\begin{align*}
    |S_1|&\ll  \frac{N^3}{q^3} \phi(q)^3 \sum_{m_3\neq 0}\sum_{\substack{(t_1,\chi_1),\\(t_2,\chi_2=\chi_1),\\(t_3,\chi_3=\chi_1)\in\mathcal{S}}}
   \left|\hat{h}_{t_1-t_3}\left(0\right)\hat{h}_{t_2-t_1}\left(0\right)\hat{h}_{t_3-t_2}\left(\frac{Nm_3}{q}\right)\right|\\
\end{align*}
On the right hand side, we split the sum into three parts \begin{align*}
    \frac{N^3}{q^3}\phi(q)^3(S_{1,1}+S_{1,2}+S_{1,3}),
\end{align*}
where \begin{align*}
    S_{1,1}&= \sum_{|m_3|>qT^{1+\epsilon}/N}\sum_{\substack{(t_1,\chi_1),\\(t_2,\chi_2=\chi_1),\\(t_3,\chi_3=\chi_1)\in\mathcal{S}}}
    \left|\hat{h}_{t_1-t_3}\left(0\right)\hat{h}_{t_2-t_1}\left(0\right)\hat{h}_{t_3-t_2}\left(\frac{Nm_3}{q}\right)\right|,\\
    S_{1,2}&=  \sum_{|m_3|\leq qT^{1+\epsilon}/N}\sum_{\substack{(t_1,\chi_1),\\(t_2,\chi_2=\chi_1),\\(t_3,\chi_3=\chi_1)\in\mathcal{S},\\ t_1\neq t_2 \ \rm{or} \ t_1\neq t_3}}
    \left|\hat{h}_{t_1-t_3}\left(0\right)\hat{h}_{t_2-t_1}\left(0\right)\hat{h}_{t_3-t_2}\left(\frac{Nm_3}{q}\right)\right|,\\
    S_{1,3}&= \sum_{|m_3|\leq qT^{1+\epsilon}/N}\sum_{(t,\chi)\in\mathcal{S}}
    \left|\hat{h}_{0}\left(0\right)\hat{h}_{0}\left(0\right)\hat{h}_{0}\left(\frac{Nm_3}{q}\right)\right|.
\end{align*}
We now apply \hyperref[nonstationary]{Non-Stationary Phase} to $\hat{h}$. In $S_{1,1}$, we consider the terms with $|m_3|>qT^{1+\epsilon}/N$, we have decay in \[
    \hat{h}_{t_3-t_2}\ll_{A}\left(\frac{Nm_3}{q}\right)\left(1+|t_3-t_2|\right)^A\left(\frac{N|m_3|}{q}\right)^{-A} \ll T^A \left(T^{1+\epsilon}\right)^{-A} \left(\frac{N|m_3|}{qT^{1+\epsilon}}\right)^{-A}.
\] Therefore, we trivially bound $\hat{h}_t{\xi}\ll 1$ to obtain a bound of $O_\epsilon(T^{-100})$ for the sum across $|m_3|>qT^{1+\epsilon}/N$.
Similarly, for $S_{1,2}$  with terms consisting $t_1\neq t_2$ or $t_1\neq t_3$, they will be $T^\epsilon$ apart, so $h_{t_2-t_1}(0)$ or $h_{t_1-t_3}(0)$ can be bounded by $O_{\epsilon}(T^{-100})$.

The final terms to bound are when $t_1=t_2=t_3$ in $S_{1,3}$, in which case \[
    \left|\hat{h}_{t_3-t_2}\left(\frac{Nm_3}{q}\right)\right| \ll_{A} \left(\frac{q}{N}\right)^{A}\ll_{\epsilon} T^{-100}.
\]
Combined, the contribution of $S_1$ is $O\epsilon(T^{-10})$.

\end{proof}


\section{$S_2$ bound}
\begin{proposition}\label{s2bound}
    We have for any integer $k$,
    \[
    S_2\lesssim_k |\mathcal{S}|^2N+|\mathcal{S}|^{2-1/k}N^2+|\mathcal{S}|^2N^2\Big(\frac{(qT)^{1/2}}{|\mathcal{S}^{3/4}|}\Big)^{1/k}.
    \]
\end{proposition}

\begin{lemma}
    We have 
    \[
    S_2 = 3\frac{N^3}{q^3} \phi(q) \hat{h}_{0}\left(0\right) \sum_{\substack{(t_1,\chi_1),\\(t_2,\chi_2)\in\mathcal{S}}} \left|\sum_{m\neq 0} \sum_{x \in \mathbb{Z}/q\mathbb{Z}}\chi_1\bar{\chi}_2(x) e\left(\frac{-mx}{q}\right)
     \hat{h}_{t_1-t_2}\left(\frac{Nm}{q}\right)\right|^2 + O(T^{-10}).
\]
\end{lemma}
\begin{proof}
    We write by symmetry \begin{align*}
        S_2= 3\frac{N^3}{q^3}\sum_{m_1,m_2\neq 0}\sum_{\substack{(t_1,\chi_1),\\(t_2,\chi_2),\\(t_3,\chi_3)\in\mathcal{S}}} &\sum_{x\in (\mathbb{Z}/q\mathbb{Z})^3}\chi_1\bar{\chi}_3(x_1)\chi_2\bar{\chi}_1(x_2)\chi_3\bar{\chi}_2(x_3) e\left(\frac{-x_1m_1-x_2m_2}{q}\right)\\
        \times \ &\hat{h}_{t_1-t_3}\left(\frac{Nm_1}{q}\right)\hat{h}_{t_2-t_1}\left(\frac{Nm_2}{q}\right)\hat{h}_{t_3-t_2}\left(0\right)
    \end{align*}
    Removing zero contributions from $\chi_2\neq \chi_3$ by orthogonality,
    we have \begin{align*}
        =3\frac{N^3}{q^3} \phi(q) \sum_{m_1,m_2\neq 0}\sum_{\substack{(t_1,\chi_1),\\(t_2,\chi_2),\\(t_3,\chi_3=\chi_2)\in\mathcal{S}}} &\sum_{x_1,x_2 \in \mathbb{Z}/q\mathbb{Z}}\chi_1\bar{\chi}_2(x_1)\chi_2\bar{\chi}_1(x_2) e\left(\frac{-x_1m_1-x_2m_2}{q}\right)\\
        \times \ &\hat{h}_{t_1-t_3}\left(\frac{Nm_1}{q}\right)\hat{h}_{t_2-t_1}\left(\frac{Nm_2}{q}\right)\hat{h}_{t_3-t_2}\left(0\right)
    \end{align*}
    Here, we can isolate contributions from the terms where $t_2\neq t_3$ (hence since $\chi_2=\chi_3$, are $T^{\epsilon}$ separated) to be $O(T^{-10})$. For the other terms, we can write
    \[
        \hat{h}_t(\xi) = \overline{\hat{h}_{-t}(-\xi)}
    \]
    to get 
    \iffalse
    $S_2$
    \begin{align*}
        = 3\frac{N^3}{q^3} \hat{h}_{0}\left(0\right)\phi(q) \sum_{m_1,m_2\neq 0}\sum_{\substack{(t_1,\chi_1),\\(t_2,\chi_2)\in\mathcal{S}}} &\sum_{x_1,x_2 \in \mathbb{Z}/q\mathbb{Z}}\chi_1\bar{\chi}_2(x_1)\chi_2\bar{\chi}_1(x_2) e\left(\frac{-x_1m_1-x_2m_2}{q}\right)\\
        \times \ &\hat{h}_{t_1-t_2}\left(\frac{Nm_1}{q}\right)\hat{h}_{t_2-t_1}\left(\frac{Nm_2}{q}\right)
    \end{align*}
    and by 
    
    we can rewrite this to get
    \fi
    \[
        S_2 = 3\frac{N^3}{q^3} \phi(q) \hat{h}_{0}\left(0\right) \sum_{\substack{(t_1,\chi_1),\\(t_2,\chi_2)\in\mathcal{S}}} \left|\sum_{m\neq 0} \sum_{x \in \mathbb{Z}/q\mathbb{Z}}\chi_1\bar{\chi}_2(x) e\left(\frac{-mx}{q}\right)
         \hat{h}_{t_1-t_2}\left(\frac{Nm}{q}\right)\right|^2 + O(T^{-10}).
    \]
\end{proof}
\begin{proof}[Proof of Proposition \ref{s2bound}]
    By the principle of non-stationary phase the terms where $(t_1,\chi_1)=(t_2,\chi_2)$ contribute $O(T^{-10})$. If $\chi_1\neq \chi_2$, then we can apply the orthogonality of characters to extend the summation in $m$ to include $m=0$ too. Lastly, if $\chi_1=\chi_2$ and $t_1\neq t_2$, we add the negligible term corresponding to $m=0$, as $\hat h_{t_1-t_2}(0)$ is negligble.
    Therefore, we have \begin{align*}
        S_2&\ll \frac{N^3}{q^3} \phi(q) \sum_{\substack{(t_1,\chi_1),\\(t_2,\chi_2)\in\mathcal{S}\\ (t_1,\chi_1)\neq (t_2,\chi_2)}} \left|\sum_{m\neq 0} \sum_{x \in \mathbb{Z}/q\mathbb{Z}}\chi_1\bar{\chi}_2(x) e\left(\frac{-mx}{q}\right)
        \hat{h}_{t_1-t_2}\left(\frac{Nm}{q}\right)\right|^2 + O(T^{-10})\\ 
        &\stackeq{Poisson} \quad \frac{N}{q} \phi(q) \sum_{\substack{(t_1,\chi_1),\\(t_2,\chi_2)\in\mathcal{S}\\ (t_1,\chi_1)\neq (t_2,\chi_2)}} \left|\sum_{n\sim N} \chi_1\bar\chi_2(n)\omega\left(\frac{n}{N}\right)^2\left(\frac{n}{N}\right)^{i(t_1-t_2)}\right|^2 + O(T^{-10})\\
        &=\frac{N}{q} \phi(q) \sum_{\substack{(t_1,\chi_1),\\(t_2,\chi_2)\in\mathcal{S}\\ (t_1,\chi_1)\neq (t_2,\chi_2)}} \left|\sum_{n\sim N} \chi_1\bar\chi_2(n)\omega\left(\frac{n}{N}\right)^2n^{i(t_1-t_2)}\right|^2 + O(T^{-10})\\
        & \leq \frac{N}{q} \phi(q) \sum_{\substack{(t_1,\chi_1),\\(t_2,\chi_2)\in\mathcal{S}}} \left|\sum_{n\sim N} \chi_1\bar\chi_2(n)\omega\left(\frac{n}{N}\right)^2n^{i(t_1-t_2)}\right|^2 + O(T^{-10}).
    \end{align*}
    Before we apply Heath-Brown's theorem, we use H\"older's inequality to get \begin{align*}
        \sum_{\substack{(t_1,\chi_1),\\(t_2,\chi_2)\in\mathcal{S}}} \left|\sum_{n\sim N} \chi_1\bar\chi_2(n)\omega\left(\frac{n}{N}\right)^2n^{i(t_1-t_2)}\right|^2\leq \left(\sum_{\substack{(t_1,\chi_1),\\(t_2,\chi_2)\in\mathcal{S}}} 1\right)^{(k-1)/k}\left(\sum_{\substack{(t_1,\chi_1),\\(t_2,\chi_2)\in\mathcal{S}}} \left|\sum_{n\sim N} \chi_1\bar\chi_2(n)\omega\left(\frac{n}{N}\right)^2n^{i(t_1-t_2)}\right|^{2k}\right)^{1/k}\\
        =|\mathcal{S}|^{2-2/k}\left(\sum_{\substack{(t_1,\chi_1),\\(t_2,\chi_2)\in\mathcal{S}}} \left|\sum_{n\asymp_k N^k} O(d(n))\chi_1\bar\chi_2(n)n^{i(t_1-t_2)}\right|^{2k}\right)^{1/k}.
    \end{align*}
    As the divisor is bounded by $d(n)=N^{o(1)}$, we apply \hyperref[heathbrown]{Heath-Brown's theorem} to bound the second term by \[
    \lesssim_k (|\mathcal{S}|^2N^k+|\mathcal{S}|N^{2k}+|\mathcal{S}|^{5/4}(qT)^{1/2}N^k)^{1/k}\ll_k|\mathcal{S}|^{2k}N+|\mathcal{S}^{1/k}|N^{2}+|\mathcal{S}|^{5/(4k)}(qT)^{1/(2k)}N.
    \]
    Therefore, combined with the previous bound on $S_2$, we have \[
    S_2\lesssim \frac{N}{q} \phi(q)|\mathcal{S}|^{2-2/k}(|\mathcal{S}|^{2k}N+|\mathcal{S}^{1/k}|N^{2}+|\mathcal{S}|^{5/(4k)}(qT)^{1/(2k)}N)
    \]
    which gives the proposition upon simplification.
\end{proof}
\iffalse by decay in $Nm/q$. \textit{We also used the fact that there are at most $\phi(q)$ characters mod $q$, so the $O(q^2)$ factor is negligible compared to $N^{-100}$}.

For the other terms where $t_1$ and $t_2$ are $T^\epsilon$ separated or $\chi_1\neq \chi_2$, we want to apply Heath Brown's theorem. 
\begin{lemma}
    Let $t>T^\epsilon$ or $\chi$ be a non-primitive character mod $q$.
    \[
        \sum_{m\neq 0} \sum_{x \in \mathbb{Z}/q\mathbb{Z}}\chi(x) e\left(\frac{-mx}{q}\right)
        \hat{h}_{t}\left(\frac{Nm}{q}\right)\ll ?
    \]
\end{lemma}
\begin{proof}
    If $\chi$ is not primitive, then \[
        \sum_{m\neq 0} \sum_{x \in \mathbb{Z}/q\mathbb{Z}}\chi(x) e\left(\frac{-mx}{q}\right)
        \hat{h}_{t}\left(\frac{Nm}{q}\right)=\sum_{m\in \mathbb{Z}} \sum_{x \in \mathbb{Z}/q\mathbb{Z}}\chi(x) e\left(\frac{-mx}{q}\right)
        \hat{h}_{t}\left(\frac{Nm}{q}\right)
    \] by the orthogonality of characters.
    When $t_1,t_2$ are at $T^\epsilon$ separated, we can add in the terms $\hat{h}_{t_1-t_2}(0)$ corresponding to $m=0$, at the cost of $O_\epsilon(T^{-100})$. Therefore, we can consider the sum across all $m\in\mathbb{Z}$ with an error of $O_\epsilon(T^{-100})$. Let $W$ be the Mellin transform of the function $\omega(x)^2$. So that
\begin{align*}
    &\frac{N^{1+it}}{q}\sum_{m\in \mathbb{Z}} \sum_{x \mod q}\chi(x) e\left(\frac{-mx}{q}\right)
    \hat{h}_{t}\left(\frac{Nm}{q}\right) \\
    =& \sum_{n} n^{it}\chi(n)\omega\left(\frac{n}{N}\right)^2\\
    =& \frac{1}{2\pi i}\int_{2-i\infty}^{2+i\infty}W(s)N^s \sum_n n^{-s+it}\chi(n) ds\\
    =& \frac{1}{2\pi i}\int_{2-i\infty}^{2+i\infty}W(s)N^s L(s-it,\chi) ds\\
    =& \frac{1}{2\pi i}\int_{-1-i\infty}^{-1+i\infty}W(s)N^sL(s-it,\chi) ds + \varepsilon(\chi)\frac{\phi(q)}{q}N^{1+it}W(1+it)
\end{align*}
where $\varepsilon$ detects if $\chi$ is principal or not. The second term arising from the (potential) pole at $1$ decays quickly so is $O_{\epsilon}(T^{-100})$ for $t>T^\epsilon$. If $\chi$ is not principal, the contribution is $0$. Therefore, this term is negligible. 
For the first term, we let $\chi$ be induced by the primitive $\chi^*$ with modulus $r$, so\[
    L(s-it,\chi)=L(s-it,\chi^*)\prod_{p|q} \left(1-\frac{\chi^*(p)}{p^s}\right).
\]
We also let \[
    G(s) =\frac{\tau(\chi^*)}{i^\delta\sqrt{r}}r^{1/2-s}\pi^{s-1/2}\frac{\Gamma(\frac{1-s+\delta}{2})}{\Gamma(\frac{s+\delta}{2})},
\]
so that $L(s-it,\chi^*)(s) = G(s-it)L(1-s+it,\overline{\chi^*})$. The integral becomes
\begin{align*}
   &\frac{1}{2\pi i}\int_{-1-i\infty}^{-1+i\infty}W(s)N^sL(s-it,\chi) ds \\=&\frac{1}{2\pi i}
   \int_{-1-i\infty}^{-1+i\infty}W(s)N^s
    G(s-it) L(1-s+it,\overline{\chi^*}) \prod_{p|q} \left(1-\frac{\chi^*(p)}{p^s}\right)ds\\
    =& \frac{1}{2\pi i}\int_{-1-i\infty}^{-1+i\infty}W(s)N^s
    G(s-it) 
    \left(\sum_{n\leq M}\frac{\overline{\chi^*}(n)}{n^{1-s+it}}+
    \sum_{n> M}\frac{\overline{\chi^*}(n)}{n^{1-s+it}}
   \right) \prod_{p|q} \left(1-\frac{\chi^*(p)}{p^s}\right)ds
\end{align*}
Where $M$ is a parameter to be determined. The summation is convergent as the real part is larger than $1$.
We thus break up the integral into two pieces according to the two summations $I_1+I_2$. Moving the line of integration of $I_1$ to $\Re(s)=1$ and $I_2$ to $\Re(s)=-2k$,
\begin{align*}
        I_1&= \frac{1}{2\pi}\int_{-\infty}^{\infty}W(1+iu)N^{1+iu}G(1+iu-it)\sum_{n\leq M}\overline{\chi^*}(n)n^{-i(u-t)}\prod_{p|q} \left(1-\frac{\chi^*(p)}{p^{1+iu}}\right)du,\\
        I_2&= \frac{1}{2\pi}\int_{-\infty}^{\infty}W(-2k+iu)N^{-2k+iu}G(-2k+iu-it)\sum_{n> M}\overline{\chi^*}(n)n^{-2k-1-i(u-t)}\prod_{p|q} \left(1-\frac{\chi^*(p)}{p^{-2k+iu}}\right)du.
\end{align*}
By the decay of $W$, we can truncate both integrals to the region $|u|\ll T^\epsilon$.
Moreover, by \textbf{Davenport multiplicative number theory chp 10}
we have \begin{align*}
    \frac{\Gamma\left(\frac{i(t-u)+\delta}{2}\right)}{\Gamma\left(\frac{1+i(u-t)+\delta}{2}\right)}\ll \frac{1}{t^{1/2}} \frac{1}{}
\end{align*}
\end{proof}
\fi
	\section{$S_3$ bound}\label{s_3chapter}
\begin{proposition}[Preliminary bound on $S_3$]\label{s_3bound}
    We have \[
    S_3 \lesssim \Big(\frac{\phi(q)}{q}\Big)^2 (qT)^2 E(\mathcal{S})^{1/2} |\mathcal{S}|^{1/2}.
    \]
\end{proposition}
Recall that \begin{align*}
	I_m\defeq\frac{N^3}{q^3}\sum_{\substack{(t_1,\chi_1),\\(t_2,\chi_2),\\(t_3,\chi_3)\in\mathcal{S}}} &\sum_{x\in (\mathbb{Z}/q\mathbb{Z})^3}\chi_1\bar{\chi}_3(x_1)\chi_2\bar{\chi}_1(x_2)\chi_3\bar{\chi}_2(x_3) e\left(\frac{-x\cdot m}{q}\right)\\
	\times \ &\hat{h}_{t_1-t_3}\left(\frac{Nm_1}{q}\right)\hat{h}_{t_2-t_1}\left(\frac{Nm_2}{q}\right)\hat{h}_{t_3-t_2}\left(\frac{Nm_3}{q}\right).
\end{align*}

By non-stationary phase, $I_m$ is negligible for the terms $qT/N\lesssim |m|$, so 
\begin{equation}
    S_3 = \sum_{0<|m_1|,|m_2|,|m_3|\lesssim qT/N} I_m + O(T^{-100}).
\end{equation}
Furthermore, by symmetry in $m_1$, $m_2$, $m_3$, we can consider the terms $|m_1|\leq |m_2|\leq |m_3|$ at a cost of a factor of $6$.
We first consider the triple summation in $\mathbb{Z}/q$. Define \begin{align*}
    A_{m,\chi_1,\chi_2,\chi_3}&\defeq \sum_{x\in (\mathbb{Z}/q\mathbb{Z})^3}\chi_1\bar{\chi}_3(x_1)\chi_2\bar{\chi}_1(x_2)\chi_3\bar{\chi}_2(x_3) e\left(\frac{-x\cdot m}{q}\right),
\end{align*}
which is the inner summation in $I_m$.
Notice that $\chi(0)=0$, so we can define the summation in $A$ to run over $(\mathbb{Z}/q\mathbb{Z})^\times$ to get 
\begin{align*}
    A_{m,\chi_1,\chi_2,\chi_3}&= \sum_{x\in ((\mathbb{Z}/q\mathbb{Z})^\times)^3}\chi_1\bar{\chi}_3(x_1)\chi_2\bar{\chi}_1(x_2)\chi_3\bar{\chi}_2(x_3) e\left(\frac{-x\cdot m}{q}\right)\\
\end{align*}
We now make the substitution $y_1=x_1x_3^{-1}, y_2=x_2x_3^{-1} \mod q$ for the summation over $x$.
We thus rewrite the sum over $x$ as 
\begin{align*}
    &\sum_{y_1,y_2,x_3\in (\mathbb{Z}/q\mathbb{Z})^\times}
    \chi_1(y_1y_2^{-1})\chi_2(y_2)\chi_3(y_1^{-1})e\left(\frac{-(y_1m_1+y_2m_2+m_3)x_3}{q}\right)\\
    =&
    \sum_{y_1,y_2\in (\mathbb{Z}/q\mathbb{Z})^\times}\chi_1(y_1)\bar{\chi}_1(y_2)\chi_2(y_2)\bar{\chi}_3(y_1)\sum_{x_3\in (\mathbb{Z}/q\mathbb{Z})^\times}e\left(\frac{-(y_1m_1+y_2m_2+m_3)x_3}{q}\right)
\end{align*}
Now we can apply M\"obius inversion on the summation in $x_3$ to get \begin{align*}
   \sum_{x_3\in (\mathbb{Z}/q\mathbb{Z})^\times}e\left(\frac{-(y_1m_1+y_2m_2+m_3)x_3}{q}\right)=& \sum_{q_0|q} \mu\left(\frac{q}{q_0}\right)\sum_{\tilde{x}_3\in \mathbb{Z}/q_0}e\left(\frac{-(y_1m_1+y_2m_2+m_3)\tilde{x}_3}{q_0}\right) \\ =& \sum_{q_0|q} \mu\left(\frac{q}{q_0}\right) q_0 \mathbb{I}_{q_0|(y_1m_1+y_2m_2+m_3)}.
\end{align*}

With this in mind, we break $I_m=\sum_{q_0|q}\mu(q/q_0)I_{m,q_0}$,\begin{align*}
I_{m,q_0}\defeq&\frac{N^3q_0}{q^2}\sum_{\substack{(t_1,\chi_1),\\(t_2,\chi_2),\\(t_3,\chi_3)\in\mathcal{S}}} \sum_{\substack{y_1,y_2 \in\mathbb{Z}/q\mathbb{Z} \\ y_1m_1+y_2m_2+m_3\equiv 0 \mod q_0}}\chi_1(y_1)\bar{\chi}_1(y_2)\chi_2(y_2)\bar{\chi}_3(y_1)\\
&\times \hat{h}_{t_1-t_3}\left(\frac{Nm_1}{q}\right)\hat{h}_{t_2-t_1}\left(\frac{Nm_2}{q}\right)\hat{h}_{t_3-t_2}\left(\frac{Nm_3}{q}\right).
\end{align*}
\begin{proposition} [Divisor reduction of $S_3$] \label{divisorreduction}We have\[
    S_3 \lesssim \sup_{q_0|q}\quad \sum_{\substack{0<|m_1|,|m_2|,\\|m_3|\lesssim qT/N}} I_{m,q_0}.
    \]
\end{proposition}
\begin{proof}
    We split the sum in $I_m$ to be across $I_{m,q_0}$ for each divisor of $q$. The divisor function grows slower than any power of $q$, so gives the proposition.
\end{proof}
We then consider the subproduct in $I_{m,q_0}$: \[
	\hat{h}_{t_1-t_3}\left(\frac{Nm_1}{q}\right)\hat{h}_{t_2-t_1}\left(\frac{Nm_2}{q}\right)\hat{h}_{t_3-t_2}\left(\frac{Nm_3}{q}\right).
\]
Expanding the Fourier transform as an integral, this expression equals \begin{align*}
	 &\int_{\reals^3}\mathbf{\tilde{\omega}}(\mathbf{u})u_1^{i(t_1-t_3)}u_2^{i(t_2-t_1)}u_3^{i(t_3-t_2)}e\left(\frac{-N\mathbf{m}\cdot \mathbf{u}}{q}\right)d\mathbf{u}\\
	 =&\int_{\reals^3}\mathbf{\tilde{\omega}}(\mathbf{u})\left(\frac{u_1}{u_2}\right)^{it_1}\left(\frac{u_2}{u_3}\right)^{it_2}\left(\frac{u_3}{u_1}\right)^{it_3}
	 e\left(\frac{-N\mathbf{m}\cdot \mathbf{u}}{q}\right)d\mathbf{u}
	 \end{align*}
	 where $\tilde{\omega}(u_1,u_2,u_3)\defeq \omega(u_1)^2\omega(u_2)^2\omega(u_3)^2$ is compactly supported. The observation is that the choice of $u_1/u_2$ and $u_2/u_3$ fixes $u_3/u_1$, so this triple integral can be rewritten in two variables. We change variables $v_1=u_1/u_3,v_2=u_2/u_3$ for the integral, which is well defined on the support of $\tilde{\omega}$. This gives us a Jacobian of $u_3^2$ and equals
	 \begin{equation}\label{im1integral}
        \begin{split}
	&\int_{\reals^3}\tilde{\omega}(v_1u_3,v_2u_3,u_3) {\left(\frac{v_1}{v_2}\right)}^{it_1} {\left(v_2\right)}^{it_2}{\left(\frac{1}{v_1}\right)}^{it_3} u_3^2 \ e\left(\frac{-N(v_1m_1+v_2m_2+m_3)u_3}{q}\right)\ dv_1\ dv_2\ du_3\\
	=&\int_{\reals^2}\int_\reals u_3^2 \ \tilde{\omega}(v_1u_3,v_2u_3,u_3) e\left(\frac{-N(v_1m_1+v_2m_2+m_3)u_3}{q}\right)  du_3 \ {\left(\frac{v_1}{v_2}\right)}^{it_1} {\left(v_2\right)}^{it_2}{\left(\frac{1}{v_1}\right)}^{it_3}  \ dv_1\ dv_2.\\
     \end{split}
\end{equation}
We define \begin{align*}
	R(v,n_1,n_2)&\defeq \sum_{(t,\chi)\in \mathcal{S}} 
	\chi({n_1})\bar{\chi}(n_2)v^{it},\\
	R(v,n)&\defeq R(v,n,1).
\end{align*}
Therefore, \begin{equation*}
    I_{m,q_0} = \frac{N^3q_0}{q^3} \sum_{\substack{y_1,y_2 \in\mathbb{Z}/q\mathbb{Z} \\ y_1m_1+y_2m_2+m_3\equiv 0 \mod q_0}}\int_{\reals^2} \tilde{I}_{u_3}(v_1,v_2)  R{\left(\frac{v_1}{v_2},y_1,y_2\right)}R {\left(v_2,y_2\right)}R{\left(\frac{1}{v_1},y_1^{-1}\right)} \ dv_1\ dv_2,
\end{equation*}
    where\[
     \tilde{I}_{u_3}(v_1,v_2)\defeq \tilde{I}_{u_3}(v_1,v_2,m)\defeq \int_\reals u_3^2 \ \tilde{\omega}(v_1u_3,v_2u_3,u_3) e\left(\frac{-N(v_1m_1+v_2m_2+m_3)u_3}{q}\right)  du_3.\]
The innermost integral $\tilde{I}_{u_3}(v_1,v_2)$ has cancellation property. By the principle of non-stationary phase through repeated integration by parts, this integral is $O_{\epsilon, A}(T^{-A})$ for any $|v_1m_1+v_2m_2+m_3|>qT^\epsilon/N$. Therefore, we can truncate the domain of the integrals in $v_1$ and $v_2$ to  $|v_1m_1+v_2m_2+m_3|\lesssim q/N$ with negligible error. On this domain, the innermost integral in $u_3$ is $O(1)$ by the trivial bound. Moreover, by the compact support of $\tilde{\omega}$ on $[1,2]\times [1,2]\times [1,2]$  the integrand of innermost integral is non-zero only if \[
v_1u_3,v_2u_3,u_3\sim N.
\]
Importantly, this requires $1/2 \leq v_1,v_2 \leq 2$, so we can further restrict the outermost integrals to this region. 
We therefore have: 
\begin{proposition}[Simplification of $I_{m,q_0}$ domain] \label{domainreduction}
    \[
    |I_{m,q_0}|\ll \frac{N^3q_0}{q^3}\sum_{\substack{y_1,y_2 \in\mathbb{Z}/q\mathbb{Z} \\ y_1m_1+y_2m_2+m_3\equiv 0 \mod q_0}}\int\displaylimits_{\substack{
        |v_1m_1+v_2m_2+m_3|\lesssim \frac{q}{N}\\
        \frac{1}{2}\leq v_1,v_2\leq 2
}}  R{\left(\frac{v_1}{v_2},y_1,y_2\right)}R {\left(v_2,y_2\right)}R{\left(\frac{1}{v_1},y_1^{-1}\right)} \ dv_1\ dv_2+O_{\epsilon}(T^{-100}).
    \]
    Moreover, if $|m_1|\leq|m_2|\leq |m_3|$, $|I_{m,q_0}|=O(T^{-100})$ unless $|m_2|\asymp|m_3|$.
\end{proposition}
\begin{proof}
    The first statement is a result of our work so far. The second part of the proposition follows from the integral bounds $|v_1m_1+v_2m_2+m_3|\lesssim q/N$
    and $v_1,v_2\asymp 1$. When $|m_1|\leq |m_2|$, we have $|v_1m_1+v_2m_2|=O(|m_2|)$. These force $|m_2| \asymp|m_3|$, or else the integral will be negligible as the domain of the integration will be outside the range $|v_1m_1+v_2m_2+m_3|\lesssim q/N$.
\end{proof}
When $|m_2|\asymp|m_3|$, the domain of integration can be written as\begin{align*}
    |v_1m_1+v_2m_2+m_3|\lesssim \frac{q}{N} \implies \left|v_2 - \frac{v_1m_1+m_3}{-m_2}\right|\lesssim \frac{q}{|m_2|N} \asymp \frac{q}{|m_3|N}.
\end{align*}
Thus, if we fix $v_1$, the integration in $v_2$ is in a ${q}/{(|m_3|N)}$-small neighborhood of $\frac{v_1m_1+m_3}{-m_2}$. In principle, we can estimate the value of the integral by evalulating at 
$v_2 = {v_1m_1+m_3}/{-m_2}$ and multiplying it by $q/(|m_3|N)$. This is made precise by splitting the sum across $m_3$ into dyadic interals $M$, and smoothing over ranges of ${q}/{(MN)}$.

Let $\tilde{\phi}$ be a smooth bump function such that equals $\tilde{\phi}=1$ on $|x|\lesssim 1$ and is supported in $|x|\lesssim 1$ (with a larger constant), so that $\|\tilde{\phi}^{(j)}\lesssim_j 1\|$ for all $j$.
We define \[
\tilde{R}_M(v,y_1,y_2) \defeq  \left( \int \frac{NM}{q}\tilde{\phi}\left(\frac{NM}{q}(v-v')\right)|R(v',y_1,y_2)|^2 dv'\right)^{1/2}.
\]
\begin{proposition} \label{dyadics_3}
    Let \[
    \tilde{I}_{m,y_1,y_2,q_0}\defeq \tilde{I}_{m,q_0}\defeq \ \int_{v_1\asymp 1} \left| \tilde{R}_M\left(\frac{m_1v_1+m_3}{-m_2v_1},y_2,y_1\right)
    \tilde{R}_M\left(\frac{m_1v_1+m_3}{-m_2},y_2\right)R\left(v_1,y_1\right)\right| dv_1.
    \]
    There is a choice of $q_0|q$ and $0<M_1\leq M \lesssim qT/N$ such that \[
        S_3\lesssim \frac{N^2q_0}{Mq^2}\sum_{|m_1|\sim M_1,|m_2|,|m_3|\sim M}\sum_{\substack{y_1,y_2 \in\mathbb{Z}/q\mathbb{Z} \\ y_1m_1+y_2m_2+m_3\equiv 0 \mod q_0}}\tilde{I}_{m,q_0}+O(T^{-100}).
    \]
\end{proposition}


\begin{proof}
By Proposition \ref{divisorreduction}, we consider the terms $|m_1|\leq |m_2|\leq|m_3|$ at the cost of a factor of $6$. By Proposition \ref{domainreduction}, we add the condition that $|m_2|\asymp|m_3|$. Expanding the sum over $m_1,m_2,m_3$ dyadically, we get for some $M_1\leq M \lesssim qT/N $ and for the value of $q_0$ that achieves supremum. \begin{align*}
    S_3\lesssim \sum_{|m_1|\sim M_1,|m_2|,|m_3|\sim M}|I_{m,q_0}| + O(T^{-100}).
\end{align*}

We now consider
\begin{align*} 
    & \int\displaylimits_{\substack{
        |v_1m_1+v_2m_2+m_3|\lesssim \frac{q}{N}\\
        \frac{1}{2}\leq v_1,v_2\leq 2
    }} \left| R\left(\frac{v_2}{v_1},y_2,y_1\right)
    R(v_2,y_2)R\left(\frac{1}{v_1},y_1^{-1}\right)\right| dv_1 \ dv_2\\
    = 
   & \int\displaylimits_{\substack{
        |v_1m_1+v_2m_2+m_3|\lesssim \frac{q}{N}\\
        \frac{1}{2}\leq v_1,v_2\leq 2
    }} \left| R\left(\frac{v_2}{v_1},y_2,y_1\right)
    R(v_2,y_2)R\left(v_1,y_1\right)\right| dv_1 \ dv_2\\
     \ll& 
    \int_{v_1\asymp 1} |R\left(v_1,y_1\right)|
    \int\displaylimits_{\left|v_2 - \frac{v_1m_1+m_3}{-m_2}\right|\lesssim \frac{q}{|m_2|N}} \left| R\left(\frac{v_2}{v_1},y_2,y_1\right)
    R(v_2,y_2)\right|  dv_2 \ dv_1\\
    \ll& \int_{v_1\asymp 1} |R\left(v_1,y_1\right)|
    \int\displaylimits_{\left|v_2 - \frac{v_1m_1+m_3}{-m_2}\right|\lesssim \frac{q}{MN}} \left| R\left(\frac{v_2}{v_1},y_2,y_1\right)
    R(v_2,y_2)\right|  dv_2 \ dv_1\\
\end{align*}
when $|m_2|\asymp M$. The inner integral, by Cauchy-Schwarz,
is \begin{align*}
    \leq & \left(\int\displaylimits_{\left|v_2 - \frac{v_1m_1+m_3}{-m_2}\right|\lesssim \frac{q}{MN}} \left| R\left(\frac{v_2}{v_1},y_2,y_1\right)\right|^2 \ dv_2 \right)^{1/2}
    \left(
    \int\displaylimits_{\left|v_2 - \frac{v_1m_1+m_3}{-m_2}\right|\lesssim \frac{q}{MN}} \left|
    R(v_2,y_2)\right|^2  dv_2\right)^{1/2}\\
    \ll& \frac{q}{MN} \tilde{R}_M\left(\frac{v_1m_1+m_3}{-m_2v_1},y_2,y_1\right)  \tilde{R}_M(\frac{v_1m_1+m_3}{-m_2},y_2)
\end{align*}
where in the last step, we used $v_1\asymp 1$.
Thus, for $|m_2|\sim M$,
\[
|I_{m,q_0}|\lesssim \frac{N^3q_0}{q^3} \frac{q}{NM}\sum_{\substack{y_1,y_2 \in\mathbb{Z}/q\mathbb{Z} \\ y_1m_1+y_2m_2+m_3\equiv 0 \mod q_0}} \tilde{I}_{m,q_0}.
\]
The proposition follows from this claim.
\end{proof}
To apply H\"older's inequality, we need to find bounds on the second and fourth moments of $R(v,y_1,y_2)$.
\begin{lemma} [Second and fourth moments of $R$]\label{secondmoment}
    Let $\mathcal{S}=\{(t_j,\chi_j)\}$, such that $\chi_j$ is a character mod $q$, and the $t$'s are contained in an interval of length $T$, and are $T^\epsilon$-separated for the same character. Then uniformly in $y_2\in \mathbb{Z}/q\mathbb{Z}$, \[
        \sum_{y_1\in (\mathbb{Z}/q\mathbb{Z})} \int_{v\asymp 1} 
        \left|R\left(v,y_1,y_2\right)\right|^2dv \ll_{\epsilon} \phi(q)|\mathcal{S}|,
    \]
     and \[
        \sum_{y_1\in (\mathbb{Z}/q\mathbb{Z})} \int_{v\asymp 1} 
        \left|R\left(v,y_1,y_2\right)\right|^4dv  \lesssim \phi(q)E(\mathcal{S}).
    \]
\end{lemma}
\begin{proof}
    For the second moment, we have \[
    |R(v,y_1,y_2)|^2 = \sum_{(t_1,\chi_1),(t_2,\chi_2)\in \mathcal{S}}
    \chi_1\bar{\chi}_2(y_1)\bar{\chi}_1{\chi}_2(y_2)v^{i(t_1-t_2)}.
    \]
    By the orthogonality of characters,\[
        \sum_{y_1\in (\mathbb{Z}/q\mathbb{Z})}|R(v,y_1,y_2)|^2 = \phi(q) \sum_{(t_1,\chi_1),(t_2,\chi_2)\in \mathcal{S}} \delta_{\chi_1\chi_2}\bar{\chi}_1{\chi}_2(y_2)v^{i(t_1-t_2)} = \phi(q)\sum_{(t_1,\chi_1),(t_2,\chi_2)\in \mathcal{S}} \delta_{\chi_1\chi_2}\mathbb{I}_{(y_2,q)=1} v^{i(t_1-t_2)},
    \]
    so it is enough to consider the second moment of $R(v,y,1)$ (and $y_2$ is coprime to $q$).
    Let $\psi$ be a bump function supported on $v\asymp 1$ and equals $1$ on the domain of integration in the lemma.
   Then, \begin{align*}
        \sum_{y\in (\mathbb{Z}/q\mathbb{Z})^\times} \int_{v\asymp 1} 
        \left|R\left(v,y\right)\right|^2dv 
        \leq&\sum_{y\in (\mathbb{Z}/q\mathbb{Z})^\times} \int 
        \psi(v)\left|R\left(v,y\right)\right|^2dv 
        \\=&
        \phi(q)\int \psi(v)
        \sum_{\substack{(t_1,\chi_1),(t_2,\chi_2)\in \mathcal{S}\\ \chi_1=\chi_2}}v^{i(t_1-t_2)}
        dv\\
        =&
        \phi(q)\sum_{\substack{(t_1,\chi_1),(t_2,\chi_2)\in \mathcal{S}\\ \chi_1=\chi_2}}\int \psi(v)
        v^{i(t_1-t_2)}
        dv.
    \end{align*}
    In the sum, the terms $t_1=t_2$ contribute $O(|\mathcal{S}|)$. If $t_1\neq t_2$, then $|t_1-t_2|\geq T^\epsilon$. The integral in this case is $O_\epsilon(T^{-100})$ and is negligible.

    Similarly for the fourth moment, it is enough to consider $R(v,y)$. We have \[
    |R(v,y)|^4 = \sum_{\substack{(t_1,\chi_1),(t_2,\chi_2),\\ (t_3,\chi_3),(t_4,\chi_4)\in \mathcal{S}}}
    \chi_1{\chi}_2\bar{\chi_3}\bar{\chi_4}(y)v^{i(t_1+t_2-t_3-t_4)}.
    \]
    So again by the orthogonality of characters, \begin{align*}
        \sum_{y\in (\mathbb{Z}/q\mathbb{Z})^\times} \int_{v\asymp 1} 
        \left|R\left(v,y\right)\right|^4dv = & \phi(q)
        \sum_{\substack{(t_1,\chi_1),(t_2,\chi_2),\\ (t_3,\chi_3),(t_4,\chi_4)\in \mathcal{S}\\ \chi_1\chi_2=\chi_3\chi_4}} \int_{v\asymp 1} v^{i(t_1+t_2-t_3-t_4)} dv.
    \end{align*}
    Similar to the previous proof, we can introduce the bump function $\psi$ for the integral, and restrict the summation to the terms $|t_1+t_2-t_3-t_4|\leq T^\epsilon$ with an error of $O_\epsilon(T^{-100})$. The remaining terms in the summation contribute $O(E(\mathcal{S}))$.
\end{proof}
\begin{lemma}\label{fourthmoment_smooth}
    Let $E(\mathcal{S})=\#\{(t_1,\chi_1),(t_2,\chi_2),(t_3,\chi_3),(t_4,\chi_4)\in \mathcal{S}  :  |t_1+t_2-t_3-t_4|\leq 1, \chi_1\chi_2=\chi_3\chi_4\}$. Then \[
        \sum_{y_1\in (\mathbb{Z}/q\mathbb{Z})^\times} \int_{v\asymp 1} 
        \left|\tilde{R}_M\left(v,y_1,y_2\right)\right|^4dv  \lesssim \phi(q)E(\mathcal{S}).
    \]
\end{lemma}
\begin{proof}
    It is enough to show the case where $y_2=1$.
    We apply Cauchy-Schwarz to \begin{align*}
        \int_{v\asymp 1} 
        \left|\tilde{R}_M\left(v,y\right)\right|^4dv  \lesssim& \int_{v\asymp 1} 
        \left(\int_{|u-v|\lesssim q/NM}
        \frac{NM}{q}|R(u,y)|^2 du\right)^2
        dv \\
        \stackon{CS}{\lesssim}& \frac{NM}{q} \int_{v\asymp 1} 
       \int_{|u-v|\lesssim q/NM}
        |R(u,y)|^4 du \ 
        dv\\
        \lesssim&  
        \int_{u\asymp 1}
         |R(u,y)|^4 du.
    \end{align*}
    Lemma \ref{secondmoment} completes the proof.
\end{proof}

\begin{proof}[{Proof of Proposition \ref{s_3bound}}]
    We exchange the summation to get \begin{align*}
    &\sum_{\substack{|m_1|\sim M_1,|m_2|,|m_3|\sim M}}\sum_{\substack{y_1,y_2 \in\mathbb{Z}/q\mathbb{Z} \\ y_1m_1+y_2m_2+m_3\equiv 0 \mod q_0}}\tilde{I}_{m,q_0}\\
    =&\sum_{\substack{y_1,y_2 \in\mathbb{Z}/q\mathbb{Z} }}\sum_{\substack{|m_1|\sim M_1,|m_2|,|m_3|\sim M\\ y_1m_1+y_2m_2+m_3\equiv 0 \mod q_0}}\tilde{I}_{m,q_0}\\
    =&\sum_{\substack{y_1,y_2 \in\mathbb{Z}/q\mathbb{Z} }}\int_{v_1\asymp 1} \left|R\left(v_1,y_1\right)\right|
    \sum_{\substack{|m_1|\sim M_1,|m_2|,|m_3|\sim M\\ y_1m_1+y_2m_2+m_3\equiv 0 \mod q_0}}\left| \tilde{R}_M\left(\frac{m_1v_1+m_3}{-m_2v_1},y_2,y_1\right)
    \tilde{R}_M\left(\frac{m_1v_1+m_3}{-m_2},y_2\right)\right| dv_1\\
    \stackon{CS}{\leq} & \Big(\sum_{\substack{y_1,y_2 \in\mathbb{Z}/q\mathbb{Z} }}\int_{v_1\asymp 1} \left|R\left(v_1,y_1\right)\right|dv_1\Big)^{1/2} 
    \\ & \quad \quad\Bigg(\sum_{\substack{y_1,y_2 \in\mathbb{Z}/q\mathbb{Z} }}\int_{v_1\asymp 1} \Big(\sum_{\substack{|m_1|\sim M_1,|m_2|,|m_3|\sim M\\ y_1m_1+y_2m_2+m_3\equiv 0 \mod q_0}}\left|\tilde{R}_M\left(\frac{m_1v_1+m_3}{-m_2v_1},y_2,y_1\right)
    \tilde{R}_M\left(\frac{m_1v_1+m_3}{-m_2},y_2\right)\right|\Big)^2 dv_1\Bigg)^{1/2}.
    \end{align*}
    The second moment of $R$ (and only considering non-zero contributions from $\gcd(q,y_1)=\gcd(q,y_2)=1$) gives \begin{align*}
        \Big(\sum_{\substack{y_1,y_2 \in\mathbb{Z}/q\mathbb{Z} }}\int_{v_1\asymp 1} \left|R\left(v_1,y_1\right)\right|dv_1\Big)^{1/2}  \lesssim (phi(q)^2 |\mathcal{S}|)^{1/2} \ll \phi(q) |\mathcal{S}|^{1/2}. 
    \end{align*}
    We apply repeated Cauchy-Schwarz in the second term to get \begin{align*}
     &   \Bigg(\sum_{\substack{y_1,y_2 \in\mathbb{Z}/q\mathbb{Z} }}\int_{v_1\asymp 1} \Big(\sum_{\substack{|m_1|\sim M_1,|m_2|,|m_3|\sim M\\ y_1m_1+y_2m_2+m_3\equiv 0 \mod q_0}}\Big|\tilde{R}_M\left(\frac{m_1v_1+m_3}{-m_2v_1},y_2,y_1\right)
    \tilde{R}_M\left(\frac{m_1v_1+m_3}{-m_2},y_2\right)\Big| dv_1\Big)^{2}\Bigg)^{1/2}\\
    \stackon{CS}{\leq}&\Bigg(\sum_{\substack{y_1,y_2 \in\mathbb{Z}/q\mathbb{Z} }}\int_{v_1\asymp 1}\Big[\sum_{\substack{|m_1|\sim M_1,|m_2|,|m_3|\sim M\\ y_1m_1+y_2m_2+m_3\equiv 0 \mod q_0}}\Big|\tilde{R}_M\left(\frac{m_1v_1+m_3}{-m_2v_1},y_2,y_1\right)\Big|^2\Big]\\ & \quad \quad
   \Big[ \sum_{\substack{|m_1|\sim M_1,|m_2|,|m_3|\sim M\\ y_1m_1+y_2m_2+m_3\equiv 0 \mod q_0}}
    \Big|\tilde{R}_M\left(\frac{m_1v_1+m_3}{-m_2},y_2\right)\Big|^2\Big] dv_1\Bigg)^{1/2}\\
    \stackon{CS}{\leq}&\Bigg(\sum_{\substack{y_1,y_2 \in\mathbb{Z}/q\mathbb{Z} }}\int_{v_1\asymp 1}\Big[\sum_{\substack{|m_1|\sim M_1,|m_2|,|m_3|\sim M\\ y_1m_1+y_2m_2+m_3\equiv 0 \mod q_0}}\Big|\tilde{R}_M\left(\frac{m_1v_1+m_3}{-m_2v_1},y_2,y_1\right)\Big|^2\Big]^2dv_1\Bigg)^{1/4}\\ & \quad \quad
    \Bigg(\sum_{\substack{y_1,y_2 \in\mathbb{Z}/q\mathbb{Z} }}\int_{v_1\asymp 1}\Big[ \sum_{\substack{|m_1|\sim M_1,|m_2|,|m_3|\sim M\\ y_1m_1+y_2m_2+m_3\equiv 0 \mod q_0}}
     \Big|\tilde{R}_M\left(\frac{m_1v_1+m_3}{-m_2},y_2\right)\Big|^2\Big]^2 dv_1\Bigg)^{1/4}\\
    \end{align*}
    We focus on one term, the computation of the second term is similar. \begin{align*}
        &\sum_{\substack{y_1,y_2 \in\mathbb{Z}/q\mathbb{Z} }}\int_{v_1\asymp 1}\Big[\sum_{\substack{|m_1|\sim M_1,|m_2|,|m_3|\sim M\\ y_1m_1+y_2m_2+m_3\equiv 0 \mod q_0}}\Big|\tilde{R}_M\left(\frac{m_1v_1+m_3}{-m_2v_1},y_2,y_1\right)\Big|^2\Big]^2dv_1\\
        \stackon{CS}{\leq}&\frac{M_1M^2}{q_0}\sum_{\substack{y_1,y_2 \in\mathbb{Z}/q\mathbb{Z} }}\int_{v_1\asymp 1}\sum_{\substack{|m_1|\sim M_1,|m_2|,|m_3|\sim M\\ y_1m_1+y_2m_2+m_3\equiv 0 \mod q_0}}\Big|\tilde{R}_M\left(\frac{m_1v_1+m_3}{-m_2v_1},y_2,y_1\right)\Big|^4dv_1
    \end{align*}
    For any choice $m_1,m_2,m_3$, we can make the change of variables $u=(m_1v_1+m_3/(-m_2v_1))$ with a Jacbian factor of $O(1)$, and the domain of integration is still $u\asymp 1$, so \begin{align*}
        &\sum_{\substack{y_1,y_2 \in\mathbb{Z}/q\mathbb{Z} }}\int_{v_1\asymp 1}\sum_{\substack{|m_1|\sim M_1,|m_2|,|m_3|\sim M\\ y_1m_1+y_2m_2+m_3\equiv 0 \mod q_0}}\Big|\tilde{R}_M\left(\frac{m_1v_1+m_3}{-m_2v_1},y_2,y_1\right)\Big|^4dv_1\\
        \ll &\frac{M_1M^2}{q_0} sum_{\substack{y_1,y_2 \in\mathbb{Z}/q\mathbb{Z} }}\int_{u\asymp 1}\Big|\tilde{R}_M\left(u,y_2,y_1\right)\Big|^4du\\
        \lesssim& \frac{M_1M^2}{q_0} \phi(q)^2 E(\mathcal{S}).
    \end{align*}
    Similarly, \begin{align*}
        \sum_{\substack{y_1,y_2 \in\mathbb{Z}/q\mathbb{Z} }}\int_{v_1\asymp 1}\Big[ \sum_{\substack{|m_1|\sim M_1,|m_2|,|m_3|\sim M\\ y_1m_1+y_2m_2+m_3\equiv 0 \mod q_0}}
     \Big|\tilde{R}_M\left(\frac{m_1v_1+m_3}{-m_2},y_2\right)\Big|^2\Big]^2 dv_1\lesssim& \frac{M_1^2M^4}{q_0^2} \phi(q)^2 E(\mathcal{S}).\end{align*}

     Therefore, we have \[
     S_3\lesssim \frac{N^2q_0}{Mq^2} \phi(q) |\mathcal{S}|^{1/2}\Big(\frac{M_1^2M^4}{q_0^2} \phi(q)^2 E(\mathcal{S})\Big)^{1/2} \ll 
     \frac{N^2q_0}{Mq^2} \phi(q)^2 |\mathcal{S}|^{1/2} \frac{M_1M^2}{q_0}E(\mathcal{S})^{1/2} \lesssim\Big(\frac{\phi(q)}{q}\Big)^2(qT)^2|\mathcal{S}|^{1/2}E(\mathcal{S})^{1/2},
     \]
    where we applied $M_1\leq M\lesssim qT/N$ in the last step.
\end{proof}
	\section{Refinement of $S_3$ bound}

\begin{proposition}[Refinement of $S_3$]
    \label{refinements_3}
    We have \[
    S_3\lesssim \phi(q)^{7/2}T^2|\mathcal{S}|^{3/2}+\phi(q)^{3}\frac{NT}{q}|\mathcal{S}|^{1/2}E(\mathcal{S})^{1/2}.
     \]
\end{proposition}
The refinement of the previous bound relies on the result for summation over affine transformation by Guth and Maynard.
\begin{lemma} \label{affineGM}
    Let $M>0$. Let $f(u)\geq 0$, supported on $u\asymp 1$, and $|\hat{f}(\xi)|\lesssim_j (|\xi|/T)^j$ for all $j$. Then \[
        \sup_{0<M_1,M_2,M_3<M} \int\Bigg( \sum_{\substack{|m_1|\sim M_1\\|m_2|\sim M_2 \\ |m_3|\ll M_3}} f\left(\frac{m_1 u+m_3}{m_2}\right)\Bigg)^2 \ du \lesssim M^6 \|f\|_{L_1}^2+M^4\|f\|_{L_2}^2.
    \] 
\end{lemma}
This is Proposition 9.1 from [GM].
For the sum in $\tilde{I}_{m,1}$, we can directly apply this lemma, exchange the summation in $\sum_\chi$ and $\sum_m$ to get \begin{align*}
    &\sum_{\substack {|m_1|\sim M_1,|m_2|,|m_3|\sim M}}\tilde{I}_{m,1}\\
   &  =\sum_{\chi}\int\displaylimits_{v_1\asymp 1}  \left|R_\chi\left(v_1\right)\right|\sum_{\substack {|m_1|\sim M_1,\\|m_2|,|m_3|\sim M}}\left| \tilde{R}_{\chi,M}\left(\frac{m_1v_1+m_3}{-m_2v_1}\right)
    \tilde{R}_{\chi,M}\left(\frac{m_1v_1+m_3}{-m_2}\right)\right|dv_1\\
    &\stackon{CS}{\leq} \Big( \sum_{\chi}\int\displaylimits_{v_1\asymp 1}  \left|R_\chi\left(v_1\right)\right|^2 dv_1 \Bigg)^{1/2} \Bigg(\sum_{\chi}\int\displaylimits_{v_1\asymp 1}\Bigg(\sum_{\substack {|m_1|\sim M_1,\\|m_2|,|m_3|\sim M}}\left| \tilde{R}_{\chi,M}\left(\frac{m_1v_1+m_3}{-m_2v_1}\right)
    \tilde{R}_{\chi,M}\left(\frac{m_1v_1+m_3}{-m_2}\right)\right|\Bigg)^2dv_1\Bigg)^{1/2}\\
    &\stackon{CS}{\lesssim} |\mathcal{S}|^{1/2} \Bigg(\sum_{\chi}\int\displaylimits_{v_1\asymp 1}\Bigg(\sum_{\substack {|m_1|\sim M_1,\\|m_2|,|m_3|\sim M}}\left| \tilde{R}_{\chi,M}\left(\frac{m_1v_1+m_3}{-m_2v_1}\right)\right|^2 \Bigg)\Bigg(\sum_{\substack {|m_1|\sim M_1,\\|m_2|,|m_3|\sim M}}\left|
    \tilde{R}_{\chi,M}\left(\frac{m_1v_1+m_3}{-m_2}\right)\right|^2\Bigg)dv_1\Bigg)^{1/2}\\
    &\stackon{CS}{\leq}|\mathcal{S}|^{1/2} \Bigg(\sum_{\chi}
    \int\displaylimits_{v_1\asymp 1}\Bigg(\sum_{\substack {|m_1|\sim M_1,\\|m_2|,|m_3|\sim M}}\left| \tilde{R}_{\chi,M}\left(\frac{m_1v_1+m_3}{-m_2v_1}\right)\right|^2\Bigg)^2 dv_1 \Bigg)^{1/4}\Bigg(\sum_{\chi}
    \int\displaylimits_{v_1\asymp 1}\Big(
    \sum_{\substack {|m_1|\sim M_1,\\|m_2|,|m_3|\sim M}}\left|
    \tilde{R}_{\chi,M}\left(\frac{m_1v_1+m_3}{-m_2}\right)\right|^2\Bigg)^2dv_1\Bigg)^{1/4}.
\end{align*} 
We can bound these terms using Proposition \ref{affineGM}. Note that $R_{\chi,M}^2$ and $R_M^2$ have rapid Fourier decay as the convolution of a smooth bump function and $R$, therefore meets the condition in the Proposition.
For the second term, we can make the change of variables $u=1/v_1$ with a Jacobian of $\asymp 1$, so that \[
\frac{m_1v_1+m_3}{-m_2v_1}= \frac{m_3u + m_1}{-m_2}.
\]
We also have \begin{align*}
    \sum_{\chi} \|R_\chi^2\|_{L_1}^2\stackon{CS}{\leq}(\sum_{\chi} \|R_\chi^2\|_{L_1})^2 \lesssim |\mathcal{S}|^2,
\end{align*}
and\begin{align*}
    \sum_{\chi} \|R_\chi^2\|_{L_2}^2=\sum_{\chi} \|R_\chi^4\|_{L_1} \lesssim E(\mathcal{S})^2.
\end{align*}
Combined, this gives a bound of \[
    \frac{N^2}{M}\sum_{\substack {|m_1|\sim M_1,|m_2|,|m_3|\sim M}}\tilde{I}_{m,1}\lesssim \frac{N^2}{M} |\mathcal{S}|^{1/2} (M^6|\mathcal{S}|^{2}+M^4E(\mathcal{S}))^{1/2}\ll N^2 M^2|\mathcal{S}|^{3/2} + N^2M |\mathcal{S}|^{1/2}E(\mathcal{S})^{3/2},
\]
which is $\lesssim (qT)^{2}|\mathcal{S}|^{3/2}+N(qT)|\mathcal{S}|^{1/2}E(\mathcal{S})^{3/2}.$


For $I_{m,2}$, we can exchange the summations in $y_1,y_2$ and in $m$ to get the sum \begin{align*}
    &\sum_{\substack {|m_1|\sim M_1,\\|m_2|,|m_3|\sim M}}\tilde{I}_{m,2}\\=&\frac{N^3}{q^2} \sum_{y_1,y_2 \in(\mathbb{Z}/q\mathbb{Z})^\times} \sum_{\substack {|m_1|\sim M_1,\\|m_2|,|m_3|\sim M\\ y_1m_1+y_2m_2+m_3\equiv 0\\ \mod q}}\int_{v_1\asymp 1} \left| \tilde{R}_M\left(\frac{m_1v_1+m_3}{-m_2v_1},y_2,y_1\right)
    \tilde{R}_M(\frac{m_1v_1+m_3}{-m_2},y_2)R\left(v_1,y_1\right)\right| dv_1.
\end{align*}
In principle, the condition $y_1m_1+y_2m_2+m_3\equiv 0$ fixes the residue class of $m_3$ based on $m_1$ and $m_2$. Therefore, we would expect a factor of $1/q^2$ if we add this summation condition to Proposition \ref{affineGM}. 
\begin{lemma}\label{affinetrans}
    Let $M>0$. Let $f(u)\geq 0$, supported on $u\asymp 1$, and $|\hat{f}(\xi)|\lesssim_j (|\xi|/T)^j$ for all $j$. Then \[
        J(f)\defeq \sup_{0<M_1,M_2,M_3<M} \int\Bigg( \sum_{\substack{|m_1|\sim M_1,|m_2|\sim M_2 , |m_3|\ll M_3\\ y_1m_1+y_2m_2+m_3\equiv 0 \mod q}} f\left(\frac{m_1 u+m_3}{m_2}\right)\Bigg)^2 \ du \lesssim M^6 \|f\|_{L_1}^2/q^3+M^4\|f\|_{L_2}^2/q^2.
    \] 
\end{lemma}
\textcolor{red}{Testing proof here...}

Let $\psi$ be a smooth bump on $|x|\lesssim 1$, such that $\psi(m_3/M_3)=1$ where $m_3\lesssim M_3$ in the summation conditiion. Furthermore, let \[
    \bar{y}\defeq \bar{y}(m_1,m_2,y_1,y_2)\defeq -m_1y_1-m_2y_2,
\]so that the condition $y_1m_1+y_2m_2+m_3\equiv 0 \mod q$ becomes $m_3=kq+\bar{y}$ for integer $k$.
For the choice of $M_1,M_2,M_3$ that the supremum in the lemma is achieved, we define \[
    g(u)\defeq\sum_{m_1\sim M_1,m_2\sim M_2} \sum_{k\in \mathbb{Z}} \psi\left(\frac{kq+\bar{y}}{M_3}\right)f\left(\frac{m_1 u+kq+\bar{y}}{m_2}\right),
\]so that we have \[
J(f)\leq \int |g(u)|^2 du \quad \stackeq{Plancheral} \quad  \int |\hat{g}(\xi)|^2 d\xi.
\]
We now compute, using the change of variables $u'=u+(kq+y)/m_1$, \begin{align*}
    \hat{g}(\xi)=&\int \sum_{m_1\sim M_1,m_2\sim M_2} \sum_{k\in \mathbb{Z}} \psi\left(\frac{kq+\bar{y}}{M_3}\right)f\left(\frac{m_1 u+kq+\bar{y}}{m_2}\right) e(-u\xi) du\\
    =& \sum_{m_1\sim M_1,m_2\sim M_2}\int \sum_{k\in \mathbb{Z}} \psi\left(\frac{kq+\bar{y}}{M_3}\right)f\left(\frac{m_1 u+kq+\bar{y}}{m_2}\right) e(-u\xi) du\\
    =& \sum_{m_1\sim M_1,m_2\sim M_2}\int \sum_{k\in \mathbb{Z}} \psi\left(\frac{kq+\bar{y}}{M_3}\right)f\left(\frac{m_1 u'}{m_2}\right) e(-u'\xi) e\left(\frac{kq+\bar{y}}{m_1}\xi\right)du'\\
    =& \sum_{m_1\sim M_1,m_2\sim M_2} \left(\int f\left(\frac{m_1 u'}{m_2}\right) e(-u'\xi) du'\right)\left(\sum_{k\in \mathbb{Z}} \psi\left(\frac{kq+\bar{y}}{M_3}\right) e\left(\frac{kq+\bar{y}}{m_1}\xi\right)\right)\\
    =& \sum_{m_1\sim M_1,m_2\sim M_2} \frac{m_2}{m_1}\hat{f}\left(\frac{m_2}{m_1}\xi\right)\left(\sum_{k\in \mathbb{Z}} \psi\left(\frac{kq+\bar{y}}{M_3}\right) e\left(\frac{kq+\bar{y}}{m_1}\xi\right)\right).\\
\end{align*}
The summation in $k$ can be rewritten through Poisson summation to get \[
    \sum_{k\in \mathbb{Z}} \psi\left(\frac{kq+\bar{y}}{M_3}\right) e\left(\frac{kq+\bar{y}}{m_1}\xi\right) = \sum_{\xi'} \frac{M_3}{q}\hat{\psi}\left(M_3\left(\frac{\xi}{m_1}-\frac{\xi'}{q}\right)\right)e\left(\frac{y\xi'}{q}\right).
\]
Therefore,
\[
\hat{g}(\xi) = \sum_{\substack{m_1\sim M_1\\m_2\sim M_2}}\sum_{\xi'}\frac{m_2}{m_1}\frac{M_3}{q}\hat{f}\left(\frac{m_2}{m_1}\xi\right)\hat\psi\left(M_3\left(\frac{\xi}{m_1}-\frac{\xi'}{q}\right)\right)e\left(\frac{-y_1m_1\xi'}{q}\right)e\left(\frac{-y_2m_2\xi'}{q}\right).
\]
By the principle of stationary phase, \[
\hat\psi\left(M_3\left(\frac{\xi}{m_1}-\frac{\xi'}{q}\right)\right)\]
is negligible except when  \[
\left|M_3\left(\frac{\xi}{m_1}-\frac{\xi'}{q}\right)\right| \lesssim 1 \implies \left|\frac{q\xi}{m_1}-\xi'\right|\lesssim \frac{q}{M_3},\]
so we can truncate the summation in $\xi'$ to be in this range.

We split \[
\int |\hat{g}(\xi)|^2 d\xi = A+B+C, 
\]
where
\begin{align*}
    A&\defeq \int_{|\xi|\leq T^\delta \frac{M_1}{M_3}}     |\hat{g}(\xi)|^2 d\xi\\
    B&\defeq \int_{T^\delta \frac{M_1}{M_3}\leq|\xi|\leq T^2}     |\hat{g}(\xi)|^2 d\xi\\
    C&\defeq \int_{T^2\leq |\xi|}     |\hat{g}(\xi)|^2 d\xi\\
\end{align*}
$C$ is negligible based on rapid decay in $\hat{f}(\xi m_2/m_1)$.
	
\section{Energy bound}
Here we provide the generalization for the orthogonal energy bound for Guth and Maynard's result. 
\begin{proposition}\label{energybound}
    \begin{align*}
        E(\mathcal{S})
        \lesssim& |\mathcal{S}|^2N^{2-2\sigma}+ |\mathcal{S}|^3N^{1-2\sigma} + |\mathcal{S}|^{9/4}(qT)^{1/2}N^{{1-2\sigma}}.
    \end{align*}
\end{proposition}
The idea for bounding energy is similar; if $\chi_1\chi_2=\chi_3\chi_4$ and $|t_1+t_2-t_3-t_4|$ is small, we should expect $|D_N(t_1+t_2-t_3,\chi_1\chi_2\bar{\chi}_3)|\simeq |D_N(t_4,\chi_4)|>N^\sigma$.
\begin{lemma}
    \[
    D_N(t,\chi)\lesssim \int_{|u-t|\lesssim 1} |D_N(u,\chi)|du + O(T^{-100}),
    \]
    uniformly in $\chi$.
\end{lemma}
\begin{proof}
    (GM)
    \begin{align*}
        D_N(t,\chi)=\sum_n \omega\left(\frac{n}{N}\right) b_n n^{it} \psi\left(\frac{\log n}{2\pi}\right)
    \end{align*}
    For other characters, we can just redefine $b'_n=b_n\chi(n)$.
\end{proof}
\begin{lemma} We have \[
    E(\mathcal{S})\lesssim N^{-2\sigma}\sum_{n_1,n_2\sim N}\left|R\left(\frac{n_1}{n_2},n_1,n_2\right) \right|^3.
    \]
    
\end{lemma}
\begin{proof}
    We have \begin{align*}    
    E(\mathcal{S})=& \sum_{\substack{(t_1,\chi_1),(t_2,\chi_2),\\ (t_3,\chi_3),(t_4,\chi_4)\in \mathcal{S}\\|t_1+t_2-t_3-t_4|\leq 1\\ \chi_1\chi_2=\chi_3\chi_4}} 1 
    \leq N^{-2\sigma}\sum_{\substack{(t_1,\chi_1),(t_2,\chi_2),\\ (t_3,\chi_3),(t_4,\chi_4)\in \mathcal{S}\\ |t_1+t_2-t_3-t_4|\leq 1\\ \chi_1\chi_2=\chi_3\chi_4}} |D_N(t_4,\chi_4)|^2.
    \end{align*}
    Now we apply the previous lemma and Cauchy-Schwarz to get \[
        |D_N(t_4,\chi_4)|^2\lesssim \int_{|u-t_4|\lesssim 1} |D_N(u,\chi_4)|^2du \lesssim  \int_{|u-t_1-t_2+t_3|\lesssim 1} |D_N(u,\chi_1\chi_2\bar{\chi}_3)|^2du,
    \]
   
    Since $\chi_1,\chi_2,\chi_3$ fixes $\chi_4$, and the $t$'s within the same character are $T^\epsilon$ separated, there is $O(1)$ possible pairs of $(t_4,\chi_4)$ for each choice of $(t_1,\chi_1),(t_2,\chi_2),(t_3,\chi_3)$, so \begin{align*}    
        E(\mathcal{S})\lesssim& N^{-2\sigma}\sum_{\substack{(t_1,\chi_1),(t_2,\chi_2),\\ (t_3,\chi_3),(t_4,\chi_4)\in \mathcal{S}\\ |t_1+t_2-t_3-t_4|\leq 1\\ \chi_1\chi_2=\chi_3\chi_4}} \int_{|u-t_1-t_2+t_3|\lesssim 1} |D_N(u,\chi_1\chi_2\bar{\chi}_3)|^2du\\
        \lesssim&  N^{-2\sigma}\sum_{\substack{(t_1,\chi_1),(t_2,\chi_2),\\ (t_3,\chi_3)\in \mathcal{S}}} \int_{|u-t_1-t_2+t_3|\lesssim 1} |D_N(u,\chi_1\chi_2\bar{\chi}_3)|^2du\\
        =& N^{-2\sigma}\sum_{\substack{(t_1,\chi_1),(t_2,\chi_2),\\ (t_3,\chi_3)\in \mathcal{S}}} \int_{|u|\lesssim 1} |D_N(t_1+t_2-t_3 + u,\chi_1\chi_2\bar{\chi}_3)|^2du\\
        =&N^{-2\sigma}\sum_{n_1,n_2} b_{n_1}\bar{b}_{n_2} \omega\left(\frac{n_1}{N}\right)\omega\left(\frac{n_2}{N}\right)\int_{|u|\lesssim 1} \left(\frac{n_1}{n_1}\right)^{iu}R\left(\frac{n_1}{n_2},n_1,n_2\right)^2 R\left(\frac{n_2}{n_1},n_2,n_1\right)du\\
        \lesssim & N^{-2\sigma}\sum_{n_1,n_2\sim N}\left|R\left(\frac{n_1}{n_2},n_1,n_2\right)^2 R\left(\frac{n_2}{n_1},n_2,n_1\right)\right|\\
        \lesssim & N^{-2\sigma}\sum_{n_1,n_2\sim N}\left|R\left(\frac{n_1}{n_2},n_1,n_2\right) \right|^3.
        \end{align*}
\end{proof}

\begin{lemma}
    We have 
\[
        \sum_{n_1,n_2\sim N}\left|R\left(\frac{n_1}{n_2},n_1,n_2\right) \right|^2\lesssim |\mathcal{S}|N^2+ |\mathcal{S}|^2N + |\mathcal{S}|^{5/4}(qT)^{1/2}N.
    \]
\end{lemma}
\begin{proof}
    From the definition of $R$, \begin{align*}
        \sum_{n_1,n_2\sim N}\left|R\left(\frac{n_1}{n_2},n_1,n_2\right) \right|^2=& \sum_{n_1,n_2\sim N}\sum_{\substack{(t_1,\chi_1),(t_2,\chi_2)\in \mathcal{S}}}
        \chi_1(n_1)\bar{\chi}_1(n_2)\left(\frac{n_1}{n_2}\right)^{it_1}\bar{\chi}_2(n_1){\chi}_2(n_2)\left(\frac{n_1}{n_2}\right)^{-it_2}
        \\
        &=\sum_{\substack{(t_1,\chi_1),(t_2,\chi_2)\in \mathcal{S}}}\left|\sum_{n\sim N}
        \chi_1(n_1)\bar{\chi}_2(n_1)n^{i(t_1-t_2)}\right|.
    \end{align*}
    A direct application of Heath Brown's Theorem \ref{heathbrown} gives \[
        \sum_{n_1,n_2\sim N}\left|R\left(\frac{n_1}{n_2},n_1,n_2\right) \right|^2\lesssim |\mathcal{S}|N^2+ |\mathcal{S}|^2N + |\mathcal{S}|^{5/4}(qT)^{1/2}N.
    \]
\end{proof}
The trivial bound for $R\leq |\mathcal{S}|$ gives \begin{align*}
    E(\mathcal{S})\lesssim& N^{-2\sigma}\sum_{n_1,n_2\sim N}\left|R\left(\frac{n_1}{n_2},n_1,n_2\right) \right|^3\\
    \lesssim &|\mathcal{S}|N^{-2\sigma}\sum_{n_1,n_2\sim N}\left|R\left(\frac{n_1}{n_2},n_1,n_2\right) \right|^2\\
    \lesssim& |\mathcal{S}|^2N^{2-2\sigma}+ |\mathcal{S}|^3N^{1-2\sigma} + |\mathcal{S}|^{9/4}(qT)^{1/2}N^{{1-2\sigma}}.
\end{align*}
\textcolor{red}{TODO: Merge with previous part to find bound}

The arguments beyond will be adaptations from GM. 

\begin{lemma}
    We have \begin{align*}
        \sum_{n_1,n_2\sim N}\left|R\left(\frac{n_1}{n_2},n_1,n_2\right) \right|^4\lesssim ?
    \end{align*}
\end{lemma}
\begin{proof}
    Let \[
    U_B=\left\{u\in\mathbb{Z} \ : \ \#\{((t_1,\chi_1),(t_2,\chi_2))\in \mathcal{S} \ : \ \lfloor {t_1-t_2}\rfloor = u \} \sim B \right\},
    \]
    so that we split the sum in $R$ as \begin{align*}
        \left|R\left(\frac{n_1}{n_2} ,n_1,n_2\right) \right|^4 =&
        \left|
        \sum_{(t_1,\chi_1),(t_2,\chi_2)\in \mathcal{S}}
        \chi_1(n_1)\bar{\chi}_2(n_2) \left(\frac{n_1}{n_2}\right)^{i(t_1-t_2)}\right|^2 \\
        =& \left|
        \sum_{j=0}^{ \lfloor \log_2 |\mathcal{S}|\rfloor} \sum_{u\in U_{2^j}}\sum_{\substack{(t_1,\chi_1),(t_2,\chi_2)\in \mathcal{S}\\ \lfloor t_1-t_2 \rfloor = u}} 
        \chi_1(n_1)\bar{\chi}_2(n_2) \left(\frac{n_1}{n_2}\right)^{i(t_1-t_2)}\right|^2\\
        \lesssim& \sum_{j=0}^{ \lfloor \log_2 |\mathcal{S}|\rfloor} \left|
         \sum_{u\in U_{2^j}}\sum_{\substack{(t_1,\chi_1),(t_2,\chi_2)\in \mathcal{S}\\ \lfloor t_1-t_2 \rfloor = u}} 
        \chi_1(n_1)\bar{\chi}_2(n_2) \left(\frac{n_1}{n_2}\right)^{i(t_1-t_2)}\right|^2
    \end{align*}
    where we applied Cauchy-Schwarz in the last step.
    Therefore, \begin{align*}
        \sum_{n_1,n_2\sim N}\left|R\left(\frac{n_1}{n_2},n_1,n_2\right) \right|^4 \lesssim & 
        \sup_{j\leq \lfloor \log_2 |\mathcal{S}|\rfloor} \sum_{n_1,n_2\sim N} \left|
        \sum_{u\in U_{2^j}}\sum_{\substack{(t_1,\chi_1),(t_2,\chi_2)\in \mathcal{S}\\ \lfloor t_1-t_2 \rfloor = u}} 
       \chi_1(n_1)\bar{\chi}_2(n_2) \left(\frac{n_1}{n_2}\right)^{i(t_1-t_2)}\right|^2\\
       \leq& \sup_{j\leq \lfloor \log_2 |\mathcal{S}|\rfloor} \sum_{n_1,n_2\sim N} \left|
       \sum_{u\in U_{2^j}}\sum_{\substack{(t_1,\chi_1),(t_2,\chi_2)\in \mathcal{S}\\ \lfloor t_1-t_2 \rfloor = u}} 
      \chi_1(n_1)\bar{\chi}_2(n_2) \left(\frac{n_1}{n_2}\right)^{i(t_1-t_2)}\right|^2
    \end{align*}
\end{proof}
	\section{Proof of reduction of theorem}
\begin{proof}[Proof of Proposition \ref{mainthm}]
    By Proposition \ref{Tracebound},
\[    |W|\ll N^2 V^{-2} + N V^{-2} S^{1/3}.\]
Recall we have $S\defeq S_2+S_3$, where 
\[
    S_2 \lesssim_k N^2|W|^2 + qTN|W|^{2-1/k} + N^2|W|^2 \Big(\frac{(qT)^{1/2}}{|W|^{3/4}}\Big)^{1/k}.
\]
    by Proposition \ref{s2bound}, and
    \[
        S_3\lesssim (qT)^2|W|^{3/2} + N(qT)|W|^{1/2}E(W)^{1/2} 
    \]
    by Proposition \ref{s3refined}.
    Finally, by Proposition \ref{energybound}, we have
    \begin{align*}
E(W)\lesssim |W|N^{4-4\sigma}+|W|^{21/8}(qT)^{1/4}N^{1-2\sigma}+|W|^3N^{1-2\sigma}.
    \end{align*}
    These are analgous statements to Proposition 4.6, Propositions 6.1 (summed with Proposition 5.1), Proposition 10.1, and Proposition 11.1 in \cite{GM2024} respectively with 
    $T$ swapped with $qT$. Thus the reduction is the same as in the proof of Proposition 3.1.
\end{proof}
	\printbibliography
\end{document} 