 \chapter{Towards a Hybrid Large Values Estimate Result}

\iffalse

\textit{Here I will document some progress with working with
the hybrid version of Guth-Maynard. This section will be removed when I send the draft to Prof. Wunsch.}
\fi
We would like to generalize Guth and Maynard's result to $L$-functions. Specifically, let $\chi$ be a \hyperref[dcharacter]{Dirichlet character}. we are interested in the the zeros of the function defined by \[
L(s,\chi)\defeq\sum_{n} \frac{\chi(n)}{n^{-s}}
\]
on $\Re(s)>1$ and its analytic continuation on the whole complex plane. The zeta function is a special case of an $L$-function with the Dirichlet character $1$ everywhere. The structure of the arguments for analytic continuation of an $L$-function, its line of symmetry along $\Re(s) = 1/2$, and the locations of zeros are very similar to that of the zeta function. This motivates the Generalized Riemann Hypothesis. \begin{conjecture} [Generalized Riemann Hypothesis]
    The \text{Generalized Riemann Hypothesis} asserts that on the critical strip, \[
    L(s,\chi)=0 \implies \Re(s)=\frac{1}{2},
    \]
    for any Dirichlet character.
\end{conjecture}
The Generalized Riemann Hypothesis leads to even stronger for primes in short intervals. Namely, fix an integer $q$, we have that \[
\sum_{\substack{n\leq N\\ n\equiv a\mod q}}\Lambda(n) = \begin{cases}
    \frac{1}{\phi(q)} N + O(x^{2+o(1)}), & \textrm{if $\gcd(a,q)=1$}\\
    o(n),& \textrm{otherwise.}
\end{cases}
\]
This means that not only that the Prime Number Theorem holds in intervals of $x^{2+\epsilon}$, the distribution of primes in each of the residual classes (coprime to $q$) are uniform at this scale too. Noticing that we can modify Huxley's proof with \[
M_{x,\chi} = \sum_{n\leq x} \chi(n) \mu(n) n^{-s},
\]
we have \[
L(s,\chi)M_{x,\chi} = \sum_{n} a_n \chi(n) \mu(n) n^{-s}.
\]

Thus, we can reproduce a similar proof on the zero density of $L$-functions. 
\begin{definition}[Zero Density for $L$-functions]
    Let $N(\sigma,\chi,T)$ denote the number of zeros of the $L$-function $L(-,\chi)$ with real part greater than $\sigma$ and imaginary part between $-T$ and $-T$. That is \[
    N(\sigma,\chi,T) \defeq \#\{\rho=\beta+i\gamma \ | \ \beta\geq \sigma, |\gamma|\leq T\}.
    \]
    For backwards compatibility with our previous definition, we take $N(\sigma,T)\defeq N(\sigma, 1,T)$.
\end{definition}
The hybrid analogs of the zero density bounds of Ingham and Huxley are known.
\begin{theorem}[Hybrid Ingham bound for zero density]
    Let $1/2\leq \sigma\leq 3/4$. We have \[
        \sum_{\chi^* }N(\sigma,\chi^*,t)\lesssim (qT)^{\frac{3(1-\sigma)}{2-\sigma}},
        \]
        where $\sum_{\chi^*}$ sums over all the primitive characters $\chi^*$ of modulus $q$.
\end{theorem}
\begin{theorem}[Hybrid Huxley bound for zero density]
    Let $3/4\leq \sigma\leq 1$. We have \[
        \sum_{\chi^* }N(\sigma,\chi^*,t)\lesssim (qT)^{\frac{3(1-\sigma)}{3\sigma-1}},
        \]        
        where $\sum_{\chi^*}$ sums over all the primitive characters $\chi^*$ of modulus $q$.
\end{theorem}

The method for detecting zeros is very similar to Huxley's proof above with the slight change in definition of $M_{x,\chi}$. This argument then reduces to bounding the number of times large values of Dirichlet polynomials can occur. Therefore we want a result in the form of the Hal\'asz inequality, this time with twisted Dirchlet polynomials:
\begin{quotation}  
Let $W=\{(t_j,\chi_j)\}$ be a set such that each $\chi_j$ is a primitive Dirichlet character of modulus $q$, and $|t_j-t_k|\geq 1$ if $j\neq k$ and $\chi_j=\chi_k$. (That is, the $t$'s are $1$-separated if the characters are the same.)Let $|b_n|\leq 1$ be a sequence of numbers indexed in $n$, and suppose also that \[
\Bigg|\sum_{n\sim N} b_n\chi_j(n)n^{it_j} \Bigg|>V.
\]
We want to find a bound on $|W|$.
\end{quotation}
\section{Statement of Main Theorem and its Reduction}
Our result is as follows: 
\begin{theorem}[Incomplete generalization of Guth-Maynard Hal\'asz Inequality]\label{incompletehybrid}
	Let $W=\{(t_j,\chi_j)\}$ be a set such that each $\chi_j$ is a primitive Dirichlet character of modulus $q<N$, and $|t_j-t_k|\geq 1$ if $j\neq k$ and $\chi_j=\chi_k$. (That is, the $t$'s are $1$-separated if the characters are the same.)Let $|b_n|\leq 1$ be a sequence of numbers indexed in $n$, and suppose also that \[
	\Bigg|\sum_{n\sim N} b_n\chi_j(n)n^{it_j} \Bigg|>V.
	\]
	Then for $q$ such that $\omega(q)=o(\log q /\log\log q),$ \[
	|W|\leq T^{o(1)}(N^2V^{-2}+ N^{18/5}V^{-4}+(qT)^{12/5}V^{-4} ).
	\]
	In particular, it holds for all primes $q=p$ and powers of primes $q=p^k$.
\end{theorem}
\begin{remark}
$o(\log q/\log \log q)$ is taken in $q\to\infty$ sense, across the subset of natural numbers. This means that it generally holds for numbers with a reasonably small number of distinct prime factors. In fact, most numbers satisfy this additional assumption on $q$. The worst edge case is when $q$ is primorial (i.e. it is the multiple of the first $k$ primes), in which case $\omega(q)=O(\log q/\log \log q)$ and is just slightly too large to bound with this method.
\end{remark}
\begin{corollary}
	The statement in Theorem \ref{incompletehybrid} holds for almost all values of $q$.
\end{corollary}
\begin{proof}
This is a direct consequence of the Erd\"os-Kac Theorem that almost all $q$ satisfy $\omega(q)\asymp \log \log q = o(\log q/\log \log q)$.
\end{proof}
\begin{proposition}[Reduction of Main Theorem]\label{mainthm}
    Let $q$ as in Theorem \ref{incompletehybrid}. Let $W=\{(t_j,\chi_j)\}$ be a set such that each $\chi_j$ is a primitive Dirichlet character of modulus $q$, and $t_j\in[0,T]$. Further assume that $|t_j-t_k|\geq T^\epsilon$ if $j\neq k$ and $\chi_j=\chi_k$. Let $|b_n|\leq 1$ be a sequence of numbers indexed in $n$, $\omega$ be a smooth bump function that equals $1$ on $[6/5,9/5]$ and has support in $[1,2]$ (thus $\omega^{(A)}\ll_A 1$ for all $A$). Let $V=N^{\sigma}$, where $[7/10,8/10]$, and $N=(qT)^{5/6}$, $T\geq q^{1/5}$. Suppose also that \[
    |D_N(t_j,\chi_j)|\defeq\Bigg|\sum_{n\sim N} \omega\Bigg(\frac{n}{N}\Bigg)b_n\chi_j(n)n^{it_j} \Bigg|>V.
\]
for all $(t_j,\chi_j)\in W$.
Then \[
|W|\ll (qT)N^{(12-20\sigma)/5+o_{\epsilon}(1)}.
\]
\end{proposition}
\begin{proof}[Proof of Theorem \ref{incompletehybrid} assuming Proposition \ref{mainthm}]
The argument is essentially identical to the Guth and Maynard's proof, except checking the condition that $T$ is not too small compared to $q$.

The result is given by Ingham and Huxley when $V\leq N^{7/10+o(1)}$ or $V\geq N^{8/10+o(1)}$. Now let $V\in [4N^{7/10},N^{8/10}]$. Then we have $N^2V^{-2}\ll N^{18/5}V^{-4}$.
We split \[
\sum_{n\sim N} b_n\chi_j(n)n^{it_j}=\sum_{N\leq n< 6N/5} b_n\chi_j(n)n^{it_j} +\sum_{6N/5 \leq n < 9N/5} b_n\chi_j(n)n^{it_j} +\sum_{9N/5 \leq n < 2N} b_n\chi_j(n)n^{it_j}.
\]
So that if \[
\Big|\sum_{n\sim N} b_n\chi_j(n)n^{it_j}\Big|>V,
\]
then at least one of the three summations in the right has magnitude at least $V/3$.
But then each of these sums can be written as dirichlet sums in the form of the lemma: \begin{align*}
	\sum_{N\leq n< 6N/5} b_n\chi_j(n)n^{it_j} &=  \sum_{N\leq n< 6N/5} \omega\Bigg(\frac{n}{11N/15}\Bigg) b_n\chi_j(n)n^{it_j}, \\
	\sum_{6N/5 \leq n < 9N/5} b_n\chi_j(n)n^{it_j} &=  \sum_{6N/5 \leq n < 9N/5} \omega\Bigg(\frac{n}{N}\Bigg) b_n\chi_j(n)n^{it_j}, \\
	\sum_{9N/5 \leq n < 2N}  b_n\chi_j(n)n^{it_j} &=  \sum_{9N/5 \leq n < 2N}  \omega\Bigg(\frac{n}{19N/15}\Bigg) b_n\chi_j(n)n^{it_j}. \\
\end{align*}
Applying the lemma on each of these three pieces and summing the three bounds gives the same bound for $\sum_{n\sim N} b_n\chi_j(n)n^{it_j}$ up to a constant factor, so it suffices to show the result assuming that
$b_n=0$ for $6N/5 \leq n < 9N/5$, in which case $b_n=b_n\omega(n/N)$.
Now suppose that $qT\leq N^{6/5}$. Then the $t_j$'s lie in an interval of length $N^{6/5}/q\geq T$. Thus in this case at a cost of $T^{\epsilon}$ we consider the $T^{\epsilon}$ disjoint subsets of $|W|$ each
being $T^{\epsilon}$ separated in the $t_j$'s. Setting $V=N^{\sigma}$ and checking that $N^{6/5}/q\geq q^{1/5},$ the proposition gives\[
|W|\leq T^{\epsilon} N^{(18-20\sigma)/5+o_{\epsilon}(1)} \leq T^{\epsilon} N^{18/5+o_{\epsilon}(1)}V^{-4}.
\] 
Letting $\epsilon\to 0$ gives the result. If $qT\geq N^{6/5}$, we divide $|W|$ into $[T/N^{6/5}]$ subsets each supported on a length of $N^{6/5}/q\geq q^{1/5}$. Therefore, we can apply the proposition (on the $T^\epsilon$ disjoint subsets) to get \[
|W|\leq T^{1+\epsilon}  N^{-6/5}(N^{18/5+o_{\epsilon}(1)}V^{-4})\leq T^{o_\epsilon(1)}N^{12/5}.
\] 
Letting $\epsilon\to 0$ slowly gives the result in this case too.
\end{proof}
	
The idea of the proof of Proposition \ref{mainthm} is similar to Guth and Maynard's proof. We can define a $|W|\times N$ matrix $M$ with entries
\[
    M_{t_j,\chi_j,n} = \chi_j(n)n^{it_j}
\]
for $(t_j,\chi_j)\in W$ and $n\sim N$, and bound its operator norm in the exact same way: taking it to the $M^*M$ to the third power and calculating its trace. We will highlight similar ideas when they come up.

\section{The matrix $M$ and its trace expansion}
We define $M$ a $|W |\times N$ matrix with entries
\[
    M_{(t_j,\chi_j),n} = \omega\Big(\frac{n}{N}\Big)\chi_j(n)n^{it_j}
\]
for $(t_j,\chi_j)\in W$ and $n\sim N$.
Thus by the same reasoning that $(M\vec{b})_j=D_N(t_j,\chi_j)$,
we want to bound the size of $W$ by trace of the matrix \[
\textrm{tr}((M^*M)^3).
\].
\begin{proposition} \label{Tracebound}
    Let \[
    P(n,(t_1,\chi_1),(t_2,\chi_2))\defeq \omega\left(\frac{n}{N}\right)^2 \left(\frac{n}{N}\right)^{i(t_2-t_1)}\chi_2\bar{\chi}_1(n),
    \]
    and \[
    S \defeq \sum_{\substack{(t_1,\chi_1),(t_2,\chi_2),\\(t_3,\chi_3)\in W\\ (*)} }\ \sum_{n_1,n_2,n_3\sim N} 
    P(n_1,(t_1,\chi_1),(t_3,\chi_3))P(n_2,(t_2,\chi_2),(t_1,\chi_1))P(n_3,(t_3,\chi_3),(t_2,\chi_2)),
    \]
    where the condition $(*)$ denotes that the three pairs are not all identical.
    We have 
    \[
        |W|\ll N^2 V^{-2} + N V^{-2} S^{1/3}.
    \]
    
\end{proposition}


\begin{lemma}\label{trace formula}
    We have 
    \[
        |W|\ll N V^{-2}\left(\frac{\rm{tr}(AA^*)}{|W|}\right) + N V^{-2}\left(\rm{tr}((MM^*)^3)-\frac{\rm{tr}(MM^*)^3}{|W|^2}\right)^{1/3}.
    \]
\end{lemma}


\begin{lemma}\label{hilberttrace}
    Let $A$ be an $m\times n$ matrix. Then 
    \[\|A\| \leq 2\left(\rm{tr}((AA^*)^3)-\frac{\rm{tr}(AA^*)^3}{m^2}\right)^{1/6}+2\left(\frac{\rm{tr}(AA^*)}{m}\right)^{1/2}.
    \]
\end{lemma}
\begin{proof}
    This is Lemma 4.2 from Guth-Maynard. \cite{GM2024}
\end{proof}


\begin{proof}[Proof of Proposition \ref{trace formula}]
    Consider the product $M\vec{b}$. The $j$-th entry is exactly $D_n(t_j,\chi_j)$. Thus we have \[
    V^2|W|\leq |M \vec{b}|^2 \leq \|M\|^2|\vec{b}|^2\leq \|M\|^2N.
    \]
    Applying Lemma \ref{hilberttrace} gives the desired result.
\end{proof}


We now compute the expression for the traces.
\begin{lemma}[Explicit Expression of Traces]
    Let \[
    P(n,(t_1,\chi_1),(t_2,\chi_2))\defeq \omega\left(\frac{n}{N}\right)^2 \left(\frac{n}{N}\right)^{i(t_2-t_1)}\chi_2\bar{\chi}_1(n).
    \]
    We have 
    \begin{align*}
        \rm{tr}(M^*M) = & \ |W|\sum_{n\sim N, \gcd (n,q)=1} \omega\left(\frac{n}{N}\right)^2, \\
        \rm{tr}((M^*M)^3)=&\sum_{\substack{(t_1,\chi_1),(t_2,\chi_2),\\(t_3,\chi_3)\in W}}\sum_{n_1,n_2,n_3\sim N} 
        P(n_1,(t_1,\chi_1),(t_3,\chi_3))P(n_2,(t_2,\chi_2),(t_1,\chi_1))P(n_3,(t_3,\chi_3),(t_2,\chi_2)).
    \end{align*}
\end{lemma}
\begin{proof}
   % We compute the traces of $MM^*$ and $(MM^*)^3$ instead, as this makes the calculations cleaner.
    We see that \begin{align*}
        (MM^*)_{(t_j,\chi_j),(t_k,\chi_k)} = \sum_{n\sim N} \omega\left(\frac{n}{N}\right)^2 n^{i(t_k-t_j)}\bar{\chi}_j\chi_k(n) = \sum_{n\sim N}P(n,(t_k,\chi_k),(t_j,\chi_j))N^{i(t_k-t_j)},
    \end{align*}
    so that \begin{align*}
        \textrm{tr}(MM^*) =& \sum_{(t_j,\chi_j)} \sum_{n\sim N} P(n,(t_j,\chi_j),(t_j,\chi_j)) \\
        =&|W|\sum_{n\sim N} |\chi_j(n)|\omega\left(\frac{n}{N}\right)^2.
    \end{align*}
    For the third power, we have 
    \begin{align*}
        (MM^*)^3_{(t_j,\chi_j),(t_k,\chi_k)} = \sum_{(t_1,\chi_1),(t_2,\chi_2)\in W}   (MM^*)_{(t_j,\chi_j),(t_1,\chi_1)} (MM^*)_{(t_1,\chi_1),(t_2,\chi_2)} (MM^*)_{(t_2,\chi_2),(t_k,\chi_k)},
    \end{align*}
    so that the trace is given by \begin{align*}
        &\sum_{\substack{(t_1,\chi_1),(t_2,\chi_2),\\(t_3,\chi_3)\in W}} (MM^*)_{(t_3,\chi_3),(t_1,\chi_1)} (MM^*)_{(t_1,\chi_1),(t_2,\chi_2)} (MM^*)_{(t_2,\chi_2),(t_3,\chi_3)}\\
        =&\sum_{\substack{(t_1,\chi_1),(t_2,\chi_2),\\(t_3,\chi_3)\in W}} \sum_{n_1,n_2,n_3\sim N} 
        P(n_1,(t_1,\chi_1),(t_3,\chi_3))P(n_2,(t_2,\chi_2),(t_1,\chi_1))P(n_3,(t_3,\chi_3),(t_2,\chi_2)).
    \end{align*}
\end{proof}

Consider the sum \[\sum_{\substack{(t_1,\chi_1),(t_2,\chi_2),\\(t_3,\chi_3)\in W}}\]  in the expression of $\rm{tr}((M^*M)^3)$.
We isolate the terms where $(t_1,\chi_1)=(t_2,\chi_2)=(t_3,\chi_3)$ in the sum, and call it $S_0$.

\begin{lemma}[Explicit Calculation of $S_0$]
    We have \[
    S_0 = |W|^2\Bigg(\sum_{n\sim N, \gcd(n,q)=1} \omega\Big(\frac{n}{N}\Big)^2 \Bigg)^2.
    \]
    In particular, \[
    S_0 =  \Big(\rm{tr}(M^*M)\Big)^3/|W|^2.
    \]
\end{lemma}
\begin{proof}
    Notice that \[
    P(n_1,(t_1,\chi_1),(t_1,\chi_1)) = \omega\Big(\frac{n_1}{N}\Big)^2 \mathbb{I}_{\gcd(n_1,q)=1}.
    \]
    Thus we have \[
    S_0 = \sum_{(t_1,\chi_1)\in W} \sum_{n_1,n_2,n_3 \sim N}\omega\Big(\frac{n_1}{N}\Big)^2 \omega\Big(\frac{n_2}{N}\Big)^2 \omega\Big(\frac{n_3}{N}\Big)^2 \mathbb{I}_{\gcd(n_1,q)=1}.\mathbb{I}_{\gcd(n_2,q)=1}.\mathbb{I}_{\gcd(n_3,q)=1}
    \]
    which gives the result.
\end{proof}
Thus assembling this lemma with Proposition \ref{trace formula} gives proposition \ref{Tracebound}.

\section{Poisson summation on $S$}
We now bound $S$ using Poisson summation. We would like to apply Poisson summation directly to $P$ in $n_1$, $n_2$, $n_3$, respectively. However, we do not have a good way to express the Dirichlet character $\chi_j\bar\chi_k$ as a $C^\infty$ function. To get around this, we just split the sum in $n$ into $q$ pieces. 
\begin{lemma}[Poisson Summation with Periodic Arithmetic Function]
    Let $f:\reals \to \complex$ be Schwartz, and $g:\naturals\to\complex$ be an arithmetic with period $q$. Then \[
    \sum_n g(n)f(n) = \frac{1}{q}\sum_{\xi} \hat{g}(\xi)\hat{f}\bigg(\frac{\xi}{q}\bigg). 
    \]
\end{lemma}
\begin{proof}
    By absolute convergence, we have \begin{align*}
        \sum_n g(n)f(n) = \sum_n \sum_{x\mod q} g(qn+x)f(qn+x)=  \sum_{x\mod q} g(x)\sum_n f(qn+x).
    \end{align*}
    Applying Poisson summation to the last expression in $n$ gives \[
        \frac{1}{q} \sum_{x\mod q} g(x) \sum_{\xi} e\bigg(\frac{-x\xi}{q}\bigg)\hat{f}\bigg(\frac{\xi}{q}\bigg)= \frac{1}{q}\sum_{\xi} \hat{g}(\xi)\hat{f}\bigg(\frac{\xi}{q}\bigg).
    \]
\end{proof}
\begin{corollary}[Poisson summation of Trace]
    Let $h_t(u)\defeq\omega(u)^2u^{it}$,
    Then \begin{align*}
        %\rm{tr}(M^*M) &= | W|\frac{N}{q}\sum_{m}\sum_{x\in \mathbb{Z}/q\mathbb{Z}}e\left(\frac{-xm}{q}\right)\hat{h}_0\left(\frac{Nm}{q}\right)\\
        S&=\sum_{\substack{(t_1,\chi_1),\\(t_2,\chi_2),\\(t_3,\chi_3)\in W \\ (*)}}
        \frac{N^3}{q^3}\sum_{m\in\mathbb{Z}^3}\widehat{{\chi_1{\overline{\chi_3}}}}(m_1)\widehat{\chi_2\overline{\chi_1}}(m_2)\widehat{\chi_3\overline{\chi_2}}(m_3)
         \hat{h}_{t_1-t_3}\left(\frac{Nm_1}{q}\right)\hat{h}_{t_2-t_1}\left(\frac{Nm_2}{q}\right)\hat{h}_{t_3-t_2}\left(\frac{Nm_3}{q}\right).
    \end{align*}
\end{corollary}

In the summation in $m$, we separate this into two terms. $S_2$, when at least one of the $m_i$'s are zero, and $S_3$, when all three $m_i$'s are non-zero.

Finally, we restate the key result that allows us to bound $\hat{h}$ in the Fourier domain. We have not modified the $\hat{h}$ function, so the proof is the same as in Lemma \ref{nonstationary}.
\begin{lemma*}[Non-stationary phase]
    \label{nonstationarynew}
    We have for any integer $A>0$\begin{align*}
        |\hat{h}_t(\xi)|\ll_A \frac{1+|t|^A}{|\xi|^A},\\
        |\hat{h}_t(\xi)|\ll_A \frac{1+|\xi|^A}{|t|^A}. 
    \end{align*}
\end{lemma*}
\section{$S_2$ Bound}

\begin{proposition}[Bound on $S_2$]\label{s2bound}
    We have \[
    S_2 \lesssim_k N^2|W|^2 + qTN|W|^{2-1/k} + N^2|W|^2 \Big(\frac{(qT)^{1/2}}{|W|^{3/4}}\Big)^{1/k}.
    \]
\end{proposition}
This bound relies on Heath-Brown's theorem. We recall its statement here.
\begin{theorem*}[Heath-Brown]
	%\label{heathbrown}
	Let $\mathcal{S}=\{(t_j,\chi_j)\}$ be one-separate, primitive characters of modulus $q$. Then 
	\[
	\sum_{\substack{(t_1,\chi_1)\\(t_2,\chi_2)}}\left|\sum_{n=1}^{N} b_n n^{-1/2 -i(t_1-t_2)}\chi_1\bar{\chi}_2(n)\right|^2 \lesssim  (|\mathcal{S}|N+ |\mathcal{S}|^2 + |\mathcal{S}|^{5/4}(qT)^{1/2}) \max_{n\leq N} |b_n|.
	\]
\end{theorem*}

With a loss of a factor of $3$, we can set $m_3=0$. Thus \begin{align*}
    %\rm{tr}(M^*M) &= | W|\frac{N}{q}\sum_{m}\sum_{x\in \mathbb{Z}/q\mathbb{Z}}e\left(\frac{-xm}{q}\right)\hat{h}_0\left(\frac{Nm}{q}\right)\\
    S_2&\ll\sum_{\substack{(t_1,\chi_1),\\(t_2,\chi_2),\\(t_3,\chi_3)\in W \\ (*)}}
    \frac{N^3}{q^3}\sum_{m_1,m_2\in\mathbb{Z}}\widehat{{\chi_1{\overline{\chi}_3}}}(m_1)\widehat{\chi_2\overline{\chi}_1}(m_2)\widehat{\chi_3\overline{\chi}_2}(0) \hat{h}_{t_1-t_3}\left(\frac{Nm_1}{q}\right)\hat{h}_{t_2-t_1}\left(\frac{Nm_2}{q}\right)\hat{h}_{t_3-t_2}\left(0\right).
\end{align*}
By the orthogonality of characters, the terms when $\chi_2\neq \chi_3$ vanish as $\widehat{\chi_3\overline{\chi_2}}(0)=0.$ But when $\chi_2=\chi_3$ and $t_3\neq t_2$, then they are $T^{\epsilon}$ apart and $\hat{h}_{t_3-t_2}(0)\ll T^{-100}$. Since we have a trivial bound on the $T^{\epsilon}$-separated $|W|$ to be $qT\leq T^6$, 
 the terms for which $(t_2,\chi_2)\neq(t_3,\chi_3)$ are negligible by Lemma \ref{nonstationarynew} and we have \begin{align*}
	 S_2\ll\sum_{\substack{(t_1,\chi_1),\\(t_2,\chi_2)\in W \\ (t_1,\chi_1)\neq (t_2,\chi_2)}}
	\frac{N^3}{q^2}\sum_{m_1,m_2\in\mathbb{Z}}\widehat{{\chi_1{\overline{\chi}_2}}}(m_1)\widehat{\chi_2\overline{\chi}_1}(m_2)
	\hat{h}_{t_1-t_2}\left(\frac{Nm_1}{q}\right)&\hat{h}_{t_2-t_1}\left(\frac{Nm_2}{q}\right)+O(T^{-100}).
\end{align*}   
Noticing that $\hat{h}_t(\xi)=\overline{\hat{h}_{-t}(-\xi)}$, we rewrite \begin{equation} \label{S2clean}
    S_2 \ll\sum_{\substack{(t_1,\chi_1),\\(t_2,\chi_2)\in W \\ (t_1,\chi_1)\neq (t_2,\chi_2)}}
    \frac{N^3}{q^2}\left|\sum_{m \in \mathbb{Z}}\widehat{{\chi_1{\overline{\chi}_2}}}(m) e\left(\frac{-xm}{q}\right)
    \hat{h}_{t_1-t_2}\left(\frac{Nm}{q}\right)\right|^2.
\end{equation}
\begin{lemma}[Approximate Functional Equation]
    Let $|t|\leq T_0$, and $\chi$ be a Dirichlet character mod $q$. Further assume that if $\chi$ is principal then $|t|\geq T^{\epsilon}$. Then we have \[
        \left|\sum_{m \in \mathbb{Z}}\hat{\bar{\chi}}(m)
        \hat{h}_{t_1-t_2}\left(\frac{Nm}{q}\right)\right|\lesssim qN^{1/2}\int_{|u|\lesssim 1}\Big|\sum_{n\lesssim \frac{qT_0}{N}}n^{-1/2-i(t+u)}\bar{\chi}(n)\Big| du + O(((qT)^{-100})).
    \]
\end{lemma}
The proof to this approximate functional equation is due to Vishal Gupta, who had been collaborating with me on this problem. The proof relies on the reflection method for $L$-functions akin to Theorem 9.15 in \cite{iwaniec}. 

\begin{proof}[Proof of Proposition \ref{s2bound}]
    Beginning from equation \ref{S2clean}, we apply the approximate functional equation to get \begin{align*}
    S_2\lesssim& N^2  \sum_{\substack{(t_1,\chi_1),\\(t_2,\chi_2)\in W \\ (t_1,\chi_1)\neq (t_2,\chi_2)}} \Bigg|
    \int_{u\lesssim 1}\Big|\sum_{n\lesssim \frac{qT}{N}}n^{-1/2-i(t_1-t_2+u)}\bar{\chi}_2\chi_1(n)\Big| du\Bigg|^2\\
    \stackon{CS}{\lesssim}& \sup_{|u|\lesssim 1} N^2  \sum_{\substack{(t_1,\chi_1),\\(t_2,\chi_2)\in W}} \Big|\sum_{n\lesssim \frac{q{T_0}}{N}}n^{-1/2-i(t_1-t_2+u)}\bar{\chi}_2\chi_1(n)\Big|^2.
    \end{align*}

    We now apply H\"older's inequality on \begin{align*}
        &\sum_{\substack{(t_1,\chi_1),\\(t_2,\chi_2)\in W}} \Big|\sum_{n\lesssim \frac{q{T_0}}{N}}n^{-1/2-i(t_1-t_2+u)}\bar{\chi}_2\chi_1(n)\Big|^2\\
        \stackon{H\"older}{\leq}&  \quad \Bigg(\sum_{\substack{(t_1,\chi_1),\\(t_2,\chi_2)\in W}} 1 \Bigg)^{(k-1)/k} \Bigg(\sum_{\substack{(t_1,\chi_1),\\(t_2,\chi_2)\in W}}
            \Big|\sum_{n\lesssim \frac{q{T_0}}{N}}n^{-1/2-i(t_1-t_2+u)}\bar{\chi}_2\chi_1(n)\Big|^{2k}\Bigg)^{1/k}\\
        =&|W|^{2-2/k}\Bigg(\sum_{\substack{(t_1,\chi_1),\\(t_2,\chi_2)\in W}}
        \Big|\sum_{n\lesssim \big(\frac{q{T_0}}{N}\big)^k} b_n n^{-1/2-i(t_1-t_2)}\bar{\chi}_2\chi_1(n)\Big|^2\Bigg)^{1/k}
    \end{align*}
    where each $|b_n|\lesssim 1$ by the divisor bound. We can thus apply Heath-Brown's theorem to get \[
    S_2 \lesssim N^2 |W|^{2-2/k}(|W|M^k+|W|^2+W^{5/4}(qT)^{1/2})^{1/k},
    \]where $M=qT/N$. This gives the proposition upon reduction.
\end{proof}