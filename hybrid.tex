\section{Towards a Hybrid Zero Density Result}
\begin{theorem}[Heath-Brown]
    Let $\mathcal{S}=\{(t_j,\chi_j)\}$ be one-separate, primitive characters of modulus $q$. Then 
    \[
        \sum_{\substack{(t_1,\chi_1)\\(t_2,\chi_2)}}\left|\sum_{n=1}^{N} b_n n^{-i(t_1-t2)}\chi_1\bar{\chi}_2(n)\right|^2 \ll  |\mathcal{S}|N^2+ |\mathcal{S}|^2N + |\mathcal{S}|^{5/4}(qT)^{1/2}N.
    \]
\end{theorem}
\textit{Here I will document some progress with working with
the hybrid version of Guth-Maynard. This section will be removed when I send the draft to Prof. Wunsch.}
Let $\omega$ be a bump function supported on \textit{[placeholder]}. Let $N=T^{[placeholder]}$. We denote \[
D_N(t,\chi) = \sum_{n\sim N} \omega\left(\frac{n}{N}\right)b_n \chi(n) n^{it}
\]
and $\mathcal{S}=\{(t_j,\chi_j)\}_{j\leq |\mathcal{S}|}$ such that 
\[
    |D_N(t_j,\chi_j)|\geq V
\]
for all $(t_j,\chi_j)\in\mathcal{S}$.
We also let $0\leq t_j\leq T$ to be $T^\epsilon$ separated, so if 
$j\neq k$ and $\chi_j=\chi_k$, $|t_j-t_k|\geq T^\epsilon$. \textcolor{red}{We can also
let $\chi_j$'s be Dirichlet characters modulo $q$ (potentially primitive if we need this hypothesis later).}

Here we write $M$ a $|\mathcal{S}\times N|$ matrix with entries
\[
    M_{t_j,\chi_j,n} = \chi_j(n)n^{it_j}
\]
for $(t_j,\chi_j)\in\mathcal{S}$ and $n\sim N$.
Thus by the same reasoning that $(M\vec{b})_j=D_N(t_j,\chi_j)$,
we want to bound the trace of the matrix \[
\textrm{tr}((M^*M)^3)
\]
which is formalized in section $4$ of Guth and Maynard's proof.
We see that \begin{align*}
    (MM^*)_{t_j,t_k} = \sum_{n\sim N} \omega\left(\frac{n}{N}\right)^2 n^{i(t_k-t_j)}\bar{\chi}_j\chi_k(n),
\end{align*}
so that \begin{align*}
    \textrm{tr}(MM^*) = |\mathcal{S}|\sum_{n\sim N} \omega\left(\frac{n}{N}\right)^2
\end{align*}
and Lemmas 4.1-4.4 (Large values controlled by singular values, bound for singular values in terms of trace,
Principle of non-stationary phase, Hilbert-Schmidt Norm estimate) can be adapted directly from the paper.

For the expansion of the trace (analog to lemma 4.5), 
\begin{align*}
    (M^*M)_{n_1,n_2} = \sum_{(t_j,\chi_j)\in \mathcal{S}} \omega\left(\frac{n_1}{N}\right)\omega\left(\frac{n_2}{N}\right)
    \bar{\chi}_j(n_1)\chi_j(n_2)n_1^{-it_j}n_2^{it_j}
\end{align*}
so that \begin{align*}
    \textrm{tr}((M^*M)^3)=\sum_{\substack{(t_1,\chi_1),\\(t_2,\chi_2),\\(t_3,\chi_3)\in\mathcal{S}}}\sum_{n_1,n_2,n_3\sim N} & 
    \omega\left(\frac{n_1}{N}\right)^2 \left(\frac{n}{N}\right)^{i(t_1-t_3)}\chi_1\bar{\chi}_3(n_1)\\
    \times \ &\omega\left(\frac{n_2}{N}\right)^2 \left(\frac{n}{N}\right)^{i(t_2-t_1)}\chi_2\bar{\chi}_1(n_2)\\
    \times \ &\omega\left(\frac{n_3}{N}\right)^2 \left(\frac{n}{N}\right)^{i(t_3-t_2)}\chi_3\bar{\chi}_2(n_3).
\end{align*}
Let $h_t(u)=\omega(u)^2u^{it}$,
so by applying Poisson summation to the inner sum, we have \begin{align*}
    \textrm{tr}((M^*M)^3)=\sum_{\substack{(t_1,\chi_1),\\(t_2,\chi_2),\\(t_3,\chi_3)\in\mathcal{S}}}
    \frac{N^3}{q^3}\sum_{m\in\mathbb{Z}^3}&\sum_{x\in (\mathbb{Z}/q\mathbb{Z})^3}\chi_1\bar{\chi}_3(x_1)\chi_2\bar{\chi}_1(x_2)\chi_3\bar{\chi}_2(x_3) e\left(\frac{-x\cdot m}{q}\right)\\
    \times \ &\hat{h}_{t_1-t_3}\left(\frac{Nm_1}{q}\right)\hat{h}_{t_2-t_1}\left(\frac{Nm_2}{q}\right)\hat{h}_{t_3-t_2}\left(\frac{Nm_3}{q}\right) 
\end{align*}
Finally, we exchange the 2 outermost integrals. And split the sum over $m$ into four parts (same as Guth Maynard) $S_0+ S_1+S_2+S_3$,
where $S_j$ runs over the values of $m$ with exactly $j$ non-zero entries.
\subsection{$S_0$ bound}
$S_0$ only has one term corresponding to $m=0$. By the principle of non-stationary phase, $\hat{h}_t(0)$ has rapid decay in $t$, so contributes $O(T^{-1000})$ except possibly when $t_1,t_2,t_3$ are not $T^\epsilon$ separated.
Moreover, by the orthogonality of characters, all three terms $\chi_a\bar{\chi}_b$ must be principal to have non-zero contribution,
so this fixes the sum to be across $(t_1,\chi_1)=(t_2,\chi_2)=(t_3,\chi_3)$ to give a $N^3\phi(q)^3|\mathcal{S}|\|\omega\|_{L_2}^6/q^3$ term.
So that \begin{align*}
    \textrm{tr}((M^*M)^3)=\frac{N^3\phi(q)^3}{q^3}|\mathcal{S}|\|\omega\|_{L_2}^6 + \sum_{m\in\mathbb{Z}^3 - \{0\}} I_m + O(T^{-100}),
\end{align*}
where \begin{align*}
    I_m=\frac{N^3}{q^3}\sum_{\substack{(t_1,\chi_1),\\(t_2,\chi_2),\\(t_3,\chi_3)\in\mathcal{S}}} &\sum_{x\in (\mathbb{Z}/q\mathbb{Z})^3}\chi_1\bar{\chi}_3(x_1)\chi_2\bar{\chi}_1(x_2)\chi_3\bar{\chi}_2(x_3) e\left(\frac{-x\cdot m}{q}\right)\\
    \times \ &\hat{h}_{t_1-t_3}\left(\frac{Nm_1}{q}\right)\hat{h}_{t_2-t_1}\left(\frac{Nm_2}{q}\right)\hat{h}_{t_3-t_2}\left(\frac{Nm_3}{q}\right).
\end{align*}
which gives the analogous Lemmas 4.5 and 4.6.
\subsection{$S_1$ bound}
\begin{proposition}
    $S_1=O\epsilon(T^{-10})$.
\end{proposition}
By symmetry, we sum $I_m$ across all $m=(0,0,m_3\neq 0)$ at a cost of a factor of $3$.
We then have \begin{align*}
    I_m=\frac{N^3}{q^3}\sum_{\substack{(t_1,\chi_1),\\(t_2,\chi_2),\\(t_3,\chi_3)\in\mathcal{S}}} &\sum_{x\in (\mathbb{Z}/q\mathbb{Z})^3}\chi_1\bar{\chi}_3(x_1)\chi_2\bar{\chi}_1(x_2)\chi_3\bar{\chi}_2(x_3) e\left(\frac{-x_3 m_3}{q}\right)\\
    \times \ &\hat{h}_{t_1-t_3}\left(0\right)\hat{h}_{t_2-t_1}\left(0\right)\hat{h}_{t_3-t_2}\left(\frac{Nm_3}{q}\right)
\end{align*}
Again by the orthogonality of characters, the only way to get non-zero contribution is when $\chi_1=\chi_2$ and $\chi_2=\chi_3$. So this reduces to\begin{align*}
    I_m&=\frac{N^3}{q^3}\sum_{\substack{(t_1,\chi_1),\\(t_2,\chi_2=\chi_1),\\(t_3,\chi_3=\chi_1)\in\mathcal{S}}} \ \phi(q)^2 \sum_{x_3\in (\mathbb{Z}/q\mathbb{Z})^{\times}}e\left(\frac{-x_3 m_3}{q}\right)\\
   & \quad  \times  \quad {h}_{t_1-t_3}\left(0\right)\hat{h}_{t_2-t_1}\left(0\right)\hat{h}_{t_3-t_2}\left(\frac{Nm_3}{q}\right)\\
   &=\frac{N^3}{q^3}\phi(q)^2 \frac{\phi(q)}{\phi\left(\frac{q}{\gcd(m_3,q)}\right)}\mu\left(\frac{q}{\gcd(m_3,q)}\right) \sum_{\substack{(t_1,\chi_1),\\(t_2,\chi_2=\chi_1),\\(t_3,\chi_3=\chi_1)\in\mathcal{S}}} \hat{h}_{t_1-t_3}\left(0\right)\hat{h}_{t_2-t_1}\left(0\right)\hat{h}_{t_3-t_2}\left(\frac{Nm_3}{q}\right)\\
\end{align*}
if 
So we trivially bound $S_1$ by\begin{align*}
    |S_1|&\ll  \frac{N^3}{q^3} \phi(q)^3 \sum_{m_3\neq 0}\sum_{\substack{(t_1,\chi_1),\\(t_2,\chi_2=\chi_1),\\(t_3,\chi_3=\chi_1)\in\mathcal{S}}}
   \left|\hat{h}_{t_1-t_3}\left(0\right)\hat{h}_{t_2-t_1}\left(0\right)\hat{h}_{t_3-t_2}\left(\frac{Nm_3}{q}\right)\right|\\
\end{align*}
By the quick decay and boundedness of $\hat{h}_t(\xi)$ in both $\xi$ and $t$ (decays in terms of $\langle \xi \rangle^{-A}\langle t \rangle^A $ or $ \langle \xi \rangle^A\langle t \rangle^{-A} $), we can bound the terms when summed across all $|m_3|>qT^{1+\epsilon}/N$. For the remaining terms, $t_1\neq t_2$, $t_1\neq t_3$ can be bounded by $O(T^{-10})$.
Finally, when $t_1=t_2=t_3$ and $|m_3|$ is small, we get decay in terms of $(N/q)^{-100}$. 

The sum over terms $|m_3|>qT^{1+\epsilon}/N$,
\begin{align*}
   & \sum_{|m_3|>qT^{1+\epsilon}/N}\sum_{\substack{(t_1,\chi_1),\\(t_2,\chi_2=\chi_1),\\(t_3,\chi_3=\chi_1)\in\mathcal{S}}} 
   \left|\hat{h}_{t_1-t_3}\left(0\right)\hat{h}_{t_2-t_1}\left(0\right)\hat{h}_{t_3-t_2}\left(\frac{Nm_3}{q}\right)\right|\\
    \ll&_{\epsilon,A}  \sum_{|m_3|>qT^{1+\epsilon}/N}\sum_{\substack{(t_1,\chi_1),\\(t_2,\chi_2=\chi_1),\\(t_3,\chi_3=\chi_1)\in\mathcal{S}}} 
    \left|T^{A}\left(\frac{Nm_3}{q} \right)^{-A}\right|\\
    \ll & |\mathcal{S}|^3 T^{-10}.
\end{align*}
For $|m_3|\leq qT^{1+\epsilon}/N$, we use the same bound for terms $t_1\neq t_2$ or $t_1\neq t_3$ by the $\hat{h}_t(0)$ terms. When $t_1=t_2=t_3$, we use $m_3\neq 0$
to bound \[
    \left|\hat{h}_{t_3-t_2}\left(\frac{Nm_3}{q}\right)\right| \ll \left(\frac{q}{N}\right)^{-100}
\]
since $|\mathcal{S}|\ll T\phi(q)$, we can set $T\gg q$ to get the contribution of $S_1$ to be $O\epsilon(T^{-10})$.
\subsection{$S_2$ bound}
We write by symmetry \begin{align*}
    S_2= 3\frac{N^3}{q^3}\sum_{m_1,m_2\neq 0}\sum_{\substack{(t_1,\chi_1),\\(t_2,\chi_2),\\(t_3,\chi_3)\in\mathcal{S}}} &\sum_{x\in (\mathbb{Z}/q\mathbb{Z})^3}\chi_1\bar{\chi}_3(x_1)\chi_2\bar{\chi}_1(x_2)\chi_3\bar{\chi}_2(x_3) e\left(\frac{-x_1m_1-x_2m_2}{q}\right)\\
    \times \ &\hat{h}_{t_1-t_3}\left(\frac{Nm_1}{q}\right)\hat{h}_{t_2-t_1}\left(\frac{Nm_2}{q}\right)\hat{h}_{t_3-t_2}\left(0\right)
\end{align*}
Removing zero contributions from $\chi_2\neq \chi_3$ by orthogonality,
we have \begin{align*}
    =3\frac{N^3}{q^3} \phi(q) \sum_{m_1,m_2\neq 0}\sum_{\substack{(t_1,\chi_1),\\(t_2,\chi_2),\\(t_3,\chi_3=\chi_2)\in\mathcal{S}}} &\sum_{x_1,x_2 \in \mathbb{Z}/q\mathbb{Z}}\chi_1\bar{\chi}_3(x_1)\chi_2\bar{\chi}_1(x_2) e\left(\frac{-x_1m_1-x_2m_2}{q}\right)\\
    \times \ &\hat{h}_{t_1-t_3}\left(\frac{Nm_1}{q}\right)\hat{h}_{t_2-t_1}\left(\frac{Nm_2}{q}\right)\hat{h}_{t_3-t_2}\left(0\right)
\end{align*}
Here, we can isolate contributions from the terms where $t_2\neq t_3$ (hence since $\chi_2=\chi_3$, are $T^{\epsilon}$ separated) to be $O(T^{-10})$. For the other terms, we can write
\[
    \hat{h}_t(\xi) = \overline{\hat{h}_{-t}(-\xi)}
\]
to get 
\iffalse
$S_2$
\begin{align*}
    = 3\frac{N^3}{q^3} \hat{h}_{0}\left(0\right)\phi(q) \sum_{m_1,m_2\neq 0}\sum_{\substack{(t_1,\chi_1),\\(t_2,\chi_2)\in\mathcal{S}}} &\sum_{x_1,x_2 \in \mathbb{Z}/q\mathbb{Z}}\chi_1\bar{\chi}_2(x_1)\chi_2\bar{\chi}_1(x_2) e\left(\frac{-x_1m_1-x_2m_2}{q}\right)\\
    \times \ &\hat{h}_{t_1-t_2}\left(\frac{Nm_1}{q}\right)\hat{h}_{t_2-t_1}\left(\frac{Nm_2}{q}\right)
\end{align*}
and by 

we can rewrite this to get
\fi
\[
    S_2 = 3\frac{N^3}{q^3} \phi(q) \hat{h}_{0}\left(0\right) \sum_{\substack{(t_1,\chi_1),\\(t_2,\chi_2)\in\mathcal{S}}} \left|\sum_{m\neq 0} \sum_{x \in \mathbb{Z}/q\mathbb{Z}}\chi_1\bar{\chi}_2(x) e\left(\frac{-mx}{q}\right)
     \hat{h}_{t_1-t_2}\left(\frac{Nm}{q}\right)\right|^2 + O(T^{-10}).
\]
By the principle of non-stationary phase we can move the terms where $|t_1-t_2|<T^\epsilon$ into $O(T^{-10})$ by decay in $Nm/q$. \textit{We also used the fact that there are at most $\phi(q)$ characters mod $q$, so the $O(q^2)$ factor is negligible compared to $N^{-100}$}.

For the other terms where $t_1$ and $t_2$ are $T^\epsilon$ separated, we want to apply Heath Brown's theorem.

\ 
\\ \ 
\\
\textit{rough work}\ \\
At the cost of $O_\epsilon(T^{-100})$ we can add in the term $\hat{h}_{t_1-t_2}(0)$ in when $t_1,t_2$ are at $T^\epsilon$ separated. Let $W$ be the Mellin transform of the function $\omega(x)^2$.
\begin{align*}
    &N^{1+it}\sum_{m\in \mathbb{Z}} \sum_{x \mod q}\chi(x) e\left(\frac{-mx}{q}\right)
    \hat{h}_{t}\left(\frac{Nm}{q}\right) \\
    =& \sum_{n} n^{it}\chi(n)\omega\left(\frac{n}{N}\right)^2\\
    =& \frac{1}{2\pi i}\int_{2-i\infty}^{2+i\infty}W(s)N^sL(s-it,\chi) ds\\
    =& \frac{1}{2\pi i}\int_{-1-i\infty}^{-1+i\infty}W(s)N^sL(s-it,\chi) ds + \varepsilon(\chi)\frac{\phi(q)}{q}N^{1+it}W(1+it)
\end{align*}
where $\varepsilon$ detects if $\chi$ is principal or not. The second term arising from the (potential) pole at $1$ decays rapidly in $t>T^\epsilon$.
For the first term, we let $\chi$ be induced by the primitive $\chi^*$ with modulus $r$, so\[
    L(s-it,\chi)=L(s-it,\chi^*)\prod_{p|q} \left(1-\frac{\chi^*(p)}{p^s}\right).
\]
We also let \[
    G(s) =\frac{\tau(\chi^*)}{i^\delta\sqrt{r}}r^{s-1/2}\pi^{1/2-s}\frac{\Gamma(\frac{1-s+\delta}{2})}{\Gamma(\frac{s+\delta}{2})},
\]
so that $L(s-it,\chi^*)(s) = G(s-it)L(1-s+it,\overline{\chi^*})$. The integral becomes
\begin{align*}
   &\frac{1}{2\pi i}\int_{-1-i\infty}^{-1+i\infty}W(s)N^sL(s-it,\chi) ds \\=&\frac{1}{2\pi i}
   \int_{-1-i\infty}^{-1+i\infty}W(s)N^s
    G(s-it) L(1-s+it,\overline{\chi^*}) \prod_{p|q} \left(1-\frac{\chi^*(p)}{p^s}\right)ds\\
    =& \frac{1}{2\pi i}\int_{-1-i\infty}^{-1+i\infty}W(s)N^s
    G(s-it) 
    \left(\sum_{n\leq M}\frac{\overline{\chi^*}(n)}{n^{1-s+it}}+
    \sum_{n> M}\frac{\overline{\chi^*}(n)}{n^{1-s+it}}
   \right) \prod_{p|q} \left(1-\frac{\chi^*(p)}{p^s}\right)ds
\end{align*}
Where $M$ is a parameter to be determined. The summation is convergent as the real part is larger than $1$.
We thus break up the integral into two pieces according to the two summations $I_1+I_2$. Moving the line of integration of $I_1$ to $\Re(s)=1$ and $I_2$ to $\Re(s)=-2k$,
\begin{align*}
        I_1&= \frac{1}{2\pi}\int_{-\infty}^{\infty}W(1+iu)N^{1+iu}G(1+iu-it)\sum_{n\leq M}\overline{\chi^*}(n)n^{-i(u-t)}\prod_{p|q} \left(1-\frac{\chi^*(p)}{p^{1+iu}}\right)du,\\
        I_2&= \frac{1}{2\pi}\int_{-\infty}^{\infty}W(-2k+iu)N^{-2k+iu}G(-2k+iu-it)\sum_{n> M}\overline{\chi^*}(n)n^{-2k-1-i(u-t)}\prod_{p|q} \left(1-\frac{\chi^*(p)}{p^{-2k+iu}}\right)du.
\end{align*}
By the decay of $W$, we can truncate both integrals to the region $|u|\ll T^\epsilon$.
Moreover, \textit{decay of gamma - might have a typo in GM paper?}
\section{$S_3$ bound}
By non-stationary phase, $I_m$ is negligible for the terms $qT/N\lesssim |m|$, so 
\begin{equation}
    S_3 = \sum_{0<|m_1|,|m_2|,|m_3|\lesssim qT/N} I_m + O(T^{-100}).
\end{equation}
We define \begin{align*}
    R(v,n_1,n_2)&\defeq \sum_{(t,\chi)\in \mathbb{S}} 
    \chi({n_1})\bar{\chi}(n_2)v^{it},\\
    R(v,n)&\defeq R(v,n,1).
\end{align*}
\begin{proposition}
    \[
    |I_m|\ll \phi(q) \frac{N^3}{q^3}  
    \sum_{y_1,y_2\in (\mathbb{Z}/q\mathbb{Z})^\times}\ \int\displaylimits_{\substack{
        |v_1m_1+v_2m_2+m_3|\lesssim \frac{q}{N}\\
        \frac{1}{2}\leq v_1,v_2\leq 2
    }} \left| R\left(\frac{v_1}{v_2},y_1,y_2\right)
    R(v_2,y_2)R\left(v_1,y_1\right)\right| dv_1 \ dv_2  + O(T^{-100}).\\
    \]
    Moreover, if $|m_1|\leq|m_2|\leq |m_3|$, $|I_m|=O(T^{-100})$ unless $|m_2|\asymp|m_3|$.
\end{proposition}
\begin{proof}
    
Recall
\begin{align*}
    I_m=\frac{N^3}{q^3}\sum_{\substack{(t_1,\chi_1),\\(t_2,\chi_2),\\(t_3,\chi_3)\in\mathcal{S}}} &\sum_{x\in (\mathbb{Z}/q\mathbb{Z})^3}\chi_1\bar{\chi}_3(x_1)\chi_2\bar{\chi}_1(x_2)\chi_3\bar{\chi}_2(x_3) e\left(\frac{-x\cdot m}{q}\right)\\
    \times \ &\hat{h}_{t_1-t_3}\left(\frac{Nm_1}{q}\right)\hat{h}_{t_2-t_1}\left(\frac{Nm_2}{q}\right)\hat{h}_{t_3-t_2}\left(\frac{Nm_3}{q}\right).
\end{align*}
Expanding the integrals, 
\begin{align*}
    I_m=\frac{N^3}{q^3}\sum_{\substack{(t_1,\chi_1),\\(t_2,\chi_2),\\(t_3,\chi_3)\in\mathcal{S}}} &\sum_{x\in (\mathbb{Z}/q\mathbb{Z})^3}\chi_1\bar{\chi}_3(x_1)\chi_2\bar{\chi}_1(x_2)\chi_3\bar{\chi}_2(x_3) e\left(\frac{-x\cdot m}{q}\right)\\
    \times \ &
    \int_{\reals^3}\mathbf{\tilde{\omega}}(\mathbf{u})u_1^{i(t_1-t_3)}u_2^{i(t_2-t_1)}u_3^{i(t_3-t_2)}e\left(\frac{-N\mathbf{m}\cdot \mathbf{u}}{q}\right)d\mathbf{u},
\end{align*}
where $\tilde{\omega}(\mathbf{u})=\omega(u_1)^2\omega(u_2)^2\omega(u_3)^2$ is compactly supported.
We now make the substitution $y_1=x_1x_3^{-1}, y_2=x_2x_3^-1 \mod q$ for the summation over $x$, and $v_1=u_1/u_3,v_2=u_2/u_3$ for the integral on the support of $\tilde{\omega}$.
We thus rewrite the sum over $x$ as 
\begin{align*}
    &\sum_{y_1,y_2,x_3\in (\mathbb{Z}/q\mathbb{Z})^\times}
    \chi_1(y_1y_2^{-1})\chi_2(y_2)\chi_3(y_1^{-1})e\left(\frac{-(y_1m_1+y_2m_2+m_3)x_3}{q}\right)\\
    =&
    \sum_{y_1,y_2\in (\mathbb{Z}/q\mathbb{Z})^\times}\chi_1(y_1)\bar{\chi}_1(y_2)\chi_2(y_)\bar{\chi}_3(y_1)\sum_{x_3\in (\mathbb{Z}/q\mathbb{Z})^\times}e\left(\frac{-(y_1m_1+y_2m_2+m_3)x_3}{q}\right),
\end{align*}
where we can use the trivial bound $\phi(q)$ for the innermost sum.
We also rewrite triple integral as 
\begin{align*}
    &\int_{\reals^3}\tilde{\omega}(v_1u_3,v_2u_3,u_3) {\left(\frac{v_1}{v_2}\right)}^{it_1} {\left(v_2\right)}^{it_2}{\left(\frac{1}{v_1}\right)}^{it_3} u_3^2 \ e\left(\frac{-N(v_1m_1+v_2m_2+m_3)u_3}{q}\right)\ dv_1\ dv_2\ du_3\\
    =&\int_{\reals^2}\int_\reals u_3^2 \ \tilde{\omega}(v_1u_3,v_2u_3,u_3) e\left(\frac{-N(v_1m_1+v_2m_2+m_3)u_3}{q}\right)  du_3 \ {\left(\frac{v_1}{v_2}\right)}^{it_1} {\left(v_2\right)}^{it_2}{\left(\frac{1}{v_1}\right)}^{it_3}  \ dv_1\ dv_2.\\
\end{align*}
The integrand of the innermost integral is non-zero only if \[
    v_1u_3,v_2u_3,u_3\sim N.
\]
Importantly, this requires $1/2 \leq v_1,v_2 \leq 2$, so we can truncate the outermost integrals to these regions. Moreover, by repeated integration by parts, this integral is $O_{\epsilon, A}(T^{-A})$ for any $|v_1m_1+v_2m_2+m_3|>qT^\epsilon/N$.
So 
\begin{align*}
    |I_m|\ll &\phi(q) \frac{N^3}{q^3}  
    \sum_{y_1,y_2\in (\mathbb{Z}/q\mathbb{Z})^\times}\left|\ \int\displaylimits_{\substack{
        |v_1m_1+v_2m_2+m_3|\lesssim \frac{q}{N}\\
        \frac{1}{2}\leq v_1,v_2\leq 2
    }} R\left(\frac{v_1}{v_2},y_1,y_2\right)
    R(v_2,y_2)R\left(\frac{1}{v_1},1,y_1\right) dv_1 \ dv_2 \right|\\ &+ O(T^{-100}).\\
\end{align*}
Since $|R(v_1^{-1},1,y_1)|=|R(v_1,y_1)|$, we have the first part of the proposition.
The second part of the proposition follows from the integral bounds $|v_1m_1+v_2m_2+m_3|\lesssim q/N$
 and $v_1,v_2\asymp 1$. These force $|m_2| \asymp|m_3|$, or else the integral will be zero.

\end{proof}

Adapting from Guth and Maynard, when $|m_2|\asymp|m_3|$, the domain of integration can be written as\begin{align*}
    |v_1m_1+v_2m_2+m_3|\lesssim \frac{q}{N} \implies \left|v_2 - \frac{v_1m_1+m_3}{-m_2}\right|\lesssim \frac{q}{|m_2|N} \asymp \frac{q}{|m_3|N}.
\end{align*}
Thus, 


\textcolor{red}{TODO}
\begin{proposition}
    There is a choice of $0<M_1\leq M \lesssim qT/N$ such that \[
        S_3\lesssim \phi(q)\frac{N^2}{M}\sum_{|m_1|\sim M_1,|m_2|,|m_3|\sim M}\tilde{I}_m+O(T^{-100}).
    \]
    where \[
    \tilde{I}_m\defeq \sum_{y_1,y_2\in (\mathbb{Z}/q\mathbb{Z})^\times}\ \int_{v_1\asymp 1} 
         \left|R\left(v_1,y_1\right) \tilde{R}_M\left(\frac{m_1v_1+m_3}{-m_2v_1},y_2,y_1\right)
        \tilde{R}_M(\frac{m_1v_1+m_3}{-m_2},y_2)\right| dv_1.
    \]
\end{proposition}


\begin{proof}
    \textcolor{red}{TODO}
\end{proof}


\begin{lemma} \label{secondmoment}
    Let $\mathbb{S}=\{(t_j,\chi_j)\}$, and the $t$'s are contained in an interval of length $T$, and are $T^\epsilon$-separated for the same character. Then \[
        \sum_{y\in (\mathbb{Z}/q\mathbb{Z})^\times} \int_{v\asymp 1} 
        \left|R\left(v,y\right)\right|^2dv \ll_{\epsilon} \phi(q)|\mathbb{S}|.
    \]
\end{lemma}
\begin{proof}
    We have \[
    |R(v,y)|^2 = \sum_{(t_1,\chi_1),(t_2,\chi_2)\in \mathbb{S}}
    \chi_1\bar{\chi}_2(y)v^{i(t_1-t_2)}.
    \]
    Let $\psi$ be a bump function supported on $v\asymp 1$ and equals $1$ on the domain of integration in the lemma.
    By orthogonality of characters, \begin{align*}
        \sum_{y\in (\mathbb{Z}/q\mathbb{Z})^\times} \int_{v\asymp 1} 
        \left|R\left(v,y\right)\right|^2dv 
        \leq&\sum_{y\in (\mathbb{Z}/q\mathbb{Z})^\times} \int 
        \psi(v)\left|R\left(v,y\right)\right|^2dv 
        \\=&
        \phi(q)\int \psi(v)
        \sum_{\substack{(t_1,\chi_1),(t_2,\chi_2)\in \mathbb{S}\\ \chi_1=\chi_2}}v^{i(t_1-t_2)}
        dv\\
        =&
        \phi(q)\sum_{\substack{(t_1,\chi_1),(t_2,\chi_2)\in \mathbb{S}\\ \chi_1=\chi_2}}\int \psi(v)
        v^{i(t_1-t_2)}
        dv.
    \end{align*}
    In the sum, the terms $t_1=t_2$ contribute $O(|\mathbb{S}|)$. If $t_1\neq t_2$, then $|t_1-t_2|\geq T^\epsilon$. The integral in this case is $O_\epsilon(T^-100)$ and is negligible.
\end{proof}
\begin{lemma}\label{fourthmoment}
    Let $E(\mathbb{S})=\#\{(t_1,\chi_1),(t_2,\chi_2),(t_3,\chi_3),(t_4,\chi_4)\in \mathbb{S}  :  |t_1+t_2-t_3-t_4|\leq 1, \chi_1\chi_2=\chi_3\chi_4\}$. Then \[
        \sum_{y\in (\mathbb{Z}/q\mathbb{Z})^\times} \int_{v\asymp 1} 
        \left|R\left(v,y\right)\right|^4dv  \lesssim \phi(q)E(\mathbb{S}).
    \]
\end{lemma}
\begin{proof}
    We have \[
    |R(v,y)|^4 = \sum_{\substack{(t_1,\chi_1),(t_2,\chi_2),\\ (t_3,\chi_3),(t_4,\chi_4)\in \mathbb{S}}}
    \chi_1{\chi}_2\bar{\chi_3}\bar{\chi_4}(y)v^{i(t_1+t_2-t_3-t_4)}.
    \]
    So again by the orthogonality of characters, \begin{align*}
        \sum_{y\in (\mathbb{Z}/q\mathbb{Z})^\times} \int_{v\asymp 1} 
        \left|R\left(v,y\right)\right|^4dv = & \phi(q)
        \sum_{\substack{(t_1,\chi_1),(t_2,\chi_2),\\ (t_3,\chi_3),(t_4,\chi_4)\in \mathbb{S}\\ \chi_1\chi_2=\chi_3\chi_4}} \int_{v\asymp 1} v^{i(t_1+t_2-t_3-t_4)} dv.
    \end{align*}
    Similar to the previous proof, we can introduce a bump function for the integral, and restrict the summation to the terms $|t_1+t_2-t_3-t_4|\leq T^\epsilon$ with an error of $O_\epsilon(T^{-100})$. The remaining terms in the summation contribute $O(E(\mathbb{S}))$.
\end{proof}
\begin{proof}[{Proof of proposition [placeholder]}]
    We first appler H\"older's inequality on the integral to get 
    \begin{align*}
        \tilde{I}_m \leq  \sum_{y_1,y_2\in (\mathbb{Z}/q\mathbb{Z})^\times}& \left(\int_{v_1\asymp 1} 
        \left|R\left(v_1,y_1\right)\right|^2dv_1\right)^{1/2} \left(\int_{v_1\asymp 1} \left|\tilde{R}_M\left(\frac{m_1v_1+m_3}{-m_2v_1},y_2,y_1\right)\right|^4 dv_1\right)^{1/4} \\& \left(\int_{v_1\asymp 1} 
       \left| \tilde{R}_M(\frac{m_1v_1+m_3}{-m_2},y_2)\right|^{4} dv_1\right)^{1/4},
    \end{align*}
    Notice the first integral is independent of $y_2$, for sum of the second and third integrals over $y_2$, we apply Cauchy-Schwarz to get\begin{align*}
        &\sum_{y_2\in (\mathbb{Z}/q\mathbb{Z})^\times}\left(\int_{v_1\asymp 1} \left|\tilde{R}_M\left(\frac{m_1v_1+m_3}{-m_2v_1},y_2,y_1\right)\right|^4 dv_1\right)^{1/4} \left(\int_{v_1\asymp 1} 
       \left| \tilde{R}_M(\frac{m_1v_1+m_3}{-m_2},y_2)\right|^{4} dv_1\right)^{1/4}\\
       \leq& \left(\sum_{y_2\in (\mathbb{Z}/q\mathbb{Z})^\times}\left(\int_{v_1\asymp 1} \left|\tilde{R}_M\left(\frac{m_1v_1+m_3}{-m_2v_1},y_2,y_1\right)\right|^4 dv_1\right)^{1/2}\right)^{1/2}\\ &\quad \quad
       \left(
       \sum_{y_2\in (\mathbb{Z}/q\mathbb{Z})^\times} \left(\int_{v_1\asymp 1}
       \left| \tilde{R}_M(\frac{m_1v_1+m_3}{-m_2},y_2)\right|^{4} dv_1 \right)^{1/2}\right)^{1/2}
       \\ 
       \leq&  \phi(q)^{\frac{1}{2}} \left(\sum_{y_2\in (\mathbb{Z}/q\mathbb{Z})^\times}\int_{v_1\asymp 1} \left|\tilde{R}_M\left(\frac{m_1v_1+m_3}{-m_2v_1},y_2,y_1\right)\right|^4 dv_1\right)^{1/4}\\ &\quad \quad
       \left(
       \sum_{y_2\in (\mathbb{Z}/q\mathbb{Z})^\times} \int_{v_1\asymp 1}
       \left| \tilde{R}_M(\frac{m_1v_1+m_3}{-m_2},y_2)\right|^{4} dv_1\right)^{1/4} \\
       \leq&  \phi(q)^{\frac{1}{2}} \left(\sum_{y_3\in (\mathbb{Z}/q\mathbb{Z})^\times}\int_{v_1\asymp 1} \left|\tilde{R}_M\left(\frac{m_1v_1+m_3}{-m_2v_1},y_3\right)\right|^4 dv_1\right)^{1/4}\\ &\quad \quad
       \left(
       \sum_{y_2\in (\mathbb{Z}/q\mathbb{Z})^\times} \int_{v_1\asymp 1}
       \left| \tilde{R}_M(\frac{m_1v_1+m_3}{-m_2},y_2)\right|^{4} dv_1\right)^{1/4}
       \\ \lesssim&
       \phi(q)E(\mathbb{S})^{\frac{1}{2}}(M/M_1)^{1/4}\leq\phi(q)E(\mathbb{S})^{\frac{1}{2}}M/M_1.
       \end{align*}
       where in the penultimate step, we made a change of variables $y_3=y_2y_1^{-1}$. In the last step we change variables of integration $u=(m_1v_1+m_3)/(-m_2v_1)$ and $u=(m_1v_1+m_3)/(-m_2)$ with a Jacobian factor of $\asymp 1$ and $\sim M/M_1$ respectively. 
       For the first integral, applying Cauchy Schwarz gives \begin{align*}
        \sum_{y_1\in (\mathbb{Z}/q\mathbb{Z})^\times} \left(\int_{v_1\asymp 1} 
        \left|R\left(v_1,y_1\right)\right|^2dv_1\right)^{1/2} 
        \leq  \phi(q)^{\frac{1}{2}}\left(\sum_{y_1\in (\mathbb{Z}/q\mathbb{Z})^\times} \int_{v_1\asymp 1} 
        \left|R\left(v_1,y_1\right)\right|^2dv_1\right)^{1/2} 
        \ll_{\epsilon} |\mathbb{S}|.
       \end{align*}
       Combined, this gives \[
       S_3 \leq 
       \]
\end{proof}

\section{Energy bound}
Here we provide the generalization for the orthogonal energy bound for Guth and Maynard's result. 

The idea for bounding energy is similar, if $\chi_1\chi_2=\chi_3\chi_4$ and $|t_1+t_2-t_3-t_4|$ is small, we should expect $|D_N(t_1+t_2-t_3,\chi_1\chi_2\bar{\chi}_3)|\simeq |D_N(t_4,\chi_4)|>N^\sigma$.
\begin{lemma}
    \[
    D_N(t,\chi)\lesssim \int_{|u-t|\lesssim 1} |D_N(u,\chi)|du + O(T^{-100}).
    \]
\end{lemma}
\begin{proof}
    \begin{align*}
        D_N(t,\chi)=\sum_n \omega\left(\frac{n}{N}\right) b_n n^{it} \psi\left(\frac{\log n}{2\pi}\right)
    \end{align*}
\end{proof}