\section{Towards a Hybrid Zero Density Result}
\begin{theorem}[Heath-Brown]
    \label{heathbrown}
    Let $\mathcal{S}=\{(t_j,\chi_j)\}$ be one-separate, primitive characters of modulus $q$. Then 
    \[
        \sum_{\substack{(t_1,\chi_1)\\(t_2,\chi_2)}}\left|\sum_{n=1}^{N} b_n n^{-i(t_1-t2)}\chi_1\bar{\chi}_2(n)\right|^2 \lesssim  |\mathcal{S}|N^2+ |\mathcal{S}|^2N + |\mathcal{S}|^{5/4}(qT)^{1/2}N.
    \]
\end{theorem}
\textit{Here I will document some progress with working with
the hybrid version of Guth-Maynard. This section will be removed when I send the draft to Prof. Wunsch.}
Let $\omega$ be a bump function supported on \textit{[placeholder]}. Let $N=T^{[placeholder]}$. We denote \[
D_N(t,\chi) = \sum_{n\sim N} \omega\left(\frac{n}{N}\right)b_n \chi(n) n^{it}
\]
and $\mathcal{S}=\{(t_j,\chi_j)\}_{j\leq |\mathcal{S}|}$ such that 
\[
    |D_N(t_j,\chi_j)|\geq V
\]
for all $(t_j,\chi_j)\in\mathcal{S}$.
We also let $0\leq t_j\leq T$ to be $T^\epsilon$ separated, so if 
$j\neq k$ and $\chi_j=\chi_k$, $|t_j-t_k|\geq T^\epsilon$. \textcolor{red}{We can also
let $\chi_j$'s be Dirichlet characters modulo $q$ (potentially primitive if we need this hypothesis later).}

Here we write $M$ a $|\mathcal{S}\times N|$ matrix with entries
\[
    M_{t_j,\chi_j,n} = \chi_j(n)n^{it_j}
\]
for $(t_j,\chi_j)\in\mathcal{S}$ and $n\sim N$.
Thus by the same reasoning that $(M\vec{b})_j=D_N(t_j,\chi_j)$,
we want to bound the trace of the matrix \[
\textrm{tr}((M^*M)^3)
\]
which is formalized in section $4$ of Guth and Maynard's proof.
We see that \begin{align*}
    (MM^*)_{t_j,t_k} = \sum_{n\sim N} \omega\left(\frac{n}{N}\right)^2 n^{i(t_k-t_j)}\bar{\chi}_j\chi_k(n),
\end{align*}
so that \begin{align*}
    \textrm{tr}(MM^*) = |\mathcal{S}|\sum_{n\sim N} \omega\left(\frac{n}{N}\right)^2
\end{align*}
and Lemmas 4.1-4.4 (Large values controlled by singular values, bound for singular values in terms of trace,
Principle of non-stationary phase, Hilbert-Schmidt Norm estimate) can be adapted directly from the paper.

For the expansion of the trace (analog to lemma 4.5), 
\begin{align*}
    (M^*M)_{n_1,n_2} = \sum_{(t_j,\chi_j)\in \mathcal{S}} \omega\left(\frac{n_1}{N}\right)\omega\left(\frac{n_2}{N}\right)
    \bar{\chi}_j(n_1)\chi_j(n_2)n_1^{-it_j}n_2^{it_j}
\end{align*}
so that \begin{align*}
    \textrm{tr}((M^*M)^3)=\sum_{\substack{(t_1,\chi_1),\\(t_2,\chi_2),\\(t_3,\chi_3)\in\mathcal{S}}}\sum_{n_1,n_2,n_3\sim N} & 
    \omega\left(\frac{n_1}{N}\right)^2 \left(\frac{n}{N}\right)^{i(t_1-t_3)}\chi_1\bar{\chi}_3(n_1)\\
    \times \ &\omega\left(\frac{n_2}{N}\right)^2 \left(\frac{n}{N}\right)^{i(t_2-t_1)}\chi_2\bar{\chi}_1(n_2)\\
    \times \ &\omega\left(\frac{n_3}{N}\right)^2 \left(\frac{n}{N}\right)^{i(t_3-t_2)}\chi_3\bar{\chi}_2(n_3).
\end{align*}
Let $h_t(u)=\omega(u)^2u^{it}$,
so by applying Poisson summation to the inner sum, we have \begin{align*}
    \textrm{tr}((M^*M)^3)=\sum_{\substack{(t_1,\chi_1),\\(t_2,\chi_2),\\(t_3,\chi_3)\in\mathcal{S}}}
    \frac{N^3}{q^3}\sum_{m\in\mathbb{Z}^3}&\sum_{x\in (\mathbb{Z}/q\mathbb{Z})^3}\chi_1\bar{\chi}_3(x_1)\chi_2\bar{\chi}_1(x_2)\chi_3\bar{\chi}_2(x_3) e\left(\frac{-x\cdot m}{q}\right)\\
    \times \ &\hat{h}_{t_1-t_3}\left(\frac{Nm_1}{q}\right)\hat{h}_{t_2-t_1}\left(\frac{Nm_2}{q}\right)\hat{h}_{t_3-t_2}\left(\frac{Nm_3}{q}\right) 
\end{align*}
Finally, we exchange the 2 outermost integrals. And split the sum over $m$ into four parts (same as Guth Maynard) $S_0+ S_1+S_2+S_3$,
where $S_j$ runs over the values of $m$ with exactly $j$ non-zero entries.

