
\section{Energy bound}
Here we provide the generalization for the orthogonal energy bound for Guth and Maynard's result. 
\begin{proposition}\label{energybound}
    Let $(qT)^{3/4}\leq N\leq (qT)$. Then
    \begin{align*}
E(W)\lesssim |W|N^{4-4\sigma}+|W|^{21/8}(qT)^{1/4}N^{1-2\sigma}+|W|^3N^{1-2\sigma}.
    \end{align*}
\end{proposition}
The idea for bounding energy is similar; if $\chi_1\chi_2=\chi_3\chi_4$ and $|t_1+t_2-t_3-t_4|$ is small, we should expect $|D_N(t_1+t_2-t_3,\chi_1\chi_2\bar{\chi}_3)|\approx |D_N(t_4,\chi_4)|>N^\sigma$.
\begin{lemma}
    \label{Dirichletint}
    \[
    D_N(t,\chi)\lesssim \int_{|u-t|\lesssim 1} |D_N(u,\chi)|du + O((qT)^{-100}),
    \]
    uniformly in $\chi$.
\end{lemma}
\begin{proof}
    This is Lemma 11.3 is \cite{GM2024}, with the restriction $\lesssim 1$ in terms of $(qT)^{\epsilon}$ instead of just $T^{\epsilon}$.
\end{proof}
\begin{lemma} We have \[
    E(W)\lesssim N^{-2\sigma}\sum_{n_1,n_2\sim N}\left|R\left(\frac{n_1}{n_2},n_1,n_2\right) \right|^3.
    \]
    
\end{lemma}
\begin{proof}
    We have \begin{align*}    
    E(W)=& \sum_{\substack{(t_1,\chi_1),(t_2,\chi_2),\\ (t_3,\chi_3),(t_4,\chi_4)\in W\\|t_1+t_2-t_3-t_4|\leq 1\\ \chi_1\chi_2=\chi_3\chi_4}} 1 
    \leq N^{-2\sigma}\sum_{\substack{(t_1,\chi_1),(t_2,\chi_2),\\ (t_3,\chi_3),(t_4,\chi_4)\in W\\ |t_1+t_2-t_3-t_4|\leq 1\\ \chi_1\chi_2=\chi_3\chi_4}} |D_N(t_4,\chi_4)|^2.
    \end{align*}
    Now we apply the previous lemma and Cauchy-Schwarz to get \[
        |D_N(t_4,\chi_4)|^2\lesssim \int_{|u-t_4|\lesssim 1} |D_N(u,\chi_4)|^2du \lesssim  \int_{|u-t_1-t_2+t_3|\lesssim 1} |D_N(u,\chi_1\chi_2\bar{\chi}_3)|^2du,
    \]
   
    Since $\chi_1,\chi_2,\chi_3$ fixes $\chi_4$, and the $t$'s within the same character are $T^\epsilon$ separated, there is $O(1)$ possible pairs of $(t_4,\chi_4)$ for each choice of $(t_1,\chi_1),(t_2,\chi_2),(t_3,\chi_3)$, so \begin{align*}    
        E(W)\lesssim& N^{-2\sigma}\sum_{\substack{(t_1,\chi_1),(t_2,\chi_2),\\ (t_3,\chi_3),(t_4,\chi_4)\in W\\ |t_1+t_2-t_3-t_4|\leq 1\\ \chi_1\chi_2=\chi_3\chi_4}} \int_{|u-t_1-t_2+t_3|\lesssim 1} |D_N(u,\chi_1\chi_2\bar{\chi}_3)|^2du\\
        \lesssim&  N^{-2\sigma}\sum_{\substack{(t_1,\chi_1),(t_2,\chi_2),\\ (t_3,\chi_3)\in W}} \int_{|u-t_1-t_2+t_3|\lesssim 1} |D_N(u,\chi_1\chi_2\bar{\chi}_3)|^2du\\
        =& N^{-2\sigma}\sum_{\substack{(t_1,\chi_1),(t_2,\chi_2),\\ (t_3,\chi_3)\in W}} \int_{|u|\lesssim 1} |D_N(t_1+t_2-t_3 + u,\chi_1\chi_2\bar{\chi}_3)|^2du\\
        =&N^{-2\sigma}\sum_{n_1,n_2} b_{n_1}\bar{b}_{n_2} \omega\left(\frac{n_1}{N}\right)\omega\left(\frac{n_2}{N}\right)\int_{|u|\lesssim 1} \left(\frac{n_1}{n_2}\right)^{iu}R\left(\frac{n_1}{n_2},n_1,n_2\right)^2 R\left(\frac{n_2}{n_1},n_2,n_1\right)du\\
        \lesssim & N^{-2\sigma}\sum_{n_1,n_2\sim N}\left|R\left(\frac{n_1}{n_2},n_1,n_2\right)^2 R\left(\frac{n_2}{n_1},n_2,n_1\right)\right|\\
        \lesssim & N^{-2\sigma}\sum_{n_1,n_2\sim N}\left|R\left(\frac{n_1}{n_2},n_1,n_2\right) \right|^3.
        \end{align*}
\end{proof}

\begin{lemma}
    We have 
\[
        \sum_{n_1,n_2\sim N}\left|R\left(\frac{n_1}{n_2},n_1,n_2\right) \right|^2\lesssim |W|N^2+ |W|^2N + |W|^{5/4}(qT)^{1/2}N.
    \]
\end{lemma}
\begin{proof}
    From the definition of $R$, \begin{align*}
        \sum_{n_1,n_2\sim N}\left|R\left(\frac{n_1}{n_2},n_1,n_2\right) \right|^2=& \sum_{n_1,n_2\sim N}\sum_{\substack{(t_1,\chi_1),(t_2,\chi_2)\in W}}
        \chi_1(n_1)\bar{\chi}_1(n_2)\left(\frac{n_1}{n_2}\right)^{it_1}\bar{\chi}_2(n_1){\chi}_2(n_2)\left(\frac{n_1}{n_2}\right)^{-it_2}
        \\
        &=\sum_{\substack{(t_1,\chi_1),(t_2,\chi_2)\in W}}\left|\sum_{n\sim N}
        \chi_1(n_1)\bar{\chi}_2(n_1)n^{i(t_1-t_2)}\right|.
    \end{align*}
    A direct application of Heath Brown's Theorem \ref{heathbrown} gives \[
        \sum_{n_1,n_2\sim N}\left|R\left(\frac{n_1}{n_2},n_1,n_2\right) \right|^2\lesssim |W|N^2+ |W|^2N + |W|^{5/4}(qT)^{1/2}N.
    \]
\end{proof}
The trivial bound for $R\leq |W|$ gives \begin{align*}
    E(W)\lesssim& N^{-2\sigma}\sum_{n_1,n_2\sim N}\left|R\left(\frac{n_1}{n_2},n_1,n_2\right) \right|^3\\
    \lesssim &|W|N^{-2\sigma}\sum_{n_1,n_2\sim N}\left|R\left(\frac{n_1}{n_2},n_1,n_2\right) \right|^2\\
    \lesssim& |W|^2N^{2-2\sigma}+ |W|^3N^{1-2\sigma} + |W|^{9/4}(qT)^{1/2}N^{{1-2\sigma}}.
\end{align*}
The arguments beyond will be adaptations from GM. These involve the refinement of the energy bound through the use of Cauchy-Schwarz\[
    \sum_{n_1,n_2\sim N}\left|R\left(\frac{n_1}{n_2},n_1,n_2\right) \right|^3\leq  \Big(\sum_{n_1,n_2\sim N}\left|R\left(\frac{n_1}{n_2},n_1,n_2\right) \right|^2\Big)^{1/2} \Big(\sum_{n_1,n_2\sim N}\left|R\left(\frac{n_1}{n_2},n_1,n_2\right) \right|^4 \Big)^{1/2}.
\]

\begin{lemma}
    We have \begin{align*}
        \sum_{n_1,n_2\sim N}  \left|R\left(\frac{n_1}{n_2} ,n_1,n_2\right) \right|^4 \lesssim E(W)N^2+ |W|^4N +  E(W)^{3/4}|W|(qT)^{1/2}N.
    \end{align*}
\end{lemma}
\begin{proof}
    Let \[
    U_B=\left\{(u,\chi) \ : u\in \mathbb{Z},\chi \textrm{ character modulus $q$}, \ \#\{((t_1,\chi_1),(t_2,\chi_2))\in W \ : \ \lfloor {t_1-t_2}\rfloor = u, \chi_1\bar\chi_2=\chi\} \sim B \right\}.
    \]
    We split the sum in $R$ as \begin{align*}
        \left|R\left(\frac{n_1}{n_2} ,n_1,n_2\right) \right|^4 =&
        \Bigg|
        \sum_{(t_1,\chi_1),(t_2,\chi_2)\in W}
        \chi_1\bar{\chi}_2(n_1)\bar\chi_1{\chi}_2(n_2) \left(\frac{n_1}{n_2}\right)^{i(t_1-t_2)}\Bigg|^2 \\
        =& \Bigg|
        \sum_{j=0}^{ \lfloor \log_2 |W|\rfloor} \sum_{(u,\chi)\in U_{2^j}}\sum_{\substack{(t_1,\chi_1),(t_2,\chi_2)\in W\\ \lfloor t_1-t_2 \rfloor = u,\\ \chi_1\bar\chi_2=\chi}} 
        \chi_1\bar{\chi}_2(n_1)\bar\chi_1{\chi}_2(n_2)  \left(\frac{n_1}{n_2}\right)^{i(t_1-t_2)}\Bigg|^2\\
        \stackon{CS}{\lesssim}& \sum_{j=0}^{ \lfloor \log_2 |W|\rfloor} \Bigg|
         \sum_{(u,\chi)\in U_{2^j}}\sum_{\substack{(t_1,\chi_1),(t_2,\chi_2)\in W\\ \lfloor t_1-t_2 \rfloor = u,\\ \chi_1\bar\chi_2=\chi}} 
         \chi(n_1)\bar\chi(n_2)  \left(\frac{n_1}{n_2}\right)^{i(t_1-t_2)}\Bigg|^2.
    \end{align*}
    Therefore, \begin{align*}
        \sum_{n_1,n_2\sim N}\left|R\left(\frac{n_1}{n_2},n_1,n_2\right) \right|^4 \lesssim & 
        \sup_{j\leq \lfloor \log_2 |W|\rfloor} \sum_{n_1,n_2\sim N} \Bigg|
        \sum_{(u,\chi)\in U_{2^j}}\sum_{\substack{(t_1,\chi_1),(t_2,\chi_2)\in W\\ \lfloor t_1-t_2 \rfloor = u,\\ \chi_1\bar\chi_2=\chi}} 
       \chi(n_1)\bar{\chi}(n_2) \left(\frac{n_1}{n_2}\right)^{i(t_1-t_2)}\Bigg|^2.\\
    \end{align*}
For the value of $j$ that achieves supremum, \begin{align*}
    &\sum_{n_1,n_2\sim N} \Bigg|
   \sum_{(u,\chi)\in U_{2^j}}\sum_{\substack{(t_1,\chi_1),(t_2,\chi_2)\in W\\ \lfloor t_1-t_2 \rfloor = u,\\ \chi_1\bar\chi_2=\chi}} 
  \chi(n_1)\bar{\chi}(n_2) \left(\frac{n_1}{n_2}\right)^{i(t_1-t_2)}\Bigg|^2\\
  \leq &\sum_{n_1,n_2\sim N} 
  \sum_{\substack{(u_{\alpha},\chi_{\alpha}),\\(u_{\beta},\chi_{\beta})\in U_{2^j}}}\sum_{\substack{(t_1,\chi_1),(t_2,\chi_2)\in W\\ \lfloor t_1-t_2 \rfloor = u_{\alpha},\\ \chi_1\bar\chi_2=\chi_{\alpha}}} \sum_{\substack{(t_3,\chi_3),(t_4,\chi_4)\in W\\ \lfloor t_3-t_4 \rfloor = u_{\beta},\\ \chi_3\bar\chi_4=\chi_{\beta}}} 
 \chi_\alpha\bar\chi_\beta(n_1)\bar{\chi}_\alpha\chi_\beta(n_2) \left(\frac{n_1}{n_2}\right)^{i(t_1-t_2-t_3+t_4)}\\
 =&\sum_{\substack{(u_{\alpha},\chi_{\alpha}),\\(u_{\beta},\chi_{\beta})\in U_{2^j}}}\sum_{\substack{(t_1,\chi_1),(t_2,\chi_2)\in W\\ \lfloor t_1-t_2 \rfloor = u_{\alpha},\\ \chi_1\bar\chi_2=\chi_{\alpha}}} \sum_{\substack{(t_3,\chi_3),(t_4,\chi_4)\in W\\ \lfloor t_3-t_4 \rfloor = u_{\beta},\\ \chi_3\bar\chi_4=\chi_{\beta}}}\Big| \sum_{n\sim N}  \chi_\alpha\bar\chi_\beta(n){n}^{i(t_1-t_2-t_3+t_4)}\Big|^2\\
 \leq & \sum_{\substack{(u_{\alpha},\chi_{\alpha}),\\(u_{\beta},\chi_{\beta})\in U_{2^j}}}
 \Bigg(\sum_{\substack{(t_1,\chi_1),(t_2,\chi_2)\in W\\ \lfloor t_1-t_2 \rfloor = u_{\alpha},\\ \chi_1\bar\chi_2=\chi_{\alpha}}} 1\Bigg)
 \Bigg(\sum_{\substack{(t_3,\chi_3),(t_4,\chi_4)\in W\\ \lfloor t_3-t_4 \rfloor = u_{\beta},\\ \chi_3\bar\chi_4=\chi_{\beta}}} 1\Bigg)
 \sup_{|s|\ll 1}\Big| \sum_{n\sim N}  \chi_\alpha\bar\chi_\beta(n){n}^{i(u_\alpha - u_\beta + s)}\Big|^2\\
 \lesssim & 2^{2j} \sum_{\substack{(u_{\alpha},\chi_{\alpha}),\\(u_{\beta},\chi_{\beta})\in U_{2^j}}} \Bigg(\int_{|s|\lesssim 1}\Big|\sum_{n\sim N}  \chi_\alpha\bar\chi_\beta(n){n}^{i(u_\alpha - u_\beta + s)}\Big|ds \Bigg)^2\\
 \stackon{CS}{\lesssim}& 2^{2j} \sum_{\substack{(u_{\alpha},\chi_{\alpha}),\\(u_{\beta},\chi_{\beta})\in U_{2^j}}} \int_{|s|\lesssim 1}\Big|\sum_{n\sim N}  \chi_\alpha\bar\chi_\beta(n){n}^{i(u_\alpha - u_\beta + s)}\Big|^2ds
\end{align*}
by Lemma \ref{Dirichletint}.
After swapping the order of summation and integration, for each fixed $s$, we can use Heath-Brown's theorem \ref{heathbrown} to bound \[
    \sum_{\substack{(u_{\alpha},\chi_{\alpha}),\\(u_{\beta},\chi_{\beta})\in U_{2^j}}}\Big|\sum_{n\sim N}  \chi_\alpha\bar\chi_\beta(n){n}^{i(u_\alpha - u_\beta + s)}\Big|^2
    \lesssim |U_{2^j}|N^2+ |U_{2^j}|^2N + |U_{2^j}|^{5/4}(qT)^{1/2}N.
\] 
Therefore, \begin{align*}
    \sum_{n_1,n_2\sim N} \left|R\left(\frac{n_1}{n_2} ,n_1,n_2\right) \right|^4 \lesssim 2^{2j}(|U_{2^j}|N^2+ |U_{2^j}|^2N + |U_{2^j}|^{5/4}(qT)^{1/2}N).
\end{align*}
Recall $U_B$ contains $(u,\chi)$ that have is representated $\sim B$ times in $(\lfloor t_1-t_2\rfloor, \chi_1\bar\chi_2)$ for $(t_1,\chi_1),(t_2,\chi_2)\in W$.
Therefore, \[
B|U_B|\ll |W|^2.
\] 
Moreover, each tuple $((t_1,\chi_1),(t_2,\chi_2),(t_3,\chi_3),(t_4,\chi_4))$ satisfying $(\lfloor t_1-t_2\rfloor, \chi_1\bar\chi_2)=(\lfloor t_3-t_4\rfloor, \chi_3\bar\chi_4)$ also satisfies the condition for energy \[
    \chi_1\bar\chi_2=\chi_3\bar\chi_4, |t_1-t_2+t_3-t_4|\lesssim 1.
\]
Therefore, \[
    B^2|U_B|\ll E(W).
\]
For $B^2|U_B|^{5/4}$, we can bound this by \[
B^2|U_B|^{5/4} = (B^{3/2}U_B^{3/4})(B^{1/2}U_B^{1/2})\ll E(W)^{3/4}|W|.
\]
So we have \[
    \sum_{n_1,n_2\sim N} \left|R\left(\frac{n_1}{n_2} ,n_1,n_2\right) \right|^4 \lesssim E(W)N^2+ |W|^4N +  E(W)^{3/4}|W|(qT)^{1/2}N.
\]

\end{proof}


We split the sum in $\sum_{n_1,n_2\sim N}  \left|R\left(\frac{n_1}{n_2} ,n_1,n_2\right) \right|^3$ according to the GCD of $n_1$ and $n_2$. Let $d$ be this GCD, and noticing that when $(q,d)>1$,  we have $R\left(n_1/n_2 ,n_1,n_2\right)=0$,\[
    \sum_{n_1,n_2\sim N}  \left|R\left(\frac{n_1}{n_2} ,n_1,n_2\right) \right|^3\leq \sum_{d\leq D}\sum_{\substack{n_1',n_2'\sim (N/d)\\(n_1',n_2')=1}}  \left|R\left(\frac{n_1'}{n_2'} ,n_1',n_2'\right) \right|^3+\sum_{d> D}\sum_{\substack{n_1',n_2'\sim (N/d)\\(n_1',n_2')=1}}  \left|R\left(\frac{n_1'}{n_2'} ,n_1',n_2'\right) \right|^3,
\]
where $D$ is a parameter to be decided.

We first take care of the large GCD terms using CS. By our two bounds on the second and fourth moments,

\begin{align*}
    \sum_{\substack{n_1',n_2'\sim (N/d)\\(n_1',n_2')=1}}  \left|R\left(\frac{n_1'}{n_2'} ,n_1',n_2'\right) \right|^3\leq& 
    \sum_{n_1',n_2' \sim (N/d)}  \left|R\left(\frac{n_1'}{n_2'} ,n_1',n_2' \right)\right|^3 \\
    \lesssim&
    \Bigg(|W|\Bigg(\frac{N}{d}\Bigg)^2+ |W|^2\frac{N}{d} + |W|^{5/4}(qT)^{1/2}\frac{N}{d}\Bigg)^{1/2}\\ &\quad
    \Bigg(E(W)\Bigg(\frac{N}{d}\Bigg)^2+ |W|^4\frac{N}{d} +  E(W)^{3/4}|W|(qT)^{1/2}\frac{N}{d}\Bigg)^{1/2}.
\end{align*}
Summing over $d>D$ (and using Cauchy-Schwarz to move the summation into the square root) gives the following result.
\begin{proposition}
    We have \begin{align*}
        \sum_{d> D}\sum_{\substack{n_1',n_2'\sim (N/d)\\(n_1',n_2')=1}}  \left|R\left(\frac{n_1'}{n_2'} ,n_1',n_2'\right) \right|^3 \lesssim & \Bigg(|W|\frac{N^2}{D}+ |W|^2N + |W|^{5/4}(qT)^{1/2}N\Bigg)^{1/2}\\ &\quad
        \Bigg(E(W)\frac{N^2}{D}+ |W|^4N+  E(W)^{3/4}|W|(qT)^{1/2}N\Bigg)^{1/2}.
    \end{align*}
\end{proposition}

When $d$ is small, we expect fractions $n_1/n_2$ for $n_1,n_2\sim N/d$ to be well-distributed across $\asymp 1$, so we would like to estimate the summation in terms of an integral. 
\begin{lemma}
    Let $n_1,n_2$ be coprime to $q$. We have for $v\asymp 1$,
    \[
    |R(v,n_1,n_2)|\lesssim T\int_{|s|\lesssim 1/T} |R(v+s)|ds.
    \]
\end{lemma}
\begin{remark}
	$qT$ is not required here. As we will see in the proof, the $1/T$ factor comes from the fact that the $t_j$'s are contained in an interval of length $T$.
\end{remark}
\begin{proof}
    Recall \[
    R(v,n_1,n_2)=\sum_j \chi_j(n_1)\chi_j(n_2)v^{i(t_j)},
    \]
    so it suffices to prove the relation for any \[
    f(v) = \sum_j c_j v^{i(t_j)},
    \]
    uniformly for $|c_j|\leq 1$. Let $\psi$ be a smooth bump that equals $1$ on $[0,1]$, then \begin{align*}
        f(v) = \sum_j c_j v^{i(t_j)} = \sum_j c_j e(t_j\log v/(2\pi))\psi(t_j/T) =  \sum_j c_j e\left(\frac{t_j\log v}{2\pi}\right)\int \hat\psi(\xi)e\left(\frac{t_j\xi}{T}\right)d\xi
    \end{align*}
    The right hand side equals \begin{align*}
        \int \sum_j c_j  \hat\psi(\xi)e\left(t_j\Big(\frac{\xi}{T} +\frac{\log v}{2\pi}\Big)\right)d\xi.
    \end{align*}
    By the rapid decay in $\hat\psi$, we can truncate the integral to the range $|\xi|\lesssim 1$ at a cost of $O((qT)^{-100})$, and use the trivial bound $\hat\psi(\xi)\ll 1$.
    Therefore, \begin{align*}
        |f(v)|\ll \int_{|\xi|\lesssim 1}\left| \sum_j c_j  e\left(t_j\Big(\frac{\xi}{T} +\frac{\log v}{2\pi}\Big)\right)\right|d\xi +O((qT)^{-100}).
    \end{align*}
    We now make the change of variables \[
    \log s= \log v + \frac{2\pi\xi}{T},
    \]
    so this gives a Jacobian factor of $T$, and \[
        |f(v)|\ll T \int_{|\log s-\log v|\lesssim 1/T}\left| \sum_j c_j  e\left(\frac{t_j\log s}{2\pi}\right)\right|ds +O((qT)^{-100})\]
        which gives the bound in the lemma upon making a second change of variables $v'=s-v$.
\end{proof}

Returning to the small GCD terms, we have for $a$ not divisible by $q$,
\begin{align*}
    \sum_{\substack{n_1',n_2'\sim (N/d)\\(n_1',n_2')=1\\n_1'/n_2'\cong a \mod q}}  \left|R\left(\frac{n_1'}{n_2'} ,n_1',n_2'\right) \right|^3
   \  \stackon{H\"older}{\ll}& \  T \sum_{\substack{n_1',n_2'\sim (N/d)\\(n_1',n_2')=1\\n_1'/n_2'\cong a \mod q}}\int\displaylimits_{|s-\frac{n_1'}{n_2'}|\lesssim 1/T}\left|R\left(s,a\right) \right|^3 ds\\
    \leq& T \int\displaylimits_{s\asymp 1} \left|R\left(s,a\right) \right|^3 \Bigg(\sum_{\substack{n_1',n_2'\sim (N/d)\\(n_1',n_2')=1\\n_1'/n_2'\cong a \mod q\\|s-\frac{n_1'}{n_2'}|\lesssim 1/T}} 1\Bigg)ds .
\end{align*}
We now consider for two distinct pairs $(n_1,n_2), (n_3,n_4)$ satisfying $\gcd(n_1,n_2)=\gcd(n_3,n_4)=1, n_1/n_2\cong n_3/n_4\cong a$, \[
\left|\frac{n_1}{n_2}-\frac{n_3}{n_4}\right|=\left|\frac{n_1n_4-n_2n_3}{n_2n_4}\right|\geq \frac{qd^2}{N^2}.
\]
Therefore, the sum is $O(1+{N^2}/({qd^2T}))$. So we have \begin{align*}
    \sum_{d\leq D}\sum_{\substack{n_1',n_2'\sim (N/d)\\(n_1',n_2')=1}}  \left|R\left(\frac{n_1'}{n_2'} ,n_1',n_2'\right) \right|^3 = &
   \sum_{a \mod q} \sum_{d\leq D}\sum_{\substack{n_1',n_2'\sim (N/d)\\(n_1',n_2')=1\\n_1'/n_2'\cong a \mod q}}  \left|R\left(\frac{n_1'}{n_2'} ,n_1',n_2'\right) \right|^3\\
   \ll& T \sum_{d\leq D}\Big(1+\frac{N^2}{qd^2T}\Big) \sum_{a \mod q}\ \int\displaylimits_{s\asymp 1} \left|R\left(s,a\right) \right|^3 ds\\
   \stackon{CS}{\ll}&T \Big(D+\frac{N^2}{qT}\Big)\Bigg(\sum_{a \mod q}\ \int\displaylimits_{s\asymp 1} \left|R\left(s,a\right) \right|^2 ds\Bigg)^{1/2}\Bigg(\sum_{a \mod q}\ \int\displaylimits_{s\asymp 1} \left|R\left(s,a\right) \right|^4 ds\Bigg)^{1/2}\\
   \lesssim& (qTD+N^2)|W|^{1/2}E(W)^{1/2}.
\end{align*}
\begin{proof}[Proof of Proposition \ref{energybound}]
    We have 
    \begin{align*}
    E(W)\lesssim& N^{-2\sigma}\sum_{n_1,n_2\sim N}\left|R\left(\frac{n_1}{n_2},n_1,n_2\right) \right|^3 \\
    {\lesssim}& (qTD+N^2)|W|^{1/2}E(W)^{1/2} + 
    \Bigg(|W|\frac{N^2}{D}+ |W|^2N + |W|^{5/4}(qT)^{1/2}N\Bigg)^{1/2}\\ &\quad
    \Bigg(E(W)\frac{N^2}{D}+ |W|^4N+  E(W)^{3/4}|W|(qT)^{1/2}N\Bigg)^{1/2}.
    \end{align*}
    Picking the choice of $D=N^2/(qT)$, this is the analog for Guth and Maynard's result in bounding energy. The remaining part is to reduce our bounds. The computations are exactly the same except replacing $T$ with $qT$. Following the same steps by considering cases when $|W|>T^{2/3}$ and $|W|<T^{2/3}$ (the explicit computation are in \cite{GM2024}, Proposition 11.1, Lemma 11.9). This reduces to\[
E(W)\lesssim |W|N^{4-4\sigma}+|W|^{21/8}(qT)^{1/4}N^{1-2\sigma}+|W|^3N^{1-2\sigma}.
\]
\end{proof}
