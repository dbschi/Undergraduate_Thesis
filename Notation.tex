\section*{Notation}
We denote summation over the natural numbers $\mathbb{N}$ to be over the positive integers. We always denote $p$ as a prime, and by extension $p_j,p_n$ etc. We denote $e(x)\defeq \exp(2\pi i x)$. For asymptotic behaviors, we write $A\ll B$ if there is an absolute constant $c$ such that $A<cB$, and $A\ll_{\epsilon} B$ if $A<cB$ with $c$ possibly depending on $\epsilon$. Similar to the notation in Guth-Maynard's paper, we write $A\asymp B$ if $A\ll B$ and $B\ll A$ and $A\sim B$ for $B< A\leq 2B$. We also write $A\lesssim B$ if $A\ll_{\epsilon} T^\epsilon B$ for any $\epsilon>0$. Words on top of a relation denote the theorem applied to derive the relation, for instance $\quad\stackeq{Poisson}\quad$ and $\stackon{CS}{\leq}$ means that Poisson summation and Cauchy-Schwarz is applied respectively.
