\section{Refinement of $S_3$}

\begin{proposition}[Refinement of $S_3$ bound] \label{s3refined}
  We have
    \[
S_3 \lesssim (qT)^{2}| W|^{3/2}+N(qT)| W|^{1/2}E( W)^{1/2}.
\]
\end{proposition}

We repeat the arguments in \ref{s_3chapter} until proposition \ref{dyadics_3}.

Recall that at this step, we expressed $v_2$ as a small neighborhood around $v_1$ by the condition \[
|v_1m_1+v_2m_2+m_3|\lesssim \frac{q}{N}.
\] 
We split up the sum in $m_1,m_2,m_3$ according to the gcd's with $q_0$. By the divisor bound, the loss from this step negligible.

Let $d_1\defeq \gcd (m_1,q_0),d_2\defeq \gcd(m_2,q_0), d_3\defeq \gcd (m_3,q_0)$. 


\begin{proposition}[Extension of Proposition \ref{domainreduction}]
    We have 
    \[
        |I_{m,q_0}|\ll \frac{N^3q_0}{q^3}\sum_{\substack{y_1,y_2 \in\mathbb{Z}/q\mathbb{Z} \\ y_1m_1+y_2m_2+m_3\equiv 0 \mod q_0}}\int\displaylimits_{\substack{
            |v_1m_1+v_2m_2+m_3|\lesssim \frac{q}{N}\\
            \frac{1}{2}\leq v_1,v_2\leq 2
    }}  R{\left(\frac{v_1}{v_2},y_1,y_2\right)}R {\left(v_2,y_2\right)}R{\left(\frac{1}{v_1},y_1^{-1}\right)} \ dv_1\ dv_2+O_{\epsilon}(T^{-100}).
        \]

        Moreover, if $|m_1|\leq|m_2|\leq |m_3|$, $|I_{m,q_0}|=O(T^{-100})$ unless $|m_2|\asymp|m_3|$. 
        
        Furthermore, $I_{m,q_0}=0$ unless $\gcd(d_1,d_2)=\gcd(d_2,d_3)=\gcd(d_3,d_1)$.
        
\end{proposition}
\begin{proof}
    It suffices to show the last statement, as we have already proved the first two. 
    Suppose $|I_{m,q_0}|\neq 0$. Then the summation condition $y_1m_1+y_2m_2+m_3\equiv 0 \mod q_0$ is fulfilled for some choice of $y_1,y_2$ which are coprime to $q_0$, as the characters vanish when $y_1,y_2$ are not coprime. We want to show that there exists a value $d$ such that $d_1/d, d_2/d, d_3/d$ are mutually coprime. We proceed by contradiction. Let $d=\gcd(d_1,d_2,d_3)$, and suppose without lost of generality (because $y_1$, $y_2$ are coprime to $q_0$) that $(d_1/d,d_2/d)= k\neq 1$. Naturally $k|(q_0/d)$ Then for some choice of $y_1,y_2, \alpha$, we have \[
    y_1m_1+y_2m_2 = -m_3 + \alpha q_0 \implies y_1 \big(\frac{m_1}{d}\big)+ y_2 \big(\frac{m_1}{d}\big) = -\frac{m_3}{d} + \frac{q_0}{d} \implies (m_3/d) \equiv 0 \mod k.
    \]
\end{proof}

We now set $d=\gcd(d_1,d_2,d_3)$, and set $k_i \defeq m_i/d$, $\tilde{d}_i\defeq d_i/d$ \[
y_1m_1+y_2m_2+m_3\equiv 0 \mod q_0 \implies y_1k_1+y_2k_2 + k_3 \equiv 0 \mod \frac{q_0}{d}.
\]
We can solve the last equation in $y_2$ in terms of $y_1$ as \[
y_2 \quad \begin{cases}
     \equiv \left[\frac{y_1k_1+k_3}{-k_2}\right]_{q_0/d_2} , & \textrm{if $\tilde{d_2}|y_1k_1+k_3$}, \\
      \textrm{has no solutions}, & \rm{otherwise}.
\end{cases}
\]
One caveat is that $k_2/\tilde{d}_2$ might not be coprime to $q_0/d$, for instance if $k_1=k_3=1,k_2=q^2$ and $q_0=q$. Therefore, we cannot naively evaluate $\left[\frac{y_1k_1+k_3}{-k_2}\right]_{q_0/d}$ to find the class of $\left[\frac{y_1k_1+k_3}{-k_2}\right]_{q_0/d_2}$. However, we can isolate the 
non-coprime parts by setting $k_2 = \tilde{k}_2 \tilde{d}_2 \omega$, where $\tilde{k}_2$ is coprime to $q_0/d$. Then the solution set for $y_2$ for when $\tilde{d}_2|y_1k_1+k_3$ is \[
    \left[\frac{y_1k_1+k_3}{-\tilde{k}_2\tilde{d}_2}\right]_{q_0/d_2}\left[\frac{1}{\omega}\right]_{q_0/d_2}+ \alpha \frac{q_0}{d_2}
\]
for integer $\alpha$. If we set $y_2\mod q$ then we can pick $0\leq \alpha < qd_2/q_0 $.


We are now ready to state the new definition for the smoothed function $\tilde{R}$, which intuitively sums up the solutions in $y_2$.

\begin{proposition}[Extension of Proposition \ref{dyadics_3}] \label{folding unfolding}
    Let
    \[
        \tilde{R}^{(q_0,d,d_2,\omega_2)}_M(u,y) \defeq \tilde{R}^{(d)}_M(u,y) \defeq \mathbb{I}_{\tilde{d}_2|y} \Bigg(\sum_{l=0}^{(qd_2/q_0)-1} \frac{q_0}{qd_2} \Big|\tilde{R}_M\Big(u,\left[\frac{y}{\tilde{d}_2}\right]_{q_0/d_2} \left[\frac{1}{\omega_2}\right]_{q_0/d_2} + l \frac{q_0}{d_2}\Big)\Big|^2\Bigg)^{1/2}.
    \]
    Then the following hold:\begin{enumerate}
        \item $\tilde{R}^{(d)}_M$ is $q_0/d$-periodic in the second argument.
        \item $\sum_{a\mod q_0/d} \int_{u\asymp 1 }|\tilde{R}^{(d)}_M(u,a)|^2 du\lesssim\frac{q_0}{d_2}| W|$.
        \item $\sum_{a\mod q_0/d} \int_{u\asymp 1 }|\tilde{R}^{(d)}_M(u,a)|^4 du\lesssim\frac{q_0}{d_2}E( W)$.
    \end{enumerate}
    Moreover, there is a choice of $d_1,d_2,d_3|q_0, q_0|q$, $0<M_1\leq M\lesssim qT/N$ such that  \[
        S_3\lesssim \frac{N^2d_2}{Mq} \sum_{\substack{\omega_1,\omega_2,\omega_3\\ (**)}}  \sum_{\substack{|m_1|\sim M_1,\\|m_2|,|m_3|\sim M\\
        (***)}}\sum_{\substack{y_1\mod q\\ (y_1,q)=1} }\int_{v_1\asymp 1}
        \tilde{R}_M^{(d)}\Big(\frac{k_1v_1+k_3}{-k_2v_1},\frac{k_1y_1+k_3}{-\tilde{k}_2y_1}\Big)\tilde{R}_M^{(d)}\Big(\frac{k_1v_1+k_3}{-k_2},\frac{k_1y_1+k_3}{-\tilde{k}_2}\Big)|R\left(v_1,y_1\right)| dv_1,
    \]
    where the summation conditions are$(**)$ on $\omega_i$'s are that $\omega_i<qT/N$ only has prime factors that are also prime factors of $d_i$, $(***)$ on $m_i$'s are that
    $m_i=d_i\tilde{k}_i\omega_i$ where $(\tilde{k}_i,q_0)=1$. 
\end{proposition}
\begin{remark}
    As a recall, we defined $d\defeq \gcd(d_1,d_2,d_3)$, $k_i\defeq m_i/d$, $\tilde{d}_i=\gcd(k_i,q_0/d)$, $\tilde{k}_i=k_i/(\tilde{d}_i\omega_i) = m_i/(d_i\omega_i)$.
\end{remark}
\begin{proof}
    The first statement is evident from construction. We have $[(y+q_0/d)/\tilde{d}_2] \equiv [y/\tilde{d}_2 + q_0/d_2] \equiv [y/\tilde{d}_2]$.
    
    For the second statement, we have \begin{align*}
        \sum_{a\mod q_0/d} \int_{u\asymp 1 } |\tilde{R}^{(d)}_M(u,a)|^2 du =&  \sum_{a\mod q_0/d} \int_{u\asymp 1 }\mathbb{I}_{\tilde{d}_2|y} \sum_{l=0}^{(qd_2/q_0)-1} \frac{q_0}{qd_2} \Big|\tilde{R}_M\Big(u,\left[\frac{a}{\tilde{d}_2}\right]_{q_0/d_2} \left[\frac{1}{\omega_2}\right]_{q_0/d_2} + l \frac{q_0}{d_2}\Big)\Big|^2\\
       =&\sum_{a\mod q_0/d_2} \int_{u\asymp 1 } \sum_{l=0}^{(qd_2/q_0)-1} \frac{q_0}{qd_2} \Big|\tilde{R}_M\Big(u,\left[a\right]_{q_0/d_2} \left[\frac{1}{\omega_2}\right]_{q_0/d_2} + l \frac{q_0}{d_2}\Big)\Big|^2\\
       =&\sum_{b\mod q} \int_{u\asymp 1 }\frac{q_0}{qd_2} \Big|\tilde{R}_M\Big(u,b\Big)\Big|^2\\
       \lesssim& \frac{q_0}{d_2} |W|.
    \end{align*}

    Similarly, the third statement follows from \begin{align*}
        \sum_{a\mod q_0/d} \int_{u\asymp 1 }|\tilde{R}^{(d)}_M(u,a)|^4 = & 
        \sum_{a\mod q_0/d} \int_{u\asymp 1 }\mathbb{I}_{\tilde{d}_2|y}\Bigg( \sum_{l=0}^{(qd_2/q_0)-1} \frac{q_0}{qd_2} \Big|\tilde{R}_M\Big(u,\left[\frac{a}{\tilde{d}_2}\right]_{q_0/d_2} \left[\frac{1}{\omega_2}\right]_{q_0/d_2} + l \frac{q_0}{d_2}\Big)\Big|^2\Bigg)^2\\
        \stackon{CS}{\leq} &\sum_{a\mod q_0/d} \int_{u\asymp 1 }\mathbb{I}_{\tilde{d}_2|y}\sum_{l=0}^{(qd_2/q_0)-1} \frac{q_0}{qd_2} \Big|\tilde{R}_M\Big(u,\left[\frac{a}{\tilde{d}_2}\right]_{q_0/d_2} \left[\frac{1}{\omega_2}\right]_{q_0/d_2} + l \frac{q_0}{d_2}\Big)\Big|^4 
        \\
        \lesssim & \frac{q_0}{d_2} E(W).
    \end{align*}

    Finally, from Proposition \ref{dyadics_3}, we choose $q_0|q$, $0<M_1\leq M\lesssim qT/N$ such that
    \[
        S_3\lesssim \frac{N^2q_0}{Mq^2}\sum_{|m_1|\sim M_1,|m_2|,|m_3|\sim M}\sum_{\substack{y_1,y_2 \in\mathbb{Z}/q\mathbb{Z} \\ y_1m_1+y_2m_2+m_3\equiv 0 \mod q_0}}\tilde{I}_{m,q_0}+O(T^{-100}).
    \]
    We split the summation in $m_1,m_2,m_3$ according to the gcd's $d_1,d_2,d_3$. By the divisor bound, we have for a choice of $d_1,d_2,d_3|q_0$,
     \[
        S_3\lesssim \frac{N^2q_0}{Mq^2}\sum_{\substack{|m_1|\sim M_1,|m_2|,|m_3|\sim M\\
        (m_1,q_0)=d_1\\(m_2,q_0)=d_2\\(m_3,q_0)=d_3}}\sum_{\substack{y_1,y_2 \in\mathbb{Z}/q\mathbb{Z} \\ y_1m_1+y_2m_2+m_3\equiv 0 \mod q_0/d}}\tilde{I}_{m,q_0}+O(T^{-100}).
    \]
    In the summation of $m_i$'s we split the $m_i$'s according to $\omega_i$'s. Such that we can write $m_i=d_i\tilde{k}_i\omega_i$ with $\tilde{k}_i$ coprime to $q_0$.
    Therefore we have \begin{align*}
        S_3\lesssim \frac{N^2q_0}{Mq^2} \sum_{\substack{\omega_1,\omega_2,\omega_3\\ (**)}}  \sum_{\substack{|m_1|\sim M_1,|m_2|,|m_3|\sim M\\
        (***)}}\sum_{\substack{y_1,y_2 \in\mathbb{Z}/q\mathbb{Z} \\ y_1m_1+y_2m_2+m_3\equiv 0 \mod q_0/d}}\tilde{I}_{m,q_0}+O(T^{-100}),
    \end{align*}
    where the conditions $(**)$ on $\omega_i$'s are that $\omega_i<qT/N$ only has prime factors that are also prime factors of $d_i$, $(***)$ on $m_i$'s are that
    $m_i=d_i\tilde{k}_i\omega_i$ where $(\tilde{k}_i,q_0)=1$. 
    Recall that \[
        \tilde{I}_{m,q_0}\defeq \ \int_{v_1\asymp 1} \left| \tilde{R}_M\left(\frac{m_1v_1+m_3}{-m_2v_1},y_2,y_1\right)
    \tilde{R}_M\left(\frac{m_1v_1+m_3}{-m_2},y_2\right)R\left(v_1,y_1\right)\right| dv_1.\]

    We now rewrite the summation in $y$, using $m=dk$, as \begin{align*}
        \sum_{\substack{y_1,y_2 \in\mathbb{Z}/q\mathbb{Z} \\ y_1k_1+y_2k_2+k_3\equiv 0 \mod q_0/d}}\tilde{I}_{d k ,q_0} =& \sum_{y_1\mod q} \sum_{\substack{y_2 \mod q,\\y_2\equiv y_1k_1+y_2k_2+k_3\equiv 0 \mod q_0/d}
        } \tilde{I}_{dk ,q_0}\\
        =&\sum_{y_1\mod q} \sum_{\substack{y_2 \mod q,\\ y_1k_1+y_2k_2+k_3\equiv 0 \mod q_0/d} }\int_{v_1\asymp 1} \Bigg| \tilde{R}_M\left(\frac{dk_1v_1+dk_3}{-dk_2v_1},y_2,y_1\right)\\
        & \quad \quad
        \tilde{R}_M\left(\frac{dk_1v_1+dk_3}{-dk_2},y_2\right)R\left(v_1,y_1\right)\Bigg| dv_1
    \end{align*}
    Keeping in mind that $R$ only has support in $y_1,y_2$ coprime to $q$, we apply Cauchy-schwarz to the summation in $y_2$ to get \begin{align*}
        &\sum_{\substack{y_2 \mod q,\\ y_1k_1+y_2k_2+k_3\equiv 0 \mod q_0/d} } \Bigg|\tilde{R}_M\left(\frac{dk_1v_1+dk_3}{-dk_2v_1},[y_2y_1^{-1}]_{q}\right)\tilde{R}_M\left(\frac{dk_1v_1+dk_3}{-dk_2},[y_2]_q\right)\Bigg|\\
        \leq &\Bigg(\sum_{\substack{y_2 \mod q,\\ y_1k_1+y_2k_2+k_3\equiv 0 \mod q_0/d} }\Big|\tilde{R}_M\left(\frac{k_1v_1+k_3}{-k_2v_1},[y_2y_1^{-1}]_{q}\right)\Big|^2\Bigg)^{1/2}
       \Bigg( \sum_{\substack{y_2 \mod q,\\ y_1k_1+y_2k_2+k_3\equiv 0 \mod q_0/d}  }
        \Big|\tilde{R}_M\left(\frac{k_1v_1+k_3}{-k_2},[y_2]_q\right)\Big|^2 \Bigg)^{1/2}\\
        =&\frac{qd_2}{q_0}\tilde{R}_M^{(d)}\Big(\frac{k_1v_1+k_3}{-k_2v_1},\frac{k_1y_1+k_3}{-\tilde{k}_2y_1}\Big)\tilde{R}_M^{(d)}\Big(\frac{k_1v_1+k_3}{-k_2},\frac{k_1y_1+k_3}{-\tilde{k}_2}\Big),
    \end{align*} 
    Combined, we have \[
        S_3\lesssim \frac{N^2d_2}{Mq} \sum_{\substack{\omega_1,\omega_2,\omega_3\\ (**)}} \sum_{\substack{|m_1|\sim M_1,\\|m_2|,|m_3|\sim M\\
        (***)}}\sum_{\substack{y_1\mod q\\ (y_1,q)=1} }\int_{v_1\asymp 1}
        \tilde{R}_M^{(d)}\Big(\frac{k_1v_1+k_3}{-k_2v_1},\frac{k_1y_1+k_3}{-\tilde{k}_2y_1}\Big)\tilde{R}_M^{(d)}\Big(\frac{k_1v_1+k_3}{-k_2},\frac{k_1y_1+k_3}{-\tilde{k}_2}\Big)|R\left(v_1,y_1\right)| dv_1
    \]
\end{proof}
\begin{corollary}\label{almostallek}
    If $\omega(q)=o(\log q /\log\log q)$, there is a choice of $d_1,d_2,d_3|q_0, q_0|q$, $0<M_1\leq M\lesssim qT/N$, $\omega_1,\omega_2,\omega_3$ such that  \[
        S_3\lesssim \frac{N^2d_2}{Mq} 
        %\sum_{\substack{\omega_1,\omega_2,\omega_3\\ (**)}}  
        \sum_{\substack{|m_1|\sim M_1,\\|m_2|,|m_3|\sim M\\
        (***)}}\sum_{\substack{y_1\mod q\\ (y_1,q)=1} }\int_{v_1\asymp 1}
        \tilde{R}_M^{(d)}\Big(\frac{k_1v_1+k_3}{-k_2v_1},\frac{k_1y_1+k_3}{-\tilde{k}_2y_1}\Big)\tilde{R}_M^{(d)}\Big(\frac{k_1v_1+k_3}{-k_2},\frac{k_1y_1+k_3}{-\tilde{k}_2}\Big)|R\left(v_1,y_1\right)| dv_1,
    \]
    where each $\omega_i<qT/N$ consists of only factors that are also prime factors of $d_i$, and the summation condition $(***)$ on $m_i$'s are that
    $m_i=d_i\tilde{k}_i\omega_i$ where $(\tilde{k}_i,q_0)=1$. 
\end{corollary}
\begin{proof}
    It suffices to show that \[
   \Sigma\defeq \sum_{\omega_1 (**)} 1 \lesssim 1.
    \]
    The conditions on $\omega_2,\omega_3$ are the same. Since $\omega_1$ consists of only prime factors of $d_i$, the number of choices of $\omega_1$ is bounded above by\[
    \prod_{p|d_1, p\leq qT/N}\log_{p} \frac{qT}{N}\leq \prod_{p|q,p\leq qT/N}\log_{p} \frac{qT}{N} \ll \prod_{p|q} \log \frac{qT}{N}.
    \]
    Therefore we have \[
    \Sigma \ll (\log qT)^{\omega(q)} = \exp(\omega(q) \times \log \log qT) \ll_{\epsilon} \exp(\epsilon \log qT)
    \]
    for any value of $\epsilon$.
\end{proof}
With this bound, we can prove the statement for almost all $q$ with a small number of distinct prime factors. This is because the Erd\"os Kac theorem implies almost all numbers $q$ satisfy $\omega(q)\asymp \log \log q$.
With some careful handling, we can prove this bound for all $q$.
\begin{proposition}\label{allek}
   There is a choice of $d_1,d_2,d_3|q_0, q_0|q$, $0<M_1\leq M\lesssim qT/N$, $\omega_1,\omega_2,\omega_3$ such that  \[
        S_3\lesssim \frac{N^2d_2}{Mq} 
        %\sum_{\substack{\omega_1,\omega_2,\omega_3\\ (**)}}  
        \sum_{\substack{|m_1|\sim M_1,\\|m_2|,|m_3|\sim M\\
        (***)}}\sum_{\substack{y_1\mod q\\ (y_1,q)=1} }\int_{v_1\asymp 1}
        \tilde{R}_M^{(d)}\Big(\frac{k_1v_1+k_3}{-k_2v_1},\frac{k_1y_1+k_3}{-\tilde{k}_2y_1}\Big)\tilde{R}_M^{(d)}\Big(\frac{k_1v_1+k_3}{-k_2},\frac{k_1y_1+k_3}{-\tilde{k}_2}\Big)|R\left(v_1,y_1\right)| dv_1,
    \]
    where each $\omega_i<qT/N$ consists of only factors that are also prime factors of $d_i$, and the summation condition $(***)$ on $m_i$'s are that
    $m_i=d_i\tilde{k}_i\omega_i$ where $(\tilde{k}_i,q_0)=1$. 
\end{proposition}
\begin{proof}
    Similar to the above corollary, it suffices to show that \[
    \Sigma\lesssim 1.
    \]
    We shall show that $\log \Sigma = o(\log qT)$, thus $\Sigma$ grows slower than any power of $qT$.

    Fix $d_1$, and let $p_1,p_2,\ldots, p_k$ the prime factors of $d_1$, not counting multiplicity. Let $s=d_1\omega_1$. Since $\omega_1$ only consists of prime factors that are prime factors of $d_1$, and $d_1|\omega_1$, the number of $s$ is equal to the number of positive integer lattice points $(x_1,\dots,x_k)$ that satisfy \[
    \sum_{i=1}^{k}x_i\log p_k\leq \log M,
    \]
    recalling that $M=qT/N= (qT)^{1/6}$. Without loss of generality, we can also assume that $p_i$ is the $i$-th smallest prime, as this leads to the upper bound. Moreover, since the product of the first $k$ primes is of order $\exp((1+o(1))k\log k ),$ we have \[
    (1+o(1))k\log k \leq \log q
    \]
    and \[
    p_k = (1+o(1))k\log k.
    \]
    We now bound the number of lattice points. Notice that each positive integer lattice point that satisfies the inequality can be identified uniquely with a unit cube completely contained inside the simplex \[
    \Delta=\{\sum_{i=1}^{k}x_i\log p_i\leq \log M,x_i\geq 0\},
    \]
    and the interiors of each of the cubes are pairwise disjoint. Thus, the number of $\omega_1$ is bounded above by the volume of the simplex \[
    \rm{Vol}\Delta = \frac{1}{k!}\prod_{i=1}^{k}\frac{\log M}{\log p_i}.
    \]
    Taking the logarithm, \begin{align*}
        \log \Sigma&\leq \log \rm{Vol}\Delta\\
        &=\int_{3/2}^{p_k}(\log \log M-\log \log y) d \pi(y) -\log k!.
    \end{align*}
    We apply integration by parts to the integral. This gives us\begin{align*}
    &\pi(p_k)(\log \log M -\log \log p_k)+\int_{3/2}^{p_k}\frac{\pi(y)}{y\log y}d y\\
    =&k\log \log M - k\log \log p_k + O\Big(\int_{3/2}^{p_k}\frac{1}{(\log y)^2}d y\Big)
    \end{align*}
    by the Prime Number Theorem. Since we have $p_k\ll \log q$ the integral behaves as $\log q / (\log \log q)^2 = o(\log qT)$. Similarly the second term is negligible by $k\ll \log q/\log \log q$. Therefore we have \[
        \int_{3/2}^{p_k}(\log \log M-\log \log y) d \pi(y)=k\log \log M+o(\log qT).
    \]
    We apply Stirling's approximation to get \[
    \log k! = k\log k + o(\log qT).
    \]
    Finally, we need to show that \[
    k\log \log M - k\log k =o( \log qT).
    \]
    
    Taking the first derivative in $k$ gives $\log \log M -\log k - 1$, we see that this function is increasing in $k$ when $k\ll \log q/\log \log q$ for sufficiently large $q$, as $\log q/\log \log q = o(\log qT)$.
    So we can bound this by evaluating at $k=(1+o(1))\log q/\log \log q.$ We also simplify using $M=q^{\alpha}$. (so that $qT=q^{6\alpha}$) \begin{align*}
        k\log \log M - k\log k &=k(\log \log M -\log k)\\&= (1+o(1))\frac{\log q}{\log \log q} \left(\log \frac{\alpha \log q \log \log q}{\log q}\right) \\&=(1+o(1))\frac{\log q}{\log \log q}\log \alpha + (1+o(1))\frac{\log q\log \log \log q}{\log \log q}
        \\&=(1+o(1))\frac{\log q \log \alpha}{\log \log q} + o(\log qT). 
    \end{align*}
    To show this is $o(\log qT)$, we consider \[
    \frac{\log q\log \alpha}{\log \log q} (\log qT)^{-1} =\frac{\log \alpha}{6\alpha \log \log q}\leq  \frac{1}{6e\log \log q} \to 0 
    \]
    as $q\to \infty$ independent of $\alpha$.

\end{proof}
\begin{proof}[Proof of Proposition \ref{s3refined}]
    From Proposition \ref{allek}, we have
    \[
    S_3\lesssim \frac{N^2d_2}{Mq} \sum_{\substack{|m_1|\sim M_1,\\|m_2|,|m_3|\sim M\\
        (***)}}\sum_{\substack{y_1\mod q\\ (y_1,q)=1} }\int_{v_1\asymp 1}
        \tilde{R}_M^{(d)}\Big(\frac{k_1v_1+k_3}{-k_2v_1},\frac{k_1y_1+k_3}{-\tilde{k}_2y_1}\Big)\tilde{R}_M^{(d)}\Big(\frac{k_1v_1+k_3}{-k_2},\frac{k_1y_1+k_3}{-\tilde{k}_2}\Big)|R\left(v_1,y_1\right)| dv_1.
    \]
    We apply Cauchy Schwarz repeatedly in the same fashion as the proof of Proposition \ref{s_3bound} to get \begin{align*}
        S_3\lesssim \frac{N^2d_2}{Mq} (\sum_{y_1\mod q}\int_{v_1\asymp 1}|R\left(v_1,y_1\right)| dv_1)^{1/2} (S_3^{(1)})^{1/2},
    \end{align*}
    where \begin{align*}
        S_3^{(1)}=&\sum_{\substack{y_1\mod q\\ (y_1,q)=1} }\int_{v_1\asymp 1}\Bigg(\sum_{\substack{|m_1|\sim M_1,\\|m_2|,|m_3|\sim M\\
        (***)}}
        \tilde{R}_M^{(d)}\Big(\frac{k_1v_1+k_3}{-k_2v_1},\frac{k_1y_1+k_3}{-\tilde{k}_2y_1}\Big)\tilde{R}_M^{(d)}\Big(\frac{k_1v_1+k_3}{-k_2},\frac{k_1y_1+k_3}{-\tilde{k}_2}\Big)\Bigg)^2 dv_1\\
        \stackon{CS}{\lesssim}& ( S_3^{(2,1)} )^{1/2}( S_3^{(2,2)} )^{1/2}.
    \end{align*}
    In the last step we define \begin{align*}
        S_3^{(2,1)}\defeq&\sum_{\substack{y_1\mod q\\ (y_1,q)=1} }\int_{v_1\asymp 1}\Bigg(\sum_{\substack{|m_1|\sim M_1,\\|m_2|,|m_3|\sim M\\
        (***)}}
        \tilde{R}_M^{(d)}\Big(\frac{k_1v_1+k_3}{-k_2v_1},\frac{k_1y_1+k_3}{-\tilde{k}_2y_1}\Big)^2\Bigg)^2 dv_1,
        \\
        S_3^{(2,2)}\defeq&\sum_{\substack{y_1\mod q\\ (y_1,q)=1} }\int_{v_1\asymp 1}\Bigg(\sum_{\substack{|m_1|\sim M_1,\\|m_2|,|m_3|\sim M\\
        (***)}}
       \tilde{R}_M^{(d)}\Big(\frac{k_1v_1+k_3}{-k_2},\frac{k_1y_1+k_3}{-\tilde{k}_2}\Big)^2\Bigg)^2 dv_1.
    \end{align*}
    We shall bound $ S_3^{(2,1)}$ The second term is similar. 
    Recall that \[
        k_i = m_i/d = \tilde{k}_i \tilde{d}_i \omega_i,
    \]  
    where $\tilde{k}_i$ is coprime to $q_0/d$, $\omega_i$ is coprime to $q_0/d_i = q_0/(d\tilde{d}_i)$.
    Thus 
    \begin{align*}
        S_3^{(2,1)}=&\sum_{\substack{y_1\mod q\\ (y_1,q)=1} }\int_{v_1\asymp 1}\Bigg(\sum_{\substack{|\tilde{k}_1|\sim M_1/(d_1\omega_1),\\|\tilde{k}_2|\sim M/(d_2\omega_2),\\|\tilde{k}_3|\sim M/(d_3\omega_3)\\
        (***)}}
        \tilde{R}_M^{(d)}\Big(\frac{\omega_1\tilde{d}_1\tilde{k}_1v_1+\omega_3\tilde{d}_3\tilde{k}_3}{-\omega_2\tilde{d}_2\tilde{k}_2v_1},\frac{\omega_1\tilde{d}_1\tilde{k}_1y_1+\omega_3\tilde{d}_3\tilde{k}_3}{-\tilde{k}_2y_1}\Big)^2\Bigg)^2 dv_1
        \\
        \asymp& \sum_{\substack{x \mod q\\ (x,q)=1} }\int_{u\asymp 1}\Bigg(\sum_{\substack{|\tilde{k}_1|\sim M_1/(d_1\omega_1),\\|\tilde{k}_2|\sim M/(d_2\omega_2),\\|\tilde{k}_3|\sim M/(d_3\omega_3)\\
        (***)}}
        \tilde{R}_M^{(d)}\Big(\frac{\omega_1\tilde{d}_1\tilde{k}_1+\omega_3\tilde{d}_3\tilde{k}_3u}{-\omega_2\tilde{d}_2\tilde{k}_2},\frac{\omega_1\tilde{d}_1\tilde{k}_1+\omega_3\tilde{d}_3\tilde{k}_3 x}{-\tilde{k}_2}\Big)^2\Bigg)^2 du
    \end{align*}
    by the change of variables $x\equiv y^{-1}\mod q$, $u=1/v_1$.
    Since the second argument is $q_0/d$ periodic, we can fold in $x$ to get \begin{align*}
        S_3^{(2,1)}\ll
        \frac{qd}{q_0} \sum_{\substack{x \mod q_0/d} }\int_{u\asymp 1}\Bigg(\sum_{\substack{|\tilde{k}_1|\sim M_1/(d_1\omega_1),\\|\tilde{k}_2|\sim M/(d_2\omega_2),\\|\tilde{k}_3|\sim M/(d_3\omega_3)\\
        (***)}}
        \tilde{R}_M^{(d)}\Big(\frac{\omega_1\tilde{d}_1\tilde{k}_1+\omega_3\tilde{d}_3\tilde{k}_3u}{-\omega_2\tilde{d}_2\tilde{k}_2},\frac{\omega_1\tilde{d}_1\tilde{k}_1+\omega_3\tilde{d}_3\tilde{k}_3 x}{-\tilde{k}_2}\Big)^2\Bigg)^2 du.
    \end{align*}
    We now make a change in variables $\tilde{u} = \omega_3\tilde{d}_3u$, and $\tilde{x}\equiv \omega_3\tilde{d_3}x$. The Jacobian factor is $1/(\omega_3\tilde{d}_3)$, and noticing that $\gcd(\omega_3\tilde{d_3},q_0/d) = \tilde{d_3}$ by the condition $\gcd(m_3,q_0) =d_3$, $\tilde{x}$ counts the multiples of $\tilde{d}_3$ by $\tilde{d}_3$ times.
    \begin{align*}
        S_3^{(2,1)}\ll
        \frac{qd}{q_0} \frac{1}{\omega_3\tilde{d_3}}\tilde{d_3} \sum_{\substack{\tilde{x} \mod q_0/d} }\int\Bigg(\sum_{\substack{|\tilde{k}_1|\sim M_1/(d_1\omega_1),\\|\tilde{k}_2|\sim M/(d_2\omega_2),\\|\tilde{k}_3|\sim M/(d_3\omega_3),\\
        \textrm{all coprime to $q_0/d$}}}
        \tilde{R}_M^{(d)}\Big(\frac{\omega_1\tilde{d}_1\tilde{k}_1+\tilde{k}_3\tilde{u}}{-\omega_2\tilde{d}_2\tilde{k}_2},\frac{\omega_1\tilde{d}_1\tilde{k}_1+\tilde{k}_3 \tilde{x}}{-\tilde{k}_2}\Big)^2\Bigg)^2 d\tilde{u}.
    \end{align*}
    We now apply corollary \ref{cor: affine} to the function $f(u,x)\defeq R_M^{(d)}(u/(-\tilde{d}_2\omega_2),x)^2$, and setting $L\defeq\omega_2\tilde{d}_2$ gives a bound, and we have bounds that $\tilde{M} = \max(ML/(d_2\omega_2) ,M_1/(d_1\omega_1),M_3/(d_3\omega_3))\leq M/d $. First we confirm that $f$ has the desired Fourier decay.
    Recall that $R_M^{(d)}$ is obtained by folding and averaging $R_M$ in the second argument, so it actually suffices to check the Fourier decay in $R_M$.
    Since $R_M^2$ is obtained from a convolution with a bump function of length $NM/q\lesssim T$, its Fourier transform in the first argument is rapidly decreasing when
    $|\xi|\gtrsim T$. So that the scaling of $1/L$ means that $f$ has Fourier transform rapidly decreasing when $|\xi|\gtrsim TL$. Since $L=\tilde{d}_2\omega_2\leq qT/(Nd)$, we 
    have the desired Fourier transform decay of $f$ when $|\xi|\gtrsim (qT)^2$.
    \begin{align*}
        S_3^{(2,1)}\lesssim &  \frac{qd}{q_0\omega_3}\left[\frac{M^6d}{d^6L^4q_0}\Big(\sum_{y \mod q_0/d } \int f(u,y) du\Big)^2  + \frac{M^4}{d^4L^2}\sum_{y\mod q_0/d} \int f(u,y)^2 du
        \right]\\
        \lesssim& \frac{qd}{q_0\omega_3}\left[\frac{M^6d}{d^6L^4q_0}\Big(\frac{q_0}{d_2}|W|L)^2  + \frac{M^4}{d^4L^2}\frac{q_0}{d_2}LE(W)
        \right]\\
        \lesssim & \frac{qd}{q_0}\left[\frac{M^6d}{d^6L^2q_0}\Big(\frac{q_0}{d_2}|W|)^2  + \frac{M^4}{d^4L}\frac{q_0}{d_2}E(W)\right].
    \end{align*}
    Where we had to introduce a factor of $L$ in the second and fourth moments as we defined $f(u,y) = \tilde{R}_M^{(d)}(-u/L,y)^2$. 
    The same bound holds for $S^{(2,2)}_3$.
    Therefore we have \begin{align*}
        S_3\lesssim &\frac{N^2 d_2}{Mq} (q|W|)^{1/2}  \left\{\frac{qd}{q_0}\left[\frac{M^6d}{L^2d^6q_0}\Big(\frac{q_0}{d_2}|W|)^2  + \frac{M^4}{d^4L}\frac{q_0}{d_2}E(W)\right]\right\}^{1/2}\\
        \lesssim& \frac{N^2M^2}{Ld^2}|W|^{3/2} + N^2M|W|^{1/2}E(W)^{1/2}\frac{d_2^{1/2}}{d^{3/2}\tilde{d}_2^{1/2}\omega_2^{1/2}}\\
        \lesssim& N^2M^2|W|^{3/2} + N^2M|W|^{1/2}E(W)^{1/2}\\
        \lesssim& (qT)^2|W|^{3/2} + N(qT)|W|^{1/2}E(W)^{1/2}.
    \end{align*}
    Where in the last two steps, we used $L=\tilde{d}_2\omega_2$, so that $d\tilde{d}_2=d_2$ cancels in the the last term, and that $M\lesssim qT/N$.
\end{proof}