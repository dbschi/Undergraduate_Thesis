\section*{Preliminaries}
Here we provide some supplementary definitions and statements of theorems. These results are well-known. 
\subsection*{Number Theory}
\begin{definition}[Dirichlet Characters]
	\label{dcharacter}
	Let $q\in\naturals$. A Dirichlet character $\chi:\mathbb{N}\to\complex$ modulus $q$ is an arithmetic function satisfying \begin{itemize}
		\item \textit{(Periodicity)} $\chi(n+q)=\chi(n)\forall n\in\naturals$.
		\item \textit{(Complete multiplicativity)} $\chi(nm)=\chi(n)\chi(m)\forall n,m\in\naturals$.
		\item $|\chi(n)|=\begin{cases}
			1, & \rm{if }\gcd(n,q)=1,\\
			0, & \rm{otherwise.}
		\end{cases}$
	\end{itemize}
\end{definition}
\begin{proposition}
	There are $\phi(q)$ Dirichlet characters of modulus $q$.
\end{proposition}
\begin{proof}
	Taking residual classes mod $q$, we see that Dirchlet characters are in one-to-one correspondence with one-dimensional representations of the multiplicative group $(\mathbb{Z}/q\mathbb{Z})^{\times}$. Since this group is abelian, all of its irreducible representations are one-dimensional. Therefore, the number of Dirichlet characters equals the number of irreducible representations of the $(\mathbb{Z}/q\mathbb{Z})^{\times}$. It is known that the sum of squares of the dimensions irreducible representations equals the order of the group, so we have \[
		\phi(q)=|(\mathbb{Z}/q\mathbb{Z})^{\times}|=\sum_{\textrm{irreducible representations } \varphi}  (\textrm{dim } \varphi)^2 = \sum_{\textrm{irreducible representations } \varphi}  1.
	\]
\end{proof}
\begin{definition}
	A Dirichlet character $\chi$ modulus $q$ is induced by another character $\chi^*$ mod $m<q$ if they agree on all $n$ such that $\gcd(q,n)=1$. A Dirchlet character is primitive if it is not induced by another character. A Dirchlet character is principal if it is induced by the character $\chi_1(n)\defeq1(n)\equiv 1$, thus corresponds to the trivial representation.
\end{definition}
\begin{theorem}[M\"obius Inversion]
	The M\"obius function $\mu$ is defined for $n\in\naturals$,
	\[
		\mu(n) = \begin{cases}
			1, \textrm{ if } n=1\\
			(-1)^k, \textrm{ if }n=p_1p_2...p_k\textrm{ for distinct }p\textrm{'s}\\
			0, \textrm {otherwise}
		\end{cases}
	\]
	Suppose we have arithmetic functions $f,g$, and that\[
		f(n) = \sum_{d|n} g(d)
	\]
	Then the M\"obius Inversion formula gives 
	\[
		g(n) = \sum_{d|n} \mu(d) f\left(\frac{n}{d}\right)
	\]
\end{theorem}
\begin{example}
	On $\Re(s)>1$, let $M_N(s) = \sum_{n\leq N} \mu(n)n^{-s}$.
	Then setting $f(n)=1$ for all $n$, $g(1)=1$, $g(n)=0$ for $n\geq 2$, we multiply $M_N$ by $\zeta$ in Dirichlet series to get\[
		\zeta(s)M_N(s) = \sum_{n} \frac{a_n}{n^{-s}},
	\]
	where $a_n=g(n)$ for all $n\leq N$.
	Similarly, letting $M_N(s) = \sum_{n\leq N} \chi(n) \mu(n) n^{-s}$ for some Dirichlet character $\chi$,
	we get \[
		L(s,\chi)M_N(s) = \sum_{n} \frac{a_n\chi(n)}{n^{-s}}
	\]
	with the same $a_n$ as in the previous equation.
\end{example}
\begin{theorem}[Erd\"os-Kac]
	Let $\omega(n)$ be the number of prime divisors of $n$, ignoring multiplicity. Let $X_N\sim \rm{unif}\{1,N\}$, the uniform distribution from $1$ to $N$, and
	\[
	Y_N\defeq \frac{\omega(X_N)-\log\log X_N}{\sqrt{\log \log X_N}}.
	\] 
	Then \[
	\lim_{N\to\infty} P(a<Y_N<b) = \frac{1}{\sqrt{2\pi}}\int_a^b e^{-t^2/2}dt.
	\]
\end{theorem}
\subsection*{Harmonic Analysis}
\begin{theorem}[Fourier Inversion]
	In Schwartz space, the Fourier transform of $\mathcal{F}:\schwartz(\reals^d)\to\schwartz(\reals^d)$ of $f\in\mathcal{S}(\reals^d)$ is given by
    \[
        \hat{f}(\mathbf{\xi})\defeq\mathcal{F}f(\mathbf{\xi})\defeq \int_{\reals^d} e(- \mathbf{\xi}\cdot \mathbf{x}) f(\mathbf{x}) \ d\mathbf{x}
    \]
    has inverse given by 
    \[
       f(\mathbf{x})= \mathcal{F}^{-1}\hat{f}(\mathbf{x})\defeq \int_{\reals^d} e(\mathbf{\xi}\cdot \mathbf{x}) \hat{f}(\mathbf{\xi}) \ d\mathbf{\xi}.
    \]
\end{theorem}
\begin{theorem}[Discrete Fourier Inversion]
	Let $g$ be an arithmetic function that is periodic $\mod q$. We define the discrete Fourier transform of $g$ to be \[
		\hat{g}(y) \defeq \sum_{x \mod q}g(x)e\big(-\frac{xy}{q}\big)
	\]
	which is periodic $\mod q$.
	This has inverse given by \[
		g(x) = \frac{1}{q}\sum_{y\mod q} \hat{g}(y)e\big(\frac{xy}{q}\big).
	\]
\end{theorem}
\begin{theorem}[Plancherel/Parsavel]
We have \[
\|f\|_{L^2}=\|\hat{f}\|_{L^2}
\]
for Schwartz functions and \[
\sum_{x\mod q}|g(x)|^2 = \frac{1}{q}\sum_{y\mod q}|\hat{g}|^2
\]
for the discrete Fourier transform for functions periodic $\mod q$.

\end{theorem}
\begin{theorem}[Mellin Inversion]
	The Mellin transform of a function $f:(0,\infty)\to\complex$
	\[
	\tilde{f}(s) \defeq \mathcal{M}f(s) \defeq \int_{0}^{\infty}f(x) x^{s-1} \ dx
	\]
	has inverse \[
	\mathcal{M}^{-1} \tilde{f} (x) = \int_{c-i\infty}^{c+i\infty} \tilde{f}(s) x^{-s}ds
	\]
	on $a<c<b$ provided that the integral $\tilde{f}$ is absolute convergent on the strip $a<\Re(s)<b$.
\end{theorem}
\begin{theorem}[Poisson Summation]
	Let $f:\reals\to\complex$ be Schwartz. Then \[
	\sum_{n\in \mathbb{Z}} f(n) = \sum_{\xi\in\mathbb{Z}} \hat{f}(\xi).
	\]
\end{theorem}