\section{Primes in Short Intervals}
We would like to answer the following question about primes in short intervals. Let $y=y(x)$. What is the smallest asymptotic behavior of $y$
such that \begin{equation}\label{shortintervalpnt}
\sum_{x\leq n \leq x+y} \Lambda(n) = (1+O(1)) y
\end{equation}
for large enough $x$? That is, what is the shortest interval such that we have the behavior of the Prime Number Theorem? If \ref{shortintervalpnt} holds for some $y$,
we say the Prime Number Theorem holds for intervals of $y$.
\begin{remark}
    This question can be rephrased into finding primes in short intervals, by including a factor of $\log x$. 
\end{remark} 
\begin{proposition}
    Assume the RH. Then the Prime Number Theorem holds in intervals of $x^{1/2+\epsilon}$.
\end{proposition}
\begin{proof}
    Assume the RH, then
    \begin{align*}
        \sum_{x\leq n \leq x+y} \Lambda(n)=
        y+O(x^{1/2}\log^2 x) = x^{1/2+\epsilon} + o(x^{1/2+\epsilon}),
    \end{align*}
    so that the sum is non-zero for large enough $x$.
\end{proof}
Recalling that the error term is related to the real part of the zeros of the Zeta function, we motivate the following definition of zero-density:
\begin{definition}
    Let $N(\sigma, T)$ denote the number of zeros of the zeta function with real part greater than $\sigma$ and imaginary part between $-T$ and $T$. That is,\[
        N(\sigma,T) = \# \{\rho = \beta + i\gamma \ | \ \beta \geq\sigma, |\gamma|\leq T\}.
    \]
\end{definition}
\begin{remark}
    The ideal scenario is that $N(\sigma,T)=0$ for all $\sigma\geq 1/2$. 
\end{remark}
\begin{theorem}[Chudakov] 
    There exists a constant $A$ such that $\zeta(\sigma+iT)\neq 0$ in the region \begin{equation*}
        \sigma > 1 - A\frac{\log \log T}{\log T}.
    \end{equation*}
\end{theorem}
\textit{add reference}
\begin{theorem}[Hoheisel] \label{Hoheisel}
    Let $A$ be defined as in the previous theorem.
    Suppose that $N(\sigma, T)\ll T^{a(1-\sigma)}\log^b T$ uniformly in $1/2\leq\sigma<1$ and in $T$. Then for all \[
        \theta > 1 - \frac{1}{a+b/A},
    \] the Prime Number Theorem holds in 
    intervals of $y=x^\theta$.
\end{theorem}
\begin{proof}
    First notice that $N(1/2,T)$ gets at least half of the zeros of height $T$, so $a\geq 2$.
    Let $y\ll x$. 
    We consider the expression \[
        S=S(x,y)=\frac{1}{y}\sum_{x\leq n \leq x+y} \Lambda(n).
    \]
    By the truncated version of the explicit formula in Theorem \ref{truncateexplcit}, we get
    \begin{align*}
        S &= 1 - \sum_{|\Im{(\rho)}|\leq T} \frac{(x+y)^\rho-x^\rho}{\rho y} + O(\frac{x}{yT}(\log xT) ^2) + O(\frac{\log x}{y}) . 
    \end{align*} 
    We focus on the sum over the non-trivial zeros with height less than $T$, and enumerate them $\rho_j$. For each $\rho_j=\sigma_j+it_j$, we apply Mean Value
    \begin{align*}
        \left|\sum_{\rho_j} \frac{(x+y)^\rho-x^\rho}{\rho y}\right|&\leq \sum_{\rho_j}\left|\frac{(x+y)^{\rho_j}-x^{\rho_j}}{\rho_j y}\right|\\
        &\ll \sum_{\rho_j} x^{\sigma_j-1}\\
        &= \sum_{\rho_j} x^{\sigma_j-1} - x^{-1} + x^{-1}\\
        &= O\left(\frac{T\log T}{x}\right) + \sum_{\rho_j} x^{\sigma_j-1} - x^{-1}.
    \end{align*}
    And by replacing $x^{\sigma_j}-1$ by an integral,\begin{align*}
        \sum_{\rho_j} x^{\sigma_j-1} - x^{-1} &=\sum_{\rho_j} \int_0^{\sigma_j}  x^{u-1} \log x \ du \\ 
        &=  \int_0^{1-A\frac{\log \log T}{\log T}} \sum_{\rho_j} \mathbbm{1}_{u\leq \sigma_j}x^{u-1} \log x \ du\\
        &= \int_0^{1-A\frac{\log \log T}{\log T}} N(u,T) x^{u-1} \log x \ du
    \end{align*}
    Where in the penulitimate step we made use of Chudaokov's bound and exchanged the order of integration and summation.
    Now we can apply the hypothesis that $N(\sigma, T)\ll T^{a(1-\sigma)}\log^b T$ for $\sigma>1/2$ and trivially $N(\sigma, T)\ll T \log T \ll T^{a(1-\sigma)}\log^b T$ for $\sigma \leq 1/2$.
    This evaluates to 
    \begin{align*}
        \sum_{\rho_j} x^{\sigma_j-1} - x^{-1} & \ll \int_0^{1-A\frac{\log \log T}{\log T}} T^{a(1-u)} \ \log^b T \  x^{u-1} \log x \ du\\
        &= \log^b T \int_0^{1-A\frac{\log \log T}{\log T}} \left(\frac{T^{a}}{x}\right)^{1-u} \log x \ du\\
        &= \frac{\log x \log^{b} T}{a \log T - \log x } \left[\frac{T^a}{x}-\left(\frac{T^a}{x}\right)^{A\frac{\log \log T}{\log T}}\right]\\
    \end{align*}
    Combined with the previous bounds, we have \[
        S=1 + O\left(\frac{T\log T}{x}\right)+ O\left( \frac{\log x \log^{b} T}{a \log T - \log x } \left[\frac{T^a}{x}-\left(\frac{T^a}{x}\right)^{A\frac{\log \log T}{\log T}}\right]\right)+ O(\frac{x}{yT}(\log xT) ^2) + O(\frac{\log x}{y}) . 
    \]
    To make all terms (except for the first) to be $o(1)$, we want to set 
    $y=x^\theta$, $T=x^k$, such that $\theta,k$ satisfy \[
        k<1,\ ak<1, \ k+\theta>1, 
    \]
    so that the second, fourth and fifth terms are $o(1)$ in $x$.
    For the second term, we can simplify \begin{align*}
        \frac{\log x \log^{b} T}{a \log T - \log x } \left[\frac{T^a}{x}-\left(\frac{T^a}{x}\right)^{A\frac{\log \log T}{\log T}}\right]
        &= \frac{k^b \log^{b}x}{ak-1}  \left[x^{ak-1}-x^{(ak-1)A\frac{\log (k \log x)}{k \log x}}\right]
        \\ &\leq \frac{k^b \log^{b}x}{1-ak} x^{ak-1}  + \frac{k^b \log^{b}x}{1-ak} \exp\left((ak-1)A\frac{\log (k \log x)}{k} \right)\\
        &\leq \frac{k^b \log^{b}x}{1-ak} x^{ak-1}  + \frac{k^b \log^{b}x}{1-ak} \exp\left((ak-1)A\frac{\log (k \log x)}{k} \right)\\
        & =O(x^{ak-1})+O\left(\left(\log  x\right)^{b+\frac{(ak-1)A}{k}}\right).
    \end{align*}
    To have decay in the last term in $x$, we need \[
        b+\frac{(ak-1)A}{k}< 0 \implies (aA+b)k<A \implies k < \frac{1}{a+\frac{b}{A}}
    \]
    We had $a>1$, so this $k$ satisfies $k<1$ and $ak<1$.
    Finally, for $k={1}/({a+bA^{-1}})+ \delta/2$ we let $\theta = 1-k+\delta/2$ to satisfy $\theta+k>1$,
    so we can find any ${1}/({a+bA^{-1}})+\delta >\theta>1-{1}/({a+bA^{-1}})$, and for $y=x^\theta$, we get\[
        \frac{1}{y}\sum_{x\leq n \leq x+y} \Lambda(n) = S = 1+o(1).
    \]

\end{proof}
Theorem \ref{Hoheisel} gives the classical way to relate the distribution of primes in short intervals to the density of zeros away from the real-half line.
\textit{todo: huxley zeta, then generalize guth maynard to hybridized} 