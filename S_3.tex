
\section{$S_3$ bound}
\begin{proposition}
    We have \[
    S_3 \lesssim \phi(q)^3 T^2 E(\mathbb{S})^{1/2} |\mathbb{S}|^{1/2}.
    \]
\end{proposition}
This bound has a refinement, which uses one of the propositions from Guth and Maynard's proof. This refinement of the $S_3$ bound is based on the same ideas as the first $S_3$ bound, so the first bound will be the main focus of the section.
\begin{proposition}[Refinement of $S_3$]
    \label{refinements_3}
    We have \[
    S_3\lesssim \phi(q)^{7/2}T^2|\mathbb{S}|^{3/2}+\phi(q)^{3}\frac{NT}{q}|\mathbb{S}|^{1/2}E(\mathbb{S})^{1/2}.
     \]
\end{proposition}
The proof of Proposition \ref{refinements_3} relies on the result for summation over affine transformation by Guth and Maynard.
\begin{lemma}
    Let $M>0$. Let $f(u)\geq 0$, supported on $u\asymp 1$, and $|\hat{f}(\xi)|\lesssim_j (|\xi|/T)^j$ for all $j$. Then \[
        \sup_{0<M_1,M_2,M_3<M} \int\Bigg( \sum_{\substack{|m_1|\sim M_1\\|m_2|\sim M_2 \\ |m_3|\ll M_3}} f\left(\frac{m_1 u+m_3}{m_2}\right)\Bigg)^2 \ du \lesssim M^6 \|f\|_{L_1}^2+M^4\|f\|_{L_2}^2.
    \] 
\end{lemma}
This is Proposition 9.1 from [GM].


By non-stationary phase, $I_m$ is negligible for the terms $qT/N\lesssim |m|$, so 
\begin{equation}
    S_3 = \sum_{0<|m_1|,|m_2|,|m_3|\lesssim qT/N} I_m + O(T^{-100}).
\end{equation}
We define \begin{align*}
    R(v,n_1,n_2)&\defeq \sum_{(t,\chi)\in \mathbb{S}} 
    \chi({n_1})\bar{\chi}(n_2)v^{it},\\
    R(v,n)&\defeq R(v,n,1).
\end{align*}
\begin{proposition}
    \label{doubleintegrals3}
    \[
    |I_m|\ll \phi(q) \frac{N^3}{q^3}  
    \sum_{y_1,y_2\in (\mathbb{Z}/q\mathbb{Z})^\times}\ \int\displaylimits_{\substack{
        |v_1m_1+v_2m_2+m_3|\lesssim \frac{q}{N}\\
        \frac{1}{2}\leq v_1,v_2\leq 2
    }} \left| R\left(\frac{v_2}{v_1},y_2,y_1\right)
    R(v_2,y_2)R\left(v_1,y_1\right)\right| dv_1 \ dv_2  + O(T^{-100}).\\
    \]
    Moreover, if $|m_1|\leq|m_2|\leq |m_3|$, $|I_m|=O(T^{-100})$ unless $|m_2|\asymp|m_3|$.
\end{proposition}
\begin{proof}
    
Recall
\begin{align*}
    I_m=\frac{N^3}{q^3}\sum_{\substack{(t_1,\chi_1),\\(t_2,\chi_2),\\(t_3,\chi_3)\in\mathcal{S}}} &\sum_{x\in (\mathbb{Z}/q\mathbb{Z})^3}\chi_1\bar{\chi}_3(x_1)\chi_2\bar{\chi}_1(x_2)\chi_3\bar{\chi}_2(x_3) e\left(\frac{-x\cdot m}{q}\right)\\
    \times \ &\hat{h}_{t_1-t_3}\left(\frac{Nm_1}{q}\right)\hat{h}_{t_2-t_1}\left(\frac{Nm_2}{q}\right)\hat{h}_{t_3-t_2}\left(\frac{Nm_3}{q}\right).
\end{align*}
Expanding the integrals, 
\begin{align*}
    I_m=\frac{N^3}{q^3}\sum_{\substack{(t_1,\chi_1),\\(t_2,\chi_2),\\(t_3,\chi_3)\in\mathcal{S}}} &\sum_{x\in (\mathbb{Z}/q\mathbb{Z})^3}\chi_1\bar{\chi}_3(x_1)\chi_2\bar{\chi}_1(x_2)\chi_3\bar{\chi}_2(x_3) e\left(\frac{-x\cdot m}{q}\right)\\
    \times \ &
    \int_{\reals^3}\mathbf{\tilde{\omega}}(\mathbf{u})u_1^{i(t_1-t_3)}u_2^{i(t_2-t_1)}u_3^{i(t_3-t_2)}e\left(\frac{-N\mathbf{m}\cdot \mathbf{u}}{q}\right)d\mathbf{u},
\end{align*}
where $\tilde{\omega}(\mathbf{u})=\omega(u_1)^2\omega(u_2)^2\omega(u_3)^2$ is compactly supported.
We now make the substitution $y_1=x_1x_3^{-1}, y_2=x_2x_3^-1 \mod q$ for the summation over $x$, and $v_1=u_1/u_3,v_2=u_2/u_3$ for the integral on the support of $\tilde{\omega}$.
We thus rewrite the sum over $x$ as 
\begin{align*}
    &\sum_{y_1,y_2,x_3\in (\mathbb{Z}/q\mathbb{Z})^\times}
    \chi_1(y_1y_2^{-1})\chi_2(y_2)\chi_3(y_1^{-1})e\left(\frac{-(y_1m_1+y_2m_2+m_3)x_3}{q}\right)\\
    =&
    \sum_{y_1,y_2\in (\mathbb{Z}/q\mathbb{Z})^\times}\chi_1(y_1)\bar{\chi}_1(y_2)\chi_2(y_)\bar{\chi}_3(y_1)\sum_{x_3\in (\mathbb{Z}/q\mathbb{Z})^\times}e\left(\frac{-(y_1m_1+y_2m_2+m_3)x_3}{q}\right),
\end{align*}
where we can use the trivial bound $\phi(q)$ for the innermost sum.
We also rewrite triple integral as 
\begin{align*}
    &\int_{\reals^3}\tilde{\omega}(v_1u_3,v_2u_3,u_3) {\left(\frac{v_1}{v_2}\right)}^{it_1} {\left(v_2\right)}^{it_2}{\left(\frac{1}{v_1}\right)}^{it_3} u_3^2 \ e\left(\frac{-N(v_1m_1+v_2m_2+m_3)u_3}{q}\right)\ dv_1\ dv_2\ du_3\\
    =&\int_{\reals^2}\int_\reals u_3^2 \ \tilde{\omega}(v_1u_3,v_2u_3,u_3) e\left(\frac{-N(v_1m_1+v_2m_2+m_3)u_3}{q}\right)  du_3 \ {\left(\frac{v_1}{v_2}\right)}^{it_1} {\left(v_2\right)}^{it_2}{\left(\frac{1}{v_1}\right)}^{it_3}  \ dv_1\ dv_2.\\
\end{align*}
The integrand of the innermost integral is non-zero only if \[
    v_1u_3,v_2u_3,u_3\sim N.
\]
Importantly, this requires $1/2 \leq v_1,v_2 \leq 2$, so we can truncate the outermost integrals to these regions. Moreover, by repeated integration by parts, this integral is $O_{\epsilon, A}(T^{-A})$ for any $|v_1m_1+v_2m_2+m_3|>qT^\epsilon/N$.
So 
\begin{align*}
    |I_m|\ll &\phi(q) \frac{N^3}{q^3}  
    \sum_{y_1,y_2\in (\mathbb{Z}/q\mathbb{Z})^\times}\left|\ \int\displaylimits_{\substack{
        |v_1m_1+v_2m_2+m_3|\lesssim \frac{q}{N}\\
        \frac{1}{2}\leq v_1,v_2\leq 2
    }} R\left(\frac{v_1}{v_2},y_1,y_2\right)
    R(v_2,y_2)R\left(\frac{1}{v_1},1,y_1\right) dv_1 \ dv_2 \right|\\ &+ O(T^{-100}).\\
\end{align*}
Since $|R(v_1^{-1},1,y_1)|=|R(v_1,y_1)|$, $| R\left(\frac{v_1}{v_2},y_1,y_2\right)| = R\left(\frac{v_2}{v_1},y_2,y_1\right)|$, we have the first part of the proposition.
The second part of the proposition follows from the integral bounds $|v_1m_1+v_2m_2+m_3|\lesssim q/N$
 and $v_1,v_2\asymp 1$. These force $|m_2| \asymp|m_3|$, or else the integral will be zero.

\end{proof}

Adapting from Guth and Maynard, when $|m_2|\asymp|m_3|$, the domain of integration can be written as\begin{align*}
    |v_1m_1+v_2m_2+m_3|\lesssim \frac{q}{N} \implies \left|v_2 - \frac{v_1m_1+m_3}{-m_2}\right|\lesssim \frac{q}{|m_2|N} \asymp \frac{q}{|m_3|N}.
\end{align*}
Thus, the integration in $v_2$ is in a small neighborhood of $\frac{v_1m_1+m_3}{-m_2}$.

Let $\tilde{\phi}$ be a smooth bump function such that equals $\tilde{\phi}=1$ on $|x|\lesssim 1$ and is supported in $|x|\lesssim 1$, with a larger constant, so that $\|\tilde{\phi}^{(j)}\lesssim_j 1\|$ for all $j$. We define \[
\tilde{R}_M(v,y_1,y_2) \defeq \left( \int \frac{NM}{q}\tilde{\phi}\left(\frac{NM}{q}(v-v')\right)|R(v',y_1,y_2)|^2 dv'\right)^{1/2}.
\]
\begin{proposition} \label{dyadics_3}
    There is a choice of $0<M_1\leq M \lesssim qT/N$ such that \[
        S_3\lesssim \phi(q)\frac{N^2}{Mq^2}\sum_{|m_1|\sim M_1,|m_2|,|m_3|\sim M}\tilde{I}_m+O(T^{-100}).
    \]
    where \[
    \tilde{I}_m\defeq \sum_{y_1,y_2\in (\mathbb{Z}/q\mathbb{Z})^\times}\ \int_{v_1\asymp 1} 
         \left|R\left(v_1,y_1\right) \tilde{R}_M\left(\frac{m_1v_1+m_3}{-m_2v_1},y_2,y_1\right)
        \tilde{R}_M\left(\frac{m_1v_1+m_3}{-m_2},y_2\right)\right| dv_1.
    \]
\end{proposition}


\begin{proof}
By Proposition \ref{doubleintegrals3}, we consider the terms $|m_1|\leq |m_2|\leq|m_3|$ at the cost of a factor of $6$, and $|m_2|\asymp|m_3|$. Expanding the sum over $m_1,m_2,m_3$ dyadically, we get for some $M_1\leq M \lesssim qT/N $ \begin{align*}
    S_3\lesssim \sum_{|m_1|\sim M_1,|m_2|,|m_3|\sim M}|I_m| + O(T^{-100})
\end{align*}
We now consider
\begin{align*} 
   & \int\displaylimits_{\substack{
        |v_1m_1+v_2m_2+m_3|\lesssim \frac{q}{N}\\
        \frac{1}{2}\leq v_1,v_2\leq 2
    }} \left| R\left(\frac{v_2}{v_1},y_2,y_1\right)
    R(v_2,y_2)R\left(v_1,y_1\right)\right| dv_1 \ dv_2\\
     \ll& 
    \int_{v_1\asymp 1} |R\left(v_1,y_1\right)|
    \int\displaylimits_{\left|v_2 - \frac{v_1m_1+m_3}{-m_2}\right|\lesssim \frac{q}{|m_2|N}} \left| R\left(\frac{v_2}{v_1},y_2,y_1\right)
    R(v_2,y_2)\right|  dv_2 \ dv_1\\
    \ll& \int_{v_1\asymp 1} |R\left(v_1,y_1\right)|
    \int\displaylimits_{\left|v_2 - \frac{v_1m_1+m_3}{-m_2}\right|\lesssim \frac{q}{MN}} \left| R\left(\frac{v_2}{v_1},y_2,y_1\right)
    R(v_2,y_2)\right|  dv_2 \ dv_1\\
\end{align*}
when $|m_2|\asymp M$. The inner integral, by Cauchy-Schwarz,
is \begin{align*}
    \leq & \left(\int\displaylimits_{\left|v_2 - \frac{v_1m_1+m_3}{-m_2}\right|\lesssim \frac{q}{MN}} \left| R\left(\frac{v_2}{v_1},y_2,y_1\right)\right|^2 \ dv_2 \right)^{1/2}
    \left(
    \int\displaylimits_{\left|v_2 - \frac{v_1m_1+m_3}{-m_2}\right|\lesssim \frac{q}{MN}} \left|
    R(v_2,y_2)\right|^2  dv_2\right)^{1/2}\\
    \ll& \frac{q}{MN} \tilde{R}_M\left(\frac{v_1m_1+m_3}{-m_2v_1},y_2,y_1\right)  \tilde{R}_M(\frac{v_1m_1+m_3}{-m_2},y_2)
\end{align*}
where in the last step, we used $v_1\asymp 1$.
Thus, for $|m_2|\sim M$,
\[
|I_m|\lesssim \phi(q)\frac{N^2}{Mq^2}\tilde{I}_m.
\]
The proposition follows from this claim.
\end{proof}


\begin{lemma} \label{secondmoment}
    Let $\mathbb{S}=\{(t_j,\chi_j)\}$, and the $t$'s are contained in an interval of length $T$, and are $T^\epsilon$-separated for the same character. Then \[
        \sum_{y\in (\mathbb{Z}/q\mathbb{Z})^\times} \int_{v\asymp 1} 
        \left|R\left(v,y\right)\right|^2dv \ll_{\epsilon} \phi(q)|\mathbb{S}|.
    \]
\end{lemma}
\begin{proof}
    We have \[
    |R(v,y)|^2 = \sum_{(t_1,\chi_1),(t_2,\chi_2)\in \mathbb{S}}
    \chi_1\bar{\chi}_2(y)v^{i(t_1-t_2)}.
    \]
    Let $\psi$ be a bump function supported on $v\asymp 1$ and equals $1$ on the domain of integration in the lemma.
    By orthogonality of characters, \begin{align*}
        \sum_{y\in (\mathbb{Z}/q\mathbb{Z})^\times} \int_{v\asymp 1} 
        \left|R\left(v,y\right)\right|^2dv 
        \leq&\sum_{y\in (\mathbb{Z}/q\mathbb{Z})^\times} \int 
        \psi(v)\left|R\left(v,y\right)\right|^2dv 
        \\=&
        \phi(q)\int \psi(v)
        \sum_{\substack{(t_1,\chi_1),(t_2,\chi_2)\in \mathbb{S}\\ \chi_1=\chi_2}}v^{i(t_1-t_2)}
        dv\\
        =&
        \phi(q)\sum_{\substack{(t_1,\chi_1),(t_2,\chi_2)\in \mathbb{S}\\ \chi_1=\chi_2}}\int \psi(v)
        v^{i(t_1-t_2)}
        dv.
    \end{align*}
    In the sum, the terms $t_1=t_2$ contribute $O(|\mathbb{S}|)$. If $t_1\neq t_2$, then $|t_1-t_2|\geq T^\epsilon$. The integral in this case is $O_\epsilon(T^-100)$ and is negligible.
\end{proof}
\begin{lemma}\label{fourthmoment}
    Let $E(\mathbb{S})=\#\{(t_1,\chi_1),(t_2,\chi_2),(t_3,\chi_3),(t_4,\chi_4)\in \mathbb{S}  :  |t_1+t_2-t_3-t_4|\leq 1, \chi_1\chi_2=\chi_3\chi_4\}$. Then \[
        \sum_{y\in (\mathbb{Z}/q\mathbb{Z})^\times} \int_{v\asymp 1} 
        \left|R\left(v,y\right)\right|^4dv  \lesssim \phi(q)E(\mathbb{S}).
    \]
\end{lemma}
\begin{proof}
    We have \[
    |R(v,y)|^4 = \sum_{\substack{(t_1,\chi_1),(t_2,\chi_2),\\ (t_3,\chi_3),(t_4,\chi_4)\in \mathbb{S}}}
    \chi_1{\chi}_2\bar{\chi_3}\bar{\chi_4}(y)v^{i(t_1+t_2-t_3-t_4)}.
    \]
    So again by the orthogonality of characters, \begin{align*}
        \sum_{y\in (\mathbb{Z}/q\mathbb{Z})^\times} \int_{v\asymp 1} 
        \left|R\left(v,y\right)\right|^4dv = & \phi(q)
        \sum_{\substack{(t_1,\chi_1),(t_2,\chi_2),\\ (t_3,\chi_3),(t_4,\chi_4)\in \mathbb{S}\\ \chi_1\chi_2=\chi_3\chi_4}} \int_{v\asymp 1} v^{i(t_1+t_2-t_3-t_4)} dv.
    \end{align*}
    Similar to the previous proof, we can introduce a bump function for the integral, and restrict the summation to the terms $|t_1+t_2-t_3-t_4|\leq T^\epsilon$ with an error of $O_\epsilon(T^{-100})$. The remaining terms in the summation contribute $O(E(\mathbb{S}))$.
\end{proof}
\begin{lemma}\label{fourthmoment_smooth}
    Let $E(\mathbb{S})=\#\{(t_1,\chi_1),(t_2,\chi_2),(t_3,\chi_3),(t_4,\chi_4)\in \mathbb{S}  :  |t_1+t_2-t_3-t_4|\leq 1, \chi_1\chi_2=\chi_3\chi_4\}$. Then \[
        \sum_{y\in (\mathbb{Z}/q\mathbb{Z})^\times} \int_{v\asymp 1} 
        \left|\tilde{R}_M\left(v,y\right)\right|^4dv  \lesssim \phi(q)E(\mathbb{S}).
    \]
\end{lemma}
\begin{proof}
    We apply Cauchy-Schwarz to \begin{align*}
        \int_{v\asymp 1} 
        \left|\tilde{R}_M\left(v,y\right)\right|^4dv  \lesssim& \int_{v\asymp 1} 
        \left(\int_{|u-v|\lesssim q/NM}
        \frac{NM}{q}|R(u)|^2 du\right)^2
        dv \\
        \lesssim& \frac{NM}{q} \int_{v\asymp 1} 
       \int_{|u-v|\lesssim q/NM}
        |R(u)|^4 du \ 
        dv\\
        \lesssim&  
        \int_{u\asymp 1}
         |R(u)|^4 du.
    \end{align*}
    Lemma \ref{fourthmoment} completes the proof.
\end{proof}
\begin{proof}[{Proof of proposition [placeholder]}]
    We first appler H\"older's inequality on the integral to get 
    \begin{align*}
        \tilde{I}_m \leq  \sum_{y_1,y_2\in (\mathbb{Z}/q\mathbb{Z})^\times}& \left(\int_{v_1\asymp 1} 
        \left|R\left(v_1,y_1\right)\right|^2dv_1\right)^{1/2} \left(\int_{v_1\asymp 1} \left|\tilde{R}_M\left(\frac{m_1v_1+m_3}{-m_2v_1},y_2,y_1\right)\right|^4 dv_1\right)^{1/4} \\& \left(\int_{v_1\asymp 1} 
       \left| \tilde{R}_M(\frac{m_1v_1+m_3}{-m_2},y_2)\right|^{4} dv_1\right)^{1/4},
    \end{align*}
    Notice the first integral is independent of $y_2$, for sum of the second and third integrals over $y_2$, we apply Cauchy-Schwarz to get\begin{align*}
        &\sum_{y_2\in (\mathbb{Z}/q\mathbb{Z})^\times}\left(\int_{v_1\asymp 1} \left|\tilde{R}_M\left(\frac{m_1v_1+m_3}{-m_2v_1},y_2,y_1\right)\right|^4 dv_1\right)^{1/4} \left(\int_{v_1\asymp 1} 
       \left| \tilde{R}_M(\frac{m_1v_1+m_3}{-m_2},y_2)\right|^{4} dv_1\right)^{1/4}\\
       \leq& \left(\sum_{y_2\in (\mathbb{Z}/q\mathbb{Z})^\times}\left(\int_{v_1\asymp 1} \left|\tilde{R}_M\left(\frac{m_1v_1+m_3}{-m_2v_1},y_2,y_1\right)\right|^4 dv_1\right)^{1/2}\right)^{1/2}\\ &\quad \quad
       \left(
       \sum_{y_2\in (\mathbb{Z}/q\mathbb{Z})^\times} \left(\int_{v_1\asymp 1}
       \left| \tilde{R}_M(\frac{m_1v_1+m_3}{-m_2},y_2)\right|^{4} dv_1 \right)^{1/2}\right)^{1/2}
       \\ 
       \leq&  \phi(q)^{\frac{1}{2}} \left(\sum_{y_2\in (\mathbb{Z}/q\mathbb{Z})^\times}\int_{v_1\asymp 1} \left|\tilde{R}_M\left(\frac{m_1v_1+m_3}{-m_2v_1},y_2,y_1\right)\right|^4 dv_1\right)^{1/4}\\ &\quad \quad
       \left(
       \sum_{y_2\in (\mathbb{Z}/q\mathbb{Z})^\times} \int_{v_1\asymp 1}
       \left| \tilde{R}_M(\frac{m_1v_1+m_3}{-m_2},y_2)\right|^{4} dv_1\right)^{1/4} \\
       \leq&  \phi(q)^{\frac{1}{2}} \left(\sum_{y_3\in (\mathbb{Z}/q\mathbb{Z})^\times}\int_{v_1\asymp 1} \left|\tilde{R}_M\left(\frac{m_1v_1+m_3}{-m_2v_1},y_3\right)\right|^4 dv_1\right)^{1/4}\\ &\quad \quad
       \left(
       \sum_{y_2\in (\mathbb{Z}/q\mathbb{Z})^\times} \int_{v_1\asymp 1}
       \left| \tilde{R}_M(\frac{m_1v_1+m_3}{-m_2},y_2)\right|^{4} dv_1\right)^{1/4}
       \\ \lesssim&
       \phi(q)E(\mathbb{S})^{\frac{1}{2}}(M/M_1)^{1/4}\leq\phi(q)E(\mathbb{S})^{\frac{1}{2}}M/M_1.
       \end{align*}
       where in the penultimate step, we made a change of variables $y_3=y_2y_1^{-1}$. In the last step we change variables of integration $u=(m_1v_1+m_3)/(-m_2v_1)$ and $u=(m_1v_1+m_3)/(-m_2)$ with a Jacobian factor of $\asymp 1$ and $\sim M/M_1$ respectively. 
       For the first integral, applying Cauchy Schwarz gives \begin{align*}
        \sum_{y_1\in (\mathbb{Z}/q\mathbb{Z})^\times} \left(\int_{v_1\asymp 1} 
        \left|R\left(v_1,y_1\right)\right|^2dv_1\right)^{1/2} 
        \leq  \phi(q)^{\frac{1}{2}}\left(\sum_{y_1\in (\mathbb{Z}/q\mathbb{Z})^\times} \int_{v_1\asymp 1} 
        \left|R\left(v_1,y_1\right)\right|^2dv_1\right)^{1/2} 
        \ll_{\epsilon}\phi(q) |\mathbb{S}|^{1/2}.
       \end{align*}
       Combined with Proposition \ref{dyadics_3}, this gives \[
       S_3 \lesssim \phi(q)^3\frac{N^2M^2}{q^2} E(\mathbb{S})^{1/2} |\mathbb{S}|^{1/2} \lesssim  \phi(q)^3 T^2 E(\mathbb{S})^{1/2} |\mathbb{S}|^{1/2}.
       \]
\end{proof}

\begin{proof}[Proof of Proposition \ref{refinements_3}]
    Recall that \begin{align*}
        S_3\lesssim \phi(q)\frac{N^2}{Mq^2}&\sum_{|m_1|\sim M_1,|m_2|,|m_3|\sim M}  \sum_{y_1,y_2\in (\mathbb{Z}/q\mathbb{Z})^\times}\ \int_{v_1\asymp 1}\\& \left|R\left(v_1,y_1\right) \tilde{R}_M\left(\frac{m_1v_1+m_3}{-m_2v_1},y_2,y_1\right)
        \tilde{R}_M\left(\frac{m_1v_1+m_3}{-m_2},y_2\right)\right| dv_1.
    \end{align*}
    We apply Cauchy Schwarz repeatedly to get \begin{align*}
    S_3 \lesssim& \phi(q)\frac{N^2}{Mq^2}\sum_{y_1\in (\mathbb{Z}/q\mathbb{Z})^\times} S_{3,1}^{1/2}\sum_{y_2\in (\mathbb{Z}/q\mathbb{Z})^\times}S_{3,2}^{1/2}\\
        \lesssim & \phi(q)^{3/2}\frac{N^2}{Mq^2}\sum_{y_1\in (\mathbb{Z}/q\mathbb{Z})^\times} S_{3,1}^{1/2}\left(\sum_{y_2\in (\mathbb{Z}/q\mathbb{Z})^\times}S_{3,2}\right)^{1/2},\\
    \end{align*}
    where \begin{align*}
        S_{3,1}= S_{3,1}(y_1) =& \int_{v_1\asymp 1} |R\left(v_1,y_1\right)|^2 dv_1,\\
        S_{3,2}=S_{3,2}(y_1,y_2) =& \int_{v_1\asymp 1}\Bigg(\sum_{|m_1|\sim M_1,|m_2|,|m_3|\sim M} \left|\tilde{R}_M\left(\frac{m_1v_1+m_3}{-m_2v_1},y_2,y_1\right)
        \tilde{R}_M\left(\frac{m_1v_1+m_3}{-m_2},y_2\right)\right|\Bigg)^2 dv_1,
    \end{align*}
    and \begin{align*}
        \sum_{y_2\in (\mathbb{Z}/q\mathbb{Z})} S_{3,2}\lesssim \sum_{y_2\in (\mathbb{Z}/q\mathbb{Z})} (S_{3,3}S_{3,4})^{1/2}\lesssim \bigg(\sum_{y_2\in (\mathbb{Z}/q\mathbb{Z})}S_{3,5} \bigg)^{1/2} \bigg(\sum_{y_2\in (\mathbb{Z}/q\mathbb{Z})}S_{3,4} \bigg)^{1/2}
    \end{align*}
    where \begin{align*}
        S_{3,3}& = \int_{v_1\asymp 1}\Bigg(\sum_{|m_1|\sim M_1,|m_2|,|m_3|\sim M} \left|\tilde{R}_M\left(\frac{m_1v_1+m_3}{-m_2v_1},y_2,y_1\right)
        \right|^2\Bigg)^2 dv_1,\\
        S_{3,4}& = \int_{v_1\asymp 1}\Bigg(\sum_{|m_1|\sim M_1,|m_2|,|m_3|\sim M} \left|
        \tilde{R}_M\left(\frac{m_1v_1+m_3}{-m_2},y_2\right)\right|^2\Bigg)^2 dv_1,\\
        S_{3,3}& = \int_{v_1\asymp 1}\Bigg(\sum_{|m_1|\sim M_1,|m_2|,|m_3|\sim M} \left|\tilde{R}_M\left(\frac{m_1v_1+m_3}{-m_2v_1},y_2\right)
        \right|^2\Bigg)^2 dv_1.\\
    \end{align*}
    The remaining arguments are similar with GM  to get 
    \begin{align*}
        \sum_{y_2\in (\mathbb{Z}/q\mathbb{Z})} S_{3,4} \lesssim & \sum_{y_2\in (\mathbb{Z}/q\mathbb{Z})} M^6 \left(\int_{v\asymp 1} 
    \left|R\left(v,y\right)\right|^2dv\right)^2 +\sum_{y_2\in (\mathbb{Z}/q\mathbb{Z})}M^4 \left(\int_{v\asymp 1} 
    \left|R\left(v,y\right)\right|^4dv\right)\\
    \lesssim &M^6\phi(q)^2|\mathbb{S}|^2+M^4\phi(q)E(\mathbb{S}).
    \end{align*}
    So we get 
    \begin{align*}
        S_3 \lesssim& \phi(q)^{3/2}\frac{N^2}{Mq^2}
        \phi(q)|\mathbb{S}|^{1/2}\left(M^6\phi(q)^2|\mathbb{S}|^2+M^4\phi(q)E(\mathbb{S})\right)^{1/2}
        \\
        \lesssim & \phi(q)^{7/2}\frac{N^2}{q^2}M^2|\mathbb{S}|^{3/2}+\phi(q)^{3}\frac{N^2}{q^2}M|\mathbb{S}|^{1/2}E(\mathbb{S})^{1/2}.
    \end{align*}
    Taking $M\lesssim qT/N$,
    \[
    S_3\lesssim \phi(q)^{7/2}T^2|\mathbb{S}|^{3/2}+\phi(q)^{3}\frac{NT}{q}|\mathbb{S}|^{1/2}E(\mathbb{S})^{1/2}.
     \]
\end{proof}