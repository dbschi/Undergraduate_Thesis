\chapter{Towards a Hybrid Zero Density Result}
\iffalse

\textit{Here I will document some progress with working with
the hybrid version of Guth-Maynard. This section will be removed when I send the draft to Prof. Wunsch.}
\fi
We would like to generalize Guth and Maynard's result to $L$-functions. Specifically, let $\chi$ be a \hyperref[dcharacter]{Dirichlet character}. we are interested in the the zeros of the function defined by \[
L(s,\chi)\defeq\sum_{n} \frac{\chi(n)}{n^{-s}}
\]
on $\Re(s)>1$ and its analytic continuation on the whole complex plane. The zeta function is a special case of an $L$-function with the Dirichlet character $1$ everywhere. The structure of the arguments for analytic continuation of an $L$-function, its line of symmetry along $\Re(s) = 1/2$, and the locations of zeros are very similar to that of the zeta function. This motivates the Generalized Riemann Hypothesis. \begin{conjecture} [Generalized Riemann Hypothesis]
    The \text{Generalized Riemann Hypothesis} asserts that on the critical strip, \[
    L(s,\chi)=0 \implies \Re(s)=\frac{1}{2},
    \]
    for any Dirichlet character.
\end{conjecture}
The Generalized Riemann Hypothesis leads to even stronger for primes in short intervals. Namely, fix an integer $q$, we have that \[
\sum_{\substack{n\leq N\\ n\equiv a\mod q}}\Lambda(n) = \begin{cases}
    \frac{1}{\phi(q)} N + O(x^{2+o(1)}), & \textrm{if $\gcd(a,q)=1$}\\
    o(n),& \textrm{otherwise.}
\end{cases}
\]
This means that not only that the Prime Number Theorem holds in intervals of $x^{2+\epsilon}$, the distribution of primes in each of the residual classes (coprime to $q$) are uniform at this scale too. Noticing that we can modify Huxley's proof with \[
M_{x,\chi} = \sum_{n\leq x} \chi(n) \mu(n) n^{-s},
\]
we have \[
L(s,\chi)M_{x,\chi} = \sum_{n} a_n \chi(n) \mu(n) n^{-s}.
\]

Thus, we can reproduce a similar proof on the zero density of $L$-functions. 
\begin{definition}[Zero Density for $L$-functions]
    Let $N(\sigma,\chi,T)$ denote the number of zeros of the $L$-function $L(-,\chi)$ with real part greater than $\sigma$ and imaginary part between $-T$ and $-T$. That is \[
    N(\sigma,\chi,T) \defeq \#\{\rho=\beta+i\gamma \ | \ \beta\geq \sigma, |\gamma|\leq T\}.
    \]
    For backwards compatibility with our previous definition, we take $N(\sigma,T)\defeq N(\sigma, 1,T)$.
\end{definition}
The hybrid analogs of the zero density bounds of Ingham and Huxley are known.
\begin{theorem}[Hybrid Ingham bound for zero density]
    Let $1/2\leq \sigma\leq 3/4$. We have \[
        \sum_{\chi^* }N(\sigma,\chi^*,t)\lesssim (qT)^{\frac{3(1-\sigma)}{2-\sigma}},
        \]
        where $\sum_{\chi^*}$ sums over all the primitive characters $\chi^*$ of modulus $q$.
\end{theorem}
\begin{theorem}[Hybrid Huxley bound for zero density]
    Let $3/4\leq \sigma\leq 1$. We have \[
        \sum_{\chi^* }N(\sigma,\chi^*,t)\lesssim (qT)^{\frac{3(1-\sigma)}{3\sigma-1}},
        \]        
        where $\sum_{\chi^*}$ sums over all the primitive characters $\chi^*$ of modulus $q$.
\end{theorem}

The method for detecting zeros is very similar to Huxley's proof above with the slight change in definition of $M_{x,\chi}$. This argument then reduces to bounding the number of times large values of Dirichlet polynomials can occur. Therefore we want a result in the form:
\begin{quotation}  
Let $\mathcal{S}=\{(t_j,\chi_j)\}$ be a set such that each $\chi_j$ is a primitive Dirichlet character of modulus $q$, and $|t_j-t_k|\geq 1$ if $j\neq k$ and $\chi_j=\chi_k$. (That is, the $t$'s are $1$-separated if the characters are the same.)Let $|b_n|\leq 1$ be a sequence of numbers indexed in $n$, and suppose also that \[
\Bigg|\sum_{n\sim N} b_n\chi_j(n)n^{it_j} \Bigg|>V.
\]
We want to find a bound on $|\mathcal{S}|$.
\end{quotation}


Our result is as follows: 
\begin{theorem}
    To be determined \textrm{\smiley}.
\end{theorem}

The idea of the proof is similar to Guth and Maynard's proof. We can define a $|\mathcal{S}\times N|$ matrix $M$ with entries
\[
    M_{t_j,\chi_j,n} = \chi_j(n)n^{it_j}
\]
for $(t_j,\chi_j)\in\mathcal{S}$ and $n\sim N$, and bound its operator norm in the exact same way: taking it to the $M^*M$ to the third power and calculating its trace. The Dirichlet characters are not nice to handle when extended to a function $\reals \to \complex$, so we instead break the sum into $q$ summations across each residue class $\mod q$, and apply Poisson summation on each piece of the sum. This gives us \begin{align*}
    \textrm{tr}((M^*M)^3)=\sum_{\substack{(t_1,\chi_1),\\(t_2,\chi_2),\\(t_3,\chi_3)\in\mathcal{S}}}
    \frac{N^3}{q^3}\sum_{m\in\mathbb{Z}^3}&\sum_{x\in (\mathbb{Z}/q\mathbb{Z})^3}\chi_1\bar{\chi}_3(x_1)\chi_2\bar{\chi}_1(x_2)\chi_3\bar{\chi}_2(x_3) e\left(\frac{-x\cdot m}{q}\right)\\
    \times \ &\hat{h}_{t_1-t_3}\left(\frac{Nm_1}{q}\right)\hat{h}_{t_2-t_1}\left(\frac{Nm_2}{q}\right)\hat{h}_{t_3-t_2}\left(\frac{Nm_3}{q}\right),
\end{align*}
where $h_t(x) \defeq \omega(u)^2 x^{it}$ has the same definition, thus its fourier transform has the same properties in decay. Similarly, we find that when we break the sum in $\mathbb{Z}^3$ into $S_0$ to $S_3$ in the same way, $S_0$ gives the main term, $S_1$ is negligible, and $S_2$ can be bounded by Heath-Brown's theorem. For the $S_3$ bound, we redefine \[
R(v,n_1,n_2)\defeq \sum_{(t,\chi)\in\mathcal{S}}\chi(n_1)\bar{\chi}(n_2) v^{it}.
\]
This additional structure in $\chi$ may look complicated when taking the $L_2$ or $L_4$ norm of $R$ in $v\asymp 1$, but this cancels out when taking \[
\sum_{n_1\in\mathbb{Z}/q\mathbb{Z}} R(v,n_1,n_2)^2 = \sum_{n_1\in\mathbb{Z}/q\mathbb{Z}}\sum_{\substack{(t_1,\chi_1),(t_2,\chi_2)\in\mathcal{S}}}\chi_1(n_1)\bar\chi_2(n_1)\bar\chi_1(n_2)\chi_2(n_2) v^{i(t_1-t_2)}.
\]
By the orthogonality of Dirichlet characters, the sum in $n_1$ vanishes except when $\chi_1=\chi_2$, in which case (assuming $\gcd(n_2,q)=1$) we can reduce \[
    \sum_{n_1\in\mathbb{Z}/q\mathbb{Z}} R(v,n_1,n_2)^2 =\phi(q) \sum_{\substack{(t_1,\chi_1),(t_2,\chi_2)\in\mathcal{S}\\\chi_1=\chi_2}}v^{i(t_1-t_2)}.
\]
When $\chi_1=\chi_2$, we have $t_1=t_2$ or they are $T^\epsilon$ separated, giving us decay in the calculation of the second moment of $R$ to be $\phi(q)|\mathcal{S}|$.
Similarly, we see the fourth power of $R$ cancels in the sum \[
    \sum_{n_1\in\mathbb{Z}/q\mathbb{Z}} R(v,n_1,n_2)^4 = \sum_{n_1\in\mathbb{Z}/q\mathbb{Z}}\sum_{\substack{(t_1,\chi_1),(t_2,\chi_2),\\(t_3,\chi_3),(t_4,\chi_4)\\\in\mathcal{S}}}\chi_1\chi_2\bar\chi_3\bar\chi_4(n_1)\bar\chi_1\bar\chi_2\chi_3\chi_4(n_2) v^{i(t_1+t_2-t_3-t_4)}.
\]
Only terms where $\chi_1\chi_2=\chi_3\chi_4$ can have non-zero contribution, giving us a natural definition for the energy of the set to be \[
E(\mathcal{S})\defeq \#\{(t_1,\chi_1),(t_2,\chi_2),(t_3,\chi_3),(t_4,\chi_4)\in \mathcal{S} \ | \ \chi_1\chi_2=\chi_3\chi_4, |t_1+t_2-t_3-t_4|\lesssim 1\}.
\]
This in turn can be bounded by a third moment of $R$: \[
 E(\mathcal{S})\lesssim N^{-2\sigma}\sum_{n_1,n_2\sim N}\left|R\left(\frac{n_1}{n_2},n_1,n_2\right) \right|^3.\]
 This expression is then bounded by Heath-Brown's theorem.

\section{Setup and Reduction of Theorem}
\begin{lemma}\label{reduction}
    Let $\mathcal{S}=\{(t_j,\chi_j)\}$ be a set such that each $\chi_j$ is a primitive Dirichlet character of modulus $q$, and $|t_j-t_k|\geq T^\epsilon$ if $j\neq k$ and $\chi_j=\chi_k$. Let $|b_n|\leq 1$ be a sequence of numbers indexed in $n$, $\omega$ be a smooth bump function that equals $1$ on $[6/5,9/5]$ and has support in $[1,2]$ (thus $\omega^{(A)}\ll_A 1$ for all $A$). Let $V=N^\sigma$, where $\sigma \in [placeholder]$, and $N=qT^\alpha$, where $\alpha \in [placeholder]$, and $q^{1+\epsilon}\ll N,T$. Suppose also that \[
    |D_N(t_j,\chi_j)|\defeq\Bigg|\sum_{n\sim N} \omega\Bigg(\frac{n}{N}\Bigg)b_n\chi_j(n)n^{it_j} \Bigg|>V.
\]
for all $(t_j,\chi_j)\in\mathcal{S}$.
Then \[
|\mathcal{S}|\lesssim \smiley.
\]
\end{lemma}
\begin{proof}[Proof of main theorem from Lemma \ref{reduction}]
    We split \[
        \sum_{n\sim N} b_n\chi_j(n)n^{it_j}=\sum_{N\leq n< 6N/5} b_n\chi_j(n)n^{it_j} +\sum_{6N/5 \leq n < 9N/5} b_n\chi_j(n)n^{it_j} +\sum_{9N/5 \leq n < 2N} b_n\chi_j(n)n^{it_j}.
    \]
    So that if \[
        \Big|\sum_{n\sim N} b_n\chi_j(n)n^{it_j}\Big|>V,
    \]
    then at least one of the three summations in the right has magnitude at least $V/3$.
    But then each of these sums can be written as dirichlet sums in the form of the lemma: \begin{align*}
        \sum_{N\leq n< 6N/5} b_n\chi_j(n)n^{it_j} &=  \sum_{N\leq n< 6N/5} \omega\Bigg(\frac{n}{11N/15}\Bigg) b_n\chi_j(n)n^{it_j}, \\
        \sum_{6N/5 \leq n < 9N/5} b_n\chi_j(n)n^{it_j} &=  \sum_{6N/5 \leq n < 9N/5} \omega\Bigg(\frac{n}{N}\Bigg) b_n\chi_j(n)n^{it_j}, \\
        \sum_{9N/5 \leq n < 2N}  b_n\chi_j(n)n^{it_j} &=  \sum_{9N/5 \leq n < 2N}  \omega\Bigg(\frac{n}{19N/15}\Bigg) b_n\chi_j(n)n^{it_j}. \\
    \end{align*}
    Applying the lemma on each of these three pieces and summing the three bounds gives the same bound for $\sum_{n\sim N} b_n\chi_j(n)n^{it_j}$ up to a constant factor.
\end{proof}
\section{The matrix $M$ and its trace expansion}
We define $M$ a $|\mathcal{S}\times N|$ matrix with entries
\[
    M_{(t_j,\chi_j),n} = \omega\Big(\frac{n}{N}\Big)\chi_j(n)n^{it_j}
\]
for $(t_j,\chi_j)\in\mathcal{S}$ and $n\sim N$.
Thus by the same reasoning that $(M\vec{b})_j=D_N(t_j,\chi_j)$,
we want to bound the size of $\mathcal{S}$ by trace of the matrix \[
\textrm{tr}((M^*M)^3).
\].

\begin{proposition} \label{setup_poisson}
    Let  \begin{align*}
        I_m\defeq\frac{N^3}{q^3}\sum_{\substack{(t_1,\chi_1),\\(t_2,\chi_2),\\(t_3,\chi_3)\in\mathcal{S}}} &\sum_{x\in (\mathbb{Z}/q\mathbb{Z})^3}\chi_1\bar{\chi}_3(x_1)\chi_2\bar{\chi}_1(x_2)\chi_3\bar{\chi}_2(x_3) e\left(\frac{-x\cdot m}{q}\right)\\
        \times \ &\hat{h}_{t_1-t_3}\left(\frac{Nm_1}{q}\right)\hat{h}_{t_2-t_1}\left(\frac{Nm_2}{q}\right)\hat{h}_{t_3-t_2}\left(\frac{Nm_3}{q}\right).
    \end{align*}
    We have \[
    |\mathcal{S}|\lesssim \frac{\phi(q)}{q}N^2V^{-2}+ NV^{-2}\left(\sum_{m\in\mathbb{Z}^3 - \{0\}} I_m\right)^{1/3}.
    \]
\end{proposition}
\begin{lemma}[Explicit Expression of Traces]
    Let \[
    P(n,(t_1,\chi_1),(t_2,\chi_2))\defeq \omega\left(\frac{n}{N}\right)^2 \left(\frac{n}{N}\right)^{i(t_2-t_1)}\chi_2\bar{\chi}_1(n).
    \]
    We have 
    \begin{align*}
        \rm{tr}(M^*M) = & \ |\mathcal{S}|\sum_{n\sim N, \gcd (n,q)=1} \omega\left(\frac{n}{N}\right)^2, \\
        \rm{tr}((M^*M)^3)=&\sum_{\substack{(t_1,\chi_1),(t_2,\chi_2),\\(t_3,\chi_3)\in\mathcal{S}}}\sum_{n_1,n_2,n_3\sim N} 
        P(n_1,(t_1,\chi_1),(t_3,\chi_3))P(n_2,(t_2,\chi_2),(t_1,\chi_1))P(n_3,(t_3,\chi_3),(t_2,\chi_2)).
    \end{align*}
\end{lemma}
\begin{proof}
    We compute the traces of $MM^*$ and $(MM^*)^3$ instead, as this makes the calculations cleaner.
    We see that \begin{align*}
        (MM^*)_{(t_j,\chi_j),(t_k,\chi_k)} = \sum_{n\sim N} \omega\left(\frac{n}{N}\right)^2 n^{i(t_k-t_j)}\bar{\chi}_j\chi_k(n) = \sum_{n\sim N}P(n,(t_k,\chi_k),(t_j,\chi_j))N^{i(t_k-t_j)},
    \end{align*}
    so that \begin{align*}
        \textrm{tr}(MM^*) =& \sum_{(t_j,\chi_j)} \sum_{n\sim N} P(n,(t_j,\chi_j),(t_j,\chi_j)) \\
        =&|\mathcal{S}|\sum_{n\sim N} |\chi_j(n)|\omega\left(\frac{n}{N}\right)^2.
    \end{align*}
    For the third power, we have 
    \begin{align*}
        (MM^*)^3_{(t_j,\chi_j),(t_k,\chi_k)} = \sum_{(t_1,\chi_1),(t_2,\chi_2)\in \mathcal{S}}   (MM^*)_{(t_j,\chi_j),(t_1,\chi_1)} (MM^*)_{(t_1,\chi_1),(t_2,\chi_2)} (MM^*)_{(t_2,\chi_2),(t_k,\chi_k)},
    \end{align*}
    so that the trace is given by \begin{align*}
        &\sum_{\substack{(t_1,\chi_1),(t_2,\chi_2),\\(t_3,\chi_3)\in\mathcal{S}}} (MM^*)_{(t_3,\chi_3),(t_1,\chi_1)} (MM^*)_{(t_1,\chi_1),(t_2,\chi_2)} (MM^*)_{(t_2,\chi_2),(t_3,\chi_3)}\\
        =&\sum_{\substack{(t_1,\chi_1),(t_2,\chi_2),\\(t_3,\chi_3)\in\mathcal{S}}} \sum_{n_1,n_2,n_3\sim N} 
        P(n_1,(t_1,\chi_1),(t_3,\chi_3))P(n_2,(t_2,\chi_2),(t_1,\chi_1))P(n_3,(t_3,\chi_3),(t_2,\chi_2)).
    \end{align*}
\end{proof}
We would like to apply Poisson summation directly to $P$ in $n_1$, $n_2$, $n_3$, respectively. However, we do not have a good way to express the Dirichlet character $\chi_j\bar\chi_k$ as a $C^\infty$ function. To get around this, we just split the sum in $n$ into $q$ pieces. 
\begin{lemma}[Poisson Summation with Periodic Arithmetic Function]
    Let $f:\reals \to \complex$ be Schwartz, and $g:\naturals\to\complex$ be an arithmetic with period $q$. Then \[
    \sum_n g(n)f(n) = \frac{1}{q}\sum_{\xi} \sum_{x \mod q} g(x) e\bigg(\frac{-x\xi}{q}\bigg)\hat{f}\bigg(\frac{\xi}{q}\bigg). 
    \]
\end{lemma}
\begin{proof}
    By absolute convergence, we have \begin{align*}
        \sum_n g(n)f(n) = \sum_n \sum_{x\mod q} g(qn+x)f(qn+x)=  \sum_{x\mod q} g(x)\sum_n f(qn+x).
    \end{align*}
    Applying Poisson summation to the last expression in $n$ gives \[
        \frac{1}{q} \sum_{x\mod q} g(x) \sum_{\xi} e\bigg(\frac{-x\xi}{q}\bigg)\hat{f}\bigg(\frac{\xi}{q}\bigg).
    \]
\end{proof}
\begin{corollary}[Poisson summation of Trace]
    Let $h_t(u)\defeq\omega(u)^2u^{it}$,
    Then \begin{align*}
        \rm{tr}(M^*M) &= |\mathcal{S}|\frac{N}{q}\sum_{m}\sum_{x\in \mathbb{Z}/q\mathbb{Z}}e\left(\frac{-xm}{q}\right)\hat{h}_0\left(\frac{Nm}{q}\right)\\
        \rm{tr}((M^*M)^3)&=\sum_{\substack{(t_1,\chi_1),\\(t_2,\chi_2),\\(t_3,\chi_3)\in\mathcal{S}}}
        \frac{N^3}{q^3}\sum_{m\in\mathbb{Z}^3}\sum_{x\in (\mathbb{Z}/q\mathbb{Z})^3}\chi_1\bar{\chi}_3(x_1)\chi_2\bar{\chi}_1(x_2)\chi_3\bar{\chi}_2(x_3) e\left(\frac{-x\cdot m}{q}\right)\\
        & \quad \quad \quad \times \ \hat{h}_{t_1-t_3}\left(\frac{Nm_1}{q}\right)\hat{h}_{t_2-t_1}\left(\frac{Nm_2}{q}\right)\hat{h}_{t_3-t_2}\left(\frac{Nm_3}{q}\right).
    \end{align*}
\end{corollary}
In the trace of $(M^*M)^3$, we split the sum in $m\defeq (m_1,m_2,m_3)$ according to the number of non-zero entries into four parts (same as Guth Maynard Sketch above) $S_0+ S_1+S_2+S_3$,
where $S_j$ runs over the values of $m$ with exactly $j$ non-zero entries.

\begin{lemma}[Isolation of main term]
    We have \begin{align*}
        \rm{tr}(M^*M) &= \frac{|\mathcal{S}|N\phi(q)}{q}\|\omega\|_{L_2}^2 + O(T^{-100}),\\
        S_0 &= \frac{|\mathcal{S}|N^3\phi(q)^3}{q^3}\|\omega\|_{L_2}^6+O(T^{-100}).
    \end{align*}
\end{lemma}
\begin{proof}
    By \hyperref[nonstationary]{Non-Stationary Phase}, we have rapid decay of $h_0(\xi)$, so we can only consider the terms with $m=0$ in the sum for $\rm{tr}(M^*M)$ at a cost of $O(T^{-100})$.
    This $m=0$ term evaluates to \[
        \frac{|\mathcal{S}|N}{q}\sum_{x\in \mathbb{Z}/q\mathbb{Z}}e\left(0\right)\hat{h}_0\left(0\right)=\frac{|\mathcal{S}|N\phi(q)}{q}\|\omega\|_{L_2}^2.
    \]
    For $S_0$, we have\[
    S_0 = \sum_{\substack{(t_1,\chi_1),\\(t_2,\chi_2),\\(t_3,\chi_3)\in\mathcal{S}}}
    \frac{N^3}{q^3}\sum_{x\in (\mathbb{Z}/q\mathbb{Z})^3}\chi_1\bar{\chi}_3(x_1)\chi_2\bar{\chi}_1(x_2)\chi_3\bar{\chi}_2(x_3)  \hat{h}_{t_1-t_3}\left(0\right)\hat{h}_{t_2-t_1}\left(0\right)\hat{h}_{t_3-t_2}\left(0\right).
    \]
    By the orthogonality of characters (as representations of $(\mathbb{Z}/q\mathbb{Z})^{\times}$),
    only the terms with $\chi_1=\chi_2=\chi_3$ can have non-zero contribution to the sum. But when $\chi_1=\chi_2=\chi_3$, $|t_i-t_j|=0$ or $|t_i-t_j|\geq T^{\epsilon}$. In the second case, we can apply \hyperref[nonstationary]{Non-Stationary Phase} to get rapid decay in $\hat{h}_{t_i-t_j}({0})$. Therefore \[
        S_0 = \sum_{(t,\chi)\in\mathcal{S}}
        \frac{N^3}{q^3}\sum_{x\in (\mathbb{Z}/q\mathbb{Z})^3} \hat{h}_{0}\left(0\right)^3+O(T^{-100})=\frac{|\mathcal{S}|N^3\phi(q)^3}{q^3}\|\omega\|_{L_2}^6+O(T^{-100}).
    \]
\end{proof}

\begin{proof}[Proof of Proposition \ref{setup_poisson}]
    We note that the inequality in equation \ref{basicineq} applies in this case, and apply Lemma \ref{gmtrace} and substitute the values we have obtained.
\end{proof}