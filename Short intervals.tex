\section{Primes in Short Intervals}
We would like to answer the following question about primes in short intervals. Let $y=y(x)$. What is the smallest asymptotic behavior of $y$
such that \begin{equation}\label{shortintervalpnt}
\sum_{x\leq n \leq x+y} \Lambda(n) \sim y
\end{equation}
for large enough $x$? That is, what is the shortest interval such that we have the behavior of the Prime Number Theorem? If \ref{shortintervalpnt} holds for some $y$,
we say the Prime Number Theorem holds for intervals of $y$.
\begin{remark}
    This question can be rephrased into finding primes in short intervals, by including a factor of $\log x$. 
\end{remark} 
\begin{proposition}
    Assume the RH. Then the Prime Number Theorem holds in intervals of $x^{1/2+\epsilon}$.
\end{proposition}
\begin{proof}
    Assume the RH, then
    \begin{align*}
        \sum_{x\leq n \leq x+y} \Lambda(n)=
        y+O(x^{1/2}\log^2 x) = x^{1/2+\epsilon} + o(x^{1/2+\epsilon}),
    \end{align*}
    so that the sum is non-zero for large enough $x$.
\end{proof}
Recalling that the error term is related to the real part of the zeros of the Zeta function, we motivate the following definition of zero-density:
\begin{definition}
    Let $N(\sigma, T)$ denote the number of zeros of the zeta function with real part greater than $\sigma$ and imaginary part between $-T$ and $T$. That is,\[
        N(\sigma,T) = \# \{\rho = \beta + i\gamma \ | \ \beta >\sigma, |\gamma|\leq T\}.
    \]
\end{definition}
\begin{remark}
    The ideal scenario is that $N(\sigma,T)=0$ for all $\sigma\geq 1/2$. 
\end{remark}
\begin{theorem}[Chudakov] 
    There exists a constant $A$ such that $\zeta(\sigma+iT)\neq 0$ in the region \begin{equation*}
        \sigma > 1 - A\frac{\log \log T}{\log T}.
    \end{equation*}
\end{theorem}
\textit{add reference}
\begin{theorem}[Hoheisel] \label{Hoheisel}
    Let $A$ be defined as in the previous theorem.
    Suppose that $N(\sigma, T)\ll T^{a(1-\sigma)}\log^b T$ uniformly in $1/2<\sigma<1$ and in $T$. Then for all \[
        \theta > 1 - \frac{1}{a+b/A},
    \] the Prime Number Theorem holds in 
    intervals of $y=x^\theta$.
\end{theorem}

\textit{here give proof}
Theorem \ref{Hoheisel} gives the classic estimate for 

huxley zeta, then generalized guth maynard to hybridized 